\documentclass[aspectratio=169]{beamer}

%\includeonlyframes{current}

\usepackage[utf8]{inputenc}
\usepackage[american]{babel}
\usepackage{amsmath,amsthm}
\usepackage{unicode}
\usepackage{array,tabularx}
\usepackage{ifthen}
\usepackage{tikz}
\usetikzlibrary{matrix,decorations,decorations.text,calc,arrows,snakes,shapes,positioning}
\usepackage{tikzsymbols}
\usepackage[backend=biber,citestyle=authoryear-comp,bibstyle=beamer,doi=false,isbn=false,url=true,maxnames=10]{biblatex}
\bibliography{refs}

\usepackage{ulem}

\mode<presentation>{%
  \usetheme{ibm}
}

\newcommand{\C}{ℂ}
\newcommand{\R}{ℝ}
\newcommand{\Z}{ℤ}
\newcommand{\N}{ℕ}
\newcommand{\Q}{ℚ}
\newcommand{\F}{\mathbb{F}}
\renewcommand{\P}{\mathbb{P}}
\renewcommand{\O}{\mathcal{O}}
\newcommand{\tildO}{\mathcal{\tilde{O}}}
\newcommand{\poly}{\operatorname{poly}}
\newcommand{\polylog}{\operatorname{polylog}}
\newcommand{\rank}{\operatorname{rank}}
\newcommand{\End}{\operatorname{End}}
\newcommand{\Hom}{\operatorname{Hom}}
\newcommand{\Gal}{\operatorname{Gal}}
\newcommand{\chr}{\operatorname{char}}
\newcommand{\Cl}{\operatorname{Cl}}
\newcommand{\GL}{\operatorname{GL}}
\renewcommand{\a}{{\mathfrak{a}}}
\renewcommand{\b}{{\mathfrak{b}}}
\newcommand{\p}{{\mathfrak{p}}}
\newcommand{\q}{{\mathfrak{q}}}
\newcommand{\g}{{\mathfrak{g}}}
\newcommand{\G}{{\mathcal{G}}}
\newcommand{\E}{{\mathcal{E}}}
\newcommand{\cyc}[1]{{〈 #1 〉}}
\newcommand{\ord}{\operatorname{ord}}
\newcommand{\mat}[1]{\left(\begin{smallmatrix}#1\end{smallmatrix}\right)}
\newcommand{\from}{\overset{\$}{\leftarrow}}

\newcommand{\bl}[1]{\textcolor{blue}{#1}}
\newcommand{\rd}[1]{\textcolor{red}{#1}}
\newcommand{\gr}[1]{\textcolor{green}{#1}}
\newcommand{\og}[1]{\textcolor{orange}{#1}}

\definecolor{light blue}{RGB}{0,102,204}

\newcommand{\myedge}[3]{
  \draw[#3] (360/\crater*#1 : \diam) to[bend right] (360/\crater*#2 : \diam);
}
\newcommand{\sk}[4]{
  \draw[very thick,blue]   (0,0) -- (0,#1);
  \draw[very thick,red]    (1,0) -- (1,#2);
  \draw[very thick,green]  (2,0) -- (2,#3);
  \draw[very thick,orange] (3,0) -- (3,#4);
}
\newcommand{\axes}[4]{
  \clip (#1,#3) rectangle (#2,#4);
  \draw [thin, gray, -latex] (#1,0) -- (#2,0);% Draw x axis
  \draw [thin, gray, -latex] (0,#3) -- (0,#4);% Draw y axis
}
\pgfkeys{/lattice/.code n args={4}{\tikzset{cm={#1,#2,#3,#4,(0,0)}}}}
\newcommand{\lattice}[2]{
  \draw[style=help lines,dashed] (#1-1,#1-1) grid[step=1] (#2+1,#2+1);
  \foreach \x in {#1,...,#2}{
    \foreach \y in {#1,...,#2}{
      \node[draw,circle,inner sep=2pt,fill] at (\x,\y) {};
      % Places a dot at those points
    }
  }
}

\newenvironment<>{goodblock}[1]{%
  \begin{actionenv}#2%
      \def\insertblocktitle{#1}%
      \par%
      \mode<presentation>{%
        \setbeamercolor{block title}{fg=green!60!black,bg=green!50!white}
       \setbeamercolor{block body}{bg=green!20!white}
     }%
      \usebeamertemplate{block begin}}
    {\par\usebeamertemplate{block end}\end{actionenv}}
\newenvironment<>{mehblock}[1]{%
  \begin{actionenv}#2%
      \def\insertblocktitle{#1}%
      \par%
      \mode<presentation>{%
        \setbeamercolor{block title}{fg=yellow!50!black,bg=yellow!50!white}
       \setbeamercolor{block body}{bg=yellow!20!white}
     }%
      \usebeamertemplate{block begin}}
    {\par\usebeamertemplate{block end}\end{actionenv}}
\newenvironment<>{badblock}[1]{%
  \begin{actionenv}#2%
      \def\insertblocktitle{#1}%
      \par%
      \mode<presentation>{%
        \setbeamercolor{block title}{fg=red!60!black,bg=red!50!white}
       \setbeamercolor{block body}{bg=red!20!white}
     }%
      \usebeamertemplate{block begin}}
    {\par\usebeamertemplate{block end}\end{actionenv}}


\newcommand\isoggraph{
    \node[circle,inner sep=.7pt,fill=black] (curve0000) at (4.1000,0.0000) {};
    \node[circle,inner sep=.7pt,fill=black] (curve0158) at (3.9895,0.9455) {};
    \node[circle,inner sep=.7pt,fill=black] (curve0410) at (3.6639,1.8401) {};
    \node[circle,inner sep=.7pt,fill=black] (curve0368) at (3.1408,2.6354) {};
    \node[circle,inner sep=.7pt,fill=black] (curve0404) at (2.4484,3.2887) {};
    \node[circle,inner sep=.7pt,fill=black] (curve0075) at (1.6239,3.7647) {};
    \node[circle,inner sep=.7pt,fill=black] (curve0144) at (0.7120,4.0377) {};
    \node[circle,inner sep=.7pt,fill=black] (curve0191) at (-0.2384,4.0931) {};
    \node[circle,inner sep=.7pt,fill=black] (curve0174) at (-1.1759,3.9278) {};
    \node[circle,inner sep=.7pt,fill=black] (curve0413) at (-2.0500,3.5507) {};
    \node[circle,inner sep=.7pt,fill=black] (curve0379) at (-2.8136,2.9822) {};
    \node[circle,inner sep=.7pt,fill=black] (curve0124) at (-3.4255,2.2530) {};
    \node[circle,inner sep=.7pt,fill=black] (curve0199) at (-3.8527,1.4023) {};
    \node[circle,inner sep=.7pt,fill=black] (curve0390) at (-4.0723,0.4760) {};
    \node[circle,inner sep=.7pt,fill=black] (curve0029) at (-4.0723,-0.4760) {};
    \node[circle,inner sep=.7pt,fill=black] (curve0220) at (-3.8527,-1.4023) {};
    \node[circle,inner sep=.7pt,fill=black] (curve0295) at (-3.4255,-2.2530) {};
    \node[circle,inner sep=.7pt,fill=black] (curve0040) at (-2.8136,-2.9822) {};
    \node[circle,inner sep=.7pt,fill=black] (curve0006) at (-2.0500,-3.5507) {};
    \node[circle,inner sep=.7pt,fill=black] (curve0245) at (-1.1759,-3.9278) {};
    \node[circle,inner sep=.7pt,fill=black] (curve0228) at (-0.2384,-4.0931) {};
    \node[circle,inner sep=.7pt,fill=black] (curve0275) at (0.7120,-4.0377) {};
    \node[circle,inner sep=.7pt,fill=black] (curve0344) at (1.6239,-3.7647) {};
    \node[circle,inner sep=.7pt,fill=black] (curve0015) at (2.4484,-3.2887) {};
    \node[circle,inner sep=.7pt,fill=black] (curve0051) at (3.1408,-2.6354) {};
    \node[circle,inner sep=.7pt,fill=black] (curve0009) at (3.6639,-1.8401) {};
    \node[circle,inner sep=.7pt,fill=black] (curve0261) at (3.9895,-0.9455) {};
    %
    \draw[color=blue] (curve0199) edge[bend right=4mm] (curve0390);
    \draw[color=blue] (curve0051) edge[bend right=4mm] (curve0009);
    \draw[color=blue] (curve0368) edge[bend right=4mm] (curve0404);
    \draw[color=blue] (curve0245) edge[bend right=4mm] (curve0228);
    \draw[color=blue] (curve0029) edge[bend right=4mm] (curve0220);
    \draw[color=blue] (curve0174) edge[bend right=4mm] (curve0413);
    \draw[color=blue] (curve0261) edge[bend right=4mm] (curve0000);
    \draw[color=blue] (curve0379) edge[bend right=4mm] (curve0124);
    \draw[color=blue] (curve0006) edge[bend right=4mm] (curve0245);
    \draw[color=blue] (curve0158) edge[bend right=4mm] (curve0410);
    \draw[color=blue] (curve0228) edge[bend right=4mm] (curve0275);
    \draw[color=blue] (curve0275) edge[bend right=4mm] (curve0344);
    \draw[color=blue] (curve0015) edge[bend right=4mm] (curve0051);
    \draw[color=blue] (curve0191) edge[bend right=4mm] (curve0174);
    \draw[color=blue] (curve0144) edge[bend right=4mm] (curve0191);
    \draw[color=blue] (curve0404) edge[bend right=4mm] (curve0075);
    \draw[color=blue] (curve0009) edge[bend right=4mm] (curve0261);
    \draw[color=blue] (curve0295) edge[bend right=4mm] (curve0040);
    \draw[color=blue] (curve0410) edge[bend right=4mm] (curve0368);
    \draw[color=blue] (curve0413) edge[bend right=4mm] (curve0379);
    \draw[color=blue] (curve0040) edge[bend right=4mm] (curve0006);
    \draw[color=blue] (curve0075) edge[bend right=4mm] (curve0144);
    \draw[color=blue] (curve0220) edge[bend right=4mm] (curve0295);
    \draw[color=blue] (curve0390) edge[bend right=4mm] (curve0029);
    \draw[color=blue] (curve0344) edge[bend right=4mm] (curve0015);
    \draw[color=blue] (curve0124) edge[bend right=4mm] (curve0199);
    \draw[color=blue] (curve0000) edge[bend right=4mm] (curve0158);
    %
    \draw[color=red] (curve0009) edge[bend right=8mm] (curve0379);
    \draw[color=red] (curve0124) edge[bend right=8mm] (curve0015);
    \draw[color=red] (curve0006) edge[bend right=8mm] (curve0368);
    \draw[color=red] (curve0228) edge[bend right=8mm] (curve0075);
    \draw[color=red] (curve0245) edge[bend right=8mm] (curve0404);
    \draw[color=red] (curve0029) edge[bend right=8mm] (curve0261);
    \draw[color=red] (curve0220) edge[bend right=8mm] (curve0000);
    \draw[color=red] (curve0368) edge[bend right=8mm] (curve0220);
    \draw[color=red] (curve0144) edge[bend right=8mm] (curve0006);
    \draw[color=red] (curve0261) edge[bend right=8mm] (curve0124);
    \draw[color=red] (curve0191) edge[bend right=8mm] (curve0245);
    \draw[color=red] (curve0015) edge[bend right=8mm] (curve0174);
    \draw[color=red] (curve0344) edge[bend right=8mm] (curve0191);
    \draw[color=red] (curve0275) edge[bend right=8mm] (curve0144);
    \draw[color=red] (curve0158) edge[bend right=8mm] (curve0390);
    \draw[color=red] (curve0295) edge[bend right=8mm] (curve0158);
    \draw[color=red] (curve0075) edge[bend right=8mm] (curve0040);
    \draw[color=red] (curve0174) edge[bend right=8mm] (curve0228);
    \draw[color=red] (curve0000) edge[bend right=8mm] (curve0199);
    \draw[color=red] (curve0379) edge[bend right=8mm] (curve0344);
    \draw[color=red] (curve0390) edge[bend right=8mm] (curve0009);
    \draw[color=red] (curve0040) edge[bend right=8mm] (curve0410);
    \draw[color=red] (curve0410) edge[bend right=8mm] (curve0029);
    \draw[color=red] (curve0404) edge[bend right=8mm] (curve0295);
    \draw[color=red] (curve0413) edge[bend right=8mm] (curve0275);
    \draw[color=red] (curve0051) edge[bend right=8mm] (curve0413);
    \draw[color=red] (curve0199) edge[bend right=8mm] (curve0051);
    %
    \draw[color=green] (curve0158) edge[bend left=10mm] (curve0144);
    \draw[color=green] (curve0009) edge[bend left=10mm] (curve0368);
    \draw[color=green] (curve0124) edge[bend left=10mm] (curve0295);
    \draw[color=green] (curve0174) edge[bend left=10mm] (curve0390);
    \draw[color=green] (curve0368) edge[bend left=10mm] (curve0174);
    \draw[color=green] (curve0344) edge[bend left=10mm] (curve0000);
    \draw[color=green] (curve0144) edge[bend left=10mm] (curve0124);
    \draw[color=green] (curve0295) edge[bend left=10mm] (curve0275);
    \draw[color=green] (curve0029) edge[bend left=10mm] (curve0245);
    \draw[color=green] (curve0051) edge[bend left=10mm] (curve0410);
    \draw[color=green] (curve0015) edge[bend left=10mm] (curve0158);
    \draw[color=green] (curve0275) edge[bend left=10mm] (curve0261);
    \draw[color=green] (curve0075) edge[bend left=10mm] (curve0379);
    \draw[color=green] (curve0379) edge[bend left=10mm] (curve0220);
    \draw[color=green] (curve0220) edge[bend left=10mm] (curve0228);
    \draw[color=green] (curve0006) edge[bend left=10mm] (curve0015);
    \draw[color=green] (curve0191) edge[bend left=10mm] (curve0199);
    \draw[color=green] (curve0228) edge[bend left=10mm] (curve0009);
    \draw[color=green] (curve0040) edge[bend left=10mm] (curve0344);
    \draw[color=green] (curve0261) edge[bend left=10mm] (curve0404);
    \draw[color=green] (curve0404) edge[bend left=10mm] (curve0413);
    \draw[color=green] (curve0390) edge[bend left=10mm] (curve0006);
    \draw[color=green] (curve0000) edge[bend left=10mm] (curve0075);
    \draw[color=green] (curve0245) edge[bend left=10mm] (curve0051);
    \draw[color=green] (curve0413) edge[bend left=10mm] (curve0029);
    \draw[color=green] (curve0199) edge[bend left=10mm] (curve0040);
    \draw[color=green] (curve0410) edge[bend left=10mm] (curve0191);
}

\newcommand\fpsqgraph{
    \node[circle,inner sep=.7pt,fill=black] (j_190_344i) at (1.0,0.0) {};
    \node[circle,inner sep=.7pt,fill=black] (j_379_325i) at (0.985615910348,0.169000820322) {};
    \node[circle,inner sep=.7pt,fill=black] (j_143_000i) at (0.942877445461,0.333139794742) {};
    \node[circle,inner sep=.7pt,fill=black] (j_304_364i) at (0.873014113161,0.487694943814) {};
    \node[circle,inner sep=.7pt,fill=black] (j_356_000i) at (0.778035754318,0.628219997296) {};
    \node[circle,inner sep=.7pt,fill=black] (j_004_000i) at (0.66067472339,0.750672305253) {};
    \node[circle,inner sep=.7pt,fill=black] (j_242_000i) at (0.524307283557,0.851529137733) {};
    \node[circle,inner sep=.7pt,fill=black] (j_234_000i) at (0.37285647778,0.927889027297) {};
    \node[circle,inner sep=.7pt,fill=black] (j_065_081i) at (0.210679269996,0.977555238948) {};
    \node[circle,inner sep=.7pt,fill=black] (j_118_209i) at (0.0424412031961,0.999098966205) {};
    \node[circle,inner sep=.7pt,fill=black] (j_125_000i) at (-0.127017819747,0.991900435259) {};
    \node[circle,inner sep=.7pt,fill=black] (j_316_000i) at (-0.292822771277,0.956166734739) {};
    \node[circle,inner sep=.7pt,fill=black] (j_426_306i) at (-0.450203744818,0.89292585815) {};
    \node[circle,inner sep=.7pt,fill=black] (j_358_000i) at (-0.594633176304,0.803997130367) {};
    \node[circle,inner sep=.7pt,fill=black] (j_241_000i) at (-0.721956093955,0.691938868978) {};
    \node[circle,inner sep=.7pt,fill=black] (j_419_000i) at (-0.828509649244,0.559974786138) {};
    \node[circle,inner sep=.7pt,fill=black] (j_061_000i) at (-0.911228490388,0.411901248244) {};
    \node[circle,inner sep=.7pt,fill=black] (j_102_000i) at (-0.967732946933,0.251978061385) {};
    \node[circle,inner sep=.7pt,fill=black] (j_190_087i) at (-0.996397488543,0.0848059244755) {};
    \node[circle,inner sep=.7pt,fill=black] (j_107_000i) at (-0.996397488543,-0.0848059244755) {};
    \node[circle,inner sep=.7pt,fill=black] (j_304_067i) at (-0.967732946933,-0.251978061385) {};
    \node[circle,inner sep=.7pt,fill=black] (j_019_000i) at (-0.911228490388,-0.411901248244) {};
    \node[circle,inner sep=.7pt,fill=black] (j_381_000i) at (-0.828509649244,-0.559974786138) {};
    \node[circle,inner sep=.7pt,fill=black] (j_319_000i) at (-0.721956093955,-0.691938868978) {};
    \node[circle,inner sep=.7pt,fill=black] (j_065_350i) at (-0.594633176304,-0.803997130367) {};
    \node[circle,inner sep=.7pt,fill=black] (j_067_000i) at (-0.450203744818,-0.89292585815) {};
    \node[circle,inner sep=.7pt,fill=black] (j_000_000i) at (-0.292822771277,-0.956166734739) {};
    \node[circle,inner sep=.7pt,fill=black] (j_315_299i) at (-0.127017819747,-0.991900435259) {};
    \node[circle,inner sep=.7pt,fill=black] (j_422_000i) at (0.0424412031961,-0.999098966205) {};
    \node[circle,inner sep=.7pt,fill=black] (j_379_106i) at (0.210679269996,-0.977555238948) {};
    \node[circle,inner sep=.7pt,fill=black] (j_189_000i) at (0.37285647778,-0.927889027297) {};
    \node[circle,inner sep=.7pt,fill=black] (j_141_042i) at (0.524307283557,-0.851529137733) {};
    \node[circle,inner sep=.7pt,fill=black] (j_141_389i) at (0.66067472339,-0.750672305253) {};
    \node[circle,inner sep=.7pt,fill=black] (j_118_222i) at (0.778035754318,-0.628219997296) {};
    \node[circle,inner sep=.7pt,fill=black] (j_315_132i) at (0.873014113161,-0.487694943814) {};
    \node[circle,inner sep=.7pt,fill=black] (j_150_000i) at (0.942877445461,-0.333139794742) {};
    \node[circle,inner sep=.7pt,fill=black] (j_426_125i) at (0.985615910348,-0.169000820322) {};
    %
    \draw[color=blue] (j_319_000i) edge (j_426_125i);
    \draw[color=orange] (j_419_000i) edge (j_141_389i);
    \draw[color=orange] (j_065_081i) edge (j_190_087i);
    \draw[color=blue] (j_102_000i) edge (j_143_000i);
    \draw[color=orange] (j_102_000i) edge (j_358_000i);
    \draw[color=blue] (j_004_000i) edge (j_102_000i);
    \draw[color=blue] (j_107_000i) edge (j_316_000i);
    \draw[color=blue] (j_143_000i) edge (j_234_000i);
    \draw[color=blue] (j_304_364i) edge (j_379_106i);
    \draw[color=blue] (j_065_081i) edge (j_426_306i);
    \draw[color=blue] (j_242_000i) edge (j_356_000i);
    \draw[color=blue] (j_125_000i) edge (j_125_000i);
    \draw[color=orange] (j_242_000i) edge (j_242_000i);
    \draw[color=orange] (j_065_350i) edge (j_141_389i);
    \draw[color=blue] (j_150_000i) edge (j_190_344i);
    \draw[color=orange] (j_379_325i) edge (j_426_125i);
    \draw[color=orange] (j_319_000i) edge (j_304_067i);
    \draw[color=blue] (j_102_000i) edge (j_315_132i);
    \draw[color=orange] (j_316_000i) edge (j_379_325i);
    \draw[color=blue] (j_356_000i) edge (j_426_125i);
    \draw[color=orange] (j_234_000i) edge (j_242_000i);
    \draw[color=blue] (j_379_106i) edge (j_379_325i);
    \draw[color=orange] (j_419_000i) edge (j_141_042i);
    \draw[color=blue] (j_000_000i) edge (j_000_000i);
    \draw[color=orange] (j_358_000i) edge (j_381_000i);
    \draw[color=orange] (j_107_000i) edge (j_190_087i);
    \draw[color=blue] (j_241_000i) edge (j_190_344i);
    \draw[color=blue] (j_422_000i) edge (j_141_389i);
    \draw[color=orange] (j_065_081i) edge (j_118_209i);
    \draw[color=blue] (j_358_000i) edge (j_422_000i);
    \draw[color=blue] (j_019_000i) edge (j_304_067i);
    \draw[color=orange] (j_379_106i) edge (j_426_306i);
    \draw[color=blue] (j_189_000i) edge (j_304_067i);
    \draw[color=blue] (j_019_000i) edge (j_304_364i);
    \draw[color=blue] (j_061_000i) edge (j_315_132i);
    \draw[color=blue] (j_381_000i) edge (j_419_000i);
    \draw[color=orange] (j_319_000i) edge (j_304_364i);
    \draw[color=blue] (j_426_125i) edge (j_426_306i);
    \draw[color=orange] (j_315_132i) edge (j_426_306i);
    \draw[color=blue] (j_065_350i) edge (j_190_087i);
    \draw[color=blue] (j_150_000i) edge (j_319_000i);
    \draw[color=blue] (j_107_000i) edge (j_189_000i);
    \draw[color=blue] (j_067_000i) edge (j_234_000i);
    \draw[color=orange] (j_102_000i) edge (j_125_000i);
    \draw[color=blue] (j_150_000i) edge (j_190_087i);
    \draw[color=orange] (j_356_000i) edge (j_422_000i);
    \draw[color=orange] (j_150_000i) edge (j_189_000i);
    \draw[color=blue] (j_141_042i) edge (j_315_132i);
    \draw[color=blue] (j_141_042i) edge (j_304_364i);
    \draw[color=blue] (j_419_000i) edge (j_419_000i);
    \draw[color=orange] (j_118_209i) edge (j_315_299i);
    \draw[color=blue] (j_065_350i) edge (j_118_209i);
    \draw[color=orange] (j_061_000i) edge (j_356_000i);
    \draw[color=orange] (j_019_000i) edge (j_241_000i);
    \draw[color=blue] (j_241_000i) edge (j_381_000i);
    \draw[color=orange] (j_143_000i) edge (j_150_000i);
    \draw[color=blue] (j_141_389i) edge (j_315_299i);
    \draw[color=blue] (j_107_000i) edge (j_118_209i);
    \draw[color=orange] (j_241_000i) edge (j_118_222i);
    \draw[color=blue] (j_118_222i) edge (j_315_299i);
    \draw[color=blue] (j_141_042i) edge (j_141_389i);
    \draw[color=orange] (j_065_350i) edge (j_190_344i);
    \draw[color=orange] (j_107_000i) edge (j_190_344i);
    \draw[color=blue] (j_356_000i) edge (j_426_306i);
    \draw[color=orange] (j_118_222i) edge (j_315_132i);
    \draw[color=orange] (j_141_042i) edge (j_426_125i);
    \draw[color=blue] (j_061_000i) edge (j_356_000i);
    \draw[color=blue] (j_125_000i) edge (j_118_222i);
    \draw[color=blue] (j_107_000i) edge (j_118_222i);
    \draw[color=blue] (j_316_000i) edge (j_379_325i);
    \draw[color=blue] (j_061_000i) edge (j_315_299i);
    \draw[color=orange] (j_067_000i) edge (j_242_000i);
    \draw[color=orange] (j_379_106i) edge (j_379_325i);
    \draw[color=orange] (j_381_000i) edge (j_315_299i);
    \draw[color=orange] (j_315_299i) edge (j_426_125i);
    \draw[color=blue] (j_358_000i) edge (j_381_000i);
    \draw[color=orange] (j_190_087i) edge (j_304_067i);
    \draw[color=blue] (j_316_000i) edge (j_379_106i);
    \draw[color=blue] (j_067_000i) edge (j_419_000i);
    \draw[color=orange] (j_316_000i) edge (j_356_000i);
    \draw[color=blue] (j_065_350i) edge (j_426_125i);
    \draw[color=orange] (j_019_000i) edge (j_234_000i);
    \draw[color=orange] (j_004_000i) edge (j_004_000i);
    \draw[color=orange] (j_189_000i) edge (j_234_000i);
    \draw[color=blue] (j_000_000i) edge (j_241_000i);
    \draw[color=blue] (j_319_000i) edge (j_426_306i);
    \draw[color=orange] (j_190_344i) edge (j_304_364i);
    \draw[color=blue] (j_241_000i) edge (j_190_087i);
    \draw[color=blue] (j_125_000i) edge (j_118_209i);
    \draw[color=blue] (j_150_000i) edge (j_189_000i);
    \draw[color=blue] (j_019_000i) edge (j_125_000i);
    \draw[color=orange] (j_316_000i) edge (j_379_106i);
    \draw[color=orange] (j_000_000i) edge (j_125_000i);
    \draw[color=blue] (j_143_000i) edge (j_065_081i);
    \draw[color=blue] (j_004_000i) edge (j_319_000i);
    \draw[color=blue] (j_019_000i) edge (j_358_000i);
    \draw[color=blue] (j_242_000i) edge (j_242_000i);
    \draw[color=blue] (j_190_344i) edge (j_379_325i);
    \draw[color=orange] (j_061_000i) edge (j_358_000i);
    \draw[color=blue] (j_065_081i) edge (j_190_344i);
    \draw[color=orange] (j_067_000i) edge (j_304_067i);
    \draw[color=orange] (j_065_081i) edge (j_141_042i);
    \draw[color=blue] (j_422_000i) edge (j_141_042i);
    \draw[color=blue] (j_102_000i) edge (j_315_299i);
    \draw[color=blue] (j_242_000i) edge (j_422_000i);
    \draw[color=blue] (j_143_000i) edge (j_065_350i);
    \draw[color=orange] (j_065_350i) edge (j_118_222i);
    \draw[color=orange] (j_125_000i) edge (j_143_000i);
    \draw[color=orange] (j_004_000i) edge (j_019_000i);
    \draw[color=orange] (j_143_000i) edge (j_422_000i);
    \draw[color=blue] (j_061_000i) edge (j_234_000i);
    \draw[color=orange] (j_419_000i) edge (j_422_000i);
    \draw[color=blue] (j_141_389i) edge (j_304_067i);
    \draw[color=blue] (j_065_081i) edge (j_118_222i);
    \draw[color=blue] (j_067_000i) edge (j_316_000i);
    \draw[color=blue] (j_118_209i) edge (j_315_132i);
    \draw[color=orange] (j_241_000i) edge (j_118_209i);
    \draw[color=orange] (j_067_000i) edge (j_304_364i);
    \draw[color=blue] (j_190_087i) edge (j_379_106i);
    \draw[color=orange] (j_102_000i) edge (j_319_000i);
    \draw[color=blue] (j_189_000i) edge (j_304_364i);
    \draw[color=blue] (j_304_067i) edge (j_379_325i);
    \draw[color=orange] (j_381_000i) edge (j_315_132i);
    \draw[color=orange] (j_141_389i) edge (j_426_306i);
    \draw[color=orange] (j_061_000i) edge (j_107_000i);
}


\title{SQIsign}
\subtitle{The new herald of isogeny based crypto}
\author{Luca De Feo}
\date[November 21, 2023, ACCESS Seminar]{November 21, 2023\\
   ACCESS --- Algebraic Coding and Cryptography Seminar Series}
\institute{IBM Research Zürich}

\begin{document}

\frame[plain]{\titlepage}

%%

\begin{frame}{Isogenies and signing}
  \begin{description}
  \item[2012]<1-> \textcolor{blue}{SIDH PoKs (DF, Jao, Plût), signatures (Yoo, Azarderakhsh, Jao)}
  \item[2017]<3-> \textcolor{purple}{Galbraith--Petit--Silva}
  \item[2019]<2-> \textcolor{red}{SeaSign (DF, Galbraith)}
  \item[]<2-> \textcolor{red}{CSI-FiSh (Beullens, Kleinjung, Vercauteren)}
  \item[2020]<3-> \textcolor{purple}{SQIsign (DF, Kohel, Leroux, Petit, Wesolowski)}
  \item[2022]<1-> \textcolor{blue}{Ternary SIDH PoKs (DF, Dobson, Galbraith, Zobernig)}
  \item[2023]<1-> \textcolor{blue}{\textit{Curves you can trust} \tiny(Basso, Codogni, Connolly, DF, Fouotsa, Lido, Morrison, Panny, Patranabis, Wesolowski)}
  \item[2024]<3-> \textcolor{purple}{SQIsignHD (Dartois, Leroux, Robert, Wesolowski)}
  \end{description}

  \bigskip
  
  \newcolumntype{C}{>{\centering\arraybackslash}X}
  \begin{tabularx}{\textwidth}{ C | C | C }
    \color{blue} SIDH & \uncover<2->{\color{red} CSIDH} & \uncover<3->{\color{purple} Deuring} \\
    \uncover<1->{Best security} & \uncover<2->{Group, ring, threshold, \dots} & \uncover<3->{Most compact}
  \end{tabularx}
\end{frame}

%%

\begin{frame}{Elliptic curves}
  \centering
  \begin{tikzpicture}[scale=3]
    \axes{-0.5}{4.5}{-0.5}{2}

    \begin{scope}[/lattice={1}{0.2}{0.4}{0.7}]
      \draw[fill,black!10] (0,0) -- (1,0) -- (1,1) -- (0,1) -- (0,0);
      \node at (0.9,-0.1) {$ω_2$};
      \node at (-0.1,0.9) {$ω_1$};
      
      \lattice{-3}{5}

      \node[draw,circle,inner sep=1pt,fill,red] at (0.8,0.2) {};
      \node[draw,circle,inner sep=1pt,fill,blue] at (0.4,0.7) {};
      
      \node[draw,circle,inner sep=1pt,fill,purple] at (1.2,0.9) {};

      \draw[thin,dotted,red] (0,0) -- (0.8,0.2) (0.4,0.7) -- (1.2,0.9);
      \draw[thin,dotted,blue] (0,0) -- (0.4,0.7) (0.8,0.2) -- (1.2,0.9);

      \node[draw,circle,inner sep=1pt,fill,purple] at (0.2,0.9) {};
    \end{scope}  
  \end{tikzpicture}
\end{frame}

%%

\begin{frame}{Torsion points}
  \large\centering
  \[E[n] \quad=\quad \{ P \in E \;\mid\; nP = 0 \} \quad\simeq\quad \Z/n\Z \times \Z/n\Z\]

  \bigskip
  
  \begin{tikzpicture}[scale=1.2]
    \axes{-0.3}{4.5}{-0.5}{3};

    \begin{scope}[/lattice={3}{0.6}{1.2}{2.1}]
      \lattice{-1}{2}

      \foreach \i in {0,...,2} {
        \foreach \j in {0,...,2} {
          \draw[red] (\i/3,\j/3) node[fill,circle,inner sep=1pt] {};
        }
      }
      \draw[red] (0,0) -- (1/3,0) (0,0) -- (0,1/3);
    \end{scope}
  \end{tikzpicture}
\end{frame}

%%

\begin{frame}
  \Large
  \begin{description}
    \setlength{\itemsep}{4em}
  \item[Isogenies =] finite-kernel group morphisms: \emph{$E \to E/K$}
  \item[Endomorphisms =] isogenies \emph{$E \to E$}
  \end{description}
\end{frame}

%%

\begin{frame}{Isogenies: an example over $\F_{11}$}
  \centering
  \begin{tikzpicture}[scale=0.4]
    \begin{scope}
      \node[anchor=center] at (0,7) {$E \;:\; y^2 = x^3 + x$};

      \uncover<-1>{
        \draw[thin,gray] (0,-6) -- (0,6);
        \draw[thin,gray] (-6,0) -- (6,0);
      }

      \foreach \x/\y in {0/0,5/3,-4/3,-3/5,-2/1,-1/3} {
        \draw[blue,fill] (\x,\y) circle (0.2) node(E_\x_\y){}
        (\x,-\y) circle (0.2) node(E_\x_-\y){};
      }

      \uncover<2->{\draw[red,fill] (0,0) circle (0.3);}
    \end{scope}

    \draw[black!10!white,thick] (10,-7) -- +(0,14);
    
    \begin{scope}[shift={(20,0)}]
      \node at (0,7) {$E' \;:\; y^2 = x^3 - 4x$};

      \uncover<-1>{
        \draw[thin,gray] (0,-6) -- (0,6);
        \draw[thin,gray] (-6,0) -- (6,0);
      }

      \foreach \x/\y in {0/0,2/0,3/2,4/2,6/4,-2/0,-1/5} {
        \draw[color=blue,fill] (\x,\y) circle (0.2) node(F_\x_\y){}
        (\x,-\y) circle (0.2) node(F_\x_-\y){};
      }
    \end{scope}

    \begin{scope}[color=red,-latex,dashed]
      \begin{uncoverenv}<2->
        \path
        (E_5_3) edge (F_3_2)
        (E_-4_3) edge (F_4_-2)
        (E_-3_5) edge (F_4_2)
        (E_-2_1) edge (F_3_-2)
        (E_-1_3) edge (F_-2_0);
      \end{uncoverenv}
      \begin{uncoverenv}<2->
        \path
        (E_5_-3) edge (F_3_-2)
        (E_-4_-3) edge (F_4_2)
        (E_-3_-5) edge (F_4_-2)
        (E_-2_-1) edge (F_3_2)
        (E_-1_-3) edge (F_-2_0);
      \end{uncoverenv}
    \end{scope}
  \end{tikzpicture}
  
  \begin{columns}
    \begin{column}{0.5\textwidth}
      \[\phi(x,y) = \left(\frac{x^2 + 1}{x},\quad y\frac{x^2-1}{x^2}\right)\]
    \end{column}
    \begin{column}{0.5\textwidth}
      \begin{itemize}
      \item<2-> Kernel generator in \alert{red}.
      \item<2-> This is a degree $2$ map.
      \item<2-> Analogous to $x\mapsto x^2$ in $\F_q^*$.
      \end{itemize}
    \end{column}
  \end{columns}
\end{frame}

%%

\begin{frame}{Some endomorphisms}
  \Large
  \begin{description}
    \setlength{\itemsep}{3em}
  \item[Scalar multiplication:] $[n] : E \to E/E[n]$
  \item[Frobenius:] $(x,y) \mapsto (x^p, y^p)$ \hfill(on curves over $\F_p$)
  \item[Automorphisms:] $(x,y) \mapsto (-x, iy)$ \hfill(on curve $y^2 = x^3 + x$)
  \end{description}
\end{frame}

%%

\begin{frame}{Endomorphisms = $2 \times 2$ matrices}
  Fix any basis \emph{$〈P,Q〉$} of \emph{$E[N]$}

  \begin{align*}
    \omega : E[N] &→ E[N]\\
    \begin{pmatrix}
      x\\y
    \end{pmatrix}
                  &↦
                    \begin{pmatrix}
                      a&b\\c&d
                    \end{pmatrix}
                              \begin{pmatrix}
                                x\\y
                              \end{pmatrix}
    \mod N
  \end{align*}
  
  \begin{block}{Tate's isogeny theorem}
    1-to-1 correspondence between endomorphisms and Galois-invariant
    maps \emph{$E[ℓ^∞] \to E[ℓ^∞]$},
    for any prime $ℓ ≠ p$.

    \[\emph{\End(E)⊗ℤ_ℓ \hookrightarrow \mathcal{M}_{2\times 2}(ℤ_ℓ)}\]
  \end{block}
\end{frame}

%% 

\begin{frame}{Endomorphisms = imaginary quadratic integers}
  \Large
  Every endomorphism satisfies a quadratic equation
  \[ω^2 - tω + n = 0\]
  with $t,n ∈ ℤ$ and $t^2 - 4n ≤ 0$.
\end{frame}

%%

\begin{frame}{Endomorphism rings}
  \Large
  $\End(E)$ is a free $ℤ$-module of rank \emph{1, 2 or 4}:

  \begin{enumerate}
  \item[1)] $\End(E) ≃ ℤ$;
  \item[2)] $\End(E)$ is isomorphic to an order in a quadratic
    imaginary field;
  \item[4)] $\End(E)$ is isomorphic to a maximal order in a quaternion
    algebra ramified at $p$ and $∞$.
  \end{enumerate}
\end{frame}

%%

\begin{frame}{An example}
  The curve of $j$-invariant \emph{$1728$}
  \[E: y^2 = x^3 + x\]
  is supersingular over $\F_p$ iff $p=-1\mod 4$.

  \begin{block}{Endomorphisms}
    \emph{$\End(E) ⊂ ℚ〈ι,π〉$}, with:
    \begin{itemize}
    \item $π$ the Frobenius endomorphism, s.t. \emph{$π^2=-p$};
    \item $ι$ the map
      \[ι(x,y) = (-x,iy),\]
      where \emph{$i∈\F_{p^2}$} is a 4-th root of unity.
      Clearly, \emph{$ι^2=-1$}.
    \end{itemize}
    And \emph{$ιπ=-πι$}.
  \end{block}
\end{frame}

%%

\begin{frame}{Isogenies = finite subgroups}
  \large
  Let $K ⊂ E$ finite, there exist a unique (separable) isogeny
  \[\emph{\phi : E → E/K}\]
  up to post-composition by an isomorphism.

  \vspace{2em}

  $\phi : E → E/K$ is
  \begin{itemize}
  \item \emph{rational} if $K$ is Galois-stable,
  \item \emph{cyclic} if $K$ is cyclic.
  \end{itemize}
\end{frame}

%%

\begin{frame}{Isogenies = Ideals}
  \large
  Let \emph{$\phi : E → E/K$} be an isogeny
  \bigskip
  \[I_\phi = \{ ω ∈ \End(E) \;\mid\; ω(K) = 0 \}\]

  \bigskip
  
  Let \emph{$I ⊂ \End(E)$} be an invertible ideal
  \begin{align*}
    E[I] &= \bigcap_{ω∈I} \ker ω,\\
    \phi_I &: E → E/E[I]
  \end{align*}

  \bigskip

  \emph{Deuring:} this is a 1-to-1 correspondence.
\end{frame}

%%

\begin{frame}{Isogenies = $2 × 2$ matrices}
  \large
  Denote by \emph{$\Hom(E, E')$} the group of isogenies $E → E'$

  \bigskip
  
  \begin{block}{Tate's isogeny theorem}
    \[\Hom(E, E') ⊗ ℤ_ℓ \quad≃\quad \Hom_k(E[ℓ^∞], E'[ℓ^∞]) \quad\hookrightarrow\quad \mathcal{M}_{2×2}(ℤ_ℓ)\]
  \end{block}

  \bigskip
  
  In particular, \emph{$\rank\Hom(E, E') = \rank\End(E)$}.
\end{frame}

%%

\begin{frame}{The Deuring correspondences}
  \centering\large
  \setlength{\tabcolsep}{2em}
  \renewcommand{\arraystretch}{1.8}
  \begin{tabular}{r l}
    \emph{Ellitpic curves} & \emph{Number fields / Quaternion algebras}\\
    \hline
    Endomorphisms & Algebraic integers\\
    Endomorphism ring & (Maximal) order\\
    Isogeny & Invertible ideal\\
    Isogeny degree & Ideal norm\\
    Isogeny composition & Ideal multiplication\\
    Isogenies \raisebox{-0.8em}{\tikz{\node (E) at (0,0) {$\bullet$}; \node (E1) at (2,0) {$\bullet$}; \draw[->] (E) edge[bend left] (E1) edge[bend right] (E1);}} & Ideal classes\\
    Dual isogeny & Conjugate ideal\\
  \end{tabular}
\end{frame}

%% 

\begin{frame}{Two computational worlds}
  \centering
  \setlength{\tabcolsep}{2em}
  \renewcommand{\arraystretch}{1.5}
  \begin{tabular}{p{0.3\textwidth} c c}
    & \emph{Ordinary / CSIDH} & \emph{Supersingular}\\
    \hline
    $\rank\Hom(E,E')$ & 2 & 4\\
    Endomorphism algebra & number field & quaternion algebra\\
    Maximal orders & one & many \\
    Ideal class\dots & \dots group & \dots set\\
    Find isogeny $E → E'$ & \alert{hard} & \alert{hard}\\
    Compute equivalent ideals & \alert{complicated} & doable\\
    Convert isogenies $\leftrightarrow$ ideals & easyish & easy\\
    Compute $\End(E)$ & easy & \alert{hard}\\
  \end{tabular}  
\end{frame}

%%

\begin{frame}{Quaternion algebras}
  (Assume $p=3 \bmod 4$) The quaternion algebra \emph{$B_{p,∞}$} is:
  \begin{itemize}
  \item A \emph{$4$-dimensional} $ℚ$-vector space with basis
    \emph{$(1,i,j,k)$}.
  \item A non-commutative \emph{division algebra}%
    \footnote{All elements have inverses.} %
    $B_{p,∞} = ℚ〈i,j〉$ with the relations:
    \[i^2 = -1, \quad j^2 = -p, \quad ij = -ji = k.\]
  \end{itemize}

  \begin{block}{Properties}
    \begin{itemize}
    \item All elements of $B_{p,∞}$ are \emph{quadratic algebraic
        numbers}.
    \item $B_{p,∞}⊗ℚ_ℓ≃\mathcal{M}_{2×2}(ℚ_ℓ)$ for all $ℓ≠p$.\\
    \item $B_{p,∞}⊗ℝ$ is isomorphic to Hamilton's quaternions.
    \item $B_{p,∞}⊗ℚ_p$ is a division algebra.
    \end{itemize}
  \end{block}
\end{frame}

%%

\begin{frame}{Norm}
  \large We define the \emph{reduced norm} of
  $B_{p,∞} = \Q〈i,j〉$ as
  \[N(α) = N(a + b\cdot i + c\cdot j + d\cdot ij) = a^2 + b^2 + p(c^2 + d^2)\]

  \bigskip
  \begin{block}{Properties}
    \begin{itemize}
    \item The norm is \emph{multiplicative}.
    \item $\sqrt{N(α - β)}$ defines a \emph{metric}.
    \item If $N(α)$ and $2a$ are integers, $α$ is called an
      \emph{algebraic integer}.
    \end{itemize}
  \end{block}
\end{frame}

%%

\begin{frame}{Ideals, orders}
  \begin{block}{Ideals}
    \begin{itemize}
    \item A full rank ($= 4$) lattice $\a \subset B_{p,∞}$ is called a
      \emph{fractional ideal}.
    \item If all elements of $\a$ are integers, it is called an
      \emph{(integral) ideal}.
    \item If $\a$ is a subring of $B_{p,∞}$, it is called an
      \emph{order}.
    \item We define $N(\a)$ as the gcd of $N(α)$ for all $α∈\a$.
    \end{itemize}
  \end{block}

  \begin{block}{Orders}
    Let $\a ⊂ B_{p,∞}$ be an ideal, its \emph{left order} is
    \[\O_L(\a) := \{α ∈ B_{p,∞} \;|\; α\a ⊂ \a\}.\]
    The \emph{right order} $\O_R(\a)$ is defined analogously.
  \end{block}
\end{frame}

%%

\begin{frame}{The Deuring correspondence}
  Let $\O,\O'\subset B_{p,\infty}$ be two \emph{maximal orders}.
  They have the same \emph{type} if there exists $\alpha$ s.t.
  \[\O = \alpha\O'\alpha^{-1}.\]

  \begin{theorem}[Deuring]
    Maximal order types of $B_{p,\infty}$ are in one-to-one
    correspondence with supersingular curves up to Galois conjugation
    in $\F_{p^2}/\F_p$.
  \end{theorem}
\end{frame}

%%

\begin{frame}{The Deuring correspondence}
  Two \emph{left ideals} $\a,\b\subset\O$ are in the same \emph{class}
  if there exists $\beta$ s.t. $\a = \b\beta$.

  \begin{block}{An equivalence of categories (Kohel, roughly)}
    \centering
    \begin{tikzpicture}
      \node (O) at (0,0) {$\O$};
      \node (O1) at (6,0) {$\O'$};
      \node (E) at (0,-1) {$E$};
      \node (E1) at (6,-1) {$E'$};
      
      \begin{scope}[gray,anchor=north]
        \node (Oc) at (-2,1) {left order};
        \node at (-2, 1.5) {$\{\alpha\in B_{p,\infty}\;|\; \alpha\a=\a\}$};
        \node (O1c) at (8,1) {right order};
        \node at (8,1.5) {$\{\alpha\in B_{p,\infty}\;|\; \a\alpha=\a\}$};
        \node (ac) at (3,1.5) {connecting ideal (class)};
        
        \node (Ec) at (-2,-2) {supersingular curve};
        \node (E1c) at (8,-2) {supersingular curve};
        \node (phic) at (3,-2) {isogeny (class)};
      \end{scope}
      
      \draw[->] (O) edge node[auto] (a) {$\a$} (O1)
      (E) edge node[auto,swap] (phi) {$\phi_\a$} (E1);
      \draw[dashed,->] (Oc) edge (O) (O1c) edge (O1) (ac) edge (a)
      (Ec) edge (E) (E1c) edge (E1) (phic) edge (phi);
    \end{tikzpicture}
  \end{block}
\end{frame}

%%

\begin{frame}{The endomorphism ring problem}
  \Large\centering
  \vfill
  Given a random supersingular curve $E$, compute $\End(E)$
  \vfill
\end{frame}

%%

\begin{frame}{Contagious knowledge}
  \centering
  \begin{tikzpicture}
    \pgfmathsetseed{12345}
    \foreach \i in {1,...,100} {
      \pgfmathparse{16*random()}
      \let\x\pgfmathresult
      \pgfmathparse{7*random()}
      \let\y\pgfmathresult
      \fill[black!20!white] (\x,\y) circle (1pt);
    }
    \foreach \i in {1,...,10} {
      \pgfmathparse{floor(15*random())}
      \let\x\pgfmathresult
      \pgfmathparse{floor(6*random())}
      \let\y\pgfmathresult
      \fill (\x,\y) circle (2pt);
      \uncover<2->{
        \foreach \rho in {0,1,2} {
          \draw[fill,-latex] (\x,\y) -- +(120*\rho:.3) circle (2pt);
          \uncover<3->{
            \foreach \sigma in {-1,1} {
              \draw[fill,-latex] (\x,\y) ++(120*\rho:.3) -- ++(120*\rho+60*\sigma:.3) circle (2pt);
            }
          }
        }
      }
    }
  \end{tikzpicture}
\end{frame}

%%

\begin{frame}{A loose analogy: Signing based on factoring}\
  {\Large \[N = pq\]}

  \bigskip
  \begin{center}
    \begin{tabular}{l | c | c}
      & $\Z/N\Z$ & $\Z/p\Z \times \Z/q\Z$\\
      \hline
      multiplication & easy & easy\\
      inversion & easy & easy\\
      square roots & \alert{hard} & easy\\
      $n$-th roots & \alert{hard} & easy\\
    \end{tabular}
  \end{center}

  \begin{block}{Rabin's signature}
    \begin{description}
    \item[Sign:] $s \gets \sqrt{H(m; r)} \mod N$,
    \item[Verify:] $s^2 \overset{?}{=} H(m; r) \mod N$.
    \end{description}
  \end{block}
\end{frame}

%%

\begin{frame}{The effective Deuring correspondence}
  \centering\large
  \setlength{\tabcolsep}{2em}
  \renewcommand{\arraystretch}{1.8}
  \begin{tabular}{c c c}
    Input & Output \\
    \hline
    random $E$ & $\End(E)$ & \alert{hard}\\
    random $E$ & $ω ∈ \End(E)$ & \alert{hard}\\
    random $E$ & $\phi : E_0\to E$ & \alert{hard}\\
    \hline
    $\End(E)$ & $E$ & easy\\
    $\End(E)$, $\End(E')$ & connecting ideal & easy\\
    \hline
    $I ⊂ \End(E)$ & $\phi_I : E → E'$ & easy\\
    $\End(E)$, $\phi: E → E'$ & $I_\phi ⊂ \End(E)$, $\End(E')$ & easy
  \end{tabular}
\end{frame}

%%

\begin{frame}{Galbraith--Petit--Silva}
  \centering
  \begin{tikzpicture}
    \node (E0) at (1,3) {$E_0$};
    \node (EA) at (0,0) {$E_A$};
    \draw [blue,dashed,->] (E0) edge (EA);
    
    \uncover<2->{
    \node (Ec) at (5,0) {$E_c$};
      \draw [blue] [->] (EA) to (Ec);
    }
    \uncover<3->{
      \draw [blue,very thick] [->] (E0) -- (Ec);
    }
    
    \matrix [right] at (6,2) {
      \node[] (l1) {}; \node (l2) [right of = l1, node distance=0.5in,label=right: secret key isogeny] {}; \draw  [dashed,blue] [-] (l1) -- (l2); \\
      \uncover<2->{\node[] (l1) {}; \node (l2) [right of = l1, node distance=0.5in,label=right: commitment isogeny (prover)] {}; \draw [blue] [->] (l1) -- (l2);} \\
    };
  \end{tikzpicture}  
\end{frame}

%%

\begin{frame}{SQIsign: Signatures from the effective Deuring correspondence}
  \begin{center}
    \begin{tikzpicture}
      \node (E0) at (1,2.5) {$E_0$};
      \uncover<2->{
        \node (E1) at (4,2.5) {$E_1$};
        \node (B) at (2.5,2.75) {$\psi$};
        \draw [blue] [->] (E0) to (E1);
      }
      \uncover<3->{
        \node (E2) at (4,1) {$E_2$};
        \node (A) at (4.25,1.75) {$\varphi$};
        \draw [red] [->] (E1) to (E2);
      }
      \node (EA) at (0,1) {$E_A$};
      \node (A) at (0.25,1.75) {$\tau$};
      \draw [blue,dashed] [->] (E0) to (EA);

      \uncover<4->{
        \node (B) at (2,1.25) {$\sigma$};
        \draw [blue,very thick] [->] (EA) -- (E2);
      }

      \matrix [right] at (6,2) {
        \node[] (l1) {}; \node (l2) [right of = l1, node distance=0.5in,label=right: commitment isogeny (prover)] {}; \draw [blue] [->] (l1) -- (l2); \\
        \node[] (l3) {}; \node (l4) [right of = l3, node distance=0.5in,label=right:challenge isogeny (verifier)] {}; \draw [red] [->] (l3) -- (l4); \\
        \node[] (l1) {}; \node (l2) [right of = l1, node distance=0.5in,label=right: response isogeny (prover)] {}; \draw [blue,very thick] [->] (l1) -- (l2); \\
        \node[] (l1) {}; \node (l2) [right of = l1, node distance=0.5in,label=right: secret key isogeny] {}; \draw  [dashed,blue] [-] (l1) -- (l2); \\
      };
    \end{tikzpicture}
  \end{center}

  \bigskip
  
  \emph{Most compact PQ signature scheme}: PK + Signature combined
  \textbf{5$\times$smaller} than Falcon.

  \begin{table}[h]
    \centering
    \begin{tabular}{ c c c c}
      Secret Key (bytes) & Public Key (bytes) & Signature (bytes) & Security \\
      \hline
      782 & 64 & 177 & NIST-1 \\
      1138 & 96 & 263 & NIST-3 \\
      1509 & 128 & 335 & NIST-5 \\      
      \hline
    \end{tabular}
  \end{table}
\end{frame}

%% 

\begin{frame}[plain]
  \centering
  \begin{tikzpicture}[remember picture,overlay]
    \begin{scope}[xscale=1.7,yshift=-15,opacity=0.8]
      \def\crater{12}
      \def\jumpa{-8}
      \def\jumpb{9}
      \def\diam{5cm}

      \foreach \i in {1,...,\crater} {
        \draw[blue] (360/\crater*\i : \diam) to[bend right] (360/\crater*\i+360/\crater : \diam);
        \draw[red] (360/\crater*\i : \diam) to[bend right] (360/\crater*\i+\jumpa*360/\crater : \diam);
        \draw[green] (360/\crater*\i : \diam) to[bend right=50] (360/\crater*\i+\jumpb*360/\crater : \diam);
      }
    \end{scope}
    
    \draw (0,0.5) node{\Huge\bf Thank you};
    \draw (0,-1.1) node{\large\url{https://defeo.lu/}};
    \draw (0,-1.8) node{\large\includegraphics[height=0.9em]{twitter.png}~\href{https://twitter.com/luca_defeo}{@luca\_defeo}};
  \end{tikzpicture}
\end{frame}

%%

\end{document}


% LocalWords:  Isogeny abelian isogenies hyperelliptic supersingular Frobenius
% LocalWords:  isogenous
