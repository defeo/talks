\documentclass[aspectratio=169]{beamer}

%\includeonlyframes{current}

\usepackage[utf8]{inputenc}
\usepackage[american]{babel}
\usepackage{amsmath,amsthm}
\usepackage{unicode}
\usepackage{array,tabularx}
\usepackage{ifthen}
\usepackage{tikz}
\usetikzlibrary{matrix,decorations,decorations.text,calc,arrows,snakes,shapes,positioning,patterns}
\usepackage{tikzsymbols}
\usepackage[backend=biber,citestyle=authoryear-comp,bibstyle=beamer,doi=false,isbn=false,url=true,maxnames=10]{biblatex}
\bibliography{refs}

\usepackage{ulem}

\mode<presentation>{%
  \usetheme{ibm}
}

\newcommand{\C}{ℂ}
\newcommand{\R}{ℝ}
\newcommand{\Z}{ℤ}
\newcommand{\N}{ℕ}
\newcommand{\Q}{ℚ}
\newcommand{\F}{\mathbb{F}}
\renewcommand{\P}{\mathbb{P}}
\renewcommand{\O}{\mathcal{O}}
\newcommand{\tildO}{\mathcal{\tilde{O}}}
\newcommand{\poly}{\operatorname{poly}}
\newcommand{\polylog}{\operatorname{polylog}}
\newcommand{\End}{\operatorname{End}}
\newcommand{\Hom}{\operatorname{Hom}}
\newcommand{\Gal}{\operatorname{Gal}}
\newcommand{\chr}{\operatorname{char}}
\newcommand{\Cl}{\operatorname{Cl}}
\newcommand{\GL}{\operatorname{GL}}
\renewcommand{\a}{{\mathfrak{a}}}
\renewcommand{\b}{{\mathfrak{b}}}
\newcommand{\p}{{\mathfrak{p}}}
\newcommand{\q}{{\mathfrak{q}}}
\newcommand{\g}{{\mathfrak{g}}}
\newcommand{\G}{{\mathcal{G}}}
\newcommand{\E}{{\mathcal{E}}}
\newcommand{\cyc}[1]{{〈 #1 〉}}
\newcommand{\ord}{\operatorname{ord}}
\newcommand{\mat}[1]{\left(\begin{smallmatrix}#1\end{smallmatrix}\right)}
\newcommand{\from}{\overset{\$}{\leftarrow}}

\newcommand{\bl}[1]{\textcolor{blue}{#1}}
\newcommand{\rd}[1]{\textcolor{red}{#1}}
\newcommand{\gr}[1]{\textcolor{green}{#1}}
\newcommand{\og}[1]{\textcolor{orange}{#1}}

\definecolor{light blue}{RGB}{0,102,204}

\newcommand{\myedge}[3]{
  \draw[#3] (360/\crater*#1 : \diam) to[bend right] (360/\crater*#2 : \diam);
}
\newcommand{\sk}[4]{
  \draw[very thick,blue]   (0,0) -- (0,#1);
  \draw[very thick,red]    (1,0) -- (1,#2);
  \draw[very thick,green]  (2,0) -- (2,#3);
  \draw[very thick,orange] (3,0) -- (3,#4);
}
\newcommand{\axes}[4]{
  \clip (#1,#3) rectangle (#2,#4);
  \draw [thin, gray, -latex] (#1,0) -- (#2,0);% Draw x axis
  \draw [thin, gray, -latex] (0,#3) -- (0,#4);% Draw y axis
}
\pgfkeys{/lattice/.code n args={4}{\tikzset{cm={#1,#2,#3,#4,(0,0)}}}}
\newcommand{\lattice}[2]{
  \draw[style=help lines,dashed] (#1-1,#1-1) grid[step=1] (#2+1,#2+1);
  \foreach \x in {#1,...,#2}{
    \foreach \y in {#1,...,#2}{
      \node[draw,circle,inner sep=2pt,fill] at (\x,\y) {};
      % Places a dot at those points
    }
  }
}

\newenvironment<>{goodblock}[1]{%
  \begin{actionenv}#2%
      \def\insertblocktitle{#1}%
      \par%
      \mode<presentation>{%
        \setbeamercolor{block title}{fg=green!60!black,bg=green!50!white}
       \setbeamercolor{block body}{bg=green!20!white}
     }%
      \usebeamertemplate{block begin}}
    {\par\usebeamertemplate{block end}\end{actionenv}}
\newenvironment<>{mehblock}[1]{%
  \begin{actionenv}#2%
      \def\insertblocktitle{#1}%
      \par%
      \mode<presentation>{%
        \setbeamercolor{block title}{fg=yellow!50!black,bg=yellow!50!white}
       \setbeamercolor{block body}{bg=yellow!20!white}
     }%
      \usebeamertemplate{block begin}}
    {\par\usebeamertemplate{block end}\end{actionenv}}
\newenvironment<>{badblock}[1]{%
  \begin{actionenv}#2%
      \def\insertblocktitle{#1}%
      \par%
      \mode<presentation>{%
        \setbeamercolor{block title}{fg=red!60!black,bg=red!50!white}
       \setbeamercolor{block body}{bg=red!20!white}
     }%
      \usebeamertemplate{block begin}}
    {\par\usebeamertemplate{block end}\end{actionenv}}


\newcommand\isoggraph{
    \node[circle,inner sep=.7pt,fill=black] (curve0000) at (4.1000,0.0000) {};
    \node[circle,inner sep=.7pt,fill=black] (curve0158) at (3.9895,0.9455) {};
    \node[circle,inner sep=.7pt,fill=black] (curve0410) at (3.6639,1.8401) {};
    \node[circle,inner sep=.7pt,fill=black] (curve0368) at (3.1408,2.6354) {};
    \node[circle,inner sep=.7pt,fill=black] (curve0404) at (2.4484,3.2887) {};
    \node[circle,inner sep=.7pt,fill=black] (curve0075) at (1.6239,3.7647) {};
    \node[circle,inner sep=.7pt,fill=black] (curve0144) at (0.7120,4.0377) {};
    \node[circle,inner sep=.7pt,fill=black] (curve0191) at (-0.2384,4.0931) {};
    \node[circle,inner sep=.7pt,fill=black] (curve0174) at (-1.1759,3.9278) {};
    \node[circle,inner sep=.7pt,fill=black] (curve0413) at (-2.0500,3.5507) {};
    \node[circle,inner sep=.7pt,fill=black] (curve0379) at (-2.8136,2.9822) {};
    \node[circle,inner sep=.7pt,fill=black] (curve0124) at (-3.4255,2.2530) {};
    \node[circle,inner sep=.7pt,fill=black] (curve0199) at (-3.8527,1.4023) {};
    \node[circle,inner sep=.7pt,fill=black] (curve0390) at (-4.0723,0.4760) {};
    \node[circle,inner sep=.7pt,fill=black] (curve0029) at (-4.0723,-0.4760) {};
    \node[circle,inner sep=.7pt,fill=black] (curve0220) at (-3.8527,-1.4023) {};
    \node[circle,inner sep=.7pt,fill=black] (curve0295) at (-3.4255,-2.2530) {};
    \node[circle,inner sep=.7pt,fill=black] (curve0040) at (-2.8136,-2.9822) {};
    \node[circle,inner sep=.7pt,fill=black] (curve0006) at (-2.0500,-3.5507) {};
    \node[circle,inner sep=.7pt,fill=black] (curve0245) at (-1.1759,-3.9278) {};
    \node[circle,inner sep=.7pt,fill=black] (curve0228) at (-0.2384,-4.0931) {};
    \node[circle,inner sep=.7pt,fill=black] (curve0275) at (0.7120,-4.0377) {};
    \node[circle,inner sep=.7pt,fill=black] (curve0344) at (1.6239,-3.7647) {};
    \node[circle,inner sep=.7pt,fill=black] (curve0015) at (2.4484,-3.2887) {};
    \node[circle,inner sep=.7pt,fill=black] (curve0051) at (3.1408,-2.6354) {};
    \node[circle,inner sep=.7pt,fill=black] (curve0009) at (3.6639,-1.8401) {};
    \node[circle,inner sep=.7pt,fill=black] (curve0261) at (3.9895,-0.9455) {};
    %
    \draw[color=blue] (curve0199) edge[bend right=4mm] (curve0390);
    \draw[color=blue] (curve0051) edge[bend right=4mm] (curve0009);
    \draw[color=blue] (curve0368) edge[bend right=4mm] (curve0404);
    \draw[color=blue] (curve0245) edge[bend right=4mm] (curve0228);
    \draw[color=blue] (curve0029) edge[bend right=4mm] (curve0220);
    \draw[color=blue] (curve0174) edge[bend right=4mm] (curve0413);
    \draw[color=blue] (curve0261) edge[bend right=4mm] (curve0000);
    \draw[color=blue] (curve0379) edge[bend right=4mm] (curve0124);
    \draw[color=blue] (curve0006) edge[bend right=4mm] (curve0245);
    \draw[color=blue] (curve0158) edge[bend right=4mm] (curve0410);
    \draw[color=blue] (curve0228) edge[bend right=4mm] (curve0275);
    \draw[color=blue] (curve0275) edge[bend right=4mm] (curve0344);
    \draw[color=blue] (curve0015) edge[bend right=4mm] (curve0051);
    \draw[color=blue] (curve0191) edge[bend right=4mm] (curve0174);
    \draw[color=blue] (curve0144) edge[bend right=4mm] (curve0191);
    \draw[color=blue] (curve0404) edge[bend right=4mm] (curve0075);
    \draw[color=blue] (curve0009) edge[bend right=4mm] (curve0261);
    \draw[color=blue] (curve0295) edge[bend right=4mm] (curve0040);
    \draw[color=blue] (curve0410) edge[bend right=4mm] (curve0368);
    \draw[color=blue] (curve0413) edge[bend right=4mm] (curve0379);
    \draw[color=blue] (curve0040) edge[bend right=4mm] (curve0006);
    \draw[color=blue] (curve0075) edge[bend right=4mm] (curve0144);
    \draw[color=blue] (curve0220) edge[bend right=4mm] (curve0295);
    \draw[color=blue] (curve0390) edge[bend right=4mm] (curve0029);
    \draw[color=blue] (curve0344) edge[bend right=4mm] (curve0015);
    \draw[color=blue] (curve0124) edge[bend right=4mm] (curve0199);
    \draw[color=blue] (curve0000) edge[bend right=4mm] (curve0158);
    %
    \draw[color=red] (curve0009) edge[bend right=8mm] (curve0379);
    \draw[color=red] (curve0124) edge[bend right=8mm] (curve0015);
    \draw[color=red] (curve0006) edge[bend right=8mm] (curve0368);
    \draw[color=red] (curve0228) edge[bend right=8mm] (curve0075);
    \draw[color=red] (curve0245) edge[bend right=8mm] (curve0404);
    \draw[color=red] (curve0029) edge[bend right=8mm] (curve0261);
    \draw[color=red] (curve0220) edge[bend right=8mm] (curve0000);
    \draw[color=red] (curve0368) edge[bend right=8mm] (curve0220);
    \draw[color=red] (curve0144) edge[bend right=8mm] (curve0006);
    \draw[color=red] (curve0261) edge[bend right=8mm] (curve0124);
    \draw[color=red] (curve0191) edge[bend right=8mm] (curve0245);
    \draw[color=red] (curve0015) edge[bend right=8mm] (curve0174);
    \draw[color=red] (curve0344) edge[bend right=8mm] (curve0191);
    \draw[color=red] (curve0275) edge[bend right=8mm] (curve0144);
    \draw[color=red] (curve0158) edge[bend right=8mm] (curve0390);
    \draw[color=red] (curve0295) edge[bend right=8mm] (curve0158);
    \draw[color=red] (curve0075) edge[bend right=8mm] (curve0040);
    \draw[color=red] (curve0174) edge[bend right=8mm] (curve0228);
    \draw[color=red] (curve0000) edge[bend right=8mm] (curve0199);
    \draw[color=red] (curve0379) edge[bend right=8mm] (curve0344);
    \draw[color=red] (curve0390) edge[bend right=8mm] (curve0009);
    \draw[color=red] (curve0040) edge[bend right=8mm] (curve0410);
    \draw[color=red] (curve0410) edge[bend right=8mm] (curve0029);
    \draw[color=red] (curve0404) edge[bend right=8mm] (curve0295);
    \draw[color=red] (curve0413) edge[bend right=8mm] (curve0275);
    \draw[color=red] (curve0051) edge[bend right=8mm] (curve0413);
    \draw[color=red] (curve0199) edge[bend right=8mm] (curve0051);
    %
    \draw[color=green] (curve0158) edge[bend left=10mm] (curve0144);
    \draw[color=green] (curve0009) edge[bend left=10mm] (curve0368);
    \draw[color=green] (curve0124) edge[bend left=10mm] (curve0295);
    \draw[color=green] (curve0174) edge[bend left=10mm] (curve0390);
    \draw[color=green] (curve0368) edge[bend left=10mm] (curve0174);
    \draw[color=green] (curve0344) edge[bend left=10mm] (curve0000);
    \draw[color=green] (curve0144) edge[bend left=10mm] (curve0124);
    \draw[color=green] (curve0295) edge[bend left=10mm] (curve0275);
    \draw[color=green] (curve0029) edge[bend left=10mm] (curve0245);
    \draw[color=green] (curve0051) edge[bend left=10mm] (curve0410);
    \draw[color=green] (curve0015) edge[bend left=10mm] (curve0158);
    \draw[color=green] (curve0275) edge[bend left=10mm] (curve0261);
    \draw[color=green] (curve0075) edge[bend left=10mm] (curve0379);
    \draw[color=green] (curve0379) edge[bend left=10mm] (curve0220);
    \draw[color=green] (curve0220) edge[bend left=10mm] (curve0228);
    \draw[color=green] (curve0006) edge[bend left=10mm] (curve0015);
    \draw[color=green] (curve0191) edge[bend left=10mm] (curve0199);
    \draw[color=green] (curve0228) edge[bend left=10mm] (curve0009);
    \draw[color=green] (curve0040) edge[bend left=10mm] (curve0344);
    \draw[color=green] (curve0261) edge[bend left=10mm] (curve0404);
    \draw[color=green] (curve0404) edge[bend left=10mm] (curve0413);
    \draw[color=green] (curve0390) edge[bend left=10mm] (curve0006);
    \draw[color=green] (curve0000) edge[bend left=10mm] (curve0075);
    \draw[color=green] (curve0245) edge[bend left=10mm] (curve0051);
    \draw[color=green] (curve0413) edge[bend left=10mm] (curve0029);
    \draw[color=green] (curve0199) edge[bend left=10mm] (curve0040);
    \draw[color=green] (curve0410) edge[bend left=10mm] (curve0191);
}

\newcommand\fpsqgraph{
    \node[circle,inner sep=.7pt,fill=black] (j_190_344i) at (1.0,0.0) {};
    \node[circle,inner sep=.7pt,fill=black] (j_379_325i) at (0.985615910348,0.169000820322) {};
    \node[circle,inner sep=.7pt,fill=black] (j_143_000i) at (0.942877445461,0.333139794742) {};
    \node[circle,inner sep=.7pt,fill=black] (j_304_364i) at (0.873014113161,0.487694943814) {};
    \node[circle,inner sep=.7pt,fill=black] (j_356_000i) at (0.778035754318,0.628219997296) {};
    \node[circle,inner sep=.7pt,fill=black] (j_004_000i) at (0.66067472339,0.750672305253) {};
    \node[circle,inner sep=.7pt,fill=black] (j_242_000i) at (0.524307283557,0.851529137733) {};
    \node[circle,inner sep=.7pt,fill=black] (j_234_000i) at (0.37285647778,0.927889027297) {};
    \node[circle,inner sep=.7pt,fill=black] (j_065_081i) at (0.210679269996,0.977555238948) {};
    \node[circle,inner sep=.7pt,fill=black] (j_118_209i) at (0.0424412031961,0.999098966205) {};
    \node[circle,inner sep=.7pt,fill=black] (j_125_000i) at (-0.127017819747,0.991900435259) {};
    \node[circle,inner sep=.7pt,fill=black] (j_316_000i) at (-0.292822771277,0.956166734739) {};
    \node[circle,inner sep=.7pt,fill=black] (j_426_306i) at (-0.450203744818,0.89292585815) {};
    \node[circle,inner sep=.7pt,fill=black] (j_358_000i) at (-0.594633176304,0.803997130367) {};
    \node[circle,inner sep=.7pt,fill=black] (j_241_000i) at (-0.721956093955,0.691938868978) {};
    \node[circle,inner sep=.7pt,fill=black] (j_419_000i) at (-0.828509649244,0.559974786138) {};
    \node[circle,inner sep=.7pt,fill=black] (j_061_000i) at (-0.911228490388,0.411901248244) {};
    \node[circle,inner sep=.7pt,fill=black] (j_102_000i) at (-0.967732946933,0.251978061385) {};
    \node[circle,inner sep=.7pt,fill=black] (j_190_087i) at (-0.996397488543,0.0848059244755) {};
    \node[circle,inner sep=.7pt,fill=black] (j_107_000i) at (-0.996397488543,-0.0848059244755) {};
    \node[circle,inner sep=.7pt,fill=black] (j_304_067i) at (-0.967732946933,-0.251978061385) {};
    \node[circle,inner sep=.7pt,fill=black] (j_019_000i) at (-0.911228490388,-0.411901248244) {};
    \node[circle,inner sep=.7pt,fill=black] (j_381_000i) at (-0.828509649244,-0.559974786138) {};
    \node[circle,inner sep=.7pt,fill=black] (j_319_000i) at (-0.721956093955,-0.691938868978) {};
    \node[circle,inner sep=.7pt,fill=black] (j_065_350i) at (-0.594633176304,-0.803997130367) {};
    \node[circle,inner sep=.7pt,fill=black] (j_067_000i) at (-0.450203744818,-0.89292585815) {};
    \node[circle,inner sep=.7pt,fill=black] (j_000_000i) at (-0.292822771277,-0.956166734739) {};
    \node[circle,inner sep=.7pt,fill=black] (j_315_299i) at (-0.127017819747,-0.991900435259) {};
    \node[circle,inner sep=.7pt,fill=black] (j_422_000i) at (0.0424412031961,-0.999098966205) {};
    \node[circle,inner sep=.7pt,fill=black] (j_379_106i) at (0.210679269996,-0.977555238948) {};
    \node[circle,inner sep=.7pt,fill=black] (j_189_000i) at (0.37285647778,-0.927889027297) {};
    \node[circle,inner sep=.7pt,fill=black] (j_141_042i) at (0.524307283557,-0.851529137733) {};
    \node[circle,inner sep=.7pt,fill=black] (j_141_389i) at (0.66067472339,-0.750672305253) {};
    \node[circle,inner sep=.7pt,fill=black] (j_118_222i) at (0.778035754318,-0.628219997296) {};
    \node[circle,inner sep=.7pt,fill=black] (j_315_132i) at (0.873014113161,-0.487694943814) {};
    \node[circle,inner sep=.7pt,fill=black] (j_150_000i) at (0.942877445461,-0.333139794742) {};
    \node[circle,inner sep=.7pt,fill=black] (j_426_125i) at (0.985615910348,-0.169000820322) {};
    %
    \draw[color=blue] (j_319_000i) edge (j_426_125i);
    \draw[color=orange] (j_419_000i) edge (j_141_389i);
    \draw[color=orange] (j_065_081i) edge (j_190_087i);
    \draw[color=blue] (j_102_000i) edge (j_143_000i);
    \draw[color=orange] (j_102_000i) edge (j_358_000i);
    \draw[color=blue] (j_004_000i) edge (j_102_000i);
    \draw[color=blue] (j_107_000i) edge (j_316_000i);
    \draw[color=blue] (j_143_000i) edge (j_234_000i);
    \draw[color=blue] (j_304_364i) edge (j_379_106i);
    \draw[color=blue] (j_065_081i) edge (j_426_306i);
    \draw[color=blue] (j_242_000i) edge (j_356_000i);
    \draw[color=blue] (j_125_000i) edge (j_125_000i);
    \draw[color=orange] (j_242_000i) edge (j_242_000i);
    \draw[color=orange] (j_065_350i) edge (j_141_389i);
    \draw[color=blue] (j_150_000i) edge (j_190_344i);
    \draw[color=orange] (j_379_325i) edge (j_426_125i);
    \draw[color=orange] (j_319_000i) edge (j_304_067i);
    \draw[color=blue] (j_102_000i) edge (j_315_132i);
    \draw[color=orange] (j_316_000i) edge (j_379_325i);
    \draw[color=blue] (j_356_000i) edge (j_426_125i);
    \draw[color=orange] (j_234_000i) edge (j_242_000i);
    \draw[color=blue] (j_379_106i) edge (j_379_325i);
    \draw[color=orange] (j_419_000i) edge (j_141_042i);
    \draw[color=blue] (j_000_000i) edge (j_000_000i);
    \draw[color=orange] (j_358_000i) edge (j_381_000i);
    \draw[color=orange] (j_107_000i) edge (j_190_087i);
    \draw[color=blue] (j_241_000i) edge (j_190_344i);
    \draw[color=blue] (j_422_000i) edge (j_141_389i);
    \draw[color=orange] (j_065_081i) edge (j_118_209i);
    \draw[color=blue] (j_358_000i) edge (j_422_000i);
    \draw[color=blue] (j_019_000i) edge (j_304_067i);
    \draw[color=orange] (j_379_106i) edge (j_426_306i);
    \draw[color=blue] (j_189_000i) edge (j_304_067i);
    \draw[color=blue] (j_019_000i) edge (j_304_364i);
    \draw[color=blue] (j_061_000i) edge (j_315_132i);
    \draw[color=blue] (j_381_000i) edge (j_419_000i);
    \draw[color=orange] (j_319_000i) edge (j_304_364i);
    \draw[color=blue] (j_426_125i) edge (j_426_306i);
    \draw[color=orange] (j_315_132i) edge (j_426_306i);
    \draw[color=blue] (j_065_350i) edge (j_190_087i);
    \draw[color=blue] (j_150_000i) edge (j_319_000i);
    \draw[color=blue] (j_107_000i) edge (j_189_000i);
    \draw[color=blue] (j_067_000i) edge (j_234_000i);
    \draw[color=orange] (j_102_000i) edge (j_125_000i);
    \draw[color=blue] (j_150_000i) edge (j_190_087i);
    \draw[color=orange] (j_356_000i) edge (j_422_000i);
    \draw[color=orange] (j_150_000i) edge (j_189_000i);
    \draw[color=blue] (j_141_042i) edge (j_315_132i);
    \draw[color=blue] (j_141_042i) edge (j_304_364i);
    \draw[color=blue] (j_419_000i) edge (j_419_000i);
    \draw[color=orange] (j_118_209i) edge (j_315_299i);
    \draw[color=blue] (j_065_350i) edge (j_118_209i);
    \draw[color=orange] (j_061_000i) edge (j_356_000i);
    \draw[color=orange] (j_019_000i) edge (j_241_000i);
    \draw[color=blue] (j_241_000i) edge (j_381_000i);
    \draw[color=orange] (j_143_000i) edge (j_150_000i);
    \draw[color=blue] (j_141_389i) edge (j_315_299i);
    \draw[color=blue] (j_107_000i) edge (j_118_209i);
    \draw[color=orange] (j_241_000i) edge (j_118_222i);
    \draw[color=blue] (j_118_222i) edge (j_315_299i);
    \draw[color=blue] (j_141_042i) edge (j_141_389i);
    \draw[color=orange] (j_065_350i) edge (j_190_344i);
    \draw[color=orange] (j_107_000i) edge (j_190_344i);
    \draw[color=blue] (j_356_000i) edge (j_426_306i);
    \draw[color=orange] (j_118_222i) edge (j_315_132i);
    \draw[color=orange] (j_141_042i) edge (j_426_125i);
    \draw[color=blue] (j_061_000i) edge (j_356_000i);
    \draw[color=blue] (j_125_000i) edge (j_118_222i);
    \draw[color=blue] (j_107_000i) edge (j_118_222i);
    \draw[color=blue] (j_316_000i) edge (j_379_325i);
    \draw[color=blue] (j_061_000i) edge (j_315_299i);
    \draw[color=orange] (j_067_000i) edge (j_242_000i);
    \draw[color=orange] (j_379_106i) edge (j_379_325i);
    \draw[color=orange] (j_381_000i) edge (j_315_299i);
    \draw[color=orange] (j_315_299i) edge (j_426_125i);
    \draw[color=blue] (j_358_000i) edge (j_381_000i);
    \draw[color=orange] (j_190_087i) edge (j_304_067i);
    \draw[color=blue] (j_316_000i) edge (j_379_106i);
    \draw[color=blue] (j_067_000i) edge (j_419_000i);
    \draw[color=orange] (j_316_000i) edge (j_356_000i);
    \draw[color=blue] (j_065_350i) edge (j_426_125i);
    \draw[color=orange] (j_019_000i) edge (j_234_000i);
    \draw[color=orange] (j_004_000i) edge (j_004_000i);
    \draw[color=orange] (j_189_000i) edge (j_234_000i);
    \draw[color=blue] (j_000_000i) edge (j_241_000i);
    \draw[color=blue] (j_319_000i) edge (j_426_306i);
    \draw[color=orange] (j_190_344i) edge (j_304_364i);
    \draw[color=blue] (j_241_000i) edge (j_190_087i);
    \draw[color=blue] (j_125_000i) edge (j_118_209i);
    \draw[color=blue] (j_150_000i) edge (j_189_000i);
    \draw[color=blue] (j_019_000i) edge (j_125_000i);
    \draw[color=orange] (j_316_000i) edge (j_379_106i);
    \draw[color=orange] (j_000_000i) edge (j_125_000i);
    \draw[color=blue] (j_143_000i) edge (j_065_081i);
    \draw[color=blue] (j_004_000i) edge (j_319_000i);
    \draw[color=blue] (j_019_000i) edge (j_358_000i);
    \draw[color=blue] (j_242_000i) edge (j_242_000i);
    \draw[color=blue] (j_190_344i) edge (j_379_325i);
    \draw[color=orange] (j_061_000i) edge (j_358_000i);
    \draw[color=blue] (j_065_081i) edge (j_190_344i);
    \draw[color=orange] (j_067_000i) edge (j_304_067i);
    \draw[color=orange] (j_065_081i) edge (j_141_042i);
    \draw[color=blue] (j_422_000i) edge (j_141_042i);
    \draw[color=blue] (j_102_000i) edge (j_315_299i);
    \draw[color=blue] (j_242_000i) edge (j_422_000i);
    \draw[color=blue] (j_143_000i) edge (j_065_350i);
    \draw[color=orange] (j_065_350i) edge (j_118_222i);
    \draw[color=orange] (j_125_000i) edge (j_143_000i);
    \draw[color=orange] (j_004_000i) edge (j_019_000i);
    \draw[color=orange] (j_143_000i) edge (j_422_000i);
    \draw[color=blue] (j_061_000i) edge (j_234_000i);
    \draw[color=orange] (j_419_000i) edge (j_422_000i);
    \draw[color=blue] (j_141_389i) edge (j_304_067i);
    \draw[color=blue] (j_065_081i) edge (j_118_222i);
    \draw[color=blue] (j_067_000i) edge (j_316_000i);
    \draw[color=blue] (j_118_209i) edge (j_315_132i);
    \draw[color=orange] (j_241_000i) edge (j_118_209i);
    \draw[color=orange] (j_067_000i) edge (j_304_364i);
    \draw[color=blue] (j_190_087i) edge (j_379_106i);
    \draw[color=orange] (j_102_000i) edge (j_319_000i);
    \draw[color=blue] (j_189_000i) edge (j_304_364i);
    \draw[color=blue] (j_304_067i) edge (j_379_325i);
    \draw[color=orange] (j_381_000i) edge (j_315_132i);
    \draw[color=orange] (j_141_389i) edge (j_426_306i);
    \draw[color=orange] (j_061_000i) edge (j_107_000i);
}


\title{Isogeny based cryptography}
\subtitle{or what's left of it}
\author{Luca De Feo}
\date[Feb 16, 2023, TII, Abu Dhabi]{February 16, 2023\\
  TII, Abu Dhabi}
\institute{IBM Research Zürich}

\begin{document}

\frame[plain]{\titlepage}

%%

\begin{frame}{Elliptic curves}
  \begin{columns}
    \begin{column}{0.4\textwidth}
      {\Large
      \[y^2 = x^3 + ax + b\]
      }
      \begin{block}{Bezout's theorem}
        Every line cuts $E$ in exactly three points (counted with
        multiplicity).
      \end{block}

      Define a \emph{group law} such that any three colinear points
      add up to zero.
    \end{column}
    \begin{column}{0.6\textwidth}
      \begin{center}
        \begin{tikzpicture}[domain=-2.4566:4,samples=100,yscale=1/2]
          \draw plot (\x,{sqrt(\x*\x*\x-4*\x+5)});
          \draw plot (\x,{-sqrt(\x*\x*\x-4*\x+5)});

          \draw[thin,gray,-latex] (0,-7) -- (0,7);
          \draw[thin,gray,-latex] (-3,0) -- (4,0);
          \draw (-3,1) -- (4,8/3+3);
          \begin{scope}[every node/.style={draw,circle,inner sep=1pt,fill},cm={1,2/3,0,0,(0,3)}]
            \node at (-2.287980,0) {};
            \node at (-0.535051,0) {};
            \node at (3.267475,0) {};
          \end{scope}
          \begin{scope}[every node/.style={yshift=0.3cm},cm={1,2/3,0,0,(0,3)}]
            \node at (-2.287980,0) {$P$};
            \node at (-0.535051,0) {$Q$};
            \node at (3.267475,0) {$R$};
          \end{scope}

          \draw[dashed] (3.267475,3.267475*2/3+3) -- (3.267475,-3.267475*2/3-3) 
          node[draw,circle,inner sep=1pt,fill] {}
          node[xshift=-0.1cm,anchor=east] {$P+Q$};
        \end{tikzpicture}
      \end{center}
    \end{column}
  \end{columns}
\end{frame}

%%

\begin{frame}{Maps: isogenies}
  \begin{theorem}
    Let $\phi:E\to E'$ be an \emph{algebraic map} between elliptic
    curves. These conditions are equivalent:
    \begin{itemize}
    \item $\phi$ is a \emph{surjective group morphism},
    \item $\phi$ is a \emph{group morphism} with \emph{finite kernel},
    \item $\phi$ is \emph{non-constant} and sends the point at infinity of
      $E$ onto the point at infinity of $E'$.
    \end{itemize}
    If they hold $\phi$ is called an \emph{isogeny}.
  \end{theorem}

  Two curves are called \emph{isogenous} if there exists an isogeny
  between them.

  \begin{block}{Example: Multiplication-by-$m$}
    On any curve, an isogeny from $E$ to itself (i.e., an
    \emph{endomorphism}):
    \begin{align*}
      [m] \;:\; E &\to E,\\
      P &\mapsto [m]P.
    \end{align*}
  \end{block}
\end{frame}

%%

\begin{frame}{Isogenies: an example over $\F_{11}$}
  \centering
  \begin{tikzpicture}[scale=0.4]
    \begin{scope}
      \node[anchor=center] at (0,7) {$E \;:\; y^2 = x^3 + x$};

      \uncover<-1>{
        \draw[thin,gray] (0,-6) -- (0,6);
        \draw[thin,gray] (-6,0) -- (6,0);
      }

      \foreach \x/\y in {0/0,5/3,-4/3,-3/5,-2/1,-1/3} {
        \draw[blue,fill] (\x,\y) circle (0.2) node(E_\x_\y){}
        (\x,-\y) circle (0.2) node(E_\x_-\y){};
      }

      \uncover<2->{\draw[red,fill] (0,0) circle (0.3);}
    \end{scope}

    \draw[black!10!white,thick] (10,-7) -- +(0,14);
    
    \begin{scope}[shift={(20,0)}]
      \node at (0,7) {$E' \;:\; y^2 = x^3 - 4x$};

      \uncover<-1>{
        \draw[thin,gray] (0,-6) -- (0,6);
        \draw[thin,gray] (-6,0) -- (6,0);
      }

      \foreach \x/\y in {0/0,2/0,3/2,4/2,6/4,-2/0,-1/5} {
        \draw[color=blue,fill] (\x,\y) circle (0.2) node(F_\x_\y){}
        (\x,-\y) circle (0.2) node(F_\x_-\y){};
      }
    \end{scope}

    \begin{scope}[color=red,-latex,dashed]
      \begin{uncoverenv}<2->
        \path
        (E_5_3) edge (F_3_2)
        (E_-4_3) edge (F_4_-2)
        (E_-3_5) edge (F_4_2)
        (E_-2_1) edge (F_3_-2)
        (E_-1_3) edge (F_-2_0);
      \end{uncoverenv}
      \begin{uncoverenv}<2->
        \path
        (E_5_-3) edge (F_3_-2)
        (E_-4_-3) edge (F_4_2)
        (E_-3_-5) edge (F_4_-2)
        (E_-2_-1) edge (F_3_2)
        (E_-1_-3) edge (F_-2_0);
      \end{uncoverenv}
    \end{scope}
  \end{tikzpicture}
  
  \begin{columns}
    \begin{column}{0.5\textwidth}
      \[\phi(x,y) = \left(\frac{x^2 + 1}{x},\quad y\frac{x^2-1}{x^2}\right)\]
    \end{column}
    \begin{column}{0.5\textwidth}
      \begin{itemize}
      \item<2-> Kernel generator in \alert{red}.
      \item<2-> This is a degree $2$ map.
      \item<2-> Analogous to $x\mapsto x^2$ in $\F_q^*$.
      \end{itemize}
    \end{column}
  \end{columns}
\end{frame}

%%

\begin{frame}{Why isogenies in crypto?\dots '90s kids}
  \large
  \begin{itemize}
    \setlength{\itemsep}{2em}
  \item OMG, point counting is tough!

  \item Isogenies transport the discrete logarithm problem from one
    curve to another.

  \item[$\to$] The birth of the isogeny graph.
  \end{itemize}
\end{frame}

%%

\begin{frame}{Crypto on isogenies?\dots Millennials}
  \large
  \begin{description}
    \setlength{\itemsep}{2em}
  \item[2006:] Charles--Goren--Lauter hash function:
    \begin{itemize}
    \item Pseudo-random walks in the \emph{supersingular
        $\ell$-isogeny graph}.
    \end{itemize}
  \item[2006:] Couveignes--Rostovtsev--Stolbunov key exchange:
    \begin{itemize}
    \item Group action of a \emph{quadratic imaginary class group} on
      a set of \emph{ordinary} curves.
    \end{itemize}
  \end{description}
\end{frame}

%%

\begin{frame}{A turning point}
  \large
  \begin{description}
    \setlength{\itemsep}{2em}
  \item[2011:] SIDH: Supersingular Isogeny Diffie--Hellman
    \begin{itemize}
    \item Same \emph{supersingular $\ell$-isogeny graph},
    \item Powerful black magic to achieve key exchange.
    \end{itemize}
  \item[2017:] SIDH aka SIKE is submitted to NIST,
    \begin{itemize}
    \item Most compact public keys and ciphertexts in the competition.
    \end{itemize}
  \end{description}
\end{frame}

%%

\begin{frame}{Cambrian explosion}
  \large
  \begin{description}
    \setlength{\itemsep}{2em}
  \item[2018:] CSIDH (Commutative SIDH?)
    \begin{itemize}
    \item A (way) more efficient version of CRS based on supersingular
      curves.
    \end{itemize}
  \item[2019:] Isogeny-based Verifiable Delay Functions
    \begin{itemize}
    \item Warning: not post-quantum.
    \end{itemize}
  \item[2020:] SQISign: Short Quaternion and Isogeny Signature
    \begin{itemize}
    \item Most compact post-quantum public key + signature known today.
    \end{itemize}
  \end{description}
\end{frame}

%%

\begin{frame}{Coming of age}
  \large

  \emph{July 5, 2022:} SIKE advances to the 4th round of NIST's competition.
\end{frame}

%%

\begin{frame}[plain]
  \centering
  \begin{tikzpicture}[remember picture,overlay]
    \node[at=(current page.center)] {
      \includegraphics[width=\paperwidth]{nist-celebrate.jpg}
    };
  \end{tikzpicture}
\end{frame}

%% 

\begin{frame}[plain]
  \includegraphics[width=\textwidth]{craigs-mail.png}
\end{frame}

%%

\begin{frame}[plain]
  \centering
  \begin{tikzpicture}[remember picture,overlay]
    \node[at=(current page.center)] {
      \includegraphics[width=\paperwidth]{quantamag.png}
    };
  \end{tikzpicture}
\end{frame}

%%

\begin{frame}[plain]
  \centering
  \begin{tikzpicture}[remember picture,overlay]
    \node[at=(current page.center)] {
      \includegraphics[width=\paperwidth]{nist-edit.jpg}
    };
  \end{tikzpicture}
\end{frame}

%% 

\begin{frame}{So, why are we in this room?}
  \LARGE
  \centering
  \emph{\url{https://issikebrokenyet.github.io/}}
\end{frame}

%%

\begin{frame}
  \centering
  \begin{tikzpicture}
    \draw[gray,-latex] (-.2,-.2) -- (-.2,7);
    \node[anchor=center] at (-2.4,3) {\parbox{5em}{\centering Torsion knowledge}};
    \draw[gray,anchor=east,font=\small,shift={(-.2,0)}]
    (0,0) -- +(-.1,0) node {None}
    (0,2.5) -- +(-.1,0) node {$\left(\begin{smallmatrix}*&*\\&*\end{smallmatrix}\right)$}
    (0,3.5) -- +(-.1,0) node {$\left(\begin{smallmatrix}*&\\&*\end{smallmatrix}\right)$}
    (0,6) -- +(-.1,0) node {Full};
    
    \draw[gray,-latex] (-.2,-.2) -- (12,-.2);
    \node[anchor=center] at (6,-1.2) {$\End(E)$ knowledge};
    \draw[gray,anchor=north,font=\small,shift={(0,-.2)}]
    (0,0) -- +(0,-.1) node {None}
    (5.5,0) -- +(0,-.1) node {Orientation}
    (11,0) -- +(0,-.1) node {Full};

    \begin{uncoverenv}<2->
      \small
      \draw[green!50!black] (0,0) node {$\bullet$} node[anchor=south west] {SQISign};
      \draw[green!50!black] (0,2.5) node {$\bullet$} node[anchor=west] {\parbox{8em}{\centering SIDH signatures (DSSP)}};
      \draw[red] (0,6) node {$\times$} node[anchor=north west] {SIDH};
      \draw[red] (11,0) node {$\times$} node[anchor=south east] {\parbox{9em}{\centering Eisenträger et al. '18\\Wesolowski '21}};
      \draw[orange] (5.5,3.5) node {$\bullet$} node[anchor=north] {\parbox{5em}{\centering CSIDH\\SCALLOP}};
      \draw[green!50!black] (4,3.5) node {$\times$} node[anchor=south] {OSIDH};
      
      \fill[pattern=north west lines,pattern color=red] (1,5.8) -- ++(0,.4) -- ++(10.2,0) -- ++(0,-5.2) -- ++(-.4,0) -- ++(0,4.8);
      \fill[pattern=north west lines,pattern color=gray] (5,-.2) -- ++(1,0) -- ++(0,1) -- ++(-1,0);
    \end{uncoverenv}

    \begin{uncoverenv}<2->
      \draw[<->] (-.8,2.8) -- +(0,.4);
      \draw[orange] (6.5,-.4) edge[->] node[below] {\tiny quantum subexp} +(4,0);
    \end{uncoverenv}
  \end{tikzpicture}
\end{frame}

%%

\begin{frame}[plain]
  \centering
  \begin{tikzpicture}[remember picture,overlay]
    \begin{scope}[xscale=1.7,yshift=-15,opacity=0.8]
      \def\crater{12}
      \def\jumpa{-8}
      \def\jumpb{9}
      \def\diam{5cm}

      \foreach \i in {1,...,\crater} {
        \draw[blue] (360/\crater*\i : \diam) to[bend right] (360/\crater*\i+360/\crater : \diam);
        \draw[red] (360/\crater*\i : \diam) to[bend right] (360/\crater*\i+\jumpa*360/\crater : \diam);
        \draw[green] (360/\crater*\i : \diam) to[bend right=50] (360/\crater*\i+\jumpb*360/\crater : \diam);
      }
    \end{scope}
    
    \draw (0,0.5) node{\Huge\bf Thank you};
    \draw (0,-0.6) node{\large\url{https://defeo.lu/}};
    \draw (0,-1.3) node{\large\includegraphics[height=0.9em]{mastodon.png}~\href{https://twitter.com/luca_defeo}{@luca\_defeo@ioc.exchange}};
    \draw (0,-1.9) node{\large\includegraphics[height=0.9em]{twitter.png}~\href{https://twitter.com/luca_defeo}{@luca\_defeo}};
  \end{tikzpicture}
\end{frame}

%%

\end{document}


% LocalWords:  Isogeny abelian isogenies hyperelliptic supersingular Frobenius
% LocalWords:  isogenous
