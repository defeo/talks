\documentclass[aspectratio=169]{beamer}

%\includeonlyframes{current}

\usepackage[utf8]{inputenc}
\usepackage[american]{babel}
\usepackage{amsmath,amsthm}
\usepackage{unicode}
\usepackage{array,tabularx}
\usepackage{ifthen}
\usepackage{tikz}
\usetikzlibrary{matrix,decorations,decorations.text,calc,arrows,snakes,shapes,positioning}
\usepackage{tikzsymbols}
\usepackage[backend=biber,citestyle=authoryear-comp,bibstyle=beamer,doi=false,isbn=false,url=true,maxnames=10]{biblatex}
\bibliography{refs}

\usepackage{ulem}

\mode<presentation>{%
  \usetheme{ibm}
}

\newcommand{\C}{ℂ}
\newcommand{\R}{ℝ}
\newcommand{\Z}{ℤ}
\newcommand{\N}{ℕ}
\newcommand{\Q}{ℚ}
\newcommand{\F}{\mathbb{F}}
\renewcommand{\P}{\mathbb{P}}
\renewcommand{\O}{\mathcal{O}}
\newcommand{\tildO}{\mathcal{\tilde{O}}}
\newcommand{\poly}{\operatorname{poly}}
\newcommand{\polylog}{\operatorname{polylog}}
\newcommand{\End}{\operatorname{End}}
\newcommand{\Hom}{\operatorname{Hom}}
\newcommand{\Gal}{\operatorname{Gal}}
\newcommand{\chr}{\operatorname{char}}
\newcommand{\Cl}{\operatorname{Cl}}
\newcommand{\GL}{\operatorname{GL}}
\renewcommand{\a}{{\mathfrak{a}}}
\renewcommand{\b}{{\mathfrak{b}}}
\newcommand{\p}{{\mathfrak{p}}}
\newcommand{\q}{{\mathfrak{q}}}
\newcommand{\g}{{\mathfrak{g}}}
\newcommand{\G}{{\mathcal{G}}}
\newcommand{\E}{{\mathcal{E}}}
\newcommand{\cyc}[1]{{〈 #1 〉}}
\newcommand{\ord}{\operatorname{ord}}
\newcommand{\mat}[1]{\left(\begin{smallmatrix}#1\end{smallmatrix}\right)}
\newcommand{\from}{\overset{\$}{\leftarrow}}

\newcommand{\bl}[1]{\textcolor{blue}{#1}}
\newcommand{\rd}[1]{\textcolor{red}{#1}}
\newcommand{\gr}[1]{\textcolor{green}{#1}}
\newcommand{\og}[1]{\textcolor{orange}{#1}}

\definecolor{light blue}{RGB}{0,102,204}

\newcommand{\myedge}[3]{
  \draw[#3] (360/\crater*#1 : \diam) to[bend right] (360/\crater*#2 : \diam);
}
\newcommand{\sk}[4]{
  \draw[very thick,blue]   (0,0) -- (0,#1);
  \draw[very thick,red]    (1,0) -- (1,#2);
  \draw[very thick,green]  (2,0) -- (2,#3);
  \draw[very thick,orange] (3,0) -- (3,#4);
}
\newcommand{\axes}[4]{
  \clip (#1,#3) rectangle (#2,#4);
  \draw [thin, gray, -latex] (#1,0) -- (#2,0);% Draw x axis
  \draw [thin, gray, -latex] (0,#3) -- (0,#4);% Draw y axis
}
\pgfkeys{/lattice/.code n args={4}{\tikzset{cm={#1,#2,#3,#4,(0,0)}}}}
\newcommand{\lattice}[2]{
  \draw[style=help lines,dashed] (#1-1,#1-1) grid[step=1] (#2+1,#2+1);
  \foreach \x in {#1,...,#2}{
    \foreach \y in {#1,...,#2}{
      \node[draw,circle,inner sep=2pt,fill] at (\x,\y) {};
      % Places a dot at those points
    }
  }
}

\newenvironment<>{goodblock}[1]{%
  \begin{actionenv}#2%
      \def\insertblocktitle{#1}%
      \par%
      \mode<presentation>{%
        \setbeamercolor{block title}{fg=green!60!black,bg=green!50!white}
       \setbeamercolor{block body}{bg=green!20!white}
     }%
      \usebeamertemplate{block begin}}
    {\par\usebeamertemplate{block end}\end{actionenv}}
\newenvironment<>{mehblock}[1]{%
  \begin{actionenv}#2%
      \def\insertblocktitle{#1}%
      \par%
      \mode<presentation>{%
        \setbeamercolor{block title}{fg=yellow!50!black,bg=yellow!50!white}
       \setbeamercolor{block body}{bg=yellow!20!white}
     }%
      \usebeamertemplate{block begin}}
    {\par\usebeamertemplate{block end}\end{actionenv}}
\newenvironment<>{badblock}[1]{%
  \begin{actionenv}#2%
      \def\insertblocktitle{#1}%
      \par%
      \mode<presentation>{%
        \setbeamercolor{block title}{fg=red!60!black,bg=red!50!white}
       \setbeamercolor{block body}{bg=red!20!white}
     }%
      \usebeamertemplate{block begin}}
    {\par\usebeamertemplate{block end}\end{actionenv}}


\newcommand\isoggraph{
    \node[circle,inner sep=.7pt,fill=black] (curve0000) at (4.1000,0.0000) {};
    \node[circle,inner sep=.7pt,fill=black] (curve0158) at (3.9895,0.9455) {};
    \node[circle,inner sep=.7pt,fill=black] (curve0410) at (3.6639,1.8401) {};
    \node[circle,inner sep=.7pt,fill=black] (curve0368) at (3.1408,2.6354) {};
    \node[circle,inner sep=.7pt,fill=black] (curve0404) at (2.4484,3.2887) {};
    \node[circle,inner sep=.7pt,fill=black] (curve0075) at (1.6239,3.7647) {};
    \node[circle,inner sep=.7pt,fill=black] (curve0144) at (0.7120,4.0377) {};
    \node[circle,inner sep=.7pt,fill=black] (curve0191) at (-0.2384,4.0931) {};
    \node[circle,inner sep=.7pt,fill=black] (curve0174) at (-1.1759,3.9278) {};
    \node[circle,inner sep=.7pt,fill=black] (curve0413) at (-2.0500,3.5507) {};
    \node[circle,inner sep=.7pt,fill=black] (curve0379) at (-2.8136,2.9822) {};
    \node[circle,inner sep=.7pt,fill=black] (curve0124) at (-3.4255,2.2530) {};
    \node[circle,inner sep=.7pt,fill=black] (curve0199) at (-3.8527,1.4023) {};
    \node[circle,inner sep=.7pt,fill=black] (curve0390) at (-4.0723,0.4760) {};
    \node[circle,inner sep=.7pt,fill=black] (curve0029) at (-4.0723,-0.4760) {};
    \node[circle,inner sep=.7pt,fill=black] (curve0220) at (-3.8527,-1.4023) {};
    \node[circle,inner sep=.7pt,fill=black] (curve0295) at (-3.4255,-2.2530) {};
    \node[circle,inner sep=.7pt,fill=black] (curve0040) at (-2.8136,-2.9822) {};
    \node[circle,inner sep=.7pt,fill=black] (curve0006) at (-2.0500,-3.5507) {};
    \node[circle,inner sep=.7pt,fill=black] (curve0245) at (-1.1759,-3.9278) {};
    \node[circle,inner sep=.7pt,fill=black] (curve0228) at (-0.2384,-4.0931) {};
    \node[circle,inner sep=.7pt,fill=black] (curve0275) at (0.7120,-4.0377) {};
    \node[circle,inner sep=.7pt,fill=black] (curve0344) at (1.6239,-3.7647) {};
    \node[circle,inner sep=.7pt,fill=black] (curve0015) at (2.4484,-3.2887) {};
    \node[circle,inner sep=.7pt,fill=black] (curve0051) at (3.1408,-2.6354) {};
    \node[circle,inner sep=.7pt,fill=black] (curve0009) at (3.6639,-1.8401) {};
    \node[circle,inner sep=.7pt,fill=black] (curve0261) at (3.9895,-0.9455) {};
    %
    \draw[color=blue] (curve0199) edge[bend right=4mm] (curve0390);
    \draw[color=blue] (curve0051) edge[bend right=4mm] (curve0009);
    \draw[color=blue] (curve0368) edge[bend right=4mm] (curve0404);
    \draw[color=blue] (curve0245) edge[bend right=4mm] (curve0228);
    \draw[color=blue] (curve0029) edge[bend right=4mm] (curve0220);
    \draw[color=blue] (curve0174) edge[bend right=4mm] (curve0413);
    \draw[color=blue] (curve0261) edge[bend right=4mm] (curve0000);
    \draw[color=blue] (curve0379) edge[bend right=4mm] (curve0124);
    \draw[color=blue] (curve0006) edge[bend right=4mm] (curve0245);
    \draw[color=blue] (curve0158) edge[bend right=4mm] (curve0410);
    \draw[color=blue] (curve0228) edge[bend right=4mm] (curve0275);
    \draw[color=blue] (curve0275) edge[bend right=4mm] (curve0344);
    \draw[color=blue] (curve0015) edge[bend right=4mm] (curve0051);
    \draw[color=blue] (curve0191) edge[bend right=4mm] (curve0174);
    \draw[color=blue] (curve0144) edge[bend right=4mm] (curve0191);
    \draw[color=blue] (curve0404) edge[bend right=4mm] (curve0075);
    \draw[color=blue] (curve0009) edge[bend right=4mm] (curve0261);
    \draw[color=blue] (curve0295) edge[bend right=4mm] (curve0040);
    \draw[color=blue] (curve0410) edge[bend right=4mm] (curve0368);
    \draw[color=blue] (curve0413) edge[bend right=4mm] (curve0379);
    \draw[color=blue] (curve0040) edge[bend right=4mm] (curve0006);
    \draw[color=blue] (curve0075) edge[bend right=4mm] (curve0144);
    \draw[color=blue] (curve0220) edge[bend right=4mm] (curve0295);
    \draw[color=blue] (curve0390) edge[bend right=4mm] (curve0029);
    \draw[color=blue] (curve0344) edge[bend right=4mm] (curve0015);
    \draw[color=blue] (curve0124) edge[bend right=4mm] (curve0199);
    \draw[color=blue] (curve0000) edge[bend right=4mm] (curve0158);
    %
    \draw[color=red] (curve0009) edge[bend right=8mm] (curve0379);
    \draw[color=red] (curve0124) edge[bend right=8mm] (curve0015);
    \draw[color=red] (curve0006) edge[bend right=8mm] (curve0368);
    \draw[color=red] (curve0228) edge[bend right=8mm] (curve0075);
    \draw[color=red] (curve0245) edge[bend right=8mm] (curve0404);
    \draw[color=red] (curve0029) edge[bend right=8mm] (curve0261);
    \draw[color=red] (curve0220) edge[bend right=8mm] (curve0000);
    \draw[color=red] (curve0368) edge[bend right=8mm] (curve0220);
    \draw[color=red] (curve0144) edge[bend right=8mm] (curve0006);
    \draw[color=red] (curve0261) edge[bend right=8mm] (curve0124);
    \draw[color=red] (curve0191) edge[bend right=8mm] (curve0245);
    \draw[color=red] (curve0015) edge[bend right=8mm] (curve0174);
    \draw[color=red] (curve0344) edge[bend right=8mm] (curve0191);
    \draw[color=red] (curve0275) edge[bend right=8mm] (curve0144);
    \draw[color=red] (curve0158) edge[bend right=8mm] (curve0390);
    \draw[color=red] (curve0295) edge[bend right=8mm] (curve0158);
    \draw[color=red] (curve0075) edge[bend right=8mm] (curve0040);
    \draw[color=red] (curve0174) edge[bend right=8mm] (curve0228);
    \draw[color=red] (curve0000) edge[bend right=8mm] (curve0199);
    \draw[color=red] (curve0379) edge[bend right=8mm] (curve0344);
    \draw[color=red] (curve0390) edge[bend right=8mm] (curve0009);
    \draw[color=red] (curve0040) edge[bend right=8mm] (curve0410);
    \draw[color=red] (curve0410) edge[bend right=8mm] (curve0029);
    \draw[color=red] (curve0404) edge[bend right=8mm] (curve0295);
    \draw[color=red] (curve0413) edge[bend right=8mm] (curve0275);
    \draw[color=red] (curve0051) edge[bend right=8mm] (curve0413);
    \draw[color=red] (curve0199) edge[bend right=8mm] (curve0051);
    %
    \draw[color=green] (curve0158) edge[bend left=10mm] (curve0144);
    \draw[color=green] (curve0009) edge[bend left=10mm] (curve0368);
    \draw[color=green] (curve0124) edge[bend left=10mm] (curve0295);
    \draw[color=green] (curve0174) edge[bend left=10mm] (curve0390);
    \draw[color=green] (curve0368) edge[bend left=10mm] (curve0174);
    \draw[color=green] (curve0344) edge[bend left=10mm] (curve0000);
    \draw[color=green] (curve0144) edge[bend left=10mm] (curve0124);
    \draw[color=green] (curve0295) edge[bend left=10mm] (curve0275);
    \draw[color=green] (curve0029) edge[bend left=10mm] (curve0245);
    \draw[color=green] (curve0051) edge[bend left=10mm] (curve0410);
    \draw[color=green] (curve0015) edge[bend left=10mm] (curve0158);
    \draw[color=green] (curve0275) edge[bend left=10mm] (curve0261);
    \draw[color=green] (curve0075) edge[bend left=10mm] (curve0379);
    \draw[color=green] (curve0379) edge[bend left=10mm] (curve0220);
    \draw[color=green] (curve0220) edge[bend left=10mm] (curve0228);
    \draw[color=green] (curve0006) edge[bend left=10mm] (curve0015);
    \draw[color=green] (curve0191) edge[bend left=10mm] (curve0199);
    \draw[color=green] (curve0228) edge[bend left=10mm] (curve0009);
    \draw[color=green] (curve0040) edge[bend left=10mm] (curve0344);
    \draw[color=green] (curve0261) edge[bend left=10mm] (curve0404);
    \draw[color=green] (curve0404) edge[bend left=10mm] (curve0413);
    \draw[color=green] (curve0390) edge[bend left=10mm] (curve0006);
    \draw[color=green] (curve0000) edge[bend left=10mm] (curve0075);
    \draw[color=green] (curve0245) edge[bend left=10mm] (curve0051);
    \draw[color=green] (curve0413) edge[bend left=10mm] (curve0029);
    \draw[color=green] (curve0199) edge[bend left=10mm] (curve0040);
    \draw[color=green] (curve0410) edge[bend left=10mm] (curve0191);
}

\newcommand\fpsqgraph{
    \node[circle,inner sep=.7pt,fill=black] (j_190_344i) at (1.0,0.0) {};
    \node[circle,inner sep=.7pt,fill=black] (j_379_325i) at (0.985615910348,0.169000820322) {};
    \node[circle,inner sep=.7pt,fill=black] (j_143_000i) at (0.942877445461,0.333139794742) {};
    \node[circle,inner sep=.7pt,fill=black] (j_304_364i) at (0.873014113161,0.487694943814) {};
    \node[circle,inner sep=.7pt,fill=black] (j_356_000i) at (0.778035754318,0.628219997296) {};
    \node[circle,inner sep=.7pt,fill=black] (j_004_000i) at (0.66067472339,0.750672305253) {};
    \node[circle,inner sep=.7pt,fill=black] (j_242_000i) at (0.524307283557,0.851529137733) {};
    \node[circle,inner sep=.7pt,fill=black] (j_234_000i) at (0.37285647778,0.927889027297) {};
    \node[circle,inner sep=.7pt,fill=black] (j_065_081i) at (0.210679269996,0.977555238948) {};
    \node[circle,inner sep=.7pt,fill=black] (j_118_209i) at (0.0424412031961,0.999098966205) {};
    \node[circle,inner sep=.7pt,fill=black] (j_125_000i) at (-0.127017819747,0.991900435259) {};
    \node[circle,inner sep=.7pt,fill=black] (j_316_000i) at (-0.292822771277,0.956166734739) {};
    \node[circle,inner sep=.7pt,fill=black] (j_426_306i) at (-0.450203744818,0.89292585815) {};
    \node[circle,inner sep=.7pt,fill=black] (j_358_000i) at (-0.594633176304,0.803997130367) {};
    \node[circle,inner sep=.7pt,fill=black] (j_241_000i) at (-0.721956093955,0.691938868978) {};
    \node[circle,inner sep=.7pt,fill=black] (j_419_000i) at (-0.828509649244,0.559974786138) {};
    \node[circle,inner sep=.7pt,fill=black] (j_061_000i) at (-0.911228490388,0.411901248244) {};
    \node[circle,inner sep=.7pt,fill=black] (j_102_000i) at (-0.967732946933,0.251978061385) {};
    \node[circle,inner sep=.7pt,fill=black] (j_190_087i) at (-0.996397488543,0.0848059244755) {};
    \node[circle,inner sep=.7pt,fill=black] (j_107_000i) at (-0.996397488543,-0.0848059244755) {};
    \node[circle,inner sep=.7pt,fill=black] (j_304_067i) at (-0.967732946933,-0.251978061385) {};
    \node[circle,inner sep=.7pt,fill=black] (j_019_000i) at (-0.911228490388,-0.411901248244) {};
    \node[circle,inner sep=.7pt,fill=black] (j_381_000i) at (-0.828509649244,-0.559974786138) {};
    \node[circle,inner sep=.7pt,fill=black] (j_319_000i) at (-0.721956093955,-0.691938868978) {};
    \node[circle,inner sep=.7pt,fill=black] (j_065_350i) at (-0.594633176304,-0.803997130367) {};
    \node[circle,inner sep=.7pt,fill=black] (j_067_000i) at (-0.450203744818,-0.89292585815) {};
    \node[circle,inner sep=.7pt,fill=black] (j_000_000i) at (-0.292822771277,-0.956166734739) {};
    \node[circle,inner sep=.7pt,fill=black] (j_315_299i) at (-0.127017819747,-0.991900435259) {};
    \node[circle,inner sep=.7pt,fill=black] (j_422_000i) at (0.0424412031961,-0.999098966205) {};
    \node[circle,inner sep=.7pt,fill=black] (j_379_106i) at (0.210679269996,-0.977555238948) {};
    \node[circle,inner sep=.7pt,fill=black] (j_189_000i) at (0.37285647778,-0.927889027297) {};
    \node[circle,inner sep=.7pt,fill=black] (j_141_042i) at (0.524307283557,-0.851529137733) {};
    \node[circle,inner sep=.7pt,fill=black] (j_141_389i) at (0.66067472339,-0.750672305253) {};
    \node[circle,inner sep=.7pt,fill=black] (j_118_222i) at (0.778035754318,-0.628219997296) {};
    \node[circle,inner sep=.7pt,fill=black] (j_315_132i) at (0.873014113161,-0.487694943814) {};
    \node[circle,inner sep=.7pt,fill=black] (j_150_000i) at (0.942877445461,-0.333139794742) {};
    \node[circle,inner sep=.7pt,fill=black] (j_426_125i) at (0.985615910348,-0.169000820322) {};
    %
    \draw[color=blue] (j_319_000i) edge (j_426_125i);
    \draw[color=orange] (j_419_000i) edge (j_141_389i);
    \draw[color=orange] (j_065_081i) edge (j_190_087i);
    \draw[color=blue] (j_102_000i) edge (j_143_000i);
    \draw[color=orange] (j_102_000i) edge (j_358_000i);
    \draw[color=blue] (j_004_000i) edge (j_102_000i);
    \draw[color=blue] (j_107_000i) edge (j_316_000i);
    \draw[color=blue] (j_143_000i) edge (j_234_000i);
    \draw[color=blue] (j_304_364i) edge (j_379_106i);
    \draw[color=blue] (j_065_081i) edge (j_426_306i);
    \draw[color=blue] (j_242_000i) edge (j_356_000i);
    \draw[color=blue] (j_125_000i) edge (j_125_000i);
    \draw[color=orange] (j_242_000i) edge (j_242_000i);
    \draw[color=orange] (j_065_350i) edge (j_141_389i);
    \draw[color=blue] (j_150_000i) edge (j_190_344i);
    \draw[color=orange] (j_379_325i) edge (j_426_125i);
    \draw[color=orange] (j_319_000i) edge (j_304_067i);
    \draw[color=blue] (j_102_000i) edge (j_315_132i);
    \draw[color=orange] (j_316_000i) edge (j_379_325i);
    \draw[color=blue] (j_356_000i) edge (j_426_125i);
    \draw[color=orange] (j_234_000i) edge (j_242_000i);
    \draw[color=blue] (j_379_106i) edge (j_379_325i);
    \draw[color=orange] (j_419_000i) edge (j_141_042i);
    \draw[color=blue] (j_000_000i) edge (j_000_000i);
    \draw[color=orange] (j_358_000i) edge (j_381_000i);
    \draw[color=orange] (j_107_000i) edge (j_190_087i);
    \draw[color=blue] (j_241_000i) edge (j_190_344i);
    \draw[color=blue] (j_422_000i) edge (j_141_389i);
    \draw[color=orange] (j_065_081i) edge (j_118_209i);
    \draw[color=blue] (j_358_000i) edge (j_422_000i);
    \draw[color=blue] (j_019_000i) edge (j_304_067i);
    \draw[color=orange] (j_379_106i) edge (j_426_306i);
    \draw[color=blue] (j_189_000i) edge (j_304_067i);
    \draw[color=blue] (j_019_000i) edge (j_304_364i);
    \draw[color=blue] (j_061_000i) edge (j_315_132i);
    \draw[color=blue] (j_381_000i) edge (j_419_000i);
    \draw[color=orange] (j_319_000i) edge (j_304_364i);
    \draw[color=blue] (j_426_125i) edge (j_426_306i);
    \draw[color=orange] (j_315_132i) edge (j_426_306i);
    \draw[color=blue] (j_065_350i) edge (j_190_087i);
    \draw[color=blue] (j_150_000i) edge (j_319_000i);
    \draw[color=blue] (j_107_000i) edge (j_189_000i);
    \draw[color=blue] (j_067_000i) edge (j_234_000i);
    \draw[color=orange] (j_102_000i) edge (j_125_000i);
    \draw[color=blue] (j_150_000i) edge (j_190_087i);
    \draw[color=orange] (j_356_000i) edge (j_422_000i);
    \draw[color=orange] (j_150_000i) edge (j_189_000i);
    \draw[color=blue] (j_141_042i) edge (j_315_132i);
    \draw[color=blue] (j_141_042i) edge (j_304_364i);
    \draw[color=blue] (j_419_000i) edge (j_419_000i);
    \draw[color=orange] (j_118_209i) edge (j_315_299i);
    \draw[color=blue] (j_065_350i) edge (j_118_209i);
    \draw[color=orange] (j_061_000i) edge (j_356_000i);
    \draw[color=orange] (j_019_000i) edge (j_241_000i);
    \draw[color=blue] (j_241_000i) edge (j_381_000i);
    \draw[color=orange] (j_143_000i) edge (j_150_000i);
    \draw[color=blue] (j_141_389i) edge (j_315_299i);
    \draw[color=blue] (j_107_000i) edge (j_118_209i);
    \draw[color=orange] (j_241_000i) edge (j_118_222i);
    \draw[color=blue] (j_118_222i) edge (j_315_299i);
    \draw[color=blue] (j_141_042i) edge (j_141_389i);
    \draw[color=orange] (j_065_350i) edge (j_190_344i);
    \draw[color=orange] (j_107_000i) edge (j_190_344i);
    \draw[color=blue] (j_356_000i) edge (j_426_306i);
    \draw[color=orange] (j_118_222i) edge (j_315_132i);
    \draw[color=orange] (j_141_042i) edge (j_426_125i);
    \draw[color=blue] (j_061_000i) edge (j_356_000i);
    \draw[color=blue] (j_125_000i) edge (j_118_222i);
    \draw[color=blue] (j_107_000i) edge (j_118_222i);
    \draw[color=blue] (j_316_000i) edge (j_379_325i);
    \draw[color=blue] (j_061_000i) edge (j_315_299i);
    \draw[color=orange] (j_067_000i) edge (j_242_000i);
    \draw[color=orange] (j_379_106i) edge (j_379_325i);
    \draw[color=orange] (j_381_000i) edge (j_315_299i);
    \draw[color=orange] (j_315_299i) edge (j_426_125i);
    \draw[color=blue] (j_358_000i) edge (j_381_000i);
    \draw[color=orange] (j_190_087i) edge (j_304_067i);
    \draw[color=blue] (j_316_000i) edge (j_379_106i);
    \draw[color=blue] (j_067_000i) edge (j_419_000i);
    \draw[color=orange] (j_316_000i) edge (j_356_000i);
    \draw[color=blue] (j_065_350i) edge (j_426_125i);
    \draw[color=orange] (j_019_000i) edge (j_234_000i);
    \draw[color=orange] (j_004_000i) edge (j_004_000i);
    \draw[color=orange] (j_189_000i) edge (j_234_000i);
    \draw[color=blue] (j_000_000i) edge (j_241_000i);
    \draw[color=blue] (j_319_000i) edge (j_426_306i);
    \draw[color=orange] (j_190_344i) edge (j_304_364i);
    \draw[color=blue] (j_241_000i) edge (j_190_087i);
    \draw[color=blue] (j_125_000i) edge (j_118_209i);
    \draw[color=blue] (j_150_000i) edge (j_189_000i);
    \draw[color=blue] (j_019_000i) edge (j_125_000i);
    \draw[color=orange] (j_316_000i) edge (j_379_106i);
    \draw[color=orange] (j_000_000i) edge (j_125_000i);
    \draw[color=blue] (j_143_000i) edge (j_065_081i);
    \draw[color=blue] (j_004_000i) edge (j_319_000i);
    \draw[color=blue] (j_019_000i) edge (j_358_000i);
    \draw[color=blue] (j_242_000i) edge (j_242_000i);
    \draw[color=blue] (j_190_344i) edge (j_379_325i);
    \draw[color=orange] (j_061_000i) edge (j_358_000i);
    \draw[color=blue] (j_065_081i) edge (j_190_344i);
    \draw[color=orange] (j_067_000i) edge (j_304_067i);
    \draw[color=orange] (j_065_081i) edge (j_141_042i);
    \draw[color=blue] (j_422_000i) edge (j_141_042i);
    \draw[color=blue] (j_102_000i) edge (j_315_299i);
    \draw[color=blue] (j_242_000i) edge (j_422_000i);
    \draw[color=blue] (j_143_000i) edge (j_065_350i);
    \draw[color=orange] (j_065_350i) edge (j_118_222i);
    \draw[color=orange] (j_125_000i) edge (j_143_000i);
    \draw[color=orange] (j_004_000i) edge (j_019_000i);
    \draw[color=orange] (j_143_000i) edge (j_422_000i);
    \draw[color=blue] (j_061_000i) edge (j_234_000i);
    \draw[color=orange] (j_419_000i) edge (j_422_000i);
    \draw[color=blue] (j_141_389i) edge (j_304_067i);
    \draw[color=blue] (j_065_081i) edge (j_118_222i);
    \draw[color=blue] (j_067_000i) edge (j_316_000i);
    \draw[color=blue] (j_118_209i) edge (j_315_132i);
    \draw[color=orange] (j_241_000i) edge (j_118_209i);
    \draw[color=orange] (j_067_000i) edge (j_304_364i);
    \draw[color=blue] (j_190_087i) edge (j_379_106i);
    \draw[color=orange] (j_102_000i) edge (j_319_000i);
    \draw[color=blue] (j_189_000i) edge (j_304_364i);
    \draw[color=blue] (j_304_067i) edge (j_379_325i);
    \draw[color=orange] (j_381_000i) edge (j_315_132i);
    \draw[color=orange] (j_141_389i) edge (j_426_306i);
    \draw[color=orange] (j_061_000i) edge (j_107_000i);
}


\title{Isogeny based cryptography}
\subtitle{the new frontier of number theoretic cryptography}
\author{Luca De Feo}
\date[Sept 29, 2021, DMV-ÖMG 2021]{September 29, 2021\\
  DMV-ÖMG Annual Meeting}
\institute{IBM Research Zürich}

\begin{document}

\frame[plain]{\titlepage}

%%

\begin{frame}{Crypto <3 Number Theory}
  \begin{itemize}
  \item[1976] Diffie--Hellman key exchange,\hfill\emph{discrete logarithm}
  \item[1977] Rivest, Shamir and Adleman invent RSA,\hfill\emph{factorization}
    \pause
  \item[1980] Miller and Koblitz introduce elliptic curve cryptography,\hfill\emph{(hyper)elliptic curves}
    \pause
  \item[1996] Hoffstein, Pipher and Silverman invent NTRU,\hfill\emph{ideal lattices}
  \item[2001] Joux' tripartite key exchange, Boneh--Franklin IBE,\hfill\emph{elliptic pairings}
  \item[2006] Couveignes--Rostovtsev--Stolbunov key exchange,\hfill\emph{complex multiplication}
  \item[2006] Charles--Goren--Lauter hash function.\hfill\emph{quaternionic multiplication}
  \end{itemize}
\end{frame}

%%

\begin{frame}{Cryptography}
  \begin{columns}
    \begin{column}{0.4\textwidth}
      \centering
      \emph{Basic goals}
      \begin{itemize}
      \item Symmetric encryption,
      \item \alert{Key exchange},
      \item Public key encryption,
      \item \alert{Authentication},
      \item Digital signatures.
      \end{itemize}
    \end{column}
    \begin{column}{0.45\textwidth}
      \centering
      \emph{Advanced goals}
      \begin{itemize}
      \item Identity/Attribute based encryption,
      \item Fully homomorphic encryption,
      \item Zero-knowledge proofs,
      \item Multi-party computation,
      \item \dots
      \end{itemize}
    \end{column}
  \end{columns}
\end{frame}

%%

\begin{frame}{Discrete logarithm}
  \begin{columns}
    \begin{column}{0.5\textwidth}
      \centering
      \begin{tikzpicture}
        \begin{scope}
          \def\crater{13}
          \def\diam{2.5cm}

          \foreach \i in {1,...,\crater} {
            \uncover<1>{
              \draw[blue] (360/\crater*\i : \diam) to[bend right] (360/\crater*\i+360/\crater : \diam);
            }
            \uncover<2->{
              \draw[blue!20!white] (360/\crater*\i : \diam) to[bend right] (360/\crater*\i+360/\crater : \diam);
            }          
          }
          
          \pgfmathparse{\crater-1}
          \let\last\pgfmathresult
          \foreach \i in {0,...,\last} {
            \uncover<-2>{
              \draw[fill] (360/\crater*\i: \diam) circle (2pt) +(360/\crater*\i: 0.4) node{$g^{\i}$};
            }
            \uncover<3>{
              \pgfmathparse{hex(random(4096,65536))}
              \let\h\pgfmathresult
              \draw[fill] (360/\crater*\i: \diam) circle (2pt) +(360/\crater*\i: 0.6) node{$\h$};
            }
            \uncover<4->{
              \draw[fill] (360/\crater*\i: \diam) circle (2pt) +(360/\crater*\i: 0.5) node{$[\i]$};
            }
          }

          \uncover<2->{
            \draw[->] (360/\crater : \diam) edge
            node[yshift=1em,rotate=360/\crater*(7+1)/2-90]{discrete log}
            node[yshift=-1em,rotate=360/\crater*(7+1)/2-90]{\alt<2>{$7$}{\alert{??}}}
            (360/\crater*7 : \diam-4);
          }
        \end{scope}  
      \end{tikzpicture} 
    \end{column}
    \begin{column}{0.3\textwidth}
      \begin{uncoverenv}<4->
        The axioms of a dlog group:
        \begin{itemize}
        \item[prod:] $[a][b] = [a+b]$,
        \item[exp:] $n[a] = [na]$.
        \end{itemize}

        \bigskip
        The hard problem:
        \begin{itemize}
        \item[dlog:] $[a] \mapsto a$.
        \end{itemize}
      \end{uncoverenv}
    \end{column}
  \end{columns}
\end{frame}

%%

\begin{frame}{Diffie--Hellman key exchange}

  \begin{center}
    \begin{tikzpicture}[x=1.4cm]
      \node at (0,0) {\bf Alice};
      \node at (7,0) {\bf Bob};
      \node at (0,-1) {pick random \alert{$a\in(\Z/N\Z)^\times$}};
      \node at (7,-1) {pick random \alert{$b\in(\Z/N\Z)^\times$}};
      \draw[->]
      (1,-2) to node[auto] {$[a]$} (6,-2);
      \draw[->] (6,-2.5) to node[auto] {$[b]$} (1,-2.5);
      \node at (3.5,-3.5) {\emph{Shared secret} is \alert{$a[b]=[ab]=b[a]$}};
    \end{tikzpicture}
  \end{center}  
\end{frame}

%%

\begin{frame}{Why isogenies?}
  
  \begin{block}{Quantum-safe crypto}
    \begin{itemize}
    \item Shortest ciphertexts and public keys for \emph{Encryption}:\hfill
      SIDH/SIKE\\
      \hfill CSIDH*
    \item Shortest public key + \emph{Signature}:\hfill SQISign
    \item Only efficient \emph{Non-Interactive Key Exchange}:\hfill CSIDH*
    \item Acceptable \emph{Threshold Signatures}:\hfill CSI-FiSh*
    \end{itemize}

    \medskip
    \rule{4em}{0.1px}\\
    \small
    *Secure parameter sizes still debated, big impact on performance.
  \end{block}
  
  \begin{block}{Time-delay crypto (not quantum safe)}
    \begin{itemize}
    \item Only efficient alternative to group-based \emph{Verifiable Delay
        Functions}\hfill Asiacrypt '19
    \item Only known instantiation of \emph{Delay Encryption}
      \hfill Eurocrypt '21
    \end{itemize}
  \end{block}
\end{frame}

%%

\begin{frame}{Brief history of isogeny-based cryptography}
  \begin{description}
  \item[1997] Couveignes introduces the \emph{Hard Homogeneous Spaces}
    framework. His work stays unpublished for 10 years.
  \item[2006] Rostovtsev \& Stolbunov independently rediscover
    Couveignes ideas, suggest isogeny-based Diffie--Hellman as a
    \emph{quantum-resistant} primitive.
  \item[2006-2010] Other isogeny-based protocols by Teske and Charles,
    Goren \& Lauter.
  \item[2011-2012] D., Jao \& Plût introduce \emph{SIDH}, an
    efficient post-quantum key exchange inspired by Couveignes,
    Rostovtsev, Stolbunov, Charles, Goren, Lauter.
  \item[2017] SIDH is submitted to the NIST competition (with the name
    \emph{SIKE}, only isogeny-based candidate).
  \item[2018] Castryck, Lange, Martindale, Panny \& Renes create an
    efficient variant of the Couveignes--Rostovtsev--Stolbunov
    protocol, named \emph{CSIDH}.
  \item[2019] Isogeny signature craze: \emph{SeaSign}
    (D. \& Galbraith; Decru, Panny \& Vercauteren), \emph{CSI-FiSh}
    (Beullens, Kleinjung \& Vercauteren), \emph{VDF} (D., Masson,
    Petit \& Sanso).
  \item[2020] Isogeny signatures get interesting: \emph{SQISign}
    (D., Kohel, Leroux, Petit, Wesolowski).\\
    SIKE is an \emph{Alternate candidate finalist} in NIST's 3rd round.
  \end{description}
\end{frame}

%%

\begin{frame}{Diffie--Hellman key exchange}

  \begin{center}
    \begin{tikzpicture}[x=1.4cm]
      \node at (0,0) {\bf Alice};
      \node at (7,0) {\bf Bob};
      \node at (0,-1) {pick random \alert{$a\in(\Z/N\Z)^\times$}};
      \node at (7,-1) {pick random \alert{$b\in(\Z/N\Z)^\times$}};
      \draw[->]
      (1,-2) to node[auto] {$[a]$} (6,-2);
      \draw[->] (6,-2.5) to node[auto] {$[b]$} (1,-2.5);
      \node at (3.5,-3.5) {\emph{Shared secret} is \alert{$a[b]=[ab]=b[a]$}};
    \end{tikzpicture}
  \end{center}  
\end{frame}

%%

\begin{frame}{What's needed for key exchange?}
  \begin{columns}
    \begin{column}{0.5\textwidth}
      \centering
      \begin{tikzpicture}
        \begin{uncoverenv}<-3>
          \begin{scope}
            \def\crater{13}
            \def\jumpa{-8}
            \def\diam{2.5cm}

            \uncover<1>{
              \foreach \i in {1,...,\crater} {
                \draw[blue!20!white] (360/\crater*\i : \diam) to[bend right] (360/\crater*\i+360/\crater : \diam);
              }
            }

            \uncover<2->{
              \pgfmathparse{int(\crater/2)}
              \let\last\pgfmathresult
              \foreach \i in {1,...,\last} {
                \pgfmathparse{mod(pow(2,\i-1),\crater)}
                \let\e\pgfmathresult
                \draw[red] (360/\crater*\e : \diam) to[bend left] (360/\crater*\e*2 : \diam);
                \draw[red] (-360/\crater*\e : \diam) to[bend right] (-360/\crater*\e*2 : \diam);
              }
            }

            \uncover<3->{
              \pgfmathparse{int(\crater/2)}
              \let\last\pgfmathresult
              \foreach \i in {1,...,\last} {
                \pgfmathparse{mod(pow(6,\i-1),\crater)}
                \let\e\pgfmathresult
                \draw[blue] (360/\crater*\e : \diam) to[bend right] (360/\crater*\e*6 : \diam);
                \draw[blue] (-360/\crater*\e : \diam) to[bend left] (-360/\crater*\e*6 : \diam);
              }
            }
            
            \pgfmathparse{\crater-1}
            \let\last\pgfmathresult
            \foreach \i in {0,...,\last} {
              \draw[fill] (360/\crater*\i: \diam) circle (2pt) +(360/\crater*\i: 0.5) node{$[\i]$};
            }
          \end{scope}
        \end{uncoverenv}
        
        \begin{uncoverenv}<4->
          \begin{scope}
            \def\crater{12}
            \def\jumpa{5}
            \def\diam{2.5cm}

            \foreach \i in {1,...,\crater} {
              \draw[red] (360/\crater*\i : \diam) to[bend right] (360/\crater*\i+360/\crater : \diam);
              \draw[blue] (360/\crater*\i : \diam) to[bend right=10] (360/\crater*\i+\jumpa*360/\crater : \diam);
            }

            \foreach \i in {1,...,\crater} {
              \pgfmathparse{int(mod(pow(2,\i-1),\crater+1))}
              \let\e\pgfmathresult
              \draw[fill] (360/\crater*\i: \diam) circle (2pt) +(360/\crater*\i: 0.5) node{$[\e]$};
            }
          \end{scope}
        \end{uncoverenv}
      \end{tikzpicture}
    \end{column}
    \begin{column}{0.4\textwidth}
      The axioms of a dlog group:
      \begin{itemize}
      \item[\sout{prod:}] \sout{$[a][b] = [a+b]$,}
      \item[exp:] $n[a] = [na]$.
      \end{itemize}

      \bigskip
      The hard problem:
      \begin{itemize}
      \item[dlog:] $[a] \mapsto a$.
      \end{itemize}

      \bigskip
      \begin{tikzpicture}
        \uncover<2->{\draw[red] (0,0) node[anchor=east]{$[a]$} -- (1,0) node[anchor=west] {$2[a]$};}
        \uncover<3->{\draw[blue] (0,-1) node[anchor=east]{$[a]$} -- (1,-1) node[anchor=west] {$6[a]$};}
      \end{tikzpicture}

      \bigskip
      \begin{uncoverenv}<5->
        Automorphism group: \emph{$(\Z/13\Z)^\times$}.
      \end{uncoverenv}
    \end{column}
  \end{columns}
\end{frame}

%%

\begin{frame}
  \begin{block}{Group action}
    \emph{$\G\circlearrowright\E$}: A (finite) set $\E$ \emph{acted
      upon} by a group $\G$ \emph{freely} and \emph{transitively}:
    \begin{align*}
      * : \G × \E &→ \E\\
      \g * E &↦ E'
    \end{align*}
    \par\begin{description}
    \item[Compatibility:] \emph{$\g' * (\g * E) = (\g'\g)*E$} for all
      $\g,\g'\in\G$ and $E\in\E$;
    \item[Identity:] \emph{$\mathfrak{e} * E = E$} if and only if
      $\mathfrak{e}\in\G$ is the identity element;
    \item[Regularity:] for all $E,E'\in\E$ there exist a \emph{unique
        $\g\in\G$} such that \emph{$\g*E'=E$}.
      \setlength{\itemsep}{2em}
    \end{description}
  \end{block}
\end{frame}

%%

\begin{frame}{Cryptographic Group Actions \small(Alamati, D., Montgomery, Patranabis 2021)}
  \begin{block}{Hard Homogeneous Space (HHS) --- Couveignes 1997 \small(eprint:2006/291)}
    \emph{$\G\circlearrowright\E$} such that $\G$ is commutative and:
    \begin{itemize}
    \item \emph{Evaluating} $E' = \g*E$ is \emph{easy};
    \item \emph{Inverting} the action is \emph{hard}.
    \end{itemize}
  \end{block}

  \begin{block}{Example}
    Let $G$ be a group of order $13$, then \emph{$(\Z/13\Z)^\times \circlearrowright G$} defined by
    \[a * g := g^a\]
    is an HHS\dots\pause
    But
    \[\alert{g^a \cdot g^b = g^{a+b}}\]
    has no interpretation as a group action!
  \end{block}
\end{frame}

%%

\begin{frame}{Key exchange from group actions}
  \begin{description}
  \item[Public parameters:]\
    \begin{itemize}
    \item A \emph{HHS $\G\circlearrowright \E$} of order $N$ (large,
      but not necessarily prime);
    \item A \emph{starting} set element $E_0\in\E$.
    \end{itemize}
  \item[Notation:] $[\a] := \a*E_0$.
  \end{description}

  \bigskip
  
  \begin{center}
    \begin{tikzpicture}[x=1.4cm]
      \node at (0,0) {\bf Alice};
      \node at (7,0) {\bf Bob};
      \node at (0,-1) {pick random \alert{$\a\in\G$}};
      \node at (7,-1) {pick random \alert{$\b\in\G$}};
      \draw[->]
      (1,-2) to node[auto] {$[\a]$} (6,-2);
      \draw[->] (6,-2.5) to node[auto] {$[\b]$} (1,-2.5);
      \node at (3.5,-3.5) {\emph{Shared secret} is \alert{$\a[\b]=[\a\b]=\b[\a]$}};
    \end{tikzpicture}
  \end{center}
\end{frame}

%%

\begin{frame}{Quantum security}

  \textbf{Fact:} Shor's algorithm \emph{does not apply} to Diffie-Hellman
  protocols from \emph{group actions}.

  \begin{block}{Subexponential attack\hfill\emph{$\exp(\sqrt{\log p\log\log p})$}}
    \begin{itemize}
    \item Reduction to the \emph{hidden shift problem} by evaluating
      the class group action in \emph{quantum
        supersposition} (subexpoential cost);
    \item Well known reduction from the hidden shift to the
      \emph{dihedral (non-abelian) hidden subgroup problem};
    \item Kuperberg's algorithm solves the dHSP with a subexponential
      number of class group evaluations.
    \item Recent work suggests that $2^{64}$-qbit security is achieved
      somewhere in $512 < \log p < 2048$.
    \end{itemize}
  \end{block}
\end{frame}

%%

\begin{frame}
  \centering
  \begin{tikzpicture}
    \draw[blue,thick] (90:3.5) -- (210:3.5) -- (330:3.5) -- (90:3.5);
    
    \node[rotate=0] at (270:2.5) {Modular functions};
    \node[rotate=60] at (150:2.5) {Class field theory};
    \node[rotate=-60] at (30:2.5) {Elliptic curves};

    \begin{uncoverenv}<1>
      \begin{scope}[gray]
        \node[rotate=0] at (270:3.2) {$j(z) = \frac{1}{q} + 744 + 196884q + \cdots$};
        \node[anchor=east] at (150:3.2) {$H(j) = j - 1728$};
        \node[anchor=west] at (30:3.2) {$y^2 = x^3 - ax - b$};
      \end{scope}
    \end{uncoverenv}

    \begin{uncoverenv}<2>
      \begin{scope}[yshift=0.5cm]
        \node[anchor=east] (C) at (150:4.5) {Abelian extensions};
        \node[anchor=east,below of=C] {of $\Q(\sqrt{-D})$};
        \node[anchor=west] (E) at (30:4.5) {Elliptic curves with};
        \node[anchor=west,below of=E] {$\End(E) \subset \Q(\sqrt{-D})$};
      \end{scope}
    \end{uncoverenv}
    
    \begin{uncoverenv}<3>
      \begin{scope}[yshift=1.5cm]
        \node[anchor=east] (C) at (150:3) {Galois group of $K/\Q(\sqrt{-D})$};
        \node[anchor=east,below of=C] (C1) {$\simeq$};
        \node[anchor=east,below of=C1] {Class group $\Cl(-D)$};
        \node[anchor=west] (E) at (30:3) {$\Cl(-D)$ acts on set of $E$ s.t.};
        \node[anchor=west,below of=E] {$\End(E) \subset \Q(\sqrt{-D})$};
      \end{scope}
    \end{uncoverenv}
    
    \begin{scope}
      \Large
      \node at (0,0.3) {\emph{Complex}};
      \node at (0,-0.5) {\emph{Multiplication}};
    \end{scope}
  \end{tikzpicture}
\end{frame}

%%

\begin{frame}{Complex tori}
  \begin{columns}
    \begin{column}{0.6\textwidth}
      \begin{tikzpicture}[scale=2]
        \axes{-1}{3.5}{-0.5}{3}

        \begin{scope}[/lattice={1}{0.2}{0.4}{0.7}]
          \begin{uncoverenv}<1>
            \draw[fill,black!10] (0,0) -- (1,0) -- (1,1) -- (0,1) -- (0,0);
            \node at (0.5,0.5) {$\C/\Lambda$};
            \node at (0.9,-0.1) {$\omega_1$};
            \node at (-0.1,0.9) {$\omega_2$};
          \end{uncoverenv}

          \lattice{-3}{4}

          \begin{uncoverenv}<2-5>
            \node[red] at (0.7,0.65) {$a$}; 
            \node[draw,circle,inner sep=1pt,fill,red] at (0.8,0.5) {};
            \node[red] at (0.2,0.9) {$b$}; 
            \node[draw,circle,inner sep=1pt,fill,red] at (0.3,0.7) {};
            
            \begin{uncoverenv}<3-4>
              \node[red] at (1.2,1.3) {$a+b$}; 
              \node[draw,circle,inner sep=1pt,fill,red] at (1.1,1.2) {};
              \begin{uncoverenv}<3>
                \draw[red,thin] (0,0) -- (0.8,0.5) -- (1.1,1.2);
                \draw[red,thin] (0,0) -- (0.3,0.7) -- (1.1,1.2);          
              \end{uncoverenv}
            \end{uncoverenv}

            \transdissolve<5>
            \begin{uncoverenv}<5>
              \node[red] at (0.2,0.3) {$a+b$}; 
              \node[draw,circle,inner sep=1pt,fill,red] at (0.1,0.2) {};
            \end{uncoverenv}
          \end{uncoverenv}
        \end{scope}  
      \end{tikzpicture}
    \end{column}
    \begin{column}{0.4\textwidth}
      \begin{onlyenv}<1>
        Let $\omega_1,\omega_2\in\C$ be linearly independent complex
        numbers. Set
        \[\emph{\Lambda = \omega_1\Z \oplus \omega_2\Z}\]

        $\C/\Lambda$ is a \emph{complex torus}.
      \end{onlyenv}

      \begin{onlyenv}<2->
        Addition law induced by addition on $\mathbb{C}$.
      \end{onlyenv}
    \end{column}
  \end{columns}
\end{frame}

%%

\begin{frame}{Homotheties}
  \begin{columns}
    \begin{column}{0.6\textwidth}
      \begin{tikzpicture}[scale=2]
        \axes{-1}{3.3}{-0.5}{3}

        \newcount\scale
        \animate<1-21>
        \animatevalue<1-21>{\scale}{0}{20}
        \begin{uncoverenv}<1-22>
          \begin{scope}[/lattice={1}{0.2}{0.4}{0.7},scale=1+\the\scale/20,rotate=\the\scale]
            \lattice{-3}{4}
            \node[red,yshift=0.2cm] at (0.8,0.5) {$a$}; 
            \node[draw,circle,inner sep=1pt,fill,red] at (0.8,0.5) {};
          \end{scope}
        \end{uncoverenv}
      \end{tikzpicture}      
    \end{column}
    \begin{column}{0.4\textwidth}
      Two lattices are \emph{homothetic} if there exist $\alpha\in\C$
      such that
      \[\emph{\alpha\Lambda_1 = \Lambda_2}\]
    \end{column}
  \end{columns}
\end{frame}  

%%

\begin{frame}{Uniformization theorem}
  \begin{center}
    One to one correspondence: \emph{Complex tori $\leftrightarrow$
      Elliptic curves over $\C$}
  \end{center}

  \begin{itemize}
  \item Isomorphic as Riemann surfaces,
  \item Isomorphic as groups,
  \item Homotheties of lattices = Isomorphisms of elliptic curves.
  \end{itemize}

  \begin{block}{The $j$-invariant}
    \[j(E) = 1728\frac{4a^3}{4a^3-27b^2}\]
    classifies curves/tori up to isomorphism/homothety.
  \end{block}
\end{frame}

%%

\begin{frame}
  \frametitle{Multiplication}

  \centering
  \begin{tikzpicture}[scale=2.2]
    \axes{-2}{5}{-0.5}{3}

    \begin{scope}[/lattice={1}{0.2}{0.4}{0.7}]
      \lattice{-4}{5}
    
      \node[red,yshift=0.2cm] at (0.8,0.6) {$a$}; 
      \draw[red] (0.8,0.6) node[fill,circle,inner sep=1pt] {};

      \begin{uncoverenv}<2>
        \node[red,yshift=0.2cm] at (2.4,1.8) {$[3]a$}; 
        \draw[red] (0,0) -- (1.6,1.2) node[fill,circle,inner sep=1pt] {} 
        -- (2.4,1.8) node[fill,circle,inner sep=1pt] {};
      \end{uncoverenv}

      \transdissolve<3>
      \begin{uncoverenv}<3>
        \node[red,yshift=0.3cm] at (0.4,0.8) {$[3]a$}; 
        \draw[red] (0.4,0.8) node[fill,circle,inner sep=1pt] {};
      \end{uncoverenv}
    \end{scope}
  \end{tikzpicture}
\end{frame}

%%

\begin{frame}
  \frametitle{Endomorphisms}

  \begin{columns}
    \begin{column}{0.65\textwidth}
      \centering
      \begin{tikzpicture}[scale=1.8]
        \axes{-0.3}{4.5}{-0.5}{4};

        \begin{scope}[/lattice={3}{0.6}{1.2}{2.1}]
          \lattice{-1}{2}

          \foreach \i in {0,...,2} {
            \foreach \j in {0,...,2} {
              \draw[red] (\i/3,\j/3) node[fill,circle,inner sep=1pt] {};
            }
          }
          \draw[red] (0,0) -- (1/3,0) node[yshift=0.2cm] {$a$};
          \draw[red] (0,0) -- (0,1/3) node[yshift=0.2cm] {$b$};
        \end{scope}
      \end{tikzpicture}  
    \end{column}
    \begin{column}{0.35\textwidth}
      Let $\alpha$ be such that \emph{$\alpha\Lambda\subset\Lambda$},
      then
      \[\phi_\alpha \;:\; z \mapsto \alpha z \mod \Lambda\]
      is an \emph{endomorphism} of $\C/\Lambda$.

      \bigskip
      
      Let $\ell$ be an integer, the kernel of $\phi_\ell$ is:
      \begin{align*}
        (\C/\Lambda)[\ell] &= \langle a,b \rangle\\
                           &\simeq (\Z/\ell\Z)^2
      \end{align*}
    \end{column}
  \end{columns}
\end{frame}

%%

\begin{frame}{Complex Multiplication (CM)}
  Endomorphisms form a subring of $\C$: indeed
  \emph{$\alpha\Lambda \subset \Lambda$} and
  \emph{$\beta\Lambda \subset \Lambda$} imply
  \begin{itemize}
  \item $(\alpha+\beta)\Lambda \subset \Lambda$,
  \item $(\alpha\beta)\Lambda \subset \Lambda$.
  \end{itemize}

  \begin{theorem}
    Let $C/\Lambda$ be a complex torus, its endomorphism ring is one
    of:
    \begin{itemize}
    \item The ring of integers $\Z$,
    \item An order in an imaginary quadratic field
      $\Q(\sqrt{-D})$.\footnote{A subring that is a lattice of
        dimension $2$.}
    \end{itemize}
  \end{theorem}

  \begin{corollary}
    For any endomorphism $\phi_\alpha$ there exist integers $t,n$ such
    that
    \[\phi_\alpha^2 - t\phi_\alpha + n = 0.\]
  \end{corollary}
\end{frame}

%%

\begin{frame}<1-3>
  \frametitle{Isogenies}

  \begin{columns}
    \begin{column}{0.65\textwidth}
      \centering
      \begin{tikzpicture}[scale=1.8]
        \axes{-0.3}{4.5}{-0.5}{4};
        
        \begin{scope}[/lattice={3}{0.6}{1.2}{2.1}]
          \uncover<1->{\lattice{-1}{2}}
          
          \begin{uncoverenv}<1-3>
            \draw[red] (0,0) -- (1/3,0) node[yshift=0.3cm] {$a$};
          \end{uncoverenv}
          \begin{uncoverenv}<4->
            \draw[red] (0,0) -- (0,1/3) node[yshift=0.3cm] {$b$};
          \end{uncoverenv}

          \begin{uncoverenv}<1-2>
            \draw[blue] (0.8,0.5) node[yshift=0.3cm] {$p$};
            \draw[blue] (0.8,0.5) node[fill,circle,inner sep=1pt] {};
          \end{uncoverenv}
        \end{scope}
        
        \begin{scope}[/lattice={1}{0.2}{1.2}{2.1}]
          \transdissolve<2>
          \begin{scope}[opacity=0.3]
            \uncover<2-4>{\lattice{-3}{5}}
          \end{scope}
          
          \transdissolve<3>
          \begin{uncoverenv}<3-5>
            \draw[blue] (0.4,0.5) node[yshift=0.3cm] {$p$};
            \draw[blue] (0.4,0.5) node[fill,circle,inner sep=1pt] {};
          \end{uncoverenv}
        \end{scope}

        \begin{scope}[/lattice={1}{0.2}{0.4}{0.7}]
          \transdissolve<5>
          \begin{scope}[opacity=0.3]
            \uncover<5->{\lattice{-3}{5}}
          \end{scope}
          
          \transdissolve<6>
          \begin{uncoverenv}<6->
            \draw[blue] (0.4,0.5) node[yshift=0.3cm] {$p$};
            \draw[blue] (0.4,0.5) node[fill,circle,inner sep=1pt] {};
          \end{uncoverenv}
        \end{scope}
        
        \begin{scope}[/lattice={3}{0.6}{1.2}{2.1}]
          \foreach \i in {0,...,2} {
            \foreach \j in {0,...,2} {
              \draw[red] (\i/3,\j/3) node[fill,circle,inner sep=1pt] {};
            }
          }
        \end{scope}
      \end{tikzpicture}  
    \end{column}
    \begin{column}{0.35\textwidth}
      \begin{onlyenv}<1-3>
        Let \emph{$\alpha\Lambda \subset \Lambda'$}, the map
        \begin{align*}
          \phi_\alpha \;:\; \C/\Lambda &\to \C/\Lambda'\\
          z &\mapsto \alpha z \mod \Lambda'
        \end{align*}
        is a morphism of complex Lie groups.

        \bigskip

        It is called an \emph{isogeny}, and it is completely
        characterized by its \emph{kernel} $\alpha^{-1}\Lambda'$.
      \end{onlyenv}
      \begin{onlyenv}<4-> 
        Taking a point $\rd{b}$ not in the kernel of \emph{$\phi$}, we
        obtain a new degree $\ell$ cover
        \[\emph{\hat{\phi}:\C/\Lambda_2\to\C/\Lambda_3}\]

        The composition \emph{$\hat{\phi}\circ\phi$} has degree
        $\ell^2$ and is \alert{homothetic to the multiplication} by
        $\ell$ map.

        \emph{$\hat{\phi}$} is called the \alert{dual isogeny} of
        \emph{$\phi$}.
      \end{onlyenv}
    \end{column}
  \end{columns}
\end{frame}

%%

\begin{frame}{Isogenies $\leftrightarrow$ ideals}
  \begin{itemize}
  \item Let $E$ be an elliptic curve/complex torus with endomorphism
    ring \emph{$\O\subset\Q(\sqrt{-D})$}.
  \item Let \emph{$G\subset E(\C)$} be a finite subgroup.
  \end{itemize}
  Define the \emph{kernel ideal}

  \[\mathrm{Ann}(G) = \{ \alpha\in\O \;|\; \alpha(G) = 0 \}.\]

  Conversely, given an ideal \emph{$\a\subset \O$}, define

  \[E[\a] = \bigcap_{\alpha\in\a}\ker\alpha.\]

  Finally, let $\mathcal{I}(\O)$ be the group of (fractional) ideals
  of $\O$ and let $\mathcal{P}(\O)$ be the subgroup of principal
  ideals, define the \emph{class group}
  \[\Cl(\O) = \mathcal{I}(\O)/\mathcal{P}(\O).\]
\end{frame}

%%

\begin{frame}{CM dictionary}
  \centering\large
  \setlength{\tabcolsep}{2em}
  \renewcommand{\arraystretch}{2}
  \begin{tabular}{r l}
    \emph{Quadratic imaginary fields} & \emph{Elliptic curves}\\
    \hline
    Integers of $\Q(\sqrt{-D})$ & Endomorphisms of $E$\\
    Integral ideals of $\Q(\sqrt{-D})$ & Isogenies of $E$\\
    Ideal classes in $\Cl(-D)$ & Isogenies \raisebox{-0.8em}{\tikz{\node (E) at (0,0) {$\bullet$}; \node (E1) at (2,0) {$\bullet$}; \draw[->] (E) edge[bend left] (E1) edge[bend right] (E1);}}\\
    Ideal norm & Isogeny degree\\
    Conjugate ideal & Dual isogeny\\
  \end{tabular}
\end{frame}

%%

\begin{frame}{The fundamental theorem of CM}
  \begin{columns}
    \begin{column}{0.6\textwidth}
      \begin{itemize}
      \item Let $E$ be an elliptic curve with CM by a quadratic imaginary
        order $\O$.
      \item Let $\a\subset\O$ be an integral ideal.
      \item Denote by $E/E[\a]$ the image curve of the unique isogeny
        $\phi_\a$ of kernel $E[\a]$.
      \end{itemize}

      \begin{theorem}
        The operator \emph{$\a * E := E/E[\a]$} defines a transitive
        action of the group of fractional ideals of $\O$ on the (finite)
        set \emph{$\E(\O)$} of elliptic curves with complex
        multiplication by $\O$.

        The action factors through principal ideals. In other words, the
        class group \emph{$\Cl(\O)$} acts regularly on \emph{$\E(\O)$}.
      \end{theorem}
    \end{column}
    \begin{column}{0.35\textwidth}
      \centering
      \begin{tikzpicture}
        \begin{scope}
          \def\crater{12}
          \def\jumpa{5}
          \def\diam{2cm}

          \foreach \i in {1,...,\crater} {
            \draw[red] (360/\crater*\i : \diam) to[bend right] (360/\crater*\i+360/\crater : \diam);
            \draw[blue] (360/\crater*\i : \diam) to[bend right=10] (360/\crater*\i+\jumpa*360/\crater : \diam);
          }

          \pgfmathparse{\crater - 1}
          \let\last\pgfmathresult
          \foreach \i in {0,...,\last} {
            \draw[fill] (360/\crater*\i: \diam) circle (2pt) +(360/\crater*\i: 0.5) node{$E_{\i}$};
          }
        \end{scope}
      \end{tikzpicture}
    \end{column}
  \end{columns}
\end{frame}

%%

\begin{frame}{Reduction at $\p$}
  \begin{center}
    Complex multiplication over $\C \sim$ Discrete log in $\Q(e^{2i\pi/N})$
  \end{center}

  \begin{theorem}
    Let $E$ be an elliptic curve over a number field $L$, with CM by
    an order $\O\subset\Q(\sqrt{-D})$.  Let $p$ be a prime split in
    $L$, denote by $E_p$ the reduction of $E$ at a place above $p$,
    and assume that $E_p$ is non-singular.
    \begin{itemize}
    \item If \emph{$\left(\frac{-D}{p}\right)=1$} then $E_p$ is said
      to be \emph{ordinary} and \emph{$\End(E_p)\simeq\O$}.
    \item If \emph{$\left(\frac{-D}{p}\right)=-1$} then $E_p$ is said
      to be \emph{supersingular} and \emph{$\O\subsetneq\End(E_p)$}.
    \end{itemize}
  \end{theorem}

  \bigskip
  
  Complex multiplication over $\F_p$: \emph{Couveignes} '06,
  \emph{Rostovtsev--Stolbunov} '06, \emph{CSIDH} '18, \emph{OSIDH}
  '20, \dots
\end{frame}

%%

\begin{frame}{Key exchange from complex multiplication}
  \begin{description}
  \item[Public parameters:]\
    \begin{itemize}
    \item A \emph{starting} curve $E_0/\F_p$ with complex
      multiplication by $\O\subset\Q(\sqrt{-D})$,
    \item \dots
    \end{itemize}
  \item[Notation:] $[\a] := \a*E_0$.
  \end{description}

  \bigskip
  
  \begin{center}
    \begin{tikzpicture}[x=1.4cm]
      \node at (0,0) {\bf Alice};
      \node at (7,0) {\bf Bob};
      \node at (0,-1) {pick random ideal \alert{$\a$}};
      \node at (7,-1) {pick random ideal \alert{$\b$}};
      \draw[->]
      (1,-2) to node[auto] {$[\a]$} (6,-2);
      \draw[->] (6,-2.5) to node[auto] {$[\b]$} (1,-2.5);
      \node at (3.5,-3.5) {\emph{Shared secret} is \alert{$\a[\b]=[\a\b]=\b[\a]$}};
    \end{tikzpicture}
  \end{center}
\end{frame}

%%

\begin{frame}{A partial converse}
  \begin{block}{Deuring's lifting theorem}
    Let $E_p$ be an elliptic curve in characteristic $p$, with an
    endomorphism $ω_p$ which is not trivial. %
    Then there exists an elliptic curve $E$ defined over a number
    field $L$, an endomorphism $ω$ of $E$, and a non-singular
    reduction of $E$ at a place $\frak{p}$ of $L$ lying above $p$,
    such that $E_p$ is isomorphic to $E(\frak{p})$, and $ω_p$
    corresponds to $ω(\frak{p})$ under the isomorphism.
  \end{block}
\end{frame}

%%

\begin{frame}{The full endomorphism ring}
  \begin{block}{Theorem (Deuring)}
    Let $E$ be a \emph{supersingular} elliptic curve, then
    \begin{itemize}
    \item $E$ is isomorphic to a curve defined over \emph{$\F_{p^2}$};
    \item Every \emph{isogeny} of $E$ is defined over \emph{$\F_{p^2}$};
    \item Every \emph{endomorphism} of $E$ is defined over
      \emph{$\F_{p^2}$};
    \item $\End(E)$ is isomorphic to a \emph{maximal order} in a
      \emph{quaternion algebra} ramified at $p$ and $∞$.
    \end{itemize}
  \end{block}

  In particular:
  \begin{itemize}
  \item If $E$ is defined over $\F_p$, then \emph{$\End_{\F_p}(E)$ is
      strictly contained in $\End(E)$}.
  \item Some endomorphisms \emph{do not commute}!
  \end{itemize}
\end{frame}

%%

\begin{frame}{An example}
  The curve of $j$-invariant \emph{$1728$}
  \[E: y^2 = x^3 + x\]
  is supersingular over $\F_p$ iff $p=-1\mod 4$.

  \begin{block}{Endomorphisms}
    \emph{$\End(E)⊗ℚ = ℚ〈ι,π〉$}, with:
    \begin{itemize}
    \item $π$ the Frobenius endomorphism, s.t. \emph{$π^2=-p$};
    \item $ι$ the map
      \[ι(x,y) = (-x,iy),\]
      where \emph{$i∈\F_{p^2}$} is a 4-th root of unity.
      Clearly, \emph{$ι^2=-1$}.
    \end{itemize}
    And \emph{$ιπ=-πι$}.
  \end{block}
\end{frame}

%%

\begin{frame}{Class group action party}
  \centering
  \begin{tikzpicture}[scale=2]
    \def\crater{11}
    \draw[fill] (360/\crater:1cm) circle (1pt);
    \uncover<1-2>{
      \draw (360/\crater:1.1cm) node[anchor=west] {$j=1728$};
    }
    \uncover<2->{
      \foreach \i in {1,...,\crater} {
        \draw[fill] (360/\crater*\i:1cm) circle (1pt);
        \draw (360/\crater*\i : 1cm) -- (360/\crater*\i+360/\crater : 1cm);
      }
      \draw (0,0) node {$\Cl(-4p)$};
    }
    \uncover<3->{
      \draw (360/\crater:1.5cm) circle (0.5cm) node {$\Cl(-4)$};
    }
    \uncover<4>{
      \draw (360/\crater*4:1.2cm) node[anchor=south east] {$j=0$};
    }
    \uncover<5->{
      \draw (360/\crater*4:1.5cm) circle (0.5cm) node {$\Cl(-3)$};
    }
    \uncover<6->{
      \begin{scope}[shift={(360/\crater*6:1.7cm)}]
        \foreach \i in {0,...,2} {
          \draw[fill] (360/\crater*6 - 120*\i - 180 : 0.7cm) circle (1pt);
          \draw (360/\crater*6 - 120*\i - 180 : 0.7cm) -- (360/\crater*6 - 120*\i+120 - 180 : 0.7cm);
        }
        \draw (0,0) node {$\Cl(-23)$};
      \end{scope}
      \begin{scope}[shift={(360/\crater*10:1.5cm)}]
        \foreach \i in {0,...,4} {
          \draw[fill] (360/\crater*10 - 72*\i - 180 : 0.5cm) circle (1pt);
          \draw (360/\crater*10 - 72*\i - 180 : 0.5cm) -- (360/\crater*10 - 72*\i+72 - 180 : 0.5cm);
        }
        \draw (0,0) node {$\Cl(-79)$};
      \end{scope}
    }
  \end{tikzpicture}
\end{frame}

%%

\begin{frame}{Quaternion algebra?! WTF?\footnote{What The Field?}}
  The quaternion algebra \emph{$B_{p,∞}$} is:
  \begin{itemize}
  \item A \emph{$4$-dimensional} $ℚ$-vector space with basis
    \emph{$(1,i,j,k)$}.
  \item A non-commutative \emph{division algebra}%
    \footnote{All elements have inverses.} %
    $B_{p,∞} = ℚ〈i,j〉$ with the relations:
    \[i^2 = a, \quad j^2 = -p, \quad ij = -ji = k,\]
    for some $a<0$ (depending on $p$).
  \item All elements of $B_{p,∞}$ are \emph{quadratic algebraic
      numbers}.
  \item $B_{p,∞}⊗ℚ_ℓ≃\mathcal{M}_{2×2}(ℚ_ℓ)$ for all $ℓ≠p$.\\
    I.e., endomorphisms restricted to $E[ℓ^e]$ are \emph{just $2×2$
      matrices $\bmod ℓ^e$}.
  \item $B_{p,∞}⊗ℝ$ is isomorphic to Hamilton's quaternions.

  \item $B_{p,∞}⊗ℚ_p$ is a division algebra.
  \end{itemize}
\end{frame}

%%

\begin{frame}{The Deuring correspondence}
  Let $\O,\O'\subset B_{p,\infty}$ be two \emph{maximal orders}.
  They have the same \emph{type} if there exists $\alpha$ s.t.
  \[\O = \alpha\O'\alpha^{-1}.\]

  \begin{theorem}[Deuring]
    Maximal order types of $B_{p,\infty}$ are in one-to-one
    correspondence with supersingular curves up to Galois conjugation
    in $\F_{p^2}/\F_p$.
  \end{theorem}
\end{frame}

%%

\begin{frame}{The Deuring correspondence}
  Two \emph{left ideals} $\a,\b\subset\O$ are in the same \emph{class}
  if there exists $\beta$ s.t. $\a = \b\beta$.

  \begin{block}{An equivalence of categories (Kohel, roughly)}
    \centering
    \begin{tikzpicture}
      \node (O) at (0,0) {$\O$};
      \node (O1) at (6,0) {$\O'$};
      \node (E) at (0,-1) {$E$};
      \node (E1) at (6,-1) {$E'$};
      
      \begin{scope}[gray,anchor=north]
        \node (Oc) at (-2,1) {left order};
        \node at (-2, 1.5) {$\{\alpha\in B_{p,\infty}\;|\; \alpha\a=\a\}$};
        \node (O1c) at (8,1) {right order};
        \node at (8,1.5) {$\{\alpha\in B_{p,\infty}\;|\; \a\alpha=\a\}$};
        \node (ac) at (3,1.5) {connecting ideal (class)};
        
        \node (Ec) at (-2,-2) {supersingular curve};
        \node (E1c) at (8,-2) {supersingular curve};
        \node (phic) at (3,-2) {isogeny (class)};
      \end{scope}
      
      \draw[->] (O) edge node[auto] (a) {$\a$} (O1)
      (E) edge node[auto,swap] (phi) {$\phi_\a$} (E1);
      \draw[dashed,->] (Oc) edge (O) (O1c) edge (O1) (ac) edge (a)
      (Ec) edge (E) (E1c) edge (E1) (phic) edge (phi);
    \end{tikzpicture}
  \end{block}
\end{frame}

%%

\begin{frame}{Supersingular isogeny graphs}
  \begin{columns}
    \begin{column}{0.6\textwidth}
      \begin{itemize}
      \item There is a \emph{unique isogeny class} of supersingular
        curves over $\bar{\F}_p$ of size \emph{$≈ p/12$}.
      \item The graph of isogenies of degree \emph{$\ell$} is
        \emph{$(\ell+1)$}-regular.
      \item It is a \emph{Ramanujan graphs}, i.e., an optimal
        \emph{expander}.
      \item Related to Hecke operators, modular forms, Brandt
        matrices\dots
      \end{itemize}

      \emph{Applications:}
      \begin{itemize}
      \item Hash functions,
      \item Key exchange (SIDH/SIKE),
      \item \dots
      \end{itemize}
    \end{column}
    \begin{column}{0.4\textwidth}
      \centering
      \begin{tikzpicture}
        \begin{scope}[every node/.style={fill,black,circle,inner sep=2pt}]
          \node at (0,0)  (1){};
          \node at (0,4) (20){};
          \node at (2,1)  (16z){};
          \node at (-2,1)  (81z){};
          \node at (-1,2) (77z){};
          \node at (1,2)  (20z){};
          \node at (-2,3)  (85z){};
          \node at (2,3)  (12z){};
        \end{scope}

        \begin{scope}[red]
          \path (1) edge (85z) edge (81z) edge (12z) edge (16z);
          \path (20) edge (85z) edge (77z) edge (20z) edge (12z);
          \path (81z) edge (85z) edge (77z) edge (16z);
          \path (85z) edge (12z);
          \path (12z) edge (16z);
          \path (16z) edge (20z);
          \path (20z) edge[bend right=10] (77z) edge[bend left=10] (77z);
        \end{scope}
      \end{tikzpicture}
      
      \small
      \emph{Figure:} $3$-isogeny graph on $\F_{97^2}$.
    \end{column}
  \end{columns}
\end{frame}

%%

\begin{frame}{SQISign: Signatures from the effective Deuring correspondence}
  \begin{center}
    \begin{tikzpicture}
      \node (E0) at (1,2.5) {$E_0$};
      \node (A) at (0.25,1.75) {$\tau$};
      \node (EA) at (0,1) {$E_A$};
      \draw [blue,dashed] [->] (E0) to (EA);
      \uncover<2->{
        \node (E1) at (4,2.5) {$E_1$};
        \node (B) at (2.5,2.75) {$\psi$};
        \draw [blue] [->] (E0) to (E1);
      }
      \uncover<3->{
        \node (E2) at (4,1) {$E_2$};
        \node (A) at (4.25,1.75) {$\varphi$};
        \draw [red] [->] (E1) to (E2);
      }
      \uncover<4->{
        \node (B) at (2,1.25) {$\sigma$};
        \draw [blue,very thick] [->] (EA) -- (E2);
      }
      % \draw [->] (E) -- (E1);
      \matrix [right] at (6,2) {
        \uncover<2->{
          \node[] (l1) {}; \node (l2) [right of = l1, node distance=0.5in,label=right: commitment isogeny (prover)] {}; \draw [blue] [->] (l1) -- (l2);
        }\\
        \uncover<3->{
          \node[] (l3) {}; \node (l4) [right of = l3, node distance=0.5in,label=right:challenge isogeny (verifier)] {}; \draw [red] [->] (l3) -- (l4);
        }\\
        \uncover<4>{
          \node[] (l1) {}; \node (l2) [right of = l1, node distance=0.5in,label=right: response isogeny (prover)] {}; \draw [blue,very thick] [->] (l1) -- (l2);
        }\\
        \node[] (l1) {}; \node (l2) [right of = l1, node distance=0.5in,label=right: secret key isogeny] {}; \draw  [dashed,blue] [-] (l1) -- (l2); \\
      };
    \end{tikzpicture}
  \end{center}

  \bigskip
  
  \emph{Most compact PQ signature scheme}: PK + Signature combined
  \textbf{5$\times$smaller} than Falcon.

  \begin{table}[h]
    \centering
    \begin{tabular}{ c c c c}
      Secret Key (bytes) & Public Key (bytes) & Signature (bytes) & Security \\
      \hline
      16 & 64 & 204 & NIST-1 \\
      \hline
    \end{tabular}
  \end{table}
\end{frame}

%%

\begin{frame}{Effective correspondences (over finite fields)}
  \begin{description}
  \item[Discrete log:] 
    \begin{tikzpicture}
      \node (g) at (0,0) {$g$};
      \node (gn) at (4,0) {$g^n$};
      \draw [->] (g) edge node[auto] {exp} (gn);
    \end{tikzpicture}
    \begin{itemize}
    \item schoolbook method
    \end{itemize}
  \item[Complex multiplication:]
    \begin{tikzpicture}
      \node (Eo) at (0,-3) {$E$};
      \node (Eo1) at (4,-3) {$E'$};
      \draw [->] (Eo) edge node[auto] {$\a\in\Cl(\O)$} (Eo1);
    \end{tikzpicture}
    \begin{itemize}
    \item Vélu '71, Elkies '92, and many others\dots
    \end{itemize}
  \item[Deuring correspondence:]
  \begin{tikzpicture}
    \node (Es) at (0,-6) {$E$};
    \node (Es1) at (4,-6) {$E'$};
    \draw [->] (Es) edge node[auto] {$\a \subset B_{p,\infty}$} (Es1);
  \end{tikzpicture}
    \begin{itemize}
    \item all of the above,
    \item Kohel, Lauter, Petit, Tignol '14 (KLPT),
    \item D., Kohel, Leroux, Petit, Wesolowski '20 (part of SQISign).
    \end{itemize}
  \end{description}
\end{frame}

%%


%%

\begin{frame}[plain]
  \centering
  \begin{tikzpicture}[remember picture,overlay]
    \begin{scope}[xscale=1.7,yshift=-15,opacity=0.8]
      \def\crater{12}
      \def\jumpa{-8}
      \def\jumpb{9}
      \def\diam{5cm}

      \foreach \i in {1,...,\crater} {
        \draw[blue] (360/\crater*\i : \diam) to[bend right] (360/\crater*\i+360/\crater : \diam);
        \draw[red] (360/\crater*\i : \diam) to[bend right] (360/\crater*\i+\jumpa*360/\crater : \diam);
        \draw[green] (360/\crater*\i : \diam) to[bend right=50] (360/\crater*\i+\jumpb*360/\crater : \diam);
      }
    \end{scope}
    
    \draw (0,0.5) node{\Huge\bf Thank you};
    \draw (0,-1.1) node{\large\url{https://defeo.lu/}};
    \draw (0,-1.8) node{\large\includegraphics[height=0.9em]{twitter.png}~\href{https://twitter.com/luca_defeo}{@luca\_defeo}};
  \end{tikzpicture}
\end{frame}

%%

\end{document}


% LocalWords:  Isogeny abelian isogenies hyperelliptic supersingular Frobenius
% LocalWords:  isogenous
