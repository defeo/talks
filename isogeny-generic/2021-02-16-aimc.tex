\documentclass[aspectratio=169]{beamer}

%\includeonlyframes{current}

\usepackage[utf8]{inputenc}
\usepackage[american]{babel}
\usepackage{amsmath,amsthm}
\usepackage{unicode}
\usepackage{array,tabularx}
\usepackage{ifthen}
\usepackage{tikz}
\usetikzlibrary{matrix,decorations,decorations.text,calc,arrows,snakes,shapes,positioning}
\usepackage{tikzsymbols}
\usepackage[backend=biber,citestyle=authoryear-comp,bibstyle=beamer,doi=false,isbn=false,url=true,maxnames=10]{biblatex}
\bibliography{refs}

\usepackage{ulem}

\mode<presentation>{%
  \usetheme{ibm}
}

\newcommand{\C}{ℂ}
\newcommand{\R}{ℝ}
\newcommand{\Z}{ℤ}
\newcommand{\N}{ℕ}
\newcommand{\Q}{ℚ}
\newcommand{\F}{\mathbb{F}}
\renewcommand{\P}{\mathbb{P}}
\renewcommand{\O}{\mathcal{O}}
\newcommand{\tildO}{\mathcal{\tilde{O}}}
\newcommand{\poly}{\operatorname{poly}}
\newcommand{\polylog}{\operatorname{polylog}}
\newcommand{\End}{\operatorname{End}}
\newcommand{\Hom}{\operatorname{Hom}}
\newcommand{\Gal}{\operatorname{Gal}}
\newcommand{\chr}{\operatorname{char}}
\newcommand{\Cl}{\operatorname{Cl}}
\newcommand{\GL}{\operatorname{GL}}
\renewcommand{\a}{{\mathfrak{a}}}
\renewcommand{\b}{{\mathfrak{b}}}
\newcommand{\g}{{\mathfrak{g}}}
\newcommand{\G}{{\mathcal{G}}}
\newcommand{\E}{{\mathcal{E}}}
\newcommand{\cyc}[1]{{〈 #1 〉}}
\newcommand{\ord}{\operatorname{ord}}
\newcommand{\mat}[1]{\left(\begin{smallmatrix}#1\end{smallmatrix}\right)}
\newcommand{\from}{\overset{\$}{\leftarrow}}

\newcommand{\bl}[1]{\textcolor{blue}{#1}}
\newcommand{\rd}[1]{\textcolor{red}{#1}}
\newcommand{\gr}[1]{\textcolor{green}{#1}}
\newcommand{\og}[1]{\textcolor{orange}{#1}}

\definecolor{light blue}{RGB}{0,102,204}

\newcommand{\myedge}[3]{
  \draw[#3] (360/\crater*#1 : \diam) to[bend right] (360/\crater*#2 : \diam);
}
\newcommand{\sk}[4]{
  \draw[very thick,blue]   (0,0) -- (0,#1);
  \draw[very thick,red]    (1,0) -- (1,#2);
  \draw[very thick,green]  (2,0) -- (2,#3);
  \draw[very thick,orange] (3,0) -- (3,#4);
}
\newcommand{\axes}[4]{
  \clip (#1,#3) rectangle (#2,#4);
  \draw [thin, gray, -latex] (#1,0) -- (#2,0);% Draw x axis
  \draw [thin, gray, -latex] (0,#3) -- (0,#4);% Draw y axis
}
\pgfkeys{/lattice/.code n args={4}{\tikzset{cm={#1,#2,#3,#4,(0,0)}}}}
\newcommand{\lattice}[2]{
  \draw[style=help lines,dashed] (#1-1,#1-1) grid[step=1] (#2+1,#2+1);
  \foreach \x in {#1,...,#2}{
    \foreach \y in {#1,...,#2}{
      \node[draw,circle,inner sep=2pt,fill] at (\x,\y) {};
      % Places a dot at those points
    }
  }
}

\newenvironment<>{goodblock}[1]{%
  \begin{actionenv}#2%
      \def\insertblocktitle{#1}%
      \par%
      \mode<presentation>{%
        \setbeamercolor{block title}{fg=green!60!black,bg=green!50!white}
       \setbeamercolor{block body}{bg=green!20!white}
     }%
      \usebeamertemplate{block begin}}
    {\par\usebeamertemplate{block end}\end{actionenv}}
\newenvironment<>{mehblock}[1]{%
  \begin{actionenv}#2%
      \def\insertblocktitle{#1}%
      \par%
      \mode<presentation>{%
        \setbeamercolor{block title}{fg=yellow!50!black,bg=yellow!50!white}
       \setbeamercolor{block body}{bg=yellow!20!white}
     }%
      \usebeamertemplate{block begin}}
    {\par\usebeamertemplate{block end}\end{actionenv}}
\newenvironment<>{badblock}[1]{%
  \begin{actionenv}#2%
      \def\insertblocktitle{#1}%
      \par%
      \mode<presentation>{%
        \setbeamercolor{block title}{fg=red!60!black,bg=red!50!white}
       \setbeamercolor{block body}{bg=red!20!white}
     }%
      \usebeamertemplate{block begin}}
    {\par\usebeamertemplate{block end}\end{actionenv}}


\newcommand\isoggraph{
    \node[circle,inner sep=.7pt,fill=black] (curve0000) at (4.1000,0.0000) {};
    \node[circle,inner sep=.7pt,fill=black] (curve0158) at (3.9895,0.9455) {};
    \node[circle,inner sep=.7pt,fill=black] (curve0410) at (3.6639,1.8401) {};
    \node[circle,inner sep=.7pt,fill=black] (curve0368) at (3.1408,2.6354) {};
    \node[circle,inner sep=.7pt,fill=black] (curve0404) at (2.4484,3.2887) {};
    \node[circle,inner sep=.7pt,fill=black] (curve0075) at (1.6239,3.7647) {};
    \node[circle,inner sep=.7pt,fill=black] (curve0144) at (0.7120,4.0377) {};
    \node[circle,inner sep=.7pt,fill=black] (curve0191) at (-0.2384,4.0931) {};
    \node[circle,inner sep=.7pt,fill=black] (curve0174) at (-1.1759,3.9278) {};
    \node[circle,inner sep=.7pt,fill=black] (curve0413) at (-2.0500,3.5507) {};
    \node[circle,inner sep=.7pt,fill=black] (curve0379) at (-2.8136,2.9822) {};
    \node[circle,inner sep=.7pt,fill=black] (curve0124) at (-3.4255,2.2530) {};
    \node[circle,inner sep=.7pt,fill=black] (curve0199) at (-3.8527,1.4023) {};
    \node[circle,inner sep=.7pt,fill=black] (curve0390) at (-4.0723,0.4760) {};
    \node[circle,inner sep=.7pt,fill=black] (curve0029) at (-4.0723,-0.4760) {};
    \node[circle,inner sep=.7pt,fill=black] (curve0220) at (-3.8527,-1.4023) {};
    \node[circle,inner sep=.7pt,fill=black] (curve0295) at (-3.4255,-2.2530) {};
    \node[circle,inner sep=.7pt,fill=black] (curve0040) at (-2.8136,-2.9822) {};
    \node[circle,inner sep=.7pt,fill=black] (curve0006) at (-2.0500,-3.5507) {};
    \node[circle,inner sep=.7pt,fill=black] (curve0245) at (-1.1759,-3.9278) {};
    \node[circle,inner sep=.7pt,fill=black] (curve0228) at (-0.2384,-4.0931) {};
    \node[circle,inner sep=.7pt,fill=black] (curve0275) at (0.7120,-4.0377) {};
    \node[circle,inner sep=.7pt,fill=black] (curve0344) at (1.6239,-3.7647) {};
    \node[circle,inner sep=.7pt,fill=black] (curve0015) at (2.4484,-3.2887) {};
    \node[circle,inner sep=.7pt,fill=black] (curve0051) at (3.1408,-2.6354) {};
    \node[circle,inner sep=.7pt,fill=black] (curve0009) at (3.6639,-1.8401) {};
    \node[circle,inner sep=.7pt,fill=black] (curve0261) at (3.9895,-0.9455) {};
    %
    \draw[color=blue] (curve0199) edge[bend right=4mm] (curve0390);
    \draw[color=blue] (curve0051) edge[bend right=4mm] (curve0009);
    \draw[color=blue] (curve0368) edge[bend right=4mm] (curve0404);
    \draw[color=blue] (curve0245) edge[bend right=4mm] (curve0228);
    \draw[color=blue] (curve0029) edge[bend right=4mm] (curve0220);
    \draw[color=blue] (curve0174) edge[bend right=4mm] (curve0413);
    \draw[color=blue] (curve0261) edge[bend right=4mm] (curve0000);
    \draw[color=blue] (curve0379) edge[bend right=4mm] (curve0124);
    \draw[color=blue] (curve0006) edge[bend right=4mm] (curve0245);
    \draw[color=blue] (curve0158) edge[bend right=4mm] (curve0410);
    \draw[color=blue] (curve0228) edge[bend right=4mm] (curve0275);
    \draw[color=blue] (curve0275) edge[bend right=4mm] (curve0344);
    \draw[color=blue] (curve0015) edge[bend right=4mm] (curve0051);
    \draw[color=blue] (curve0191) edge[bend right=4mm] (curve0174);
    \draw[color=blue] (curve0144) edge[bend right=4mm] (curve0191);
    \draw[color=blue] (curve0404) edge[bend right=4mm] (curve0075);
    \draw[color=blue] (curve0009) edge[bend right=4mm] (curve0261);
    \draw[color=blue] (curve0295) edge[bend right=4mm] (curve0040);
    \draw[color=blue] (curve0410) edge[bend right=4mm] (curve0368);
    \draw[color=blue] (curve0413) edge[bend right=4mm] (curve0379);
    \draw[color=blue] (curve0040) edge[bend right=4mm] (curve0006);
    \draw[color=blue] (curve0075) edge[bend right=4mm] (curve0144);
    \draw[color=blue] (curve0220) edge[bend right=4mm] (curve0295);
    \draw[color=blue] (curve0390) edge[bend right=4mm] (curve0029);
    \draw[color=blue] (curve0344) edge[bend right=4mm] (curve0015);
    \draw[color=blue] (curve0124) edge[bend right=4mm] (curve0199);
    \draw[color=blue] (curve0000) edge[bend right=4mm] (curve0158);
    %
    \draw[color=red] (curve0009) edge[bend right=8mm] (curve0379);
    \draw[color=red] (curve0124) edge[bend right=8mm] (curve0015);
    \draw[color=red] (curve0006) edge[bend right=8mm] (curve0368);
    \draw[color=red] (curve0228) edge[bend right=8mm] (curve0075);
    \draw[color=red] (curve0245) edge[bend right=8mm] (curve0404);
    \draw[color=red] (curve0029) edge[bend right=8mm] (curve0261);
    \draw[color=red] (curve0220) edge[bend right=8mm] (curve0000);
    \draw[color=red] (curve0368) edge[bend right=8mm] (curve0220);
    \draw[color=red] (curve0144) edge[bend right=8mm] (curve0006);
    \draw[color=red] (curve0261) edge[bend right=8mm] (curve0124);
    \draw[color=red] (curve0191) edge[bend right=8mm] (curve0245);
    \draw[color=red] (curve0015) edge[bend right=8mm] (curve0174);
    \draw[color=red] (curve0344) edge[bend right=8mm] (curve0191);
    \draw[color=red] (curve0275) edge[bend right=8mm] (curve0144);
    \draw[color=red] (curve0158) edge[bend right=8mm] (curve0390);
    \draw[color=red] (curve0295) edge[bend right=8mm] (curve0158);
    \draw[color=red] (curve0075) edge[bend right=8mm] (curve0040);
    \draw[color=red] (curve0174) edge[bend right=8mm] (curve0228);
    \draw[color=red] (curve0000) edge[bend right=8mm] (curve0199);
    \draw[color=red] (curve0379) edge[bend right=8mm] (curve0344);
    \draw[color=red] (curve0390) edge[bend right=8mm] (curve0009);
    \draw[color=red] (curve0040) edge[bend right=8mm] (curve0410);
    \draw[color=red] (curve0410) edge[bend right=8mm] (curve0029);
    \draw[color=red] (curve0404) edge[bend right=8mm] (curve0295);
    \draw[color=red] (curve0413) edge[bend right=8mm] (curve0275);
    \draw[color=red] (curve0051) edge[bend right=8mm] (curve0413);
    \draw[color=red] (curve0199) edge[bend right=8mm] (curve0051);
    %
    \draw[color=green] (curve0158) edge[bend left=10mm] (curve0144);
    \draw[color=green] (curve0009) edge[bend left=10mm] (curve0368);
    \draw[color=green] (curve0124) edge[bend left=10mm] (curve0295);
    \draw[color=green] (curve0174) edge[bend left=10mm] (curve0390);
    \draw[color=green] (curve0368) edge[bend left=10mm] (curve0174);
    \draw[color=green] (curve0344) edge[bend left=10mm] (curve0000);
    \draw[color=green] (curve0144) edge[bend left=10mm] (curve0124);
    \draw[color=green] (curve0295) edge[bend left=10mm] (curve0275);
    \draw[color=green] (curve0029) edge[bend left=10mm] (curve0245);
    \draw[color=green] (curve0051) edge[bend left=10mm] (curve0410);
    \draw[color=green] (curve0015) edge[bend left=10mm] (curve0158);
    \draw[color=green] (curve0275) edge[bend left=10mm] (curve0261);
    \draw[color=green] (curve0075) edge[bend left=10mm] (curve0379);
    \draw[color=green] (curve0379) edge[bend left=10mm] (curve0220);
    \draw[color=green] (curve0220) edge[bend left=10mm] (curve0228);
    \draw[color=green] (curve0006) edge[bend left=10mm] (curve0015);
    \draw[color=green] (curve0191) edge[bend left=10mm] (curve0199);
    \draw[color=green] (curve0228) edge[bend left=10mm] (curve0009);
    \draw[color=green] (curve0040) edge[bend left=10mm] (curve0344);
    \draw[color=green] (curve0261) edge[bend left=10mm] (curve0404);
    \draw[color=green] (curve0404) edge[bend left=10mm] (curve0413);
    \draw[color=green] (curve0390) edge[bend left=10mm] (curve0006);
    \draw[color=green] (curve0000) edge[bend left=10mm] (curve0075);
    \draw[color=green] (curve0245) edge[bend left=10mm] (curve0051);
    \draw[color=green] (curve0413) edge[bend left=10mm] (curve0029);
    \draw[color=green] (curve0199) edge[bend left=10mm] (curve0040);
    \draw[color=green] (curve0410) edge[bend left=10mm] (curve0191);
}

\newcommand\fpsqgraph{
    \node[circle,inner sep=.7pt,fill=black] (j_190_344i) at (1.0,0.0) {};
    \node[circle,inner sep=.7pt,fill=black] (j_379_325i) at (0.985615910348,0.169000820322) {};
    \node[circle,inner sep=.7pt,fill=black] (j_143_000i) at (0.942877445461,0.333139794742) {};
    \node[circle,inner sep=.7pt,fill=black] (j_304_364i) at (0.873014113161,0.487694943814) {};
    \node[circle,inner sep=.7pt,fill=black] (j_356_000i) at (0.778035754318,0.628219997296) {};
    \node[circle,inner sep=.7pt,fill=black] (j_004_000i) at (0.66067472339,0.750672305253) {};
    \node[circle,inner sep=.7pt,fill=black] (j_242_000i) at (0.524307283557,0.851529137733) {};
    \node[circle,inner sep=.7pt,fill=black] (j_234_000i) at (0.37285647778,0.927889027297) {};
    \node[circle,inner sep=.7pt,fill=black] (j_065_081i) at (0.210679269996,0.977555238948) {};
    \node[circle,inner sep=.7pt,fill=black] (j_118_209i) at (0.0424412031961,0.999098966205) {};
    \node[circle,inner sep=.7pt,fill=black] (j_125_000i) at (-0.127017819747,0.991900435259) {};
    \node[circle,inner sep=.7pt,fill=black] (j_316_000i) at (-0.292822771277,0.956166734739) {};
    \node[circle,inner sep=.7pt,fill=black] (j_426_306i) at (-0.450203744818,0.89292585815) {};
    \node[circle,inner sep=.7pt,fill=black] (j_358_000i) at (-0.594633176304,0.803997130367) {};
    \node[circle,inner sep=.7pt,fill=black] (j_241_000i) at (-0.721956093955,0.691938868978) {};
    \node[circle,inner sep=.7pt,fill=black] (j_419_000i) at (-0.828509649244,0.559974786138) {};
    \node[circle,inner sep=.7pt,fill=black] (j_061_000i) at (-0.911228490388,0.411901248244) {};
    \node[circle,inner sep=.7pt,fill=black] (j_102_000i) at (-0.967732946933,0.251978061385) {};
    \node[circle,inner sep=.7pt,fill=black] (j_190_087i) at (-0.996397488543,0.0848059244755) {};
    \node[circle,inner sep=.7pt,fill=black] (j_107_000i) at (-0.996397488543,-0.0848059244755) {};
    \node[circle,inner sep=.7pt,fill=black] (j_304_067i) at (-0.967732946933,-0.251978061385) {};
    \node[circle,inner sep=.7pt,fill=black] (j_019_000i) at (-0.911228490388,-0.411901248244) {};
    \node[circle,inner sep=.7pt,fill=black] (j_381_000i) at (-0.828509649244,-0.559974786138) {};
    \node[circle,inner sep=.7pt,fill=black] (j_319_000i) at (-0.721956093955,-0.691938868978) {};
    \node[circle,inner sep=.7pt,fill=black] (j_065_350i) at (-0.594633176304,-0.803997130367) {};
    \node[circle,inner sep=.7pt,fill=black] (j_067_000i) at (-0.450203744818,-0.89292585815) {};
    \node[circle,inner sep=.7pt,fill=black] (j_000_000i) at (-0.292822771277,-0.956166734739) {};
    \node[circle,inner sep=.7pt,fill=black] (j_315_299i) at (-0.127017819747,-0.991900435259) {};
    \node[circle,inner sep=.7pt,fill=black] (j_422_000i) at (0.0424412031961,-0.999098966205) {};
    \node[circle,inner sep=.7pt,fill=black] (j_379_106i) at (0.210679269996,-0.977555238948) {};
    \node[circle,inner sep=.7pt,fill=black] (j_189_000i) at (0.37285647778,-0.927889027297) {};
    \node[circle,inner sep=.7pt,fill=black] (j_141_042i) at (0.524307283557,-0.851529137733) {};
    \node[circle,inner sep=.7pt,fill=black] (j_141_389i) at (0.66067472339,-0.750672305253) {};
    \node[circle,inner sep=.7pt,fill=black] (j_118_222i) at (0.778035754318,-0.628219997296) {};
    \node[circle,inner sep=.7pt,fill=black] (j_315_132i) at (0.873014113161,-0.487694943814) {};
    \node[circle,inner sep=.7pt,fill=black] (j_150_000i) at (0.942877445461,-0.333139794742) {};
    \node[circle,inner sep=.7pt,fill=black] (j_426_125i) at (0.985615910348,-0.169000820322) {};
    %
    \draw[color=blue] (j_319_000i) edge (j_426_125i);
    \draw[color=orange] (j_419_000i) edge (j_141_389i);
    \draw[color=orange] (j_065_081i) edge (j_190_087i);
    \draw[color=blue] (j_102_000i) edge (j_143_000i);
    \draw[color=orange] (j_102_000i) edge (j_358_000i);
    \draw[color=blue] (j_004_000i) edge (j_102_000i);
    \draw[color=blue] (j_107_000i) edge (j_316_000i);
    \draw[color=blue] (j_143_000i) edge (j_234_000i);
    \draw[color=blue] (j_304_364i) edge (j_379_106i);
    \draw[color=blue] (j_065_081i) edge (j_426_306i);
    \draw[color=blue] (j_242_000i) edge (j_356_000i);
    \draw[color=blue] (j_125_000i) edge (j_125_000i);
    \draw[color=orange] (j_242_000i) edge (j_242_000i);
    \draw[color=orange] (j_065_350i) edge (j_141_389i);
    \draw[color=blue] (j_150_000i) edge (j_190_344i);
    \draw[color=orange] (j_379_325i) edge (j_426_125i);
    \draw[color=orange] (j_319_000i) edge (j_304_067i);
    \draw[color=blue] (j_102_000i) edge (j_315_132i);
    \draw[color=orange] (j_316_000i) edge (j_379_325i);
    \draw[color=blue] (j_356_000i) edge (j_426_125i);
    \draw[color=orange] (j_234_000i) edge (j_242_000i);
    \draw[color=blue] (j_379_106i) edge (j_379_325i);
    \draw[color=orange] (j_419_000i) edge (j_141_042i);
    \draw[color=blue] (j_000_000i) edge (j_000_000i);
    \draw[color=orange] (j_358_000i) edge (j_381_000i);
    \draw[color=orange] (j_107_000i) edge (j_190_087i);
    \draw[color=blue] (j_241_000i) edge (j_190_344i);
    \draw[color=blue] (j_422_000i) edge (j_141_389i);
    \draw[color=orange] (j_065_081i) edge (j_118_209i);
    \draw[color=blue] (j_358_000i) edge (j_422_000i);
    \draw[color=blue] (j_019_000i) edge (j_304_067i);
    \draw[color=orange] (j_379_106i) edge (j_426_306i);
    \draw[color=blue] (j_189_000i) edge (j_304_067i);
    \draw[color=blue] (j_019_000i) edge (j_304_364i);
    \draw[color=blue] (j_061_000i) edge (j_315_132i);
    \draw[color=blue] (j_381_000i) edge (j_419_000i);
    \draw[color=orange] (j_319_000i) edge (j_304_364i);
    \draw[color=blue] (j_426_125i) edge (j_426_306i);
    \draw[color=orange] (j_315_132i) edge (j_426_306i);
    \draw[color=blue] (j_065_350i) edge (j_190_087i);
    \draw[color=blue] (j_150_000i) edge (j_319_000i);
    \draw[color=blue] (j_107_000i) edge (j_189_000i);
    \draw[color=blue] (j_067_000i) edge (j_234_000i);
    \draw[color=orange] (j_102_000i) edge (j_125_000i);
    \draw[color=blue] (j_150_000i) edge (j_190_087i);
    \draw[color=orange] (j_356_000i) edge (j_422_000i);
    \draw[color=orange] (j_150_000i) edge (j_189_000i);
    \draw[color=blue] (j_141_042i) edge (j_315_132i);
    \draw[color=blue] (j_141_042i) edge (j_304_364i);
    \draw[color=blue] (j_419_000i) edge (j_419_000i);
    \draw[color=orange] (j_118_209i) edge (j_315_299i);
    \draw[color=blue] (j_065_350i) edge (j_118_209i);
    \draw[color=orange] (j_061_000i) edge (j_356_000i);
    \draw[color=orange] (j_019_000i) edge (j_241_000i);
    \draw[color=blue] (j_241_000i) edge (j_381_000i);
    \draw[color=orange] (j_143_000i) edge (j_150_000i);
    \draw[color=blue] (j_141_389i) edge (j_315_299i);
    \draw[color=blue] (j_107_000i) edge (j_118_209i);
    \draw[color=orange] (j_241_000i) edge (j_118_222i);
    \draw[color=blue] (j_118_222i) edge (j_315_299i);
    \draw[color=blue] (j_141_042i) edge (j_141_389i);
    \draw[color=orange] (j_065_350i) edge (j_190_344i);
    \draw[color=orange] (j_107_000i) edge (j_190_344i);
    \draw[color=blue] (j_356_000i) edge (j_426_306i);
    \draw[color=orange] (j_118_222i) edge (j_315_132i);
    \draw[color=orange] (j_141_042i) edge (j_426_125i);
    \draw[color=blue] (j_061_000i) edge (j_356_000i);
    \draw[color=blue] (j_125_000i) edge (j_118_222i);
    \draw[color=blue] (j_107_000i) edge (j_118_222i);
    \draw[color=blue] (j_316_000i) edge (j_379_325i);
    \draw[color=blue] (j_061_000i) edge (j_315_299i);
    \draw[color=orange] (j_067_000i) edge (j_242_000i);
    \draw[color=orange] (j_379_106i) edge (j_379_325i);
    \draw[color=orange] (j_381_000i) edge (j_315_299i);
    \draw[color=orange] (j_315_299i) edge (j_426_125i);
    \draw[color=blue] (j_358_000i) edge (j_381_000i);
    \draw[color=orange] (j_190_087i) edge (j_304_067i);
    \draw[color=blue] (j_316_000i) edge (j_379_106i);
    \draw[color=blue] (j_067_000i) edge (j_419_000i);
    \draw[color=orange] (j_316_000i) edge (j_356_000i);
    \draw[color=blue] (j_065_350i) edge (j_426_125i);
    \draw[color=orange] (j_019_000i) edge (j_234_000i);
    \draw[color=orange] (j_004_000i) edge (j_004_000i);
    \draw[color=orange] (j_189_000i) edge (j_234_000i);
    \draw[color=blue] (j_000_000i) edge (j_241_000i);
    \draw[color=blue] (j_319_000i) edge (j_426_306i);
    \draw[color=orange] (j_190_344i) edge (j_304_364i);
    \draw[color=blue] (j_241_000i) edge (j_190_087i);
    \draw[color=blue] (j_125_000i) edge (j_118_209i);
    \draw[color=blue] (j_150_000i) edge (j_189_000i);
    \draw[color=blue] (j_019_000i) edge (j_125_000i);
    \draw[color=orange] (j_316_000i) edge (j_379_106i);
    \draw[color=orange] (j_000_000i) edge (j_125_000i);
    \draw[color=blue] (j_143_000i) edge (j_065_081i);
    \draw[color=blue] (j_004_000i) edge (j_319_000i);
    \draw[color=blue] (j_019_000i) edge (j_358_000i);
    \draw[color=blue] (j_242_000i) edge (j_242_000i);
    \draw[color=blue] (j_190_344i) edge (j_379_325i);
    \draw[color=orange] (j_061_000i) edge (j_358_000i);
    \draw[color=blue] (j_065_081i) edge (j_190_344i);
    \draw[color=orange] (j_067_000i) edge (j_304_067i);
    \draw[color=orange] (j_065_081i) edge (j_141_042i);
    \draw[color=blue] (j_422_000i) edge (j_141_042i);
    \draw[color=blue] (j_102_000i) edge (j_315_299i);
    \draw[color=blue] (j_242_000i) edge (j_422_000i);
    \draw[color=blue] (j_143_000i) edge (j_065_350i);
    \draw[color=orange] (j_065_350i) edge (j_118_222i);
    \draw[color=orange] (j_125_000i) edge (j_143_000i);
    \draw[color=orange] (j_004_000i) edge (j_019_000i);
    \draw[color=orange] (j_143_000i) edge (j_422_000i);
    \draw[color=blue] (j_061_000i) edge (j_234_000i);
    \draw[color=orange] (j_419_000i) edge (j_422_000i);
    \draw[color=blue] (j_141_389i) edge (j_304_067i);
    \draw[color=blue] (j_065_081i) edge (j_118_222i);
    \draw[color=blue] (j_067_000i) edge (j_316_000i);
    \draw[color=blue] (j_118_209i) edge (j_315_132i);
    \draw[color=orange] (j_241_000i) edge (j_118_209i);
    \draw[color=orange] (j_067_000i) edge (j_304_364i);
    \draw[color=blue] (j_190_087i) edge (j_379_106i);
    \draw[color=orange] (j_102_000i) edge (j_319_000i);
    \draw[color=blue] (j_189_000i) edge (j_304_364i);
    \draw[color=blue] (j_304_067i) edge (j_379_325i);
    \draw[color=orange] (j_381_000i) edge (j_315_132i);
    \draw[color=orange] (j_141_389i) edge (j_426_306i);
    \draw[color=orange] (j_061_000i) edge (j_107_000i);
}


\title{Isogeny Based Cryptography}
\subtitle{the new frontier of number theoretic cryptography}
\author{Luca De Feo}
\date[Feb 17, 2021, AIMC 2021]{February 17, 2021\\
  Annual Iranian Mathematics Conference 2021}
\institute{IBM Research Zürich}

\begin{document}

\frame[plain]{\titlepage}

%%

\begin{frame}{Elliptic curves}
  \begin{columns}
    \begin{column}{0.4\textwidth}
      Let $k$ be a field of characteristic $\ne 2,3$.

      An \emph{elliptic curve \textit{defined over $k$}} is the locus
      in the \emph{projective space} $\P^2(\bar{k})$ of an equation
      \[\emph{Y^2Z = X^3 + aXZ^2 + bZ^3},\]
      where $a,b\in k$ and $4a^3+27b^2\ne 0$.

      \begin{itemize}
      \item $\O=(0:1:0)$ is the \emph{point at infinity};
      \item $y^2 = x^3 + ax + b$ is the \emph{affine Weierstrass equation}.
      \end{itemize}
    \end{column}
    \begin{column}{0.6\textwidth}
      \begin{center}
        \begin{tikzpicture}[domain=-2.4566:4,samples=100,yscale=1/2]
          \draw plot (\x,{sqrt(\x*\x*\x-4*\x+5)});
          \draw plot (\x,{-sqrt(\x*\x*\x-4*\x+5)});

          \draw[thin,gray,-latex] (0,-7) -- (0,7);
          \draw[thin,gray,-latex] (-3,0) -- (4,0);
        \end{tikzpicture}
      \end{center}
    \end{column}
  \end{columns}
\end{frame}

%%

\begin{frame}{The group law}
  \begin{columns}
    \begin{column}{0.4\textwidth}
      \begin{block}{Bezout's theorem}
        Every line cuts $E$ in exactly three points (counted with
        multiplicity).
      \end{block}

      Define a \emph{group law} such that any three colinear points
      add up to zero.

      \begin{itemize}
      \item The law is \emph{algebraic}\\ (it has \textit{formulas});
      \item The law is \emph{commutative};
      \item $\O$ is the \emph{group identity};
      \item \emph{Opposite points} have the same $x$-value.
      \end{itemize}
    \end{column}
    \begin{column}{0.6\textwidth}
      \begin{center}
        \begin{tikzpicture}[domain=-2.4566:4,samples=100,yscale=1/2]
          \draw plot (\x,{sqrt(\x*\x*\x-4*\x+5)});
          \draw plot (\x,{-sqrt(\x*\x*\x-4*\x+5)});

          \draw[thin,gray,-latex] (0,-7) -- (0,7);
          \draw[thin,gray,-latex] (-3,0) -- (4,0);
          \draw (-3,1) -- (4,8/3+3);
          \begin{scope}[every node/.style={draw,circle,inner sep=1pt,fill},cm={1,2/3,0,0,(0,3)}]
            \node at (-2.287980,0) {};
            \node at (-0.535051,0) {};
            \node at (3.267475,0) {};
          \end{scope}
          \begin{scope}[every node/.style={yshift=0.3cm},cm={1,2/3,0,0,(0,3)}]
            \node at (-2.287980,0) {$P$};
            \node at (-0.535051,0) {$Q$};
            \node at (3.267475,0) {$R$};
          \end{scope}

          \draw[dashed] (3.267475,3.267475*2/3+3) -- (3.267475,-3.267475*2/3-3) 
          node[draw,circle,inner sep=1pt,fill] {}
          node[xshift=-0.1cm,anchor=east] {$P+Q$};
        \end{tikzpicture}
      \end{center}
    \end{column}
  \end{columns}
\end{frame}

%%

\begin{frame}{Why do cryptographers care? (Diffie--Hellman key exchange)}
  \begin{description}
  \item[Goal:] Alice and Bob have never met before. They are chatting
    over a public channel, and want to agree on a \emph{shared secret}
    to start a private conversation.
  \item[Setup:] They agree on a (large) cyclic group
    $G=\langle g\rangle$ of order $N$.
  \end{description}

  \bigskip
  \begin{center}
    \begin{tikzpicture}[x=1.4cm]
      \node at (0,0) {\bf Alice};
      \node at (7,0) {\bf Bob};
      \node at (0,-1) {pick random \alert{$a\in\Z/N\Z$}};
      \node at (0,-1.5) {compute $A=g^a$};
      \node at (7,-1) {pick random \alert{$b\in\Z/N\Z$}};
      \node at (7,-1.5) {compute $B=g^b$};
      \draw[->]
      (1,-2) to node[auto] {$A$} (6,-2);
      \draw[->] (6,-2.5) to node[auto] {$B$} (1,-2.5);
      \node at (3.5,-3.5) {\emph{Shared secret} is \alert{$B^a=g^{ab}=A^b$}};
    \end{tikzpicture}
  \end{center}
\end{frame}

%%

\begin{frame}{Brief history of DH key exchange}
  \begin{description}
  \item[1976] Diffie \& Hellman publish \emph{New directions in
      cryptography}, suggest using $G=\F_p^\ast$.
  \item[1978] Pollard publishes his \emph{discrete logarithm}
    algorithm ($O(\sqrt{\# G})$ complexity).
  \item[1980] Miller and Koblitz independently suggest using
    \emph{elliptic curves} $G=E(\F_p)$.
  \item[1994] Shor publishes his \emph{quantum discrete logarithm /
      factoring} algorithm.
  \item[2005] NSA standardizes elliptic curve key agreement (ECDH) and
    signatures ECDSA.
  \item[2017] $\sim 70\%$ of web traffic is secured by ECDH and/or
    ECDSA.
  \item[2017] NIST launches \emph{post-quantum competition}, says
    ``not to bother moving to elliptic curves, if you haven't yet''.
  \item[2020] NIST calls the finalists for the competition. Elliptic
    curves are still running, thanks to \emph{SIKE}, the
    \emph{Supersingular Isogeny Key Encapsulation} scheme.
  \end{description}
\end{frame}

%%

\begin{frame}
  \frametitle{Elliptic curves}

  \begin{columns}
    \begin{column}{0.7\textwidth}
      \centering
      \begin{tikzpicture}[scale=2]
        \axes{-1}{3.5}{-0.5}{3}

        \begin{scope}[/lattice={1}{0.2}{0.4}{0.7}]
          \begin{uncoverenv}<1>
            \draw[fill,black!10] (0,0) -- (1,0) -- (1,1) -- (0,1) -- (0,0);
            \node at (0.5,0.5) {$\C/\Lambda$};
            \node at (0.9,-0.1) {$\omega_1$};
            \node at (-0.1,0.9) {$\omega_2$};
          \end{uncoverenv}

          \lattice{-3}{4}

          \begin{uncoverenv}<2-5>
            \node[red] at (0.7,0.65) {$a$}; 
            \node[draw,circle,inner sep=1pt,fill,red] at (0.8,0.5) {};
            \node[red] at (0.2,0.9) {$b$}; 
            \node[draw,circle,inner sep=1pt,fill,red] at (0.3,0.7) {};
            
            \begin{uncoverenv}<3-4>
              \node[red] at (1.2,1.3) {$a+b$}; 
              \node[draw,circle,inner sep=1pt,fill,red] at (1.1,1.2) {};
              \begin{uncoverenv}<3>
                \draw[red,thin] (0,0) -- (0.8,0.5) -- (1.1,1.2);
                \draw[red,thin] (0,0) -- (0.3,0.7) -- (1.1,1.2);          
              \end{uncoverenv}
            \end{uncoverenv}

            \transdissolve<5>
            \begin{uncoverenv}<5>
              \node[red] at (0.2,0.3) {$a+b$}; 
              \node[draw,circle,inner sep=1pt,fill,red] at (0.1,0.2) {};
            \end{uncoverenv}
          \end{uncoverenv}
        \end{scope}  
      \end{tikzpicture}
    \end{column}
    \begin{column}{0.3\textwidth}
      \begin{onlyenv}<1>
        Let $\omega_1,\omega_2\in\C$ be linearly independent complex
        numbers. Set
        \[\emph{\Lambda = \omega_1\Z \oplus \omega_2\Z}\]

        $\C/\Lambda$ is an elliptic curve.
      \end{onlyenv}

      \begin{onlyenv}<2->
        Addition law induced by addition on $\mathbb{C}$.
      \end{onlyenv}
    \end{column}
  \end{columns}
\end{frame}

%%

\begin{frame}
  \frametitle{Multiplication}

  \centering
  \begin{tikzpicture}[scale=2.2]
    \axes{-2}{5}{-0.5}{3}

    \begin{scope}[/lattice={1}{0.2}{0.4}{0.7}]
      \lattice{-4}{5}
    
      \node[red,yshift=0.2cm] at (0.8,0.6) {$a$}; 
      \draw[red] (0.8,0.6) node[fill,circle,inner sep=1pt] {};

      \begin{uncoverenv}<2>
        \node[red,yshift=0.2cm] at (2.4,1.8) {$[3]a$}; 
        \draw[red] (0,0) -- (1.6,1.2) node[fill,circle,inner sep=1pt] {} 
        -- (2.4,1.8) node[fill,circle,inner sep=1pt] {};
      \end{uncoverenv}

      \transdissolve<3>
      \begin{uncoverenv}<3>
        \node[red,yshift=0.3cm] at (0.4,0.8) {$[3]a$}; 
        \draw[red] (0.4,0.8) node[fill,circle,inner sep=1pt] {};
      \end{uncoverenv}
    \end{scope}
  \end{tikzpicture}
\end{frame}

%%

\begin{frame}
  \frametitle{Torsion subgroups}

  \begin{columns}
    \begin{column}{0.65\textwidth}
      \centering
      \begin{tikzpicture}[scale=1.8]
        \axes{-0.3}{4.5}{-0.5}{4};

        \begin{scope}[/lattice={3}{0.6}{1.2}{2.1}]
          \lattice{-1}{2}

          \foreach \i in {0,...,2} {
            \foreach \j in {0,...,2} {
              \draw[red] (\i/3,\j/3) node[fill,circle,inner sep=1pt] {};
            }
          }
          \draw[red] (0,0) -- (1/3,0) node[yshift=0.2cm] {$a$};
          \draw[red] (0,0) -- (0,1/3) node[yshift=0.2cm] {$b$};
        \end{scope}
      \end{tikzpicture}  
    \end{column}
    \begin{column}{0.35\textwidth}
      The $\ell$-torsion subgroup is made up by the points
      \[\emph{\left(\frac{i\omega_1}{\ell},\frac{j\omega_2}{\ell}\right)}\]

      It is a group of rank two
      \begin{alertenv}
        \begin{align*}
          E[\ell] &= \langle a,b \rangle\\
          &\simeq (\Z/\ell\Z)^2
        \end{align*}
      \end{alertenv}
    \end{column}
  \end{columns}
\end{frame}

%%

\begin{frame}
  \frametitle{Isogenies}

  \begin{columns}
    \begin{column}{0.65\textwidth}
      \centering
      \begin{tikzpicture}[scale=1.8]
        \axes{-0.3}{4.5}{-0.5}{4};
        
        \begin{scope}[/lattice={3}{0.6}{1.2}{2.1}]
          \uncover<1->{\lattice{-1}{2}}
          
          \begin{uncoverenv}<1-3>
            \draw[red] (0,0) -- (1/3,0) node[yshift=0.3cm] {$a$};
          \end{uncoverenv}
          \begin{uncoverenv}<4->
            \draw[red] (0,0) -- (0,1/3) node[yshift=0.3cm] {$b$};
          \end{uncoverenv}

          \begin{uncoverenv}<1-2>
            \draw[blue] (0.8,0.5) node[yshift=0.3cm] {$p$};
            \draw[blue] (0.8,0.5) node[fill,circle,inner sep=1pt] {};
          \end{uncoverenv}
        \end{scope}
        
        \begin{scope}[/lattice={1}{0.2}{1.2}{2.1}]
          \transdissolve<2>
          \begin{scope}[opacity=0.3]
            \uncover<2-4>{\lattice{-3}{5}}
          \end{scope}
          
          \transdissolve<3>
          \begin{uncoverenv}<3-5>
            \draw[blue] (0.4,0.5) node[yshift=0.3cm] {$p$};
            \draw[blue] (0.4,0.5) node[fill,circle,inner sep=1pt] {};
          \end{uncoverenv}
        \end{scope}

        \begin{scope}[/lattice={1}{0.2}{0.4}{0.7}]
          \transdissolve<5>
          \begin{scope}[opacity=0.3]
            \uncover<5->{\lattice{-3}{5}}
          \end{scope}
          
          \transdissolve<6>
          \begin{uncoverenv}<6->
            \draw[blue] (0.4,0.5) node[yshift=0.3cm] {$p$};
            \draw[blue] (0.4,0.5) node[fill,circle,inner sep=1pt] {};
          \end{uncoverenv}
        \end{scope}
        
        \begin{scope}[/lattice={3}{0.6}{1.2}{2.1}]
          \foreach \i in {0,...,2} {
            \foreach \j in {0,...,2} {
              \draw[red] (\i/3,\j/3) node[fill,circle,inner sep=1pt] {};
            }
          }
        \end{scope}
      \end{tikzpicture}  
    \end{column}
    \begin{column}{0.35\textwidth}
      \begin{onlyenv}<1-3>
        Let $\rd{a}\in\C/\Lambda_1$ be an $\ell$-torsion point, and let
        \[\emph{\Lambda_2 = a\Z\oplus\Lambda_1}\]
        Then \emph{$\Lambda_1\subset\Lambda_2$} and we define a degree
        $\ell$ cover
        \[\emph{\phi:\C/\Lambda_1\to\C/\Lambda_2}\]

        \emph{$\phi$} is a morphism of complex Lie groups and is called an
        \alert{isogeny}.
      \end{onlyenv}
      \begin{onlyenv}<4-> 
        Taking a point $\rd{b}$ not in the kernel of \emph{$\phi$}, we
        obtain a new degree $\ell$ cover
        \[\emph{\hat{\phi}:\C/\Lambda_2\to\C/\Lambda_3}\]

        The composition \emph{$\hat{\phi}\circ\phi$} has degree
        $\ell^2$ and is \alert{homothetic to the multiplication} by
        $\ell$ map.

        \emph{$\hat{\phi}$} is called the \alert{dual isogeny} of
        \emph{$\phi$}.
      \end{onlyenv}
    \end{column}
  \end{columns}
\end{frame}

%% 

\begin{frame}{What is \alt<2->{\xout{scalar multiplication} an
      isogeny}{scalar multiplication}?}

  \begin{overlayarea}{\textwidth}{4em}
    \Large
    \[
      \alt<3->{ϕ}{[n]}
      \;:\; P \mapsto
      \alt<3->{ϕ(P)}{\underbrace{P + P + \cdots + P}_{n\text{ times}}}\]
  \end{overlayarea}
  
  \begin{itemize}
  \item A map \emph{$E\to \alt<4->{\xout{E} E'}{E\phantom{\xout{}}}$},
  \item a \emph{group morphism},
  \item with \emph{finite kernel}\\
    \alt<5->{(\xout{the torsion group $E[n]≃(ℤ/nℤ)^2$} any
      finite subgroup \emph{$H\subset E$})}{(the torsion group
      \emph{$E[n]\simeq(ℤ/nℤ)^2$})},
  \item \emph{surjective} (in the algebraic closure),
  \item given by \emph{rational maps} of degree \alt<6->{\xout{$n^2$}
      \emph{$\#H$}}{\emph{$n^2$}}.
  \end{itemize}

  \begin{uncoverenv}<7->
    (Separable) isogenies $⇔$ finite subgroups:
    \alert{\[0 → H → E \overset{ϕ}{→} E' \to 0\]}
    The kernel \emph{$H$} determines the image curve \emph{$E'$} up to
    isomorphism \[\emph{E/H\overset{\text{\tiny def}}{=}E'}.\]
  \end{uncoverenv}
\end{frame}

%%

\begin{frame}{Isogenies: an example over $\F_{11}$}
  \centering
  \begin{tikzpicture}[scale=0.4]
    \begin{scope}
      \node[anchor=center] at (0,7) {$E \;:\; y^2 = x^3 + x$};

      \uncover<-1>{
        \draw[thin,gray] (0,-6) -- (0,6);
        \draw[thin,gray] (-6,0) -- (6,0);
      }

      \foreach \x/\y in {0/0,5/3,-4/3,-3/5,-2/1,-1/3} {
        \draw[blue,fill] (\x,\y) circle (0.2) node(E_\x_\y){}
        (\x,-\y) circle (0.2) node(E_\x_-\y){};
      }

      \uncover<2->{\draw[red,fill] (0,0) circle (0.3);}
    \end{scope}

    \draw[black!10!white,thick] (10,-7) -- +(0,14);
    
    \begin{scope}[shift={(20,0)}]
      \node at (0,7) {$E' \;:\; y^2 = x^3 - 4x$};

      \uncover<-1>{
        \draw[thin,gray] (0,-6) -- (0,6);
        \draw[thin,gray] (-6,0) -- (6,0);
      }

      \foreach \x/\y in {0/0,2/0,3/2,4/2,6/4,-2/0,-1/5} {
        \draw[color=blue,fill] (\x,\y) circle (0.2) node(F_\x_\y){}
        (\x,-\y) circle (0.2) node(F_\x_-\y){};
      }
    \end{scope}

    \begin{scope}[color=red,-latex,dashed]
      \begin{uncoverenv}<2->
        \path
        (E_5_3) edge (F_3_2)
        (E_-4_3) edge (F_4_-2)
        (E_-3_5) edge (F_4_2)
        (E_-2_1) edge (F_3_-2)
        (E_-1_3) edge (F_-2_0);
      \end{uncoverenv}
      \begin{uncoverenv}<2->
        \path
        (E_5_-3) edge (F_3_-2)
        (E_-4_-3) edge (F_4_2)
        (E_-3_-5) edge (F_4_-2)
        (E_-2_-1) edge (F_3_2)
        (E_-1_-3) edge (F_-2_0);
      \end{uncoverenv}
    \end{scope}
  \end{tikzpicture}
  
  \begin{columns}
    \begin{column}{0.5\textwidth}
      \[\phi(x,y) = \left(\frac{x^2 + 1}{x},\quad y\frac{x^2-1}{x^2}\right)\]
    \end{column}
    \begin{column}{0.5\textwidth}
      \begin{itemize}
      \item<2-> Kernel generator in \alert{red}.
      \item<2-> This is a degree $2$ map.
      \item<2-> Analogous to $x\mapsto x^2$ in $\F_q^*$.
      \end{itemize}
    \end{column}
  \end{columns}
\end{frame}

%%

\begin{frame}{Up to \emph{isomorphism}}
  \begin{center}
    \begin{tikzpicture}[domain=-2.4566:4,samples=100]
      \newcount\zoomout
      \animatevalue<5-7>{\zoomout}{0}{10}
      \begin{uncoverenv}<-7>
        \begin{scope}[scale=1-0.09*\zoomout]
          \begin{scope}
            \draw[thin,gray,-latex] (0,-4) -- (0,4);
            \draw[thin,gray,-latex] (-4.2,0) -- (7,0);
          \end{scope}
          
          \newcount\xstretch
          \newcount\ystretch
          \newcount\slant
          \animatevalue<1-2>{\xstretch}{0}{4}
          \animatevalue<2-3>{\ystretch}{0}{4}
          \animatevalue<3-4>{\slant}{0}{4}      
          \begin{scope}[yscale=0.55-0.05*\the\ystretch,xscale=1+0.1*\the\xstretch,xslant=0.02*\slant]
            \draw plot (\x,{sqrt(\x*\x*\x-4*\x+5)});
            \draw plot (\x,{-sqrt(\x*\x*\x-4*\x+5)});

            \begin{uncoverenv}<-5>
              \draw (-3,1) -- (4,8/3+3);
              \begin{scope}[every node/.style={draw,circle,inner sep=1pt,fill},cm={1,2/3,0,0,(0,3)}]
                \node at (-2.287980,0) {};
                \node at (-0.535051,0) {};
                \node at (3.267475,0) {};
              \end{scope}
              \begin{scope}[every node/.style={yshift=0.3cm},cm={1,2/3,0,0,(0,3)}]
                \node at (-2.287980,0) {$P$};
                \node at (-0.535051,0) {$Q$};
                \node at (3.267475,0) {$R$};
              \end{scope}
              \draw[dashed] (3.267475,3.267475*2/3+3) -- (3.267475,-3.267475*2/3-3) 
              node[draw,circle,inner sep=1pt,fill] {}
              node[xshift=-0.1cm,anchor=east] {$P+Q$};
            \end{uncoverenv}
          \end{scope}

          \begin{uncoverenv}<5>
            \node[anchor=west] at (-4,-3) {\Large\alert{$y^2=x^3+ax+b \quad\longrightarrow\quad j\equiv 1728\frac{4a^3}{4a^3+27b^2}$}};
          \end{uncoverenv}
        \end{scope}
      \end{uncoverenv}
      
      \begin{uncoverenv}<8->
        \draw[fill] (0,0) circle (2pt) node[anchor=north] {$j=1728$};
        \uncover<9>{
          \draw (0.1,0) edge[bend left,->] node[auto] {$\phi$} (7,0);
        }
        \uncover<9->{
          \draw[fill] (7.1,0) circle (2pt) node[anchor=north] {$j=287496$};
        }
        \uncover<10->{
          \draw (0.1,0) edge[bend left,<->,red,very thick] (7,0);
        }
      \end{uncoverenv}
    \end{tikzpicture}
  \end{center}  
\end{frame}

%%

\begin{frame}{The beauty and the beast {\quad\footnotesize(credit: Lorenz Panny)}}
  \begin{overlayarea}{\textwidth}{\textheight}
    \smallskip
    \begin{center}

      \alt<1>{
        Components of particular isogeny graphs look like this:
      }{
        At this time, there are \underline{\smash{two distinct families}} of systems:
      }
      \par\vspace{1ex}

      \begin{minipage}{.49\textwidth}\centering
        \begin{tikzpicture}[scale=.6,>=stealth,shorten >=.2mm,shorten <=.2mm,rotate=90,line width=.6pt]
          \isoggraph
        \end{tikzpicture}

        \vspace{.2ex}

        \only<2->{
          $\F_p$ \\[.7ex]
          \textbf{CSIDH}
          \,\raisebox{.2ex}{\small{[pron.: sea-side]}}
          \\ {\small
            \url{https://csidh.isogeny.org}
            \par}
        }
      \end{minipage}
      % 
      \begin{minipage}{.49\textwidth}\centering
        \begin{tikzpicture}[scale=2.456,>=stealth,shorten >=.2mm,shorten <=.2mm,rotate=90,line width=.6pt]
          \fpsqgraph
        \end{tikzpicture}

        \vspace{.2ex}

        \only<2->{
          $\F_{p^2}$ \\[.7ex]
          \textbf{SIDH}
          \\ {\small
            \url{https://sike.org}
            \par}
        }
      \end{minipage}

      \only<1>{
        \vspace{3ex}

        \textit{Which of these is good for crypto?}
      }

    \end{center}
  \end{overlayarea}
\end{frame}

%%

\begin{frame}{Brief history of isogeny-based cryptography}
  \begin{description}
  \item[1997] Couveignes introduces the \emph{Hard Homogeneous Spaces}
    framework. His work stays unpublished for 10 years.
  \item[2006] Rostovtsev \& Stolbunov independently rediscover
    Couveignes ideas, suggest isogeny-based Diffie--Hellman as a
    \emph{quantum-resistant} primitive.
  \item[2006-2010] Other isogeny-based protocols by Teske and Charles,
    Goren \& Lauter.
  \item[2011-2012] D., Jao \& Plût introduce \emph{SIDH}, an
    efficient post-quantum key exchange inspired by Couveignes,
    Rostovtsev, Stolbunov, Charles, Goren, Lauter.
  \item[2017] SIDH is submitted to the NIST competition (with the name
    \emph{SIKE}, only isogeny-based candidate).
  \item[2018] Castryck, Lange, Martindale, Panny \& Renes create an
    efficient variant of the Couveignes--Rostovtsev--Stolbunov
    protocol, named \emph{CSIDH}.
  \item[2019] Isogeny signature craze: \emph{SeaSign}
    (D. \& Galbraith; Decru, Panny \& Vercauteren), \emph{CSI-FiSh}
    (Beullens, Kleinjung \& Vercauteren), \emph{VDF} (D., Masson,
    Petit \& Sanso).
  \item[2020] Isogeny signatures get interesting: \emph{SQISign}
    (D., Kohel, Leroux, Petit, Wesolowski).\\
    SIKE is an \emph{Alternate candidate finalist} in NIST's 3rd round.
  \end{description}
\end{frame}

%%

\begin{frame}
  \centering
  \begin{tikzpicture}
    \draw[blue,thick] (90:3.5) -- (210:3.5) -- (330:3.5) -- (90:3.5);
    
    \node[rotate=0] at (270:2.5) {Modular functions};
    \node[rotate=60] at (150:2.5) {Class field theory};
    \node[rotate=-60] at (30:2.5) {Elliptic curves};

    \begin{uncoverenv}<1>
      \begin{scope}[gray]
        \node[rotate=0] at (270:3.2) {$j(z) = \frac{1}{q} + 744 + 196884q + \cdots$};
        \node[anchor=east] at (150:3.2) {$H(j) = j - 1728$};
        \node[anchor=west] at (30:3.2) {$y^2 = x^3 - ax - b$};
      \end{scope}
    \end{uncoverenv}

    \begin{uncoverenv}<2>
      \begin{scope}[yshift=0.5cm]
        \node[anchor=east] (C) at (150:4.5) {Abelian extensions};
        \node[anchor=east,below of=C] {of $\Q(\sqrt{-D})$};
        \node[anchor=west] (E) at (30:4.5) {Elliptic curves with};
        \node[anchor=west,below of=E] {$\End(E) \subset \Q(\sqrt{-D})$};
      \end{scope}
    \end{uncoverenv}
    
    \begin{uncoverenv}<3>
      \begin{scope}[yshift=1.5cm]
        \node[anchor=east] (C) at (150:3) {Galois group of $K/\Q(\sqrt{-D})$};
        \node[anchor=east,below of=C] (C1) {$\simeq$};
        \node[anchor=east,below of=C1] {Class group $\Cl(-D)$};
        \node[anchor=west] (E) at (30:3) {$\Cl(-D)$ acts on set of $E$ s.t.};
        \node[anchor=west,below of=E] {$\End(E) \subset \Q(\sqrt{-D})$};
      \end{scope}
    \end{uncoverenv}
    
    \begin{scope}
      \Large
      \node at (0,0.3) {\emph{Complex}};
      \node at (0,-0.5) {\emph{Multiplication}};
    \end{scope}
  \end{tikzpicture}
\end{frame}

%%

\begin{frame}{Complex multiplication dictionary}
  \centering\large
  \setlength{\tabcolsep}{2em}
  \renewcommand{\arraystretch}{2}
  \begin{tabular}{r l}
    \emph{Quadratic imaginary fields} & \emph{Elliptic curves}\\
    \hline
    Integers of $\Q(\sqrt{-D})$ & Endomorphisms of $E$\\
    Integral ideals of $\Q(\sqrt{-D})$ & Isogenies of $E$\\
    Ideal classes in $\Cl(-D)$ & Isogenies \raisebox{-0.8em}{\tikz{\node (E) at (0,0) {$\bullet$}; \node (E1) at (2,0) {$\bullet$}; \draw[->] (E) edge[bend left] (E1) edge[bend right] (E1);}}\\
    Ideal norm & Isogeny degree\\
    Conjugate ideal & Dual isogeny\\
  \end{tabular}
\end{frame}

%%

\begin{frame}
  \begin{block}{Group action}
    \emph{$\G\circlearrowright\E$}: A (finite) set $\E$ \emph{acted
      upon} by a group $\G$ \emph{faithfully} and \emph{transitively}:
    \begin{align*}
      * : \G × \E &→ \E\\
      \g * E &↦ E'
    \end{align*}
    \par\begin{description}
    \item[Compatibility:] \emph{$\g' * (\g * E) = (\g'\g)*E$} for all
      $\g,\g'\in\G$ and $E\in\E$;
    \item[Identity:] \emph{$\mathfrak{e} * E = E$} if and only if
      $\mathfrak{e}\in\G$ is the identity element;
    \item[Transitivity:] for all $E,E'\in\E$ there exist a
      \emph{unique $\g\in\G$} such that \emph{$\g*E'=E$}.
      \setlength{\itemsep}{2em}
    \end{description}
  \end{block}
  
  \begin{block}{Hard Homogeneous Space (HHS) --- Couveignes 1996}
    \emph{$\G\circlearrowright\E$} such that $\G$ is commutative and:
    \begin{itemize}
    \item \emph{Evaluating} $E' = \g*E$ is \emph{easy};
    \item \emph{Inverting} the action is \emph{hard}.
    \end{itemize}
  \end{block}
\end{frame}

% %%

\begin{frame}{HHS Diffie--Hellman}
  \begin{description}
  \item[Goal:] Alice and Bob have never met before. They are chatting
    over a public channel, and want to agree on a \emph{shared secret}
    to start a private conversation.
  \item[Setup:] They agree on a (large) \emph{HHS
      $\G\circlearrowright \E$} of order $N$.
  \end{description}

  \begin{center}
    \begin{tikzpicture}[x=1.4cm]
      \node at (0,0) {\bf Alice};
      \node at (7,0) {\bf Bob};
      \node at (0,-1) {pick random \alert{$\a\in\G$}};
      \node at (0,-1.5) {compute $E_A=\a*E_0$};
      \node at (7,-1) {pick random \alert{$\b\in\G$}};
      \node at (7,-1.5) {compute $E_B=\b*E_0$};
      \draw[->]
      (1,-2) to node[auto] {$E_A$} (6,-2);
      \draw[->] (6,-2.5) to node[auto] {$E_B$} (1,-2.5);
      \node at (3.5,-3.5) {\emph{Shared secret} is \alert{$\a*E_B=(\a\b)*E_0=\b*E_A$}};
    \end{tikzpicture}
  \end{center}  
\end{frame}

%%

\begin{frame}{HHSDH from complex multiplication}
  \begin{columns}
    \begin{column}{0.6\textwidth}
      \textbf{Obstacles:}
      \begin{itemize}
      \item The \emph{group size} of $\Cl(-D)$ is \emph{unknown}.
      \item Only ideals of small norm (\emph{isogenies of small degree})
        are efficient to evaluate.
      \end{itemize}

      \textbf{Solution:}
      \begin{itemize}
      \item Restrict to elements of $\Cl(-D)$ of the form
        \[\g = \prod \a_i^{e_i}\]
        for a basis of $\a_i$ of \emph{small norm}.
      \item Equivalent to doing \emph{isogeny walks} of \emph{smooth
          degree}.
      \end{itemize}
    \end{column}
    \begin{column}{0.4\textwidth}
      \centering
      \begin{tikzpicture}
        \begin{scope}
          \def\crater{12}
          \def\jumpa{-8}
          \def\jumpb{9}
          \def\diam{2cm}

          \foreach \i in {1,...,\crater} {
            \draw[blue] (360/\crater*\i : \diam) to[bend right] (360/\crater*\i+360/\crater : \diam);
            \draw[red] (360/\crater*\i : \diam) to[bend right] (360/\crater*\i+\jumpa*360/\crater : \diam);
            \draw[green] (360/\crater*\i : \diam) to[bend right=50] (360/\crater*\i+\jumpb*360/\crater : \diam);
          }
          \foreach \i in {1,...,\crater} {
            \draw[fill] (360/\crater*\i: \diam) circle (2pt) +(360/\crater*\i: 0.4) node{$E_{\i}$};
          }
        \end{scope}
      \end{tikzpicture}
    \end{column}
  \end{columns}
\end{frame}

%% 

\begin{frame}
  \frametitle{Couveignes/Rostovtsev--Stolbunov/CSIDH key exchange}

  \begin{columns}
    \begin{column}{0.55\textwidth}
      \begin{tikzpicture}
        \begin{scope}
          \def\crater{12}
          \def\jumpa{-8}
          \def\jumpb{9}
          \def\diam{2.5cm}
          
          \foreach \i in {1,...,\crater} {
            \pgfmathparse{int(mod(2^\i,13))}
            \let\exp\pgfmathresult
            \draw[fill] (360/\crater*\i: \diam) circle (2pt);
          }
          \uncover<2,6->{
            % Alice 1
            \myedge{0}{1}{blue}\myedge{1}{5}{red}\myedge{5}{6}{blue}\myedge{6}{3}{green}
          }
          \uncover<3,5>{
            % Bob 1
            \begin{scope}[dashed,thick]
              \myedge{0}{4}{red}\myedge{4}{8}{red}\myedge{8}{5}{green}\myedge{5}{6}{blue}
            \end{scope}
          }
          \uncover<5>{
            % Alice 2
            \myedge{6}{7}{blue}\myedge{7}{11}{red}\myedge{11}{0}{blue}\myedge{0}{9}{green}
          }
          \uncover<6->{
            % Bob 2
            \begin{scope}[dashed,thick]
              \myedge{3}{7}{red}\myedge{7}{11}{red}\myedge{11}{8}{green}\myedge{8}{9}{blue}
            \end{scope}
          }

          \draw (0 : \diam + 0.4cm) node {$E_0$};
          \uncover<2->{\draw (360/\crater*3 : \diam + 0.4cm) node {$E_A$};}
          \uncover<3->{\draw (360/\crater*6 : \diam + 0.4cm) node {$E_B$};}
          \uncover<5->{\draw (360/\crater*9 : \diam + 0.4cm) node {$E_{BA}\uncover<6->{=E_{AB}}$};}
        \end{scope}
      \end{tikzpicture}  
    \end{column}    
    \begin{column}{0.45\textwidth}
      \textbf{Public parameters:}
      \begin{itemize}
      \item A starting curve $E_0/\F_p$;
      \item A set of small prime degree isogenies.
      \end{itemize}
      \begin{enumerate}
      \item<2-> \textbf{Alice} takes a \alert{secret} random walk
        \emph{$ϕ_A:E_0\to E_A$} of length \emph{$O(\log p)$};
      \item<3-> \textbf{Bob} does the same;
      \item<4-> They publish \emph{$E_A$} and \emph{$E_B$};
      \item<5-> \textbf{Alice} repeats her secret walk \emph{$ϕ_A$}
        starting from \emph{$E_B$}.
      \item<6-> \textbf{Bob} repeats his secret walk \emph{$ϕ_B$}
        starting from \emph{$E_A$}.
      \end{enumerate}
    \end{column}
  \end{columns}
\end{frame}

%%

\begin{frame}{Quantum security}

  \textbf{Fact:} Shor's algorithm \emph{does not apply} to Diffie-Hellman
  protocols from \emph{group actions}.

  \begin{block}{Subexponential attack\hfill\emph{$\exp(\sqrt{\log p\log\log p})$}}
    \begin{itemize}
    \item Reduction to the \emph{hidden shift problem} by evaluating
      the class group action in \emph{quantum
        supersposition} (subexpoential cost);
    \item Well known reduction from the hidden shift to the
      \emph{dihedral (non-abelian) hidden subgroup problem};
    \item Kuperberg's algorithm solves the dHSP with a subexponential
      number of class group evaluations.
    \item Recent work suggests that $2^{64}$-qbit security is achieved
      somewhere in $512 < \log p < 2048$.
    \end{itemize}
  \end{block}
\end{frame}

%%

\begin{frame}{Supersingular curves}
  \begin{block}{Theorem (Deuring)}
    Let $E$ be an elliptic curve defined over a field $k$ of
    characteristic $p$.\\
    $\End(E)$ is isomorphic to one of the following:
    \begin{itemize}
    \item $\Z$, only if $p=0$
      \begin{flushright}
        $E$ is \emph{ordinary}.
      \end{flushright}
    \item An order $\O$ in a quadratic imaginary field:
      \begin{flushright}
        $E$ is \emph{ordinary} with \emph{complex multiplication} by
        $\O$.
      \end{flushright}
    \item Only if $p>0$, a maximal order in a quaternion
      algebra\footnote{(ramified at $p$ and $\infty$)}:
      \begin{flushright}
        $E$ is \emph{supersingular}.
      \end{flushright}
    \end{itemize}
  \end{block}
\end{frame}

%%

\begin{frame}
  \frametitle{Key exchange with supersingular curves (Jao \& D. 2011)}
  
  \begin{description}
  \item[Good news:] there is no action of a commutative class group.
  \item[Bad news:] there is no action of a commutative class group.
  \item[Idea:] Let \bl{Alice} and \rd{Bob} walk in two
    \emph{different isogeny graphs} on the \emph{same vertex set}.
  \end{description}

  \begin{columns}
    \begin{column}{0.6\textwidth}
      \centering
      \begin{tikzpicture}[scale=1.4]
        \begin{scope}[every node/.style={fill,black,circle,inner sep=2pt}]
          \node at (0,0)  (1){};
          \node at (0,4) (20){};
          \node at (2,1)  (16z){};
          \node at (-2,1)  (81z){};
          \node at (-1,2) (77z){};
          \node at (1,2)  (20z){};
          \node at (-2,3)  (85z){};
          \node at (2,3)  (12z){};
        \end{scope}

        \begin{uncoverenv}<1->
          \begin{scope}[blue,every loop/.style={looseness=50}]
            \path (1) edge (20) edge (16z) edge (81z);
            \path (20) edge[loop left] (20) edge[loop right] (20);
            \path (16z) edge (81z) edge (77z);
            \path (81z) edge (20z);
            \path (77z) edge (20z) edge (85z);
            \path (20z) edge (12z);
            \path (12z) edge[bend right=10] (85z) edge[bend left=10] (85z);
          \end{scope}
        \end{uncoverenv}
        
        \begin{uncoverenv}<1->
          \begin{scope}[red]
            \path (1) edge (85z) edge (81z) edge (12z) edge (16z);
            \path (20) edge (85z) edge (77z) edge (20z) edge (12z);
            \path (81z) edge (85z) edge (77z) edge (16z);
            \path (85z) edge (12z);
            \path (12z) edge (16z);
            \path (16z) edge (20z);
            \path (20z) edge[bend right=10] (77z) edge[bend left=10] (77z);
          \end{scope}
        \end{uncoverenv}
      \end{tikzpicture}
    \end{column}
    \begin{column}{0.4\textwidth}
      \small
      \emph{Figure:} \bl{$2$}- and \rd{$3$}-isogeny graphs on $\F_{97^2}$.
    \end{column}
  \end{columns}
\end{frame}

%%

\begin{frame}
  \frametitle{Key exchange with supersingular curves (Jao \& D. 2011)}

  \begin{itemize}
  \item Fix small primes \bl{$\ell_A$}, \rd{$\ell_B$};
  \item \emph{No canonical labeling} of the \bl{$\ell_A$}- and
    \rd{$\ell_B$}-isogeny graphs; \emph{however\dots}
  \end{itemize}

  \begin{center}
    \bf
    Walk of length \bl{$e_A$}\\
    $=$\\
    Isogeny of degree \bl{$\ell_A^{e_A}$}\\
    $=$\\
    Kernel \bl{$\langle P\rangle\subset E[\ell_A^{e_A}]$}
  \end{center}
  
  \begin{center}
    \begin{tikzpicture}
      \begin{scope}
        \draw (0,1.2) node[anchor=east,blue] {$\ker\phi=〈P〉\subset E[\ell_A^{e_A}]$};
        \draw (0,0.4) node[anchor=east,red] {$\ker\psi=〈Q〉\subset E[\ell_B^{e_B}]$};
        \draw (0,-0.4) node[anchor=east,blue] {$\ker\phi' = 〈\rd{\psi}(P)〉$};
        \draw (0,-1.2) node[anchor=east,red] {$\ker\psi' = 〈\bl{\phi}(Q)〉$};
      \end{scope}
      \begin{scope}[xshift=4.5cm,coils/.style={-angle 90,decorate,decoration={coil,aspect=0,amplitude=1pt}}]
        \large
        \node[matrix of nodes, ampersand replacement=\&, column sep=3cm, row sep=1.5cm] (diagram) {
          |(E)| $E$ \& |(Es)| $E/〈\bl{P}〉$ \\
          |(Ep)| {$E/〈\rd{Q}〉$} \& |(Eps)| {$E/〈\bl{P},\rd{Q}〉$}\\
        };
        \path[->,blue] (E) edge[coils] node[auto] {$\phi$} (Es);
        \path[->,blue] (Ep) edge[coils] node[auto,swap] {$\phi'$} (Eps);
        \path[->,red] (E) edge[coils] node[auto,swap] {$\psi$} (Ep);
        \path[->,red] (Es) edge[coils] node[auto] {$\psi'$} (Eps);
      \end{scope}
    \end{tikzpicture}
  \end{center}
\end{frame}

%%

\begin{frame}{From 10 minutes to 10ms in 20 years}
  \begin{tikzpicture}[xscale=1.3,gray,every node/.style={anchor=south}]
    \draw[->] (0,0) -- (11,0);
    \uncover<1->{
      \node at (1,-0.5) {1997};
      \draw (1,-0.1) -- +(0,0.3) node[blue]{Couveignes' key exchange};
    }
    \uncover<2->{
      \node at (3,-0.5) {2006};
      \draw (3,-0.1) -- +(0,1.3) node[blue]{Rostovstev \& Stolbunov (> 5 min)};
    }
    \uncover<3->{
      \node at (4,-0.5) {2011};
      \draw (4,-0.1) -- +(0,2.3) node[red]{SIDH (500ms) \tiny(Jao and D.)};
    }
    \uncover<4->{
      \node at (5,-0.5) {2012};
      \draw (5,-0.1) -- +(0,3.3) node[red]{SIDH (50ms) \tiny(D., Jao, Plût)};
    }
    \uncover<5->{
      \node at (6,-0.5) {2016};
      \draw (6,-0.1) -- +(0,4.3) node[red]{SIDH (30ms) \tiny(Costello, Longa, Naherig)};
    }
    \uncover<6->{
      \node at (7,-0.5) {2017};
      \draw (7,-0.1) -- +(0,5.3) node[red]{SIKE (10ms) \tiny(NIST candidate)};
    }
    \uncover<7->{
      \node at (8,-0.5) {2018};
      \draw (8,-0.1) -- +(0,2.3) node[blue]{CSIDH (50ms)};
    }
    \uncover<8>{
      \node at (9,-0.5) {2019};
      \draw (9,-0.1) -- +(0,3.3) node[blue]{CSIDH (35ms) \tiny(Meyer, Reith)};
    }
  \end{tikzpicture}
\end{frame}

%%

\begin{frame}{Contemporary research}
  \large
  \begin{itemize}
  \item Efficient \emph{signature schemes} and \emph{proofs of
      knowledge};
  \item Quaternionic multiplication $\to$ \emph{SQISign};
  \item Higher dimensional abelian varieties;
  \item Cryptanalysis;
  \item \emph{Side-channel} protections;
  \item Lower complexity bounds and \emph{delay protocols};
  \item Trusted generation of random supersingular curves;
  \item Prime searches;
  \item \dots
  \end{itemize}
\end{frame}

%%

\begin{frame}[plain]
  \centering
  \begin{tikzpicture}[remember picture,overlay]
    \begin{scope}[xscale=1.7,yshift=-15,opacity=0.8]
      \def\crater{12}
      \def\jumpa{-8}
      \def\jumpb{9}
      \def\diam{5cm}

      \foreach \i in {1,...,\crater} {
        \draw[blue] (360/\crater*\i : \diam) to[bend right] (360/\crater*\i+360/\crater : \diam);
        \draw[red] (360/\crater*\i : \diam) to[bend right] (360/\crater*\i+\jumpa*360/\crater : \diam);
        \draw[green] (360/\crater*\i : \diam) to[bend right=50] (360/\crater*\i+\jumpb*360/\crater : \diam);
      }
    \end{scope}
    
    \draw (0,0.5) node{\Huge\bf Thank you};
    \draw (0,-1.1) node{\large\url{https://defeo.lu/}};
    \draw (0,-1.8) node{\large\includegraphics[height=0.9em]{twitter.png}~\href{https://twitter.com/luca_defeo}{@luca\_defeo}};
  \end{tikzpicture}
\end{frame}

%%

\end{document}


% LocalWords:  Isogeny abelian isogenies hyperelliptic supersingular Frobenius
% LocalWords:  isogenous
