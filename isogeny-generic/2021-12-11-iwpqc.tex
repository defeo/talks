\documentclass[aspectratio=169]{beamer}

%\includeonlyframes{current}

\usepackage[utf8]{inputenc}
\usepackage[american]{babel}
\usepackage{amsmath,amsthm}
\usepackage{unicode}
\usepackage{array,tabularx}
\usepackage{ifthen}
\usepackage{tikz}
\usetikzlibrary{matrix,decorations,decorations.text,calc,arrows,snakes,shapes,positioning}
\usepackage{tikzsymbols}
\usepackage[backend=biber,citestyle=authoryear-comp,bibstyle=beamer,doi=false,isbn=false,url=true,maxnames=10]{biblatex}
\bibliography{refs}
\usepackage{multimedia}

\usepackage{savesym}
\savesymbol{checkmark}
\usepackage{dingbat}
\restoresymbol{ding}{checkmark}

\usepackage{ulem}

\mode<presentation>{%
  \usetheme{ibm}
}

\newcommand{\C}{ℂ}
\newcommand{\R}{ℝ}
\newcommand{\Z}{ℤ}
\newcommand{\N}{ℕ}
\newcommand{\Q}{ℚ}
\newcommand{\F}{\mathbb{F}}
\renewcommand{\P}{\mathbb{P}}
\renewcommand{\O}{\mathcal{O}}
\newcommand{\tildO}{\mathcal{\tilde{O}}}
\newcommand{\poly}{\operatorname{poly}}
\newcommand{\polylog}{\operatorname{polylog}}
\newcommand{\End}{\operatorname{End}}
\newcommand{\Hom}{\operatorname{Hom}}
\newcommand{\Gal}{\operatorname{Gal}}
\newcommand{\chr}{\operatorname{char}}
\newcommand{\Cl}{\operatorname{Cl}}
\newcommand{\GL}{\operatorname{GL}}
\renewcommand{\a}{{\mathfrak{a}}}
\renewcommand{\b}{{\mathfrak{b}}}
\newcommand{\p}{{\mathfrak{p}}}
\newcommand{\q}{{\mathfrak{q}}}
\newcommand{\g}{{\mathfrak{g}}}
\newcommand{\G}{{\mathcal{G}}}
\newcommand{\E}{{\mathcal{E}}}
\newcommand{\cyc}[1]{{〈 #1 〉}}
\newcommand{\ord}{\operatorname{ord}}
\newcommand{\mat}[1]{\left(\begin{smallmatrix}#1\end{smallmatrix}\right)}
\newcommand{\from}{\overset{\$}{\leftarrow}}

\newcommand{\bl}[1]{\textcolor{blue}{#1}}
\newcommand{\rd}[1]{\textcolor{red}{#1}}
\newcommand{\gr}[1]{\textcolor{green}{#1}}
\newcommand{\og}[1]{\textcolor{orange}{#1}}

\definecolor{light blue}{RGB}{0,102,204}

\newcommand{\myedge}[3]{
  \draw[#3] (360/\crater*#1 : \diam) to[bend right] (360/\crater*#2 : \diam);
}
\newcommand{\sk}[4]{
  \draw[very thick,blue]   (0,0) -- (0,#1);
  \draw[very thick,red]    (1,0) -- (1,#2);
  \draw[very thick,green]  (2,0) -- (2,#3);
  \draw[very thick,orange] (3,0) -- (3,#4);
}
\newcommand{\axes}[4]{
  \clip (#1,#3) rectangle (#2,#4);
  \draw [thin, gray, -latex] (#1,0) -- (#2,0);% Draw x axis
  \draw [thin, gray, -latex] (0,#3) -- (0,#4);% Draw y axis
}
\pgfkeys{/lattice/.code n args={4}{\tikzset{cm={#1,#2,#3,#4,(0,0)}}}}
\newcommand{\lattice}[2]{
  \draw[style=help lines,dashed] (#1-1,#1-1) grid[step=1] (#2+1,#2+1);
  \foreach \x in {#1,...,#2}{
    \foreach \y in {#1,...,#2}{
      \node[draw,circle,inner sep=2pt,fill] at (\x,\y) {};
      % Places a dot at those points
    }
  }
}

\newenvironment<>{goodblock}[1]{%
  \begin{actionenv}#2%
      \def\insertblocktitle{#1}%
      \par%
      \mode<presentation>{%
        \setbeamercolor{block title}{fg=green!60!black,bg=green!50!white}
       \setbeamercolor{block body}{bg=green!20!white}
     }%
      \usebeamertemplate{block begin}}
    {\par\usebeamertemplate{block end}\end{actionenv}}
\newenvironment<>{mehblock}[1]{%
  \begin{actionenv}#2%
      \def\insertblocktitle{#1}%
      \par%
      \mode<presentation>{%
        \setbeamercolor{block title}{fg=yellow!50!black,bg=yellow!50!white}
       \setbeamercolor{block body}{bg=yellow!20!white}
     }%
      \usebeamertemplate{block begin}}
    {\par\usebeamertemplate{block end}\end{actionenv}}
\newenvironment<>{badblock}[1]{%
  \begin{actionenv}#2%
      \def\insertblocktitle{#1}%
      \par%
      \mode<presentation>{%
        \setbeamercolor{block title}{fg=red!60!black,bg=red!50!white}
       \setbeamercolor{block body}{bg=red!20!white}
     }%
      \usebeamertemplate{block begin}}
    {\par\usebeamertemplate{block end}\end{actionenv}}


\newcommand\isoggraph{
    \node[circle,inner sep=.7pt,fill=black] (curve0000) at (4.1000,0.0000) {};
    \node[circle,inner sep=.7pt,fill=black] (curve0158) at (3.9895,0.9455) {};
    \node[circle,inner sep=.7pt,fill=black] (curve0410) at (3.6639,1.8401) {};
    \node[circle,inner sep=.7pt,fill=black] (curve0368) at (3.1408,2.6354) {};
    \node[circle,inner sep=.7pt,fill=black] (curve0404) at (2.4484,3.2887) {};
    \node[circle,inner sep=.7pt,fill=black] (curve0075) at (1.6239,3.7647) {};
    \node[circle,inner sep=.7pt,fill=black] (curve0144) at (0.7120,4.0377) {};
    \node[circle,inner sep=.7pt,fill=black] (curve0191) at (-0.2384,4.0931) {};
    \node[circle,inner sep=.7pt,fill=black] (curve0174) at (-1.1759,3.9278) {};
    \node[circle,inner sep=.7pt,fill=black] (curve0413) at (-2.0500,3.5507) {};
    \node[circle,inner sep=.7pt,fill=black] (curve0379) at (-2.8136,2.9822) {};
    \node[circle,inner sep=.7pt,fill=black] (curve0124) at (-3.4255,2.2530) {};
    \node[circle,inner sep=.7pt,fill=black] (curve0199) at (-3.8527,1.4023) {};
    \node[circle,inner sep=.7pt,fill=black] (curve0390) at (-4.0723,0.4760) {};
    \node[circle,inner sep=.7pt,fill=black] (curve0029) at (-4.0723,-0.4760) {};
    \node[circle,inner sep=.7pt,fill=black] (curve0220) at (-3.8527,-1.4023) {};
    \node[circle,inner sep=.7pt,fill=black] (curve0295) at (-3.4255,-2.2530) {};
    \node[circle,inner sep=.7pt,fill=black] (curve0040) at (-2.8136,-2.9822) {};
    \node[circle,inner sep=.7pt,fill=black] (curve0006) at (-2.0500,-3.5507) {};
    \node[circle,inner sep=.7pt,fill=black] (curve0245) at (-1.1759,-3.9278) {};
    \node[circle,inner sep=.7pt,fill=black] (curve0228) at (-0.2384,-4.0931) {};
    \node[circle,inner sep=.7pt,fill=black] (curve0275) at (0.7120,-4.0377) {};
    \node[circle,inner sep=.7pt,fill=black] (curve0344) at (1.6239,-3.7647) {};
    \node[circle,inner sep=.7pt,fill=black] (curve0015) at (2.4484,-3.2887) {};
    \node[circle,inner sep=.7pt,fill=black] (curve0051) at (3.1408,-2.6354) {};
    \node[circle,inner sep=.7pt,fill=black] (curve0009) at (3.6639,-1.8401) {};
    \node[circle,inner sep=.7pt,fill=black] (curve0261) at (3.9895,-0.9455) {};
    %
    \draw[color=blue] (curve0199) edge[bend right=4mm] (curve0390);
    \draw[color=blue] (curve0051) edge[bend right=4mm] (curve0009);
    \draw[color=blue] (curve0368) edge[bend right=4mm] (curve0404);
    \draw[color=blue] (curve0245) edge[bend right=4mm] (curve0228);
    \draw[color=blue] (curve0029) edge[bend right=4mm] (curve0220);
    \draw[color=blue] (curve0174) edge[bend right=4mm] (curve0413);
    \draw[color=blue] (curve0261) edge[bend right=4mm] (curve0000);
    \draw[color=blue] (curve0379) edge[bend right=4mm] (curve0124);
    \draw[color=blue] (curve0006) edge[bend right=4mm] (curve0245);
    \draw[color=blue] (curve0158) edge[bend right=4mm] (curve0410);
    \draw[color=blue] (curve0228) edge[bend right=4mm] (curve0275);
    \draw[color=blue] (curve0275) edge[bend right=4mm] (curve0344);
    \draw[color=blue] (curve0015) edge[bend right=4mm] (curve0051);
    \draw[color=blue] (curve0191) edge[bend right=4mm] (curve0174);
    \draw[color=blue] (curve0144) edge[bend right=4mm] (curve0191);
    \draw[color=blue] (curve0404) edge[bend right=4mm] (curve0075);
    \draw[color=blue] (curve0009) edge[bend right=4mm] (curve0261);
    \draw[color=blue] (curve0295) edge[bend right=4mm] (curve0040);
    \draw[color=blue] (curve0410) edge[bend right=4mm] (curve0368);
    \draw[color=blue] (curve0413) edge[bend right=4mm] (curve0379);
    \draw[color=blue] (curve0040) edge[bend right=4mm] (curve0006);
    \draw[color=blue] (curve0075) edge[bend right=4mm] (curve0144);
    \draw[color=blue] (curve0220) edge[bend right=4mm] (curve0295);
    \draw[color=blue] (curve0390) edge[bend right=4mm] (curve0029);
    \draw[color=blue] (curve0344) edge[bend right=4mm] (curve0015);
    \draw[color=blue] (curve0124) edge[bend right=4mm] (curve0199);
    \draw[color=blue] (curve0000) edge[bend right=4mm] (curve0158);
    %
    \draw[color=red] (curve0009) edge[bend right=8mm] (curve0379);
    \draw[color=red] (curve0124) edge[bend right=8mm] (curve0015);
    \draw[color=red] (curve0006) edge[bend right=8mm] (curve0368);
    \draw[color=red] (curve0228) edge[bend right=8mm] (curve0075);
    \draw[color=red] (curve0245) edge[bend right=8mm] (curve0404);
    \draw[color=red] (curve0029) edge[bend right=8mm] (curve0261);
    \draw[color=red] (curve0220) edge[bend right=8mm] (curve0000);
    \draw[color=red] (curve0368) edge[bend right=8mm] (curve0220);
    \draw[color=red] (curve0144) edge[bend right=8mm] (curve0006);
    \draw[color=red] (curve0261) edge[bend right=8mm] (curve0124);
    \draw[color=red] (curve0191) edge[bend right=8mm] (curve0245);
    \draw[color=red] (curve0015) edge[bend right=8mm] (curve0174);
    \draw[color=red] (curve0344) edge[bend right=8mm] (curve0191);
    \draw[color=red] (curve0275) edge[bend right=8mm] (curve0144);
    \draw[color=red] (curve0158) edge[bend right=8mm] (curve0390);
    \draw[color=red] (curve0295) edge[bend right=8mm] (curve0158);
    \draw[color=red] (curve0075) edge[bend right=8mm] (curve0040);
    \draw[color=red] (curve0174) edge[bend right=8mm] (curve0228);
    \draw[color=red] (curve0000) edge[bend right=8mm] (curve0199);
    \draw[color=red] (curve0379) edge[bend right=8mm] (curve0344);
    \draw[color=red] (curve0390) edge[bend right=8mm] (curve0009);
    \draw[color=red] (curve0040) edge[bend right=8mm] (curve0410);
    \draw[color=red] (curve0410) edge[bend right=8mm] (curve0029);
    \draw[color=red] (curve0404) edge[bend right=8mm] (curve0295);
    \draw[color=red] (curve0413) edge[bend right=8mm] (curve0275);
    \draw[color=red] (curve0051) edge[bend right=8mm] (curve0413);
    \draw[color=red] (curve0199) edge[bend right=8mm] (curve0051);
    %
    \draw[color=green] (curve0158) edge[bend left=10mm] (curve0144);
    \draw[color=green] (curve0009) edge[bend left=10mm] (curve0368);
    \draw[color=green] (curve0124) edge[bend left=10mm] (curve0295);
    \draw[color=green] (curve0174) edge[bend left=10mm] (curve0390);
    \draw[color=green] (curve0368) edge[bend left=10mm] (curve0174);
    \draw[color=green] (curve0344) edge[bend left=10mm] (curve0000);
    \draw[color=green] (curve0144) edge[bend left=10mm] (curve0124);
    \draw[color=green] (curve0295) edge[bend left=10mm] (curve0275);
    \draw[color=green] (curve0029) edge[bend left=10mm] (curve0245);
    \draw[color=green] (curve0051) edge[bend left=10mm] (curve0410);
    \draw[color=green] (curve0015) edge[bend left=10mm] (curve0158);
    \draw[color=green] (curve0275) edge[bend left=10mm] (curve0261);
    \draw[color=green] (curve0075) edge[bend left=10mm] (curve0379);
    \draw[color=green] (curve0379) edge[bend left=10mm] (curve0220);
    \draw[color=green] (curve0220) edge[bend left=10mm] (curve0228);
    \draw[color=green] (curve0006) edge[bend left=10mm] (curve0015);
    \draw[color=green] (curve0191) edge[bend left=10mm] (curve0199);
    \draw[color=green] (curve0228) edge[bend left=10mm] (curve0009);
    \draw[color=green] (curve0040) edge[bend left=10mm] (curve0344);
    \draw[color=green] (curve0261) edge[bend left=10mm] (curve0404);
    \draw[color=green] (curve0404) edge[bend left=10mm] (curve0413);
    \draw[color=green] (curve0390) edge[bend left=10mm] (curve0006);
    \draw[color=green] (curve0000) edge[bend left=10mm] (curve0075);
    \draw[color=green] (curve0245) edge[bend left=10mm] (curve0051);
    \draw[color=green] (curve0413) edge[bend left=10mm] (curve0029);
    \draw[color=green] (curve0199) edge[bend left=10mm] (curve0040);
    \draw[color=green] (curve0410) edge[bend left=10mm] (curve0191);
}

\newcommand\fpsqgraph{
    \node[circle,inner sep=.7pt,fill=black] (j_190_344i) at (1.0,0.0) {};
    \node[circle,inner sep=.7pt,fill=black] (j_379_325i) at (0.985615910348,0.169000820322) {};
    \node[circle,inner sep=.7pt,fill=black] (j_143_000i) at (0.942877445461,0.333139794742) {};
    \node[circle,inner sep=.7pt,fill=black] (j_304_364i) at (0.873014113161,0.487694943814) {};
    \node[circle,inner sep=.7pt,fill=black] (j_356_000i) at (0.778035754318,0.628219997296) {};
    \node[circle,inner sep=.7pt,fill=black] (j_004_000i) at (0.66067472339,0.750672305253) {};
    \node[circle,inner sep=.7pt,fill=black] (j_242_000i) at (0.524307283557,0.851529137733) {};
    \node[circle,inner sep=.7pt,fill=black] (j_234_000i) at (0.37285647778,0.927889027297) {};
    \node[circle,inner sep=.7pt,fill=black] (j_065_081i) at (0.210679269996,0.977555238948) {};
    \node[circle,inner sep=.7pt,fill=black] (j_118_209i) at (0.0424412031961,0.999098966205) {};
    \node[circle,inner sep=.7pt,fill=black] (j_125_000i) at (-0.127017819747,0.991900435259) {};
    \node[circle,inner sep=.7pt,fill=black] (j_316_000i) at (-0.292822771277,0.956166734739) {};
    \node[circle,inner sep=.7pt,fill=black] (j_426_306i) at (-0.450203744818,0.89292585815) {};
    \node[circle,inner sep=.7pt,fill=black] (j_358_000i) at (-0.594633176304,0.803997130367) {};
    \node[circle,inner sep=.7pt,fill=black] (j_241_000i) at (-0.721956093955,0.691938868978) {};
    \node[circle,inner sep=.7pt,fill=black] (j_419_000i) at (-0.828509649244,0.559974786138) {};
    \node[circle,inner sep=.7pt,fill=black] (j_061_000i) at (-0.911228490388,0.411901248244) {};
    \node[circle,inner sep=.7pt,fill=black] (j_102_000i) at (-0.967732946933,0.251978061385) {};
    \node[circle,inner sep=.7pt,fill=black] (j_190_087i) at (-0.996397488543,0.0848059244755) {};
    \node[circle,inner sep=.7pt,fill=black] (j_107_000i) at (-0.996397488543,-0.0848059244755) {};
    \node[circle,inner sep=.7pt,fill=black] (j_304_067i) at (-0.967732946933,-0.251978061385) {};
    \node[circle,inner sep=.7pt,fill=black] (j_019_000i) at (-0.911228490388,-0.411901248244) {};
    \node[circle,inner sep=.7pt,fill=black] (j_381_000i) at (-0.828509649244,-0.559974786138) {};
    \node[circle,inner sep=.7pt,fill=black] (j_319_000i) at (-0.721956093955,-0.691938868978) {};
    \node[circle,inner sep=.7pt,fill=black] (j_065_350i) at (-0.594633176304,-0.803997130367) {};
    \node[circle,inner sep=.7pt,fill=black] (j_067_000i) at (-0.450203744818,-0.89292585815) {};
    \node[circle,inner sep=.7pt,fill=black] (j_000_000i) at (-0.292822771277,-0.956166734739) {};
    \node[circle,inner sep=.7pt,fill=black] (j_315_299i) at (-0.127017819747,-0.991900435259) {};
    \node[circle,inner sep=.7pt,fill=black] (j_422_000i) at (0.0424412031961,-0.999098966205) {};
    \node[circle,inner sep=.7pt,fill=black] (j_379_106i) at (0.210679269996,-0.977555238948) {};
    \node[circle,inner sep=.7pt,fill=black] (j_189_000i) at (0.37285647778,-0.927889027297) {};
    \node[circle,inner sep=.7pt,fill=black] (j_141_042i) at (0.524307283557,-0.851529137733) {};
    \node[circle,inner sep=.7pt,fill=black] (j_141_389i) at (0.66067472339,-0.750672305253) {};
    \node[circle,inner sep=.7pt,fill=black] (j_118_222i) at (0.778035754318,-0.628219997296) {};
    \node[circle,inner sep=.7pt,fill=black] (j_315_132i) at (0.873014113161,-0.487694943814) {};
    \node[circle,inner sep=.7pt,fill=black] (j_150_000i) at (0.942877445461,-0.333139794742) {};
    \node[circle,inner sep=.7pt,fill=black] (j_426_125i) at (0.985615910348,-0.169000820322) {};
    %
    \draw[color=blue] (j_319_000i) edge (j_426_125i);
    \draw[color=orange] (j_419_000i) edge (j_141_389i);
    \draw[color=orange] (j_065_081i) edge (j_190_087i);
    \draw[color=blue] (j_102_000i) edge (j_143_000i);
    \draw[color=orange] (j_102_000i) edge (j_358_000i);
    \draw[color=blue] (j_004_000i) edge (j_102_000i);
    \draw[color=blue] (j_107_000i) edge (j_316_000i);
    \draw[color=blue] (j_143_000i) edge (j_234_000i);
    \draw[color=blue] (j_304_364i) edge (j_379_106i);
    \draw[color=blue] (j_065_081i) edge (j_426_306i);
    \draw[color=blue] (j_242_000i) edge (j_356_000i);
    \draw[color=blue] (j_125_000i) edge (j_125_000i);
    \draw[color=orange] (j_242_000i) edge (j_242_000i);
    \draw[color=orange] (j_065_350i) edge (j_141_389i);
    \draw[color=blue] (j_150_000i) edge (j_190_344i);
    \draw[color=orange] (j_379_325i) edge (j_426_125i);
    \draw[color=orange] (j_319_000i) edge (j_304_067i);
    \draw[color=blue] (j_102_000i) edge (j_315_132i);
    \draw[color=orange] (j_316_000i) edge (j_379_325i);
    \draw[color=blue] (j_356_000i) edge (j_426_125i);
    \draw[color=orange] (j_234_000i) edge (j_242_000i);
    \draw[color=blue] (j_379_106i) edge (j_379_325i);
    \draw[color=orange] (j_419_000i) edge (j_141_042i);
    \draw[color=blue] (j_000_000i) edge (j_000_000i);
    \draw[color=orange] (j_358_000i) edge (j_381_000i);
    \draw[color=orange] (j_107_000i) edge (j_190_087i);
    \draw[color=blue] (j_241_000i) edge (j_190_344i);
    \draw[color=blue] (j_422_000i) edge (j_141_389i);
    \draw[color=orange] (j_065_081i) edge (j_118_209i);
    \draw[color=blue] (j_358_000i) edge (j_422_000i);
    \draw[color=blue] (j_019_000i) edge (j_304_067i);
    \draw[color=orange] (j_379_106i) edge (j_426_306i);
    \draw[color=blue] (j_189_000i) edge (j_304_067i);
    \draw[color=blue] (j_019_000i) edge (j_304_364i);
    \draw[color=blue] (j_061_000i) edge (j_315_132i);
    \draw[color=blue] (j_381_000i) edge (j_419_000i);
    \draw[color=orange] (j_319_000i) edge (j_304_364i);
    \draw[color=blue] (j_426_125i) edge (j_426_306i);
    \draw[color=orange] (j_315_132i) edge (j_426_306i);
    \draw[color=blue] (j_065_350i) edge (j_190_087i);
    \draw[color=blue] (j_150_000i) edge (j_319_000i);
    \draw[color=blue] (j_107_000i) edge (j_189_000i);
    \draw[color=blue] (j_067_000i) edge (j_234_000i);
    \draw[color=orange] (j_102_000i) edge (j_125_000i);
    \draw[color=blue] (j_150_000i) edge (j_190_087i);
    \draw[color=orange] (j_356_000i) edge (j_422_000i);
    \draw[color=orange] (j_150_000i) edge (j_189_000i);
    \draw[color=blue] (j_141_042i) edge (j_315_132i);
    \draw[color=blue] (j_141_042i) edge (j_304_364i);
    \draw[color=blue] (j_419_000i) edge (j_419_000i);
    \draw[color=orange] (j_118_209i) edge (j_315_299i);
    \draw[color=blue] (j_065_350i) edge (j_118_209i);
    \draw[color=orange] (j_061_000i) edge (j_356_000i);
    \draw[color=orange] (j_019_000i) edge (j_241_000i);
    \draw[color=blue] (j_241_000i) edge (j_381_000i);
    \draw[color=orange] (j_143_000i) edge (j_150_000i);
    \draw[color=blue] (j_141_389i) edge (j_315_299i);
    \draw[color=blue] (j_107_000i) edge (j_118_209i);
    \draw[color=orange] (j_241_000i) edge (j_118_222i);
    \draw[color=blue] (j_118_222i) edge (j_315_299i);
    \draw[color=blue] (j_141_042i) edge (j_141_389i);
    \draw[color=orange] (j_065_350i) edge (j_190_344i);
    \draw[color=orange] (j_107_000i) edge (j_190_344i);
    \draw[color=blue] (j_356_000i) edge (j_426_306i);
    \draw[color=orange] (j_118_222i) edge (j_315_132i);
    \draw[color=orange] (j_141_042i) edge (j_426_125i);
    \draw[color=blue] (j_061_000i) edge (j_356_000i);
    \draw[color=blue] (j_125_000i) edge (j_118_222i);
    \draw[color=blue] (j_107_000i) edge (j_118_222i);
    \draw[color=blue] (j_316_000i) edge (j_379_325i);
    \draw[color=blue] (j_061_000i) edge (j_315_299i);
    \draw[color=orange] (j_067_000i) edge (j_242_000i);
    \draw[color=orange] (j_379_106i) edge (j_379_325i);
    \draw[color=orange] (j_381_000i) edge (j_315_299i);
    \draw[color=orange] (j_315_299i) edge (j_426_125i);
    \draw[color=blue] (j_358_000i) edge (j_381_000i);
    \draw[color=orange] (j_190_087i) edge (j_304_067i);
    \draw[color=blue] (j_316_000i) edge (j_379_106i);
    \draw[color=blue] (j_067_000i) edge (j_419_000i);
    \draw[color=orange] (j_316_000i) edge (j_356_000i);
    \draw[color=blue] (j_065_350i) edge (j_426_125i);
    \draw[color=orange] (j_019_000i) edge (j_234_000i);
    \draw[color=orange] (j_004_000i) edge (j_004_000i);
    \draw[color=orange] (j_189_000i) edge (j_234_000i);
    \draw[color=blue] (j_000_000i) edge (j_241_000i);
    \draw[color=blue] (j_319_000i) edge (j_426_306i);
    \draw[color=orange] (j_190_344i) edge (j_304_364i);
    \draw[color=blue] (j_241_000i) edge (j_190_087i);
    \draw[color=blue] (j_125_000i) edge (j_118_209i);
    \draw[color=blue] (j_150_000i) edge (j_189_000i);
    \draw[color=blue] (j_019_000i) edge (j_125_000i);
    \draw[color=orange] (j_316_000i) edge (j_379_106i);
    \draw[color=orange] (j_000_000i) edge (j_125_000i);
    \draw[color=blue] (j_143_000i) edge (j_065_081i);
    \draw[color=blue] (j_004_000i) edge (j_319_000i);
    \draw[color=blue] (j_019_000i) edge (j_358_000i);
    \draw[color=blue] (j_242_000i) edge (j_242_000i);
    \draw[color=blue] (j_190_344i) edge (j_379_325i);
    \draw[color=orange] (j_061_000i) edge (j_358_000i);
    \draw[color=blue] (j_065_081i) edge (j_190_344i);
    \draw[color=orange] (j_067_000i) edge (j_304_067i);
    \draw[color=orange] (j_065_081i) edge (j_141_042i);
    \draw[color=blue] (j_422_000i) edge (j_141_042i);
    \draw[color=blue] (j_102_000i) edge (j_315_299i);
    \draw[color=blue] (j_242_000i) edge (j_422_000i);
    \draw[color=blue] (j_143_000i) edge (j_065_350i);
    \draw[color=orange] (j_065_350i) edge (j_118_222i);
    \draw[color=orange] (j_125_000i) edge (j_143_000i);
    \draw[color=orange] (j_004_000i) edge (j_019_000i);
    \draw[color=orange] (j_143_000i) edge (j_422_000i);
    \draw[color=blue] (j_061_000i) edge (j_234_000i);
    \draw[color=orange] (j_419_000i) edge (j_422_000i);
    \draw[color=blue] (j_141_389i) edge (j_304_067i);
    \draw[color=blue] (j_065_081i) edge (j_118_222i);
    \draw[color=blue] (j_067_000i) edge (j_316_000i);
    \draw[color=blue] (j_118_209i) edge (j_315_132i);
    \draw[color=orange] (j_241_000i) edge (j_118_209i);
    \draw[color=orange] (j_067_000i) edge (j_304_364i);
    \draw[color=blue] (j_190_087i) edge (j_379_106i);
    \draw[color=orange] (j_102_000i) edge (j_319_000i);
    \draw[color=blue] (j_189_000i) edge (j_304_364i);
    \draw[color=blue] (j_304_067i) edge (j_379_325i);
    \draw[color=orange] (j_381_000i) edge (j_315_132i);
    \draw[color=orange] (j_141_389i) edge (j_426_306i);
    \draw[color=orange] (j_061_000i) edge (j_107_000i);
}


\title{Isogeny-based cryptography}
\subtitle{old (and new) assumptions}
\author{Luca De Feo}
\date[Dec 11, 2021, IWPQC]{December 11, 2021\\
  International Workshop on Post-quantum Cryptography}
\institute{IBM Research Zürich}

\begin{document}

\frame[plain]{\titlepage}

%%

\begin{frame}{Isogeny epochs}
  \begin{description}
  \item[The Big Bang]<+->\
    \begin{itemize}
    \item[1997] Couveignes presents the first isogeny based scheme at
      the ENS seminar in Paris.
    \end{itemize}
  \item[The Dark Age]<+->
  \item[Star formation]<+->\
    \begin{itemize}
    \item[2006] Charles--Goren--Lauter hash function;
    \item[2006] Rostovtsev and Stolbunov independently rediscover
      Couveignes' idea;
    \item[2011] Jao and De Feo introduce SIDH.
    \end{itemize}
  \item[Inflection point]<+->\
    \begin{itemize}
    \item[2016] NIST announces plans for the Post-Quantum Competition.
    \end{itemize}
  \item[Accelerating expansion]<+->\
    \begin{itemize}
    \item[2018] Castryck, Lange, Martindale, Panny, and Renes
      introduce CSIDH;
    \item[2019] Beullens, Kleinjung and Vercauteren introduce CSI-FiSh;
    \item[2020] De Feo, Kohel, Leroux, Petit and Wesolowski introduce SQISign\dots
    \end{itemize}
  \end{description}
\end{frame}

%%

\begin{frame}{Where did isogenies come from?}
  \begin{description}
  \item[1940s] André Weil invents the name for his Foundations of
    Algebraic Geometry;
  \item[1985] Schoof's elliptic point counting algorithm;
  \item[1990s] Schoof--Elkies--Atkin (SEA) point counting
    algorithm\footnote{with important contributions by Couveignes,
      Lercier, Morain, \dots}\\
    \dots \emph{computing isogenies} becomes a thing!
  \item[2000s] People\footnote{Galbraith--Hess--Smart, Menezes--Teske,
      Jao--Miller--Venkatesan, and many others\dots} use isogenies to
    reduce discrete logarithms between elliptic curves\\
    \dots \emph{isogeny graphs} become mainstream!
  \end{description}
\end{frame}

%%

\begin{frame}
  \frametitle{Maps: what's \alt<2->{\xout{scalar multiplication} an
      isogeny}{scalar multiplication}?}

  \begin{overlayarea}{\textwidth}{4em}
    \Large
    \[
      \alt<3->{\phi}{[n]}
      \;:\; P \mapsto
      \alt<3->{\phi(P)}{\underbrace{P + P + \cdots + P}_{n\text{ times}}}\]
  \end{overlayarea}
  
  \begin{itemize}
  \item A map \emph{$E\to \alt<4->{\xout{E} E'}{E\phantom{\xout{}}}$},
  \item a \emph{group morphism},
  \item with \emph{finite kernel}\\
    \alt<5->{(\xout{the torsion group $E[n]\simeq(ℤ/nℤ)^2$} any
      finite subgroup \emph{$H\subset E$})}{(the torsion group
      \emph{$E[n]\simeq(ℤ/nℤ)^2$})},
  \item \emph{surjective} (in the algebraic closure),
  \item given by \emph{rational maps} of degree \alt<6->{\xout{$n^2$}
      \emph{$\#H$}}{\emph{$n^2$}}.
  \end{itemize}
\end{frame}

%% 

\begin{frame}{Isogenies: an example over $\F_{11}$}
  \centering
  \begin{tikzpicture}[scale=0.4]
    \begin{scope}
      \node[anchor=center] at (0,7) {$E \;:\; y^2 = x^3 + x$};

      \uncover<-1>{
        \draw[thin,gray] (0,-6) -- (0,6);
        \draw[thin,gray] (-6,0) -- (6,0);
      }

      \foreach \x/\y in {0/0,5/3,-4/3,-3/5,-2/1,-1/3} {
        \draw[blue,fill] (\x,\y) circle (0.2) node(E_\x_\y){}
        (\x,-\y) circle (0.2) node(E_\x_-\y){};
      }

      \uncover<2->{\draw[red,fill] (0,0) circle (0.3);}
    \end{scope}

    \draw[black!10!white,thick] (10,-7) -- +(0,14);
    
    \begin{scope}[shift={(20,0)}]
      \node at (0,7) {$E' \;:\; y^2 = x^3 - 4x$};

      \uncover<-1>{
        \draw[thin,gray] (0,-6) -- (0,6);
        \draw[thin,gray] (-6,0) -- (6,0);
      }

      \foreach \x/\y in {0/0,2/0,3/2,4/2,6/4,-2/0,-1/5} {
        \draw[color=blue,fill] (\x,\y) circle (0.2) node(F_\x_\y){}
        (\x,-\y) circle (0.2) node(F_\x_-\y){};
      }
    \end{scope}

    \begin{scope}[color=red,-latex,dashed]
      \begin{uncoverenv}<2->
        \path
        (E_5_3) edge (F_3_2)
        (E_-4_3) edge (F_4_-2)
        (E_-3_5) edge (F_4_2)
        (E_-2_1) edge (F_3_-2)
        (E_-1_3) edge (F_-2_0);
      \end{uncoverenv}
      \begin{uncoverenv}<2->
        \path
        (E_5_-3) edge (F_3_-2)
        (E_-4_-3) edge (F_4_2)
        (E_-3_-5) edge (F_4_-2)
        (E_-2_-1) edge (F_3_2)
        (E_-1_-3) edge (F_-2_0);
      \end{uncoverenv}
    \end{scope}
  \end{tikzpicture}
  
  \begin{columns}
    \begin{column}{0.5\textwidth}
      \[\phi(x,y) = \left(\frac{x^2 + 1}{x},\quad y\frac{x^2-1}{x^2}\right)\]
    \end{column}
    \begin{column}{0.5\textwidth}
      \begin{itemize}
      \item<2-> Kernel generator in \alert{red}.
      \item<2-> This is a degree $2$ map.
      \item<2-> Analogous to $x\mapsto x^2$ in $\F_q^*$.
      \end{itemize}
    \end{column}
  \end{columns}
\end{frame}

%%

\begin{frame}{Reducing discrete logarithms}

  \begin{block}{Serre-Tate theorem}
    Two elliptic curves $E,E'$ defined over a finite field $\F_q$ are
    \emph{isogenous}
    (over $\F_q$) iff \emph{$\#E(\F_q) = \#E'(\F_q)$}.
  \end{block}
  
  \begin{center}
    \begin{tikzpicture}
      \path (0,0) node[anchor=east] {$E$} (6,0) node[anchor=west] {$E'$};
      \path[gray] (-0.8,0) node[anchor=east] {weak curve}
      (6.8,0) node[anchor=west] {strong curve};

      \draw[->] (0,0) -- (0.5,-0.2);
      \draw[->] (6,0) -- (5.5,0.2);
      \draw[->] (0.5,-0.2) -- (1,0.2);
      \draw[->] (5.5,0.2) -- (5,-0.2);
      \begin{scope}[densely dotted,coils/.style={decorate,decoration={coil,aspect=0,amplitude=2pt}}]
        \draw[coils] (1,0.2) -- (3,0.4);
        \draw[coils] (5,-0.2) -- (3,0.4);
        \draw[-angle 90,coils] (3,0.4) -- (3, -0.4) node[anchor=north] {$E''$};
      \end{scope}
    \end{tikzpicture}
  \end{center}
  
  \begin{block}{Fourth root attacks (Galbraith--Hess--Smart, Menezes--Teske)}
    \begin{itemize}
    \item Start two random walks from the two curves and wait for a
      collision.
    \item Over \emph{$\F_q$}, the average size of an isogeny class is
      \emph{$h_\Delta\sim\sqrt{q}$}.
    \item A collision is expected after \emph{$O(\sqrt{h_\Delta}) =
        O(q^{\frac{1}{4}})$} steps.
    \end{itemize}
  \end{block}
\end{frame}

%%

\begin{frame}{Up to \emph{isomorphism}}
  \begin{center}
    \begin{tikzpicture}[domain=-2.4566:4,samples=100]
      \newcount\zoomout
      \transduration<15-21>{0.5}
      \animatevalue<15-20>{\zoomout}{0}{10}
      \begin{uncoverenv}<-20>
        \begin{scope}[scale=1-0.09*\zoomout]
          \begin{scope}
            \draw[thin,gray,-latex] (0,-4) -- (0,4);
            \draw[thin,gray,-latex] (-4.2,0) -- (7,0);
          \end{scope}
          
          \newcount\xstretch
          \newcount\ystretch
          \newcount\slant
          \transduration<1-13>{0.5}
          \animatevalue<1-5>{\xstretch}{0}{4}
          \animatevalue<5-9>{\ystretch}{0}{4}
          \animatevalue<9-13>{\slant}{0}{4}      
          \begin{scope}[yscale=0.55-0.05*\the\ystretch,xscale=1+0.1*\the\xstretch,xslant=0.02*\slant]
            \draw plot (\x,{sqrt(\x*\x*\x-4*\x+5)});
            \draw plot (\x,{-sqrt(\x*\x*\x-4*\x+5)});

            \begin{uncoverenv}<-18>
              \draw (-3,1) -- (4,8/3+3);
              \begin{scope}[every node/.style={draw,circle,inner sep=1pt,fill},cm={1,2/3,0,0,(0,3)}]
                \node at (-2.287980,0) {};
                \node at (-0.535051,0) {};
                \node at (3.267475,0) {};
              \end{scope}
              \begin{scope}[every node/.style={yshift=0.3cm},cm={1,2/3,0,0,(0,3)}]
                \node at (-2.287980,0) {$P$};
                \node at (-0.535051,0) {$Q$};
                \node at (3.267475,0) {$R$};
              \end{scope}
              \draw[dashed] (3.267475,3.267475*2/3+3) -- (3.267475,-3.267475*2/3-3) 
              node[draw,circle,inner sep=1pt,fill] {}
              node[xshift=-0.1cm,anchor=east] {$P+Q$};
            \end{uncoverenv}
          \end{scope}

          \begin{uncoverenv}<14>
            \node[anchor=west] at (-4,-3) {\Large\alert{$y^2=x^3+ax+b \quad\longrightarrow\quad j\equiv 1728\frac{4a^3}{4a^3+27b^2}$}};
          \end{uncoverenv}
        \end{scope}
      \end{uncoverenv}
      
      \begin{uncoverenv}<21->
        \draw[fill] (0,0) circle (2pt) node[anchor=north] {$j=1728$};
        \uncover<22>{
          \draw (0.1,0) edge[bend left,->] node[auto] {$\phi$} (7,0);
        }
        \uncover<22->{
          \draw[fill] (7.1,0) circle (2pt) node[anchor=north] {$j=287496$};
        }
        \uncover<23->{
          \draw (0.1,0) edge[bend left,<->,red,very thick] (7,0);
        }
      \end{uncoverenv}
    \end{tikzpicture}
  \end{center}  
\end{frame}

%%

\begin{frame}{Isogeny graphs}
  \begin{block}{}
    \begin{description}
    \item[1986] Mestre's \emph{\it``La méthode des graphes''};
    \item[1996] Kohel's PhD thesis \emph{\it``Endomorphism rings of elliptic
      curves over finite fields''};
    \item[2002] Fouquet and Morain's \emph{\it``Isogeny volcanoes and the SEA
      algorithm''};
    \item \dots
    \end{description}
  \end{block}
  
  \smallskip
  
  \begin{columns}
    \begin{column}{0.5\textwidth}
      \begin{itemize}
      \item \emph{Vertices are curves} up to isomorphism,
      \item \emph{Edges are isogenies} up to isomorphism.
      \end{itemize}
    \end{column}      
    \begin{column}{0.5\textwidth}
      \centering
      \begin{tikzpicture}
        \begin{scope}
          \def\crater{7}
          \foreach \i in {1,...,\crater} {
            \draw[fill] (360/\crater*\i:1cm) circle (5pt);
            \draw (360/\crater*\i : 1cm) -- (360/\crater*\i+360/\crater : 1cm);
            \foreach \j in {-1,1} {
              \draw[fill] (360/\crater*\i : 1cm) -- (360/\crater*\i + \j*360/\crater/4 : 2cm) circle (3pt);
              \foreach \k in {-1,0,1} {
                \draw[fill] (360/\crater*\i + \j*360/\crater/4 : 2cm) --
                (360/\crater*\i + + \j*360/\crater/4 + \k*360/\crater/6 : 2.5cm) circle (1pt);
              }
            }
          }
        \end{scope}
      \end{tikzpicture}
    \end{column}
  \end{columns}
\end{frame}

%%

\begin{frame}{Cryptanalysis $\to$ cryptography}
  \begin{description}
  \item[1993]<+-> Menezes--Okamoto--Vanstone use \emph{pairings to
      break discrete logarithms} of supersingular curves;
  \item[2000]<+-> Joux and Boneh--Franklin initiate \emph{pairing based
      cryptography}.
    \bigskip
  \item[1999]<+-> Galbraith--Hess--Smart use \emph{isogeny graphs} to
    extend the impact of \emph{Weil descent attacks};
  \item[2006]<+-> Charles--Goren--Lauter and Rostovtsev--Stolbunov
    initiate \emph{isogeny based cryptography}.
  \end{description}
\end{frame}

%%

\begin{frame}{The beauty and the beast {\quad\footnotesize(credit: Lorenz Panny)}}
    \smallskip
    \begin{center}

      Cryptographically interesting isogeny graphs

      \par\vspace{1ex}

      \begin{minipage}{.49\textwidth}\centering
        \begin{tikzpicture}[scale=.6,>=stealth,shorten >=.2mm,shorten <=.2mm,rotate=90,line width=.6pt]
          \isoggraph
        \end{tikzpicture}

        \vspace{.2ex}

        Complex multiplication graphs \\[.7ex]
        \small Couveignes--Rostovtsev--Stolbunov, CSIDH, CSI-FiSh, \dots
      \end{minipage}
      % 
      \begin{minipage}{.49\textwidth}\centering
        \begin{tikzpicture}[scale=2.456,>=stealth,shorten >=.2mm,shorten <=.2mm,rotate=90,line width=.6pt]
          \fpsqgraph
        \end{tikzpicture}

        \vspace{.2ex}

        Full supersingular graphs \\[.7ex]
        \small SIDH, SQISign, \dots
      \end{minipage}
    \end{center}
\end{frame}

%%

\begin{frame}{The isogeny path problem}
  \centering
  \movie[width=13.5cm,height=7.875cm,loop,poster,autostart,borderwidth=0pt]{}{PathFinding.mp4}
\end{frame}

%%

\begin{frame}{The isogeny path problem}
  \begin{block}{The mother of all problems}
    Given (supersingular) elliptic curves \emph{$E$} and \emph{$E'$}
    over a finite field, find a \emph{chain of low-degree isogenies}
    \[\phi_n\circ\cdots\circ\phi_1 \;:\; E\to E'.\]
  \end{block}

  \begin{description}[leftmargin=1em]
    \setlength{\itemsep}{1em}
  \item[CSIDH/CSI-FiSh]
    \begin{itemize}
    \item $E$ and $E'$ are supersingular curves \emph{defined over $\F_p$}.
      \hfill\uncover<2->{\alert{\large\rightpointleft}}
    \end{itemize}
  \item[SIDH/SIKE] 
    \begin{itemize}
    \item Additional information given on a \emph{secret short
        chain} $E\to E'$;
    \item In instantiations, $E$ is fixed and \emph{special} (in a
      cryptanalytic sense).
    \end{itemize}
  \item[SQISign]
    \begin{itemize}
    \item Every signature leaks information on a secret chain
      $E\to E'$,\\ unless a \emph{10-pages-long computational
        assumption} holds.
    \item + Random Oracle Model (rewinding).
    \end{itemize}
  \item[Time-delay]
    \begin{itemize}
    \item A \emph{long chain} $E\to E'$ is \emph{publicly known}, find
      a shorter one.
    \item + Pairing assumptions (not post-quantum).
    \end{itemize}
  \end{description}
\end{frame}

%%

\begin{frame}
  \centering
  \begin{tikzpicture}
    \draw[blue,thick] (90:3.5) -- (210:3.5) -- (330:3.5) -- (90:3.5);
    
    \node[rotate=0] at (270:2.5) {Modular functions};
    \node[rotate=60] at (150:2.5) {Class field theory};
    \node[rotate=-60] at (30:2.5) {Elliptic curves};

    \begin{uncoverenv}<1>
      \begin{scope}[gray]
        \node[rotate=0] at (270:3.2) {$j(z) = \frac{1}{q} + 744 + 196884q + \cdots$};
        \node[anchor=east] at (150:3.2) {$H(j) = j - 1728$};
        \node[anchor=west] at (30:3.2) {$y^2 = x^3 - ax - b$};
      \end{scope}
    \end{uncoverenv}

    \begin{uncoverenv}<2>
      \begin{scope}[yshift=0.5cm]
        \node[anchor=east] (C) at (150:4.5) {Abelian extensions};
        \node[anchor=east,below of=C] {of $\Q(\sqrt{-D})$};
        \node[anchor=west] (E) at (30:4.5) {Elliptic curves with};
        \node[anchor=west,below of=E] {$\End(E) \subset \Q(\sqrt{-D})$};
      \end{scope}
    \end{uncoverenv}
    
    \begin{uncoverenv}<3>
      \begin{scope}[yshift=1.5cm]
        \node[anchor=east] (C) at (150:3) {Galois group of $K/\Q(\sqrt{-D})$};
        \node[anchor=east,below of=C] (C1) {$\simeq$};
        \node[anchor=east,below of=C1] {Ideal class group $\Cl(-D)$};
        \node[anchor=west] (E) at (30:3) {$\Cl(-D)$ acts on set of $E$ s.t.};
        \node[anchor=west,below of=E] {$\End(E) \subset \Q(\sqrt{-D})$};
      \end{scope}
    \end{uncoverenv}
    
    \begin{scope}
      \Large
      \node at (0,0.3) {\emph{Complex}};
      \node at (0,-0.5) {\emph{Multiplication}};
    \end{scope}
  \end{tikzpicture}
\end{frame}

%%

\begin{frame}
  \begin{block}{Group action}
    \emph{$\G\circlearrowright\E$}: A (finite) set $\E$ \emph{acted
      upon} by a group $\G$ \emph{faithfully} and \emph{transitively}:
    \begin{align*}
      * : \G × \E &→ \E\\
      \g * E &↦ E'
    \end{align*}
    \par\begin{description}
    \item[Compatibility:] \emph{$\g' * (\g * E) = (\g'\g)*E$} for all
      $\g,\g'\in\G$ and $E\in\E$;
    \item[Identity:] \emph{$\mathfrak{e} * E = E$} if and only if
      $\mathfrak{e}\in\G$ is the identity element;
    \item[Transitivity:] for all $E,E'\in\E$ there exist a
      \emph{unique $\g\in\G$} such that \emph{$\g*E'=E$}.
      \setlength{\itemsep}{2em}
    \end{description}
  \end{block}
  
  \begin{block}{Hard Homogeneous Space (HHS) --- Couveignes 1997}
    \emph{$\G\circlearrowright\E$} such that $\G$ is commutative and:
    \begin{itemize}
    \item \emph{Evaluating} $E' = \g*E$ is \emph{easy};
    \item \emph{Inverting} the action is \emph{hard}.
    \end{itemize}
  \end{block}
\end{frame}

% %%

\begin{frame}{HHS Diffie--Hellman}
  \begin{description}
  \item[Goal:] Alice and Bob have never met before. They are chatting
    over a public channel, and want to agree on a \emph{shared secret}
    to start a private conversation.
  \item[Setup:] They agree on a (large) \emph{HHS
      $\G\circlearrowright \E$} of order $N$.
  \end{description}

  \begin{center}
    \begin{tikzpicture}[x=1.4cm]
      \node at (0,0) {\bf Alice};
      \node at (7,0) {\bf Bob};
      \node at (0,-1) {pick random \alert{$\a\in\G$}};
      \node at (0,-1.5) {compute $E_A=\a*E_0$};
      \node at (7,-1) {pick random \alert{$\b\in\G$}};
      \node at (7,-1.5) {compute $E_B=\b*E_0$};
      \draw[->]
      (1,-2) to node[auto] {$E_A$} (6,-2);
      \draw[->] (6,-2.5) to node[auto] {$E_B$} (1,-2.5);
      \node at (3.5,-3.5) {\emph{Shared secret} is \alert{$\a*E_B=(\a\b)*E_0=\b*E_A$}};
    \end{tikzpicture}
  \end{center}  
\end{frame}

%%

\begin{frame}{Complex multiplication dictionary}
  \centering\large
  \setlength{\tabcolsep}{2em}
  \renewcommand{\arraystretch}{2}
  \begin{tabular}{r l}
    \emph{Quadratic imaginary fields} & \emph{Elliptic curves}\\
    \hline
    Integers of $\Q(\sqrt{-D})$ & Endomorphisms of $E$\\
    Integral ideals of $\Q(\sqrt{-D})$ & Isogenies of $E$\\
    Ideal classes in $\Cl(-D)$ & Isogenies \raisebox{-0.8em}{\tikz{\node (E) at (0,0) {$\bullet$}; \node (E1) at (2,0) {$\bullet$}; \draw[->] (E) edge[bend left] (E1) edge[bend right] (E1);}}\\
    Ideal norm & Isogeny degree\\
    Conjugate ideal & Dual isogeny\\
  \end{tabular}
\end{frame}

%% 

\begin{frame}{HHSDH from complex multiplication}
  \centering
  \movie[width=13cm,height=7.875cm,showcontrols,open,poster,autostart,borderwidth=0pt]{}{CSIDH.mp4}
\end{frame}

%%

\begin{frame}{HHSDH from complex multiplication}
  \centering
  \movie[width=13cm,height=7.875cm,showcontrols,open,poster,autostart,borderwidth=0pt]{}{KeyExchange.mp4}
\end{frame}

%%

\begin{frame}{Quantum security}

  \textbf{Fact:} Shor's algorithm \emph{does not apply} to Diffie-Hellman
  protocols from \emph{group actions}.

  \begin{block}{Subexponential attack\hfill\emph{$\exp(\sqrt{\log p\log\log p})$}}
    \begin{itemize}
    \item Reduction to the \emph{hidden shift problem} by evaluating
      the class group action in \emph{quantum
        supersposition} (subexpoential cost);
    \item Well known reduction from the hidden shift to the
      \emph{dihedral (non-abelian) hidden subgroup problem};
    \item Kuperberg's algorithm solves the dHSP with a subexponential
      number of class group evaluations.
    \item Recent work suggests that $2^{64}$-qbit security is achieved
      somewhere in $512 < \log p < 2048$.
    \end{itemize}
  \end{block}
\end{frame}

%%

\begin{frame}{More applications of post-quantum group actions}
  \large
  \begin{itemize}
  \item Efficient \emph{non-interactive} key exchange;\hfill\emph{CSIDH}
  \item Decent signatures;\hfill\emph{CSI-FiSh}
  \item Threshold and ring signatures;\footnote{De Feo--Meyer, Beullens--Katsumata--Pintore}
  \item Hash proof systems, OT, VRFs, KDM-secure symmetric encryption,
    \dots\footnote{De Feo--Alamati--Montgomery--Patranabis}
  \item \dots
  \end{itemize}
\end{frame}

%%

\begin{frame}{The isogeny path problem}
  \begin{block}{The mother of all problems}
    Given (supersingular) elliptic curves \emph{$E$} and \emph{$E'$}
    over a finite field, find a \emph{chain of low-degree isogenies}
    \[\phi_n\circ\cdots\circ\phi_1 \;:\; E\to E'.\]
  \end{block}

  \begin{description}[leftmargin=1em]
    \setlength{\itemsep}{1em}
  \item[CSIDH/CSI-FiSh]
    \begin{itemize}
    \item $E$ and $E'$ are supersingular curves \emph{defined over $\F_p$}.
      \hfill\uncover<-1>{\alert{\large\rightpointleft}}
    \end{itemize}
  \item[SIDH/SIKE] 
    \begin{itemize}
    \item Additional information given on a \emph{secret short
        chain} $E\to E'$;
      \hfill\uncover<2->{\alert{\large\rightpointleft}}
    \item In instantiations, $E$ is fixed and \emph{special} (in a
      cryptanalytic sense).
    \end{itemize}
  \item[SQISign]
    \begin{itemize}
    \item Every signature leaks information on a secret chain
      $E\to E'$,\\ unless a \emph{10-pages-long computational
        assumption} holds.
    \item + Random Oracle Model (rewinding).
    \end{itemize}
  \item[Time-delay]
    \begin{itemize}
    \item A \emph{long chain} $E\to E'$ is \emph{publicly known}, find
      a shorter one.
    \item + Pairing assumptions (not post-quantum).
    \end{itemize}
  \end{description}
\end{frame}

%%

\begin{frame}{The full supersingular graph}
  \begin{columns}
    \begin{column}{0.55\textwidth}
      \begin{itemize}
        \setlength{\itemsep}{2em}
      \item All supersingular curves are defined over $\F_{p^2}$;
      \item The $\ell$-isogeny graph is \emph{$(\ell+1)$-regular and
          connected} for any prime $\ell\ne p$;
      \item Complex multiplication does not apply,\\
        \emph{good luck!}
      \end{itemize}
    \end{column}
    \begin{column}{0.45\textwidth}
      \centering
      \begin{tikzpicture}[scale=1.4]
        \begin{scope}[every node/.style={fill,black,circle,inner sep=2pt}]
          \node at (0,0)  (1){};
          \node at (0,4) (20){};
          \node at (2,1)  (16z){};
          \node at (-2,1)  (81z){};
          \node at (-1,2) (77z){};
          \node at (1,2)  (20z){};
          \node at (-2,3)  (85z){};
          \node at (2,3)  (12z){};
        \end{scope}

        \begin{uncoverenv}<1->
          \begin{scope}[blue,every loop/.style={looseness=50}]
            \path (1) edge (20) edge (16z) edge (81z);
            \path (20) edge[loop left] (20) edge[loop right] (20);
            \path (16z) edge (81z) edge (77z);
            \path (81z) edge (20z);
            \path (77z) edge (20z) edge (85z);
            \path (20z) edge (12z);
            \path (12z) edge[bend right=10] (85z) edge[bend left=10] (85z);
          \end{scope}
        \end{uncoverenv}
        
        \begin{uncoverenv}<1->
          \begin{scope}[red]
            \path (1) edge (85z) edge (81z) edge (12z) edge (16z);
            \path (20) edge (85z) edge (77z) edge (20z) edge (12z);
            \path (81z) edge (85z) edge (77z) edge (16z);
            \path (85z) edge (12z);
            \path (12z) edge (16z);
            \path (16z) edge (20z);
            \path (20z) edge[bend right=10] (77z) edge[bend left=10] (77z);
          \end{scope}
        \end{uncoverenv}
      \end{tikzpicture}
      
      \small
      \emph{Figure:} \bl{$2$}- and \rd{$3$}-isogeny graphs on $\F_{97^2}$.
    \end{column}
  \end{columns}
\end{frame}

%%

\begin{frame}{Making things that do not want to commute commute}
  \large
  \begin{tikzpicture}[x=15em,y=10em]
    \node at (-.5,.3) {}; \node at (1.5,-1.3) {};
    \node(E0) at (0,0) {$E$};
    \node(EA) at (1,0) {\alt<4->{$E/\ker\phi_A$}{$E_A$}};
    \node(EB) at (0,-1) {\alt<4->{$E/\ker\phi_B$}{$E_B$}};
    \node(ES) at (1,-1) {\temporal<2>{}{??}{\alt<5->{$E/(\ker\phi_{A}\oplus\ker\phi_B)$}{$E_{AB}$}}};
    \draw[->]
    (E0) edge node[auto] {\small\alt<6->{$\left(\begin{smallmatrix}\rd{0}&0&0&0\\0&1&0&0\\0&0&1&0\\0&0&0&1\end{smallmatrix}\right)$}{$\phi_A$}} (EA)
    edge node[auto,swap] {\small\alt<6->{$\left(\begin{smallmatrix}1&0&0&0\\0&1&0&0\\0&0&1&0\\0&0&0&\rd{0}\end{smallmatrix}\right)$}{$\phi_B$}} (EB);
    \begin{uncoverenv}<2->
      \draw[<-]
      (ES) edge node[auto,swap] {\small\temporal<3-5>{}{$[\phi_A]_*\phi_B$}{$\left(\begin{smallmatrix}1&0&0\\0&1&0\\0&0&\rd{0}\end{smallmatrix}\right)$}} (EA)
      edge node[auto] {\small\temporal<3-5>{}{$[\phi_B]_*\phi_A$}{$\left(\begin{smallmatrix}\rd{0}&0&0\\0&1&0\\0&0&1\end{smallmatrix}\right)$}} (EB);
    \end{uncoverenv}
    \begin{uncoverenv}<6->
      \draw[->,dashed]
      (E0) edge node[auto] {$\left(\begin{smallmatrix}\rd{0}&0&0&0\\0&1&0&0\\0&0&1&0\\0&0&0&\rd{0}\end{smallmatrix}\right)$} (ES);
    \end{uncoverenv}
  \end{tikzpicture}
\end{frame}

%%

\begin{frame}{SIKE}
  \centering\large
  \begin{tikzpicture}
    \begin{scope}
      \draw (0,1.2) node[anchor=east,blue] {$A\subset E[2^{e_A}]$};
      \draw (0,0.4) node[anchor=east,red] {$B\subset E[3^{e_B}]$};
      \draw (0,-0.4) node[anchor=east,blue] {$\ker[\phi_A]_*\rd{\phi_B} = \phi_A(\rd{B})$};
      \draw (0,-1.2) node[anchor=east,red] {$\ker[\phi_B]_*\bl{\phi_A} = \phi_B(\bl{A})$};
    \end{scope}
    \begin{scope}[xshift=5cm,coils/.style={-angle 90,decorate,decoration={coil,aspect=0,amplitude=1pt}}]
      \node[matrix of nodes, ampersand replacement=\&, column sep=3cm, row sep=1.5cm] (diagram) {
        |(E)| $E$ \& |(Es)| $E/\bl{A}$ \\
        |(Ep)| {$E/\rd{B}$} \& |(Eps)| {$E/(\bl{A}\oplus\rd{B})$}\\
      };
      \path[->,blue] (E) edge[coils] node[auto] {$\phi_A$} (Es);
      \path[->,blue] (Ep) edge[coils] node[auto,swap] {$\rd{[\phi_B]_*}\phi_A$} (Eps);
      \path[->,red] (E) edge[coils] node[auto,swap] {$\phi_B$} (Ep);
      \path[->,red] (Es) edge[coils] node[auto] {$\bl{[\phi_A]_*}\phi_B$} (Eps);
    \end{scope}
  \end{tikzpicture}
\end{frame}

%%

\begin{frame}{Is all algebraic structure really lost?}
  \begin{block}{Quadratic orientations}
    $\O$ a quadratic imaginary order
    \[\emph{\Q(\sqrt{-D})} \qquad\supset\qquad \emph{\O} \qquad\subset\qquad \emph{\End(E)}\]
    Complex multiplication group action from $\Cl(\O)$.
  \end{block}

  \begin{block}{Deuring correspondence}
    When $E$ is \emph{supersingular}:
    \begin{itemize}
    \item $\End(E)$ isomorphic to a maximal order in a quaternion
      algebra;
    \item $E$ has infinitely many (primitive) orientations;
    \item \emph{Quaternionic multiplication} acts on the set of
      supersingular curves.
    \end{itemize}
  \end{block}
\end{frame}

%%

\begin{frame}{The Deuring correspondence}
  \centering
  \movie[width=13.5cm,height=7.875cm,loop,poster,autostart,borderwidth=0pt]{}{Cycles.mp4}
\end{frame}

%%

\begin{frame}{The Deuring correspondence}
  \begin{itemize}
    \setlength{\itemsep}{2em}
  \item A foundation for the Charles--Goren--Lauter hash function;
  \item A framework to study the security of SIDH;\footnote{de
      Quehen--Kutas--Leonardi--Martindale--Panny--Petit--Stange.}
  \item An equivalence between the \emph{isogeny path problem} and the
    \emph{endomorphism ring
      problem};\footnote{Galbraith--Petit--Shani--Ti,
      Eisenträger--Hallgren--Lauter--Morrison--Petit, Wesolowski.}
  \item A tool to build new schemes.\footnote{Galbraith--Petit--Silva,
      De Feo--Kohel--Leroux--Petit--Wesolowski,
      De~Feo--Delpech~de~Saint~Guilhem--Fouotsa--Kutas--Leroux--Petit--Silva--Wesolowski}
  \end{itemize}
\end{frame}

%%

\begin{frame}{Cryptanalysis $\to$ cryptography}
  \begin{description}
  \item[1993] Menezes--Okamoto--Vanstone use \emph{pairings to
      break discrete logarithms} of supersingular curves;
  \item[2000] Joux and Boneh--Franklin initiate \emph{pairing based
      cryptography}.
    \bigskip
  \item[1999] Galbraith--Hess--Smart use \emph{isogeny graphs} to
    extend the impact of \emph{Weil descent attacks};
  \item[2006]<+-> Charles--Goren--Lauter and Rostovtsev--Stolbunov
    initiate \emph{isogeny based cryptography}.
    \bigskip
  \item[2014]<+-> Kohel--Lauter--Petit--Tignol solve one direction of
    the \emph{Deuring correspondence} (quaternions $\to$ curves) to
    \emph{break hash functions};
  \item[2017]<+-> Galbraith--Petit--Silva construct a \emph{signature
      scheme based on the Deuring correspondence};
  \item[2020]<+-> De Feo--Kohel--Leroux--Petit--Wesolowski introduce
    \emph{SQISign}.
  \end{description}
\end{frame}

%%

\begin{frame}{Are isogenies for real?}
  \begin{columns}
    \begin{column}{0.5\textwidth}
      Ask this guy

      \bigskip
      \url{https://youtu.be/EAe5dqWcxh4}

      \bigskip
      
      He'll tell you:
      \begin{itemize}
      \item SIKE is ready for standardization;
      \item CSIDH, CSI-FiSh, SQISign, require more work, but still
        hold promise.
      \end{itemize}
    \end{column}
    \begin{column}{0.5\textwidth}
      \centering
      \includegraphics[width=\textwidth]{pirate.jpg}

      \small
      Peculiar invited speaker at RWC '21
    \end{column}
  \end{columns}
\end{frame}

%%

\begin{frame}[plain]
  \centering
  \begin{tikzpicture}[remember picture,overlay]
    \begin{scope}[xscale=1.7,yshift=-15,opacity=0.8]
      \def\crater{12}
      \def\jumpa{-8}
      \def\jumpb{9}
      \def\diam{5cm}

      \foreach \i in {1,...,\crater} {
        \draw[blue] (360/\crater*\i : \diam) to[bend right] (360/\crater*\i+360/\crater : \diam);
        \draw[red] (360/\crater*\i : \diam) to[bend right] (360/\crater*\i+\jumpa*360/\crater : \diam);
        \draw[green] (360/\crater*\i : \diam) to[bend right=50] (360/\crater*\i+\jumpb*360/\crater : \diam);
      }
    \end{scope}
    
    \draw (0,0.5) node{\Huge\bf Thank you};
    \draw (0,-1.1) node{\large\url{https://defeo.lu/}};
    \draw (0,-1.8) node{\large\includegraphics[height=0.9em]{twitter.png}~\href{https://twitter.com/luca_defeo}{@luca\_defeo}};
  \end{tikzpicture}
\end{frame}

%%

\end{document}


% LocalWords:  Isogeny abelian isogenies hyperelliptic supersingular Frobenius
% LocalWords:  isogenous
