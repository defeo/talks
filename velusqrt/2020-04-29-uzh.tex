\documentclass[aspectratio=169]{beamer}

\usepackage[T1]{fontenc}
\usepackage[utf8]{inputenc}
\usepackage{unicode}
\usepackage[american]{babel}
\usepackage{amsmath,amsthm}
\usepackage{array}
\usepackage{multirow}
\usepackage{tikz}
\usetikzlibrary{matrix,decorations,decorations.text,calc,arrows,snakes,shapes,positioning}
\usepackage{multimedia}

\usepackage[arrow,matrix,curve,color,pdftex]{xy}
\definecolor{lightgray}{gray}{0.8}
\definecolor{lightergray}{gray}{0.95}

\usepackage{sourcesanspro}
\usepackage[amssymb,amsfonts]{concmath}
\usefonttheme[onlymath]{serif}
\usepackage{ulem}

\mode<presentation>{%
  \usetheme{ibm}
}

\newcommand{\F}{\mathbb{F}}
\newcommand{\Z}{\mathbb{Z}}
\newcommand{\Q}{\mathbb{Q}}
\DeclareMathOperator{\bigO}{O}
\DeclareMathOperator{\tildO}{\tilde{O}}
\DeclareMathOperator{\Res}{Res}


\title{$\sqrt{\!\text{élu's formulas}}$}
\subtitle{Faster Evaluation of Isogenies of Large Prime Degree}
\author{Luca De Feo}
\date[April 29, 2020, U Zürich]{April 29, 2020, the University of Zürich eSeminar}
\institute[IBM Research Zürich]{IBM Research Zürich\\[0.5em]
  \normalsize joint work with D.J.~Bernstein, A.~Leroux, B.~Smith}

\begin{document}

\frame[plain]{\titlepage}

%%

\begin{frame}{Why isogenies?}
  \begin{columns}
    \begin{column}{0.45\textwidth}
      Six families still in NIST post-quantum competition:

      \bigskip
      \begin{tabular}{ >{\color{structure}}l@{\hspace{2em}} c c}
      Lattices     & 9 encryption & \textcolor{black!30}{3 signature} \\
      Codes        & 7 encryption\\
      Multivariate & & \textcolor{black!30}{4 signature}\\
      Isogenies    & \alert{1 encryption}\\
      Hash-based   & & \textcolor{black!30}{1 signature}\\
      MPC          & & \textcolor{black!30}{1 signature}
      \end{tabular}
  \end{column}
    \begin{column}{0.45\textwidth}<2->
      \centering
      \begin{overlayarea}{0.8\textwidth}{\textheight}
      \begin{onlyenv}<2>
        \begin{tikzpicture}
          \node at (3,-0.5) {\bf Public key size};
          \node at (3,-1) {NIST-1 level (AES128)};
          \node at (3,-1.5) {\footnotesize (not to scale)};
          \draw (0.5,0) -- (5.5,0);
          \fill[structure] (1,0) -- (2,0) -- (2,4) -- (1,4);
          \node at (1.5,4.5) {\parbox{2cm}{\centering \emph{Codes}\\1 -- 300 KB}};
          \fill[structure] (2.5,0) -- (3.5,0) -- (3.5,1) -- (2.5,1);
          \node at (3,1.8) {\parbox{2cm}{\centering \emph{Lattices}\\0.5 -- 10 KB}};
          \fill[structure] (4,0) -- (5,0) -- (5,0.2) -- (4,0.2);
          \node at (4.5,0.7) {\parbox{2cm}{\centering \emph{Isogenies}\\209 B}};
        \end{tikzpicture}
      \end{onlyenv}
      \transdissolve<3>
      \begin{onlyenv}<3->
        \begin{tikzpicture}
          \node at (3,-0.5) {\bf Encryption performance};
          \node at (3,-1) {NIST-1 level (AES128)};
          \node at (3,-1.5) {\footnotesize (not to scale)};
          \draw (0.5,0) -- (5.5,0);
          \fill[structure] (1,0) -- (2,0) -- (2,0.2) -- (1,0.2);
          \node at (1.5,0.7) {\parbox{2cm}{\centering \emph{Codes}\\1 Mcycles}};
          \fill[structure] (2.5,0) -- (3.5,0) -- (3.5,0.3) -- (2.5,0.3);
          \node at (3,1) {\parbox{2cm}{\centering \emph{Lattices}\\0.5 -- 5 Mcycles}};
          \fill[structure] (4,0) -- (5,0) -- (5,4) -- (4,4);
          \node at (4.5,4.5) {\parbox{2cm}{\centering \emph{Isogenies}\\190 Mcycles}};
        \end{tikzpicture}
      \end{onlyenv}
    \end{overlayarea}
  \end{column}
  \end{columns}
\end{frame}

%%

\begin{frame}{Iso-what?!}
  \begin{block}{Keywords}
    \begin{itemize}
    \item<1-> An \emph{isogeny} is a \emph{map} between two
      \emph{elliptic curves};
    \item<2-> It is a \emph{group morphism}:
      \[ϕ(P+Q) = ϕ(P)+ϕ(Q);\]
    \item<3-> It is an \emph{algebraic map}:
      \[ϕ(x,y) = \left( \frac{g(x)}{h(x)}, y\left(\frac{g(x)}{h(x)}\right)' \right);\]
    \item<4-> It is entirely determined by its \emph{kernel} (i.e., by a single \emph{point});
    \item<5-> Isogeny \emph{degree $=$} size of the kernel \emph{$=$}
      order of kernel generator \emph{$\approx$} size of the polynomials;
    \end{itemize}
  \end{block}
\end{frame}

%%

\begin{frame}{Isogenies: an example over $\F_{11}$}
  \centering
  \begin{tikzpicture}[scale=0.4]
    \begin{scope}
      \node[anchor=center] at (0,7) {$E \;:\; y^2 = x^3 + x$};

      \uncover<-1>{
        \draw[thin,gray] (0,-6) -- (0,6);
        \draw[thin,gray] (-6,0) -- (6,0);
      }

      \foreach \x/\y in {0/0,5/3,-4/3,-3/5,-2/1,-1/3} {
        \draw[blue,fill] (\x,\y) circle (0.2) node(E_\x_\y){}
        (\x,-\y) circle (0.2) node(E_\x_-\y){};
      }

      \uncover<2->{\draw[red,fill] (0,0) circle (0.3);}
    \end{scope}

    \draw[black!10!white,thick] (10,-7) -- +(0,14);
    
    \begin{scope}[shift={(20,0)}]
      \node at (0,7) {$E' \;:\; y^2 = x^3 - 4x$};

      \uncover<-1>{
        \draw[thin,gray] (0,-6) -- (0,6);
        \draw[thin,gray] (-6,0) -- (6,0);
      }

      \foreach \x/\y in {0/0,2/0,3/2,4/2,6/4,-2/0,-1/5} {
        \draw[color=blue,fill] (\x,\y) circle (0.2) node(F_\x_\y){}
        (\x,-\y) circle (0.2) node(F_\x_-\y){};
      }
    \end{scope}

    \begin{scope}[color=red,-latex,dashed]
      \begin{uncoverenv}<2->
        \path
        (E_5_3) edge (F_3_2)
        (E_-4_3) edge (F_4_-2)
        (E_-3_5) edge (F_4_2)
        (E_-2_1) edge (F_3_-2)
        (E_-1_3) edge (F_-2_0);
      \end{uncoverenv}
      \begin{uncoverenv}<2->
        \path
        (E_5_-3) edge (F_3_-2)
        (E_-4_-3) edge (F_4_2)
        (E_-3_-5) edge (F_4_-2)
        (E_-2_-1) edge (F_3_2)
        (E_-1_-3) edge (F_-2_0);
      \end{uncoverenv}
    \end{scope}
  \end{tikzpicture}
  
  \begin{columns}
    \begin{column}{0.5\textwidth}
      \[\phi(x,y) = \left(\frac{x^2 + 1}{x},\quad y\frac{x^2-1}{x^2}\right)\]
    \end{column}
    \begin{column}{0.5\textwidth}
      \begin{itemize}
      \item<2-> Kernel generator in \alert{red}.
      \item<2-> This is a degree $2$ map.
      \item<2-> Analogous to $x\mapsto x^2$ in $\F_q^*$.
      \end{itemize}
    \end{column}
  \end{columns}
\end{frame}

%%

\begin{frame}{Algebraic complexity}
  \Large
  \centering
  \vfill
  In this talk, \emph{$k$} is an \emph{arbitrary} field.
  \vfill
  All complexities are \emph{in terms of operations over $k$}.
  \vfill
\end{frame}

%%

\begin{frame}{Anatomy of an isogeny (short Weierstrass form)}
  \transdissolve<2-13>
  \large
  \centering
  \begin{tikzpicture}
    \node at (-6,0) {$\phi(x,y) = \Biggl($};
    \node (x) at (-2.5,0) {$\displaystyle
      \frac{
        \only<-6>{x^4 - x^3 + 11x^2 + 9x + 12}
        \only<7->{g(x)}
      }{
        \only<-3>{\phantom{(}x^3-x^2-x+1\phantom{)}}
        \only<4-5>{(x + 1)(x - 1)^2}
        \only<6->{h(x)}
      },$};
    \node (y) at (3.5,0) {$\displaystyle y
      \only<-7>{\frac{x^5 - x^4 - 14x^3 - 26x^2 - 67x - 21}{x^5 - x^4 - 2x^3 + 2x^2 + x - 1}}
      \only<8->{\left(\frac{g(x)}{h(x)}\right)'}$};
    \node at (7,0) {$\Biggr)$};
    
    \begin{scope}[thick,red]
      \uncover<2-6>{\draw (-4.5,0.4) circle (0.2) +(0,0.2) -- +(0,1) +(0,1.2) node {degree};}
      \uncover<2>{\draw (-3.6,-0.2) circle (0.2) +(0,-0.2) -- +(0,-1) +(0,-1.2) node {degree $-1$};}
      \uncover<3-4>{
        \draw (-2.5,-0.3) ellipse (1.7 and 0.4) +(0,-0.4) -- +(0,-1.2) +(0,-1.4) node {kernel polynomial};
      }
      \uncover<5>{
        \draw (-3.3,-0.3) ellipse (0.7 and 0.4) +(-0.5,-0.3) -- +(-1,-0.6) +(-1.2,-0.8) node {Point of order $2$};
        \draw (-1.9,-0.3) ellipse (0.9 and 0.4) +(0.2,-0.4) -- +(1,-1.5) +(1.1,-1.7) node {Two points of order $4$};
      }
      \uncover<6->{
        \draw (-2.5,-0.6) -- +(0,-1) +(0,-1.5) node {$h(x) = \displaystyle\prod_{P\in K\setminus\{0\}}\bigl(x-x(P)\bigr)$};
      }
      \uncover<7->{
        \draw (-2.5,0.6) -- +(0,1) +(0,1.5) node {computed by Vélu--Elkies--Kohel formulas};
      }
    \end{scope}
  \end{tikzpicture}

  \medskip
  
  \begin{uncoverenv}<9->
    \begin{description}
    \item[Input:] \alt<16->{Kernel generator \emph{$P∈E(k)$}}{Finite
        kernel \emph{$K⊂E(k)$}} of order \emph{$n$}, \uncover<10->{a
        point \emph{$Q ∈ E(k)$};}
    \item[Output:] \alt<11->{The image point \emph{$ϕ(Q)$};}{Rational
        fractions \emph{$ϕ(x,y)$};}
    \item[Complexity:] \emph{\alt<12->{$\bigO(n)$}{$\tildO(n)$}}
      operations over $k$.
      \hfill
      \large\alert{
        \uncover<13->{Fast?}
        \uncover<14->{YES!}
        \hfill
        \uncover<15->{Optimal?}
        \uncover<17->{NO!}
        \hfill
        \strut
      }
    \end{description}
  \end{uncoverenv}
\end{frame}

%%

\begin{frame}{One step back\dots}
  \large
  \transdissolve<2-8>

  \begin{columns}
    \begin{column}{0.4\textwidth}
      \uncover<3->{Assume \emph{$n = ab$}}

      \begin{tikzpicture}[right]
        \node at (0,1) {};
        \node at (0,0) {$\displaystyle P(\alt<2->{α}{X}) =
          \alt<8->{\Res_Y(G,B)}{
            \prod_{i=0}^{\alt<4->{a}{n}-1}
            \alt<6->{G(i)}{
              \alert<5>{
                \uncover<4->{\prod_{j=0}^{b-1}}
                (\alt<2->{α}{X} - i \alt<4->{- j \cdot a)}{)}
              }
            }
          }$};

        \uncover<5->{
          \node at (0, -2) {$\displaystyle G(Y) = \prod_{j=0}^{b-1} (α - Y - j \cdot a)$};
        }
        \uncover<7->{
          \node at (0, -4) {$\displaystyle B(Y) = \prod_{i=0}^{a-1}(Y-i)$};
        }
      \end{tikzpicture}
    \end{column}
    \begin{column}{0.55\textwidth}
      \begin{uncoverenv}<9->
        \begin{enumerate}
        \item Compute \emph{baby steps} $B(Y)$
          \hfill\uncover<10->{\emph{$\tildO(a)$}}
        \item Compute \emph{giant steps} $G(Y)$
          \hfill\uncover<10->{\emph{$\tildO(b)$}}
        \item Compute resultant by\\
          \emph{multi-point evaluation}
          \hfill\uncover<10->{\emph{$\tildO(a+b)$}}
        \end{enumerate}
      \end{uncoverenv}

      \bigskip
      
      \begin{uncoverenv}<11->
        \begin{description}
        \item[Pollard '74, Strassen '76:] deterministic integer
          factorization in $O(n^{1/4})$.
        \item[Chudnovsky$^2$ '88:] $n$-th term of a holonomic sequence.
        \item[Bostan '20:] $n$-th term of a $q$-holonomic sequence.
        \end{description}
      \end{uncoverenv}
    \end{column}
  \end{columns}
\end{frame}

%%

\begin{frame}{Why did the trick work?}
  \transdissolve<3-5>
  \large
  \begin{itemize}
  \item An arithmetic decomposition of the root set:
    \emph{\alt<3->{$g^i ↦ g^i g^{aj}$}{$i ↦ i + a\cdot j$}};
  \item Efficient to compute \emph{giant steps}:
    \alt<4->{$1, (g^a), (g^a)^2, \dots,
      (g^{a})^{b-1}$}{$0, a, 2a, \dots, (b-1)a$};
  \item \alert<5->{Only univariate polynomials\alt<5->{???}{.}}
  \end{itemize}

  \bigskip
  \centering
  \uncover<2->{\alert<-4>{Wait! It works for any algebraic group!}} \uncover<5->{Or does it?}
\end{frame}

%%

\begin{frame}{\dots and two steps forward}
  \transdissolve<2-5>
  \hspace{4em}
  \begin{tikzpicture}[right]
    \uncover<1>{
      \node at (0,0) {$\displaystyle h(x(Q)) = \prod_{P ∈ K}^{\phantom{m}} \bigl(x(Q)-x(P)\bigr)$};
    }
    \uncover<2>{
      \node at (0,0) {$\displaystyle h(x(Q)) = \prod_{i=1}^n \bigl(x(Q)-x([i]P)\bigr)$};
    }
    \uncover<3->{
      \node at (0,0) {$\displaystyle h(x(Q)) =
        \prod_{i=1}^a \prod_{j=1}^b
        \bigl(x(Q) - x([i]P + [a\cdot j]P)\bigr)$};
    }
    
    \uncover<5->{
      \node at (0, -2) {$\displaystyle G(\alert<6->{\mathcal{Y}}) =
        \prod_{j=0}^{b-1} \bigl(x(Q) - x(\alert<6->{\mathcal{Y}} + [a\cdot j]P)\bigr)$};
    }
    \uncover<4->{
      \node at (0, -4) {$\displaystyle B(Y) = \prod_{i=0}^{a-1}\bigl(Y-x([i]P)\bigr)$};
    }
  \end{tikzpicture}
\end{frame}

%%

\begin{frame}{Biquadratic relations {\normalsize(Montgomery model)}}
  Let $E\;:\;y^2 = x^3 + Ax^2 + x$, let \emph{$P,Q\in E$},
  
  \begin{align*}
    \bigl(X - \alert{x(P+Q)})(X - \alert{x(P-Q)}\bigr) 
    &=
    X^2\\
    &-2
    \frac{(\emph{x(P)x(Q)} + 1)(\emph{x(P)} + \emph{x(Q)}) + 2A\emph{x(P)x(Q)}}{(\emph{x(P)} - \emph{x(Q)})^2}X\\
    &+ 
    \frac{(\emph{x(P)x(Q)} - 1)^2}{(\emph{x(P)} - \emph{x(Q)})^2}
  \end{align*}

  \bigskip
  
  assuming $0∉\{P,Q,P+Q,P-Q\}$.
\end{frame}

%%

\begin{frame}{Back to polynomials}
  \transdissolve<2-6>
  \hspace{4em}
  \begin{tikzpicture}[right]
    \uncover<1>{
      \node at (0,0) {$\displaystyle h(x(Q)) =
        \prod_{i=1}^a \prod_{j=1}^b
        \bigl(x(Q) - \alert{x([i]P + [a\cdot j]P)}\bigr)$};
    }
    \uncover<2-5>{
      \node at (0,0) {$\displaystyle h(x(Q)) =
        \prod_{i\in I}^{\phantom{a}} \prod_{j\in J}^{\phantom{b}}
        \bigl(x(Q) - \alert{x([i]P + [j]P)}\bigr)\bigl(x(Q) - \alert{x([i]P - [j]P)}\bigr)$};
    }
    \uncover<6>{
      \node at (0,0) {$\displaystyle h(x(Q)) = \Res_Y(G,B)\phantom{\prod_{i}^{a}}$};
    }

    \uncover<-2>{
      \node at (0, -2) {$\displaystyle G(\mathcal{Y}) =
        \prod_{j=0}^{b-1} \bigl(x(Q) - \alert{x(\mathcal{Y} + [a\cdot j]P)}\bigr)$};
    }
    \uncover<3>{
      \node at (0, -2) {$\displaystyle G(\mathcal{Y}) =
        \prod_{j\in J}^{\phantom{b-1}} \bigl(x(Q) - \alert{x(\mathcal{Y} + [j]P)}\bigr)\bigl(x(Q) - \alert{x(\mathcal{Y} - [j]P)}\bigr)$};
    }
    \uncover<4->{
      \node at (0, -2) {$\displaystyle G(\alt<5->{\!Y}{\mathcal{Y}}) =
        \prod_{j\in J}^{\phantom{b-1}} \left(x(Q)^2 
        + \frac{f_1(\alt<5->{\phantom{x(}\!Y\phantom{)}}{\emph{x(\mathcal{Y})}}, \emph{x([j]P)})}
        {f_0(\alt<5->{\phantom{x(}\!Y\phantom{)}}{\emph{x(\mathcal{Y})}}, \emph{x([j]P)})}x(Q)
        + \frac{f_2(\alt<5->{\phantom{x(}\!Y\phantom{)}}{\emph{x(\mathcal{Y})}}, \emph{x([j]P)})}
        {f_0(\alt<5->{\phantom{x(}\!Y\phantom{)}}{\emph{x(\mathcal{Y})}}, \emph{x([j]P)})}
        \right)$};
    }

    \uncover<1>{
      \node at (0, -4) {$\displaystyle B(Y) = \prod_{i=0}^{a-1}\bigl(Y-x([i]P)\bigr)$};
    }
    \uncover<2->{
      \node at (0, -4) {$\displaystyle B(Y) = \prod_{i\in I}^{\phantom{a-1}}\bigl(Y-\emph{x([i]P)}\bigr)$};
    }
  \end{tikzpicture}
\end{frame}

%%

\begin{frame}{Additional remarks}
  \Large
  \begin{itemize}
  \item Same technique to evaluate the isogeny \emph{numerator}.
  \item Similar technique to compute \emph{image curve} equation.
  \item Compatible with \emph{projective coordinates}.
  \item Similar techniques for \emph{$y$-coordinate} (formal derivatives).
  \item Also applies to kernel points defined over \emph{algebraic
      extensions of $k$}\\
    \textcolor{gray}{(but in the worst (generic) case it provides no gain)}
  \item Might extend to \emph{theta-functions} more general than $x(\cdot)$.
  \end{itemize}
\end{frame}

%%

\begin{frame}{Why is this important?}
  \large
  
  Every \emph{efficient} isogeny based cryptosystem evaluates tons of
  isogenies:

  \bigskip
  
  \begin{description}
  \item[SIDH (De Feo, Jao, Plût '12):] only isogenies of degree $2$ and $3$.\hfill\uncover<2->{\emph{meh}}
  \item[CSIDH (Castryck, Lange, Martindale, Panny, Renes '18):] degrees up to 587.\hfill\uncover<2->{\emph{got a chance}}
  \item[CSURF (Castryck, Decru '20):] degrees up to 389.\hfill\uncover<2->{\emph{got a chance}}
  \item[B-SIDH (Costello '19):] degrees in the millions!\hfill\uncover<2->{\emph{wow!}}
  \item[others:] Galbraith, Petit, Silva '17,\\
    Delpech de Saint Guilhem, Kutas, Petit, Silva
    '19,\hfill\uncover<2->{\emph{needs work}}\\
    \dots
  \end{description}
\end{frame}

%%

\begin{frame}
  \begin{columns}
    \begin{column}{0.5\textwidth}
      \scalebox{0.6}{\input xyisogeny512 }
    \end{column}
    \begin{column}{0.4\textwidth}
      Performance of new (\alert{red}) against old (\emph{blue}) algorithm, in three
      different implementations.

      Time to evaluate an isogeny with base field CSIDH-512. $x$-axis:
      isogeny degree, $y$-axis:

      \begin{description}
      \item[Top:] Cycle counts of pure C implementation based on Flint.
      \item[Middle:] Cycle counts of assembly optimized implementation
        based on original CSIDH-512.
      \item[Bottom:] $\F_p$ multiplication counts of the assembly
        optimized implementation.
      \end{description}
    \end{column}
  \end{columns}
\end{frame}

%%

\begin{frame}{CSIDH results}
  \large
  \begin{itemize}
  \item Assembly implementation improves CSIDH-512 by \emph{1\%},
    CSIDH-1024 by \emph{8\%}.
  \item Slightly smaller gain expected on CSURF-512.
  \item Not constant-time.
  \item We only applied algebraic algorithms for \emph{polynomial multiplication}.\\
    \alert{There is still some room for improvement.}
  \end{itemize}
\end{frame}

%%

\begin{frame}{Very large degrees}
  \begin{columns}
    \begin{column}{0.75\textwidth}
      \scalebox{0.9}{\input xyisogenyjulia }
    \end{column}
    \begin{column}{0.25\textwidth}
      Performance comparison of new (\alert{red}) against old
      (\emph{blue}) algorithm in a Julia/Nemo implementation on a
      256-bits base field.

      \medskip
      
      $x$-axis: isogeny degree.

      \medskip
      $y$-axis: cycle counts.
    \end{column}
  \end{columns}
\end{frame}

%%

\begin{frame}{B-SIDH results}
  \large
  
  \emph{B-SIDH:} a variation on SIDH with optimal field size

  \medskip
  
  \begin{itemize}
  \item First \emph{NIST level 1} secure instantiation of B-SIDH (256 bits base field).
  \item High level, unoptimized implementation in Julia/Nemo.
  \item Largest isogeny degree: \emph{$6548911 \approx 2^{23}$}.
  \item Alice completes one round of key exchange in \emph{0.56s},\\
    against \emph{2s} for old algorithm.
  \item Bob completes one round of key exchange in \emph{10s},\\
    against \emph{10 minutes} for old algorithm!
  \end{itemize}
\end{frame}

%%

\begin{frame}{Implementation details}
  \large
  
  Code (5 different implementations!) at: \emph{\url{https://velusqrt.isogeny.org}}

  \medskip
  \begin{itemize}
  \item Lots of \emph{exploitable symmetries} due to the Montgomery
    form,
  \item \emph{Interesting challenge:} fast polynomial multiplication for\\
    small degree polynomials over moderately sized fields:
    \begin{itemize}
    \item Naive, Karatsuba,
    \item \emph{Middle product} algorithms.
    \item Toom-Cook? For what sizes?
    \item We did not explore \emph{Kronecker substitution} at all!
    \end{itemize}
  \item \emph{Scaled remainder trees} to reduce number of
    inversions in multi-point evaluation.
  \end{itemize}
\end{frame}

%%

\begin{frame}{Open questions}
  \large
  \begin{itemize}
  \item Constant time implementation.
  \item Impact CSIDH? On isogeny action crypto in general?
  \item What about memory-constrained architectures?
  \item Lower bounds? See also VDFs\dots
  \end{itemize}

  \bigskip
  \centering
  \emph{\url{https://velusqrt.isogeny.org}}
\end{frame}

%% 

\begin{frame}[plain]
  \centering
  \begin{tikzpicture}[remember picture,overlay]
    \begin{scope}[xscale=1.7,yshift=-15,opacity=0.8]
      \def\crater{12}
      \def\jumpa{-8}
      \def\jumpb{9}
      \def\diam{5cm}

      \foreach \i in {1,...,\crater} {
        \draw[blue] (360/\crater*\i : \diam) to[bend right] (360/\crater*\i+360/\crater : \diam);
        \draw[red] (360/\crater*\i : \diam) to[bend right] (360/\crater*\i+\jumpa*360/\crater : \diam);
        \draw[green] (360/\crater*\i : \diam) to[bend right=50] (360/\crater*\i+\jumpb*360/\crater : \diam);
      }
    \end{scope}
    
    \draw (0,0.5) node{\Huge\bf Thank you};
    \draw (0,-1.1) node{\large\url{https://defeo.lu/}};
    \draw (0,-1.8) node{\large\includegraphics[height=0.9em]{twitter.png}~\href{https://twitter.com/luca_defeo}{@luca\_defeo}};
  \end{tikzpicture}
\end{frame}

\end{document}


% LocalWords:  Isogeny abelian isogenies hyperelliptic supersingular Frobenius
% LocalWords:  isogenous
