\documentclass{beamer}

%\includeonlyframes{current}

\usepackage[T1]{fontenc}
\usepackage[utf8]{inputenc}
\usepackage[american]{babel}
\usepackage{amsmath,amssymb,amsthm}
\usepackage{tikz,pgflibraryarrows,pgflibraryplotmarks,pgflibrarysnakes,pgflibraryshapes}
\usepackage[backend=biber,citestyle=authoryear-comp,bibstyle=beamer,doi=false,isbn=false,url=false,maxnames=10]{biblatex}
\bibliography{defeo}

\mode<presentation>{%
  \usetheme{Boadilla}
}
\beamertemplatenavigationsymbolsempty

\usepackage{sourcesanspro}
\usepackage[amssymb,amsfonts]{concmath}
\usefonttheme[onlymath]{serif}

\renewcommand{\emph}[1]{{\usebeamercolor[fg]{structure}#1}}

%\let\footcite\footnote

\newcommand{\C}{\mathbb{C}}
\newcommand{\R}{\mathbb{R}}
\newcommand{\Z}{\mathbb{Z}}
\newcommand{\N}{\mathbb{N}}
\newcommand{\Q}{\mathbb{Q}}
\newcommand{\F}{\mathbb{F}}
\renewcommand{\P}{\mathbb{P}}
\renewcommand{\O}{\mathcal{O}}
\newcommand{\tildO}{\mathcal{\tilde{O}}}
\newcommand{\End}{\operatorname{End}}
\newcommand{\chr}{\operatorname{char}}
\newcommand{\Cl}{\operatorname{Cl}}
\renewcommand{\a}{{\mathfrak{a}}}
\renewcommand{\b}{{\mathfrak{b}}}
\newcommand{\cyc}[1]{{\langle #1 \rangle}}
\newcommand{\ord}{\operatorname{ord}}

\usetikzlibrary{matrix,decorations,decorations.text,calc}

\pgfkeys{/triangle/.code=\tikzset{x={(-0.5cm,-0.866cm)},y={(1cm,0cm)}}}
\pgfkeys{/lattice/.code n args={4}{\tikzset{cm={#1,#2,#3,#4,(0,0)}}}}

\newcommand{\axes}[4]{
  \clip (#1,#3) rectangle (#2,#4);
  \draw [thin, gray, -latex] (#1,0) -- (#2,0);% Draw x axis
  \draw [thin, gray, -latex] (0,#3) -- (0,#4);% Draw y axis
}

\newcommand{\lattice}[2]{
  \draw[style=help lines,dashed] (#1-1,#1-1) grid[step=1] (#2+1,#2+1);
  \foreach \x in {#1,...,#2}{
    \foreach \y in {#1,...,#2}{
      \node[draw,circle,inner sep=2pt,fill] at (\x,\y) {};
      % Places a dot at those points
    }
  }
}

\newcommand{\bl}[1]{\textcolor{blue}{#1}}
\newcommand{\rd}[1]{\textcolor{red}{#1}}
\newcommand{\gr}[1]{\textcolor{green}{#1}}

% This command defines a triangle of dots of given height
\newcommand{\dottriangle}[2][\i-\j]{%
  \foreach \i in {0,...,#2} {%
    \foreach \j in {0,...,\i} {%
      \draw(\i,\j) node{#1};%
    }%
  }}


\title{Isogeny graphs in cryptography}
\author[Luca De Feo]{Luca De Feo}
\date[Mar 19--23, 2018 --- Post-Scryptum]{March 19--23, 2018, Post-Scryptum Spring School, Les 7 Laux}
\institute[U Paris Saclay]{Université Paris Saclay, UVSQ \& Inria}

\begin{document}

\frame[plain]{
  \begin{tikzpicture}[remember picture,overlay,opacity=0.5]
%    \node[at=(current page.center),yshift=-8mm]{\includegraphics[width=0.7\textwidth]{stroop.jpg}};
  \end{tikzpicture}
  {\bf \titlepage}
  \vspace{1.5cm}
  \begin{center}
    Slides online at~~~\emph{\url{http://defeo.lu/docet/}}
  \end{center}
}

%% 

\begin{frame}
  \frametitle{Overview}
  \tableofcontents  
\end{frame}

%%
%% 

\section{Foundations}

\subsection{Elliptic curves}

\begin{frame}{Projective space}
  \begin{definition}[Projective space]
    Let $\bar{k}$ an algebraically closed field, the \emph{projective
      space} $\P^n(\bar{k})$ is the set of non-null $n+1$-tuples
    $(x_0,\dots,x_n)\in \bar{k}^n$ modulo the equivalence relation
    \[(x_0,\dots,x_n) \sim (\lambda x_0, \dots, \lambda x_n) \qquad
      \text{with $\lambda\in \bar{k}\setminus\{0\}$}.\]
    A class is denoted by $(x_0:\cdots:x_n)$.
  \end{definition}

  Picture here
\end{frame}

%%

\begin{frame}{Weierstrass equations}
  \begin{columns}
    \begin{column}{0.4\textwidth}
      Let $k$ be a field of characteristic $\ne 2,3$.

      An \emph{elliptic curve \textit{defined over $k$}} is the locus
      in $\P^2(\bar{k})$ of an equation
      \[\emph{Y^2Z = X^3 + aXZ^2 + bZ^3},\]
      where $a,b\in k$ and $4a^3+27b^2\ne 0$.

      \begin{itemize}
      \item<2-> $\O=(0:1:0)$ is the \emph{point at infinity};
      \item<3-> $y^2 = x^3 + ax + b$ is the \emph{affine equation}.
      \end{itemize}
    \end{column}
    \begin{column}{0.6\textwidth}
      \begin{center}
        \begin{tikzpicture}[domain=-2.4566:4,samples=100,yscale=1/2]
          \draw plot (\x,{sqrt(\x*\x*\x-4*\x+5)});
          \draw plot (\x,{-sqrt(\x*\x*\x-4*\x+5)});

          \draw[thin,gray,-latex] (0,-7) -- (0,7);
          \draw[thin,gray,-latex] (-3,0) -- (4,0);
        \end{tikzpicture}
      \end{center}
    \end{column}
  \end{columns}
\end{frame}

%%

\begin{frame}{The group law}
  \begin{columns}
    \begin{column}{0.4\textwidth}
      \begin{block}{Bezout's theorem}
        Every line cuts $E$ in exactly three points (counted with
        multiplicity).
      \end{block}

      Define a \emph{group law} such that any three colinear points
      add up to zero.

      \begin{itemize}
      \item<2-> The law is \emph{algebraic}\\ (it has \textit{formulas});
      \item<3-> The law is \emph{commutative};
      \item<3-> $\O$ is the \emph{group identity};
      \item<3-> \emph{Opposite points} have the same $x$-value.
      \end{itemize}
    \end{column}
    \begin{column}{0.6\textwidth}
      \begin{center}
        \begin{tikzpicture}[domain=-2.4566:4,samples=100,yscale=1/2]
          \draw plot (\x,{sqrt(\x*\x*\x-4*\x+5)});
          \draw plot (\x,{-sqrt(\x*\x*\x-4*\x+5)});

          \draw[thin,gray,-latex] (0,-7) -- (0,7);
          \draw[thin,gray,-latex] (-3,0) -- (4,0);
          \draw (-3,1) -- (4,8/3+3);
          \begin{scope}[every node/.style={draw,circle,inner sep=1pt,fill},cm={1,2/3,0,0,(0,3)}]
            \node at (-2.287980,0) {};
            \node at (-0.535051,0) {};
            \node at (3.267475,0) {};
          \end{scope}
          \begin{scope}[every node/.style={yshift=0.3cm},cm={1,2/3,0,0,(0,3)}]
            \node at (-2.287980,0) {$P$};
            \node at (-0.535051,0) {$Q$};
            \node at (3.267475,0) {$R$};
          \end{scope}

          \draw[dashed] (3.267475,3.267475*2/3+3) -- (3.267475,-3.267475*2/3-3) 
          node[draw,circle,inner sep=1pt,fill] {}
          node[xshift=-0.1cm,anchor=east] {$P+Q$};
        \end{tikzpicture}
      \end{center}
    \end{column}
  \end{columns}
\end{frame}

%%

\begin{frame}{Group structure}
  
\end{frame}

%%

\begin{frame}{Isomorphisms}
  j-invariant, twists?
\end{frame}

%%

\begin{frame}{Isogenies}
  
\end{frame}

%%

\begin{frame}{Curves over finite fields}
  Frobenius, Hasse, Sato-Tate
\end{frame}

%%
%% 

\subsection{Isogenies}

\begin{frame}{Complex tori}
  
\end{frame}

%%

\begin{frame}{Curves over $\C$}
  j-invariant, $\p$, uniformization
\end{frame}

%%

\begin{frame}{Isogenies}
  dual, degree
\end{frame}

%%
%% 

\subsection{Complex multiplication}

\begin{frame}{The endomorphism ring}
  orders, quaternions, Deuring
\end{frame}

%%

\begin{frame}{Complex multiplication}
  Hasse
\end{frame}

%%

\begin{frame}{Isogeny volcanoes}
  
\end{frame}

%%

\begin{frame}{Quaternions?}
  
\end{frame}

%%
%% 

\section{Isogeny-based cryptography}
\subsection{Isogeny walks}

%%
%% 

\subsection{Key exchange from ordinary graphs}

%%
%% 

\subsection{Key exchange from supersingular graphs}

\begin{frame}{The scheme}
  
\end{frame}

%%

\begin{frame}{Attacks}
  
\end{frame}

%%

\begin{frame}{SIKE}
  
\end{frame}

%%

\begin{frame}{Implementation}
  
\end{frame}

%%

\begin{frame}{Side channels}
  
\end{frame}

%% 

\begin{frame}{Key compression? Signatures?}
  
\end{frame}

%%
%% 

%%

\begin{frame}
  \centering
  \begin{tikzpicture}
    \begin{scope}[xscale=1.2,black!60]
      \def\crater{7}
      \foreach \i in {1,...,\crater} {
        \draw[fill] (360/\crater*\i:3cm) circle (5pt);
        \draw (360/\crater*\i : 3cm) -- (360/\crater*\i+360/\crater : 3cm);
        \foreach \j in {-1,1} {
          \draw[fill] (360/\crater*\i : 3cm) -- (360/\crater*\i + \j*360/\crater/4 : 4cm) circle (3pt);
          \foreach \k in {-1,0,1} {
            \draw[fill] (360/\crater*\i + \j*360/\crater/4 : 4cm) --
            (360/\crater*\i + + \j*360/\crater/4 + \k*360/\crater/6 : 4.5cm) circle (1pt);
          }
        }
      }
    \end{scope}
    
    \draw (0,1) node{\Huge\bf Thank you};
    \draw (0,-0.6) node{\large\url{http://defeo.lu/}};
    \draw (0,-1.3) node{\large\includegraphics[height=0.9em]{twitter.png}~\href{https://twitter.com/luca_defeo}{@luca\_defeo}};
  \end{tikzpicture}
\end{frame}

%%
%%

\begin{frame}[allowframebreaks]
  \frametitle{References}

  \defbibfilter{books}{\type{book} \or \type{booklet} \or \type{thesis}
    \or \type{report} \or \type{collection} \or \type{manual}
    \or \type{periodical} \or \type{proceedings}}
  \defbibfilter{articles}{\not \(\type{book} \or \type{booklet} \or \type{thesis}
    \or \type{report} \or \type{collection} \or \type{manual}
    \or \type{periodical} \or \type{proceedings}\)}

  \beamertemplatebookbibitems
  \printbibliography[filter=books]
  \beamertemplatearticlebibitems
  \printbibliography[filter=articles]
\end{frame}

\end{document}


% LocalWords:  Isogeny abelian isogenies hyperelliptic supersingular Frobenius
% LocalWords:  isogenous


