\documentclass{beamer}

\usepackage[T1]{fontenc}
\usepackage[utf8]{inputenc}
\usepackage[american]{babel}
\usepackage{amsmath,amsthm}
\usepackage{tikz}
\usetikzlibrary{matrix,decorations,decorations.text,calc,arrows,snakes,shapes,positioning}

\usepackage[nosfdefault]{comicneue}
\usepackage{sourcesanspro}
\usepackage[amssymb,amsfonts]{concmath}
\usefonttheme[onlymath]{serif}
\usepackage{ulem}

\mode<presentation>{%
  \usetheme{Boadilla}
}
\beamertemplatenavigationsymbolsempty
\usecolortheme[RGB={110,117,124}]{structure}
\setbeamercolor*{title}{fg=blue!50!black}
\setbeamercolor*{titlegraphic}{fg=blue!20!black}

\renewcommand{\emph}[1]{{\usebeamercolor[fg]{structure}#1}}

%\let\footcite\footnote

\title[VDFs from Isogenies and Pairings]{Verifiable Delay Functions from Isogenies and Pairings}
\author[Luca De Feo]{
  Luca De Feo\\[1em]
  joint work with J.~Burdges, S.~Masson, C.~Petit, A.~Sanso}
\date[SIAM AG 2019]{July 13, 2019, SIAM AG, Bern\\
  \medskip
  Slides online at \url{https://defeo.lu/docet}}
\institute[UVSQ]{Université Paris Saclay -- UVSQ, France}

\begin{document}

\frame[plain]{\titlepage}

%%

\begin{frame}
  \centering\huge\bf
  
  Tired of *SIDH?
  \vfill

  \pause

  Enough quantum FUD?
  \vfill
  
  \pause

  Ready for a new buzzword?
\end{frame}

%%

{
  \setbeamercolor{background canvas}{bg=black}
  \begin{frame}[plain]
    \begin{tikzpicture}[remember picture,overlay]
      \node(pic)[at=(current page.center)] {
        \includegraphics[width=\paperwidth]{hodl.jpg}
      };
    \end{tikzpicture}
  \end{frame}
}

%%

\begin{frame}{Distributed lottery}
  Participants \textbf{A, B, \dots, Z} want to agree on a random
  winning ticket.

  \begin{block}{Flawed protocol}
    \begin{itemize}
    \item Each participant \emph{$x$} broadcasts a random string
      \emph{$s_x$};
    \item Winning ticket is \emph{$H(s_A, \dots, s_Z)$}.
    \end{itemize}
  \end{block}

  \pause

  \begin{block}{Fixes}
    \begin{itemize}
    \item Make the hash function \textbf{sloooooooooooooooooooooooooooow};
    \item<3> Make it possible to verify \emph{$w = H(s_A, \dots, s_Z)$} \textbf{fast}.
    \end{itemize}
  \end{block}
\end{frame}

%%

\begin{frame}{Verifiable Delay Functions (Boneh, Bonneau, Bünz, Fisch 2018)}
  \begin{block}{Wanted}
    Function (family) \emph{$f:X\to Y$} s.t.:

    \begin{itemize}
    \item Evaluating $f(x)$ takes \emph{long time}:
      \begin{itemize}
      \item \emph{uniformly} long time,
      \item on almost all random inputs $x$,
      \item even after having seen many values of $f(x')$,
      \item even given \emph{massive number of processors};
      \end{itemize}
    \item Verifying $y=f(x)$ is \emph{efficient}:
      \begin{itemize}
      \item ideally, exponential separation between evaluation and
        verification.
      \end{itemize}
    \end{itemize}
  \end{block}

  \centering
  \pause\vfill
  \textbf{Exercise}
  \pause\vfill
  Think of a function you like with these properties
  \pause\vfill
  Got it?
  \pause\vfill
  \textbf{You're probably wrong!}
\end{frame}

%%

\begin{frame}{Sequentiality}
  Ideal functionality:

  \[y = f(x) = \underbrace{H(H(\cdots(H(x))))}_{T \text{ times}}\]

  \begin{itemize}
  \item \emph{Sequential} assuming hash output ``unpredictability'',
  \item but how do you verify?
  \end{itemize}
\end{frame}

%%

\begin{frame}{VDFs from groups of unknown order}
  \begin{block}{Setup}
    A group of \emph{unknown order}, e.g.:
    \begin{itemize}
    \item \emph{$\mathbb{Z}/N\mathbb{Z}$} with $N=pq$ an RSA modulus, $p,q$ \emph{unknown}\\
      (e.g., generated by some trusted authority),
    \item \emph{Class group} of imaginary quadratic order.
    \end{itemize}
  \end{block}

  \begin{block}{Evaluation}
    With \emph{delay parameter} $T$:
    \begin{align*}
      f:G &\longrightarrow G\\
      x &\longmapsto x^{2^T}
    \end{align*}
    Conjecturally, fastest algorithm is repeated squaring.
  \end{block}

  \begin{block}{Verification (Wesolowski 2019, Pietrzak 2019)}
    \centering\pause\bf\Large
    Aha!
  \end{block}
\end{frame}

%%

\begin{frame}{Isogeny <3 Pairing}
  \begin{block}{}
    Let \emph{$\phi: E\to E'$}, let \emph{$P\in E[N]$} and
    {$Q\in E'[N]$}. Then
    
    \[e_N(P,\hat\phi(Q)) = e_N(\phi(P),Q)\]
  \end{block}
  
  \centering
  \begin{tikzpicture}[ampersand replacement=\&]
    \matrix(m)[matrix of math nodes,row sep=3em,column sep=4em]{
      X_1 \times X_2 \& X_1 \times X_2\\
      X_1 \times X_2 \& \mathbb{F}_{p^k}\\
    };
    \draw[->]
    (m-1-1) edge node[above]{\small$\phi\times 1$} (m-1-2)
    (m-1-1) edge node[left]{\small$1\times\hat\phi$} (m-2-1)
    (m-1-2) edge node[right]{\small$e_N$} (m-2-2)
    (m-2-1) edge node[below]{\small$e_N$} (m-2-2);
  \end{tikzpicture}

  \pause
  
  \begin{block}{Idea \#1}
    Use the equation for a BLS-like signature scheme: US patent
    8,250,367 (Broker, Charles, Lauter).
  \end{block}
\end{frame}

%%

\begin{frame}{Isogeny VDF}
  Assume $\deg\phi = 2^T$
  \vfill
  \[e_N(\phi(P),\phi(Q)) = e_N(P,Q)^{2^T}\]
  \vfill
  \begin{description}
  \item[Right side:] known group structure:
    \emph{$2^T \;\;\to\;\; 2^T\bmod p^k-1$};
  \item[Left side:] can evaluate $\phi$ in \emph{less than $T$ steps}?
  \end{description}
\end{frame}

%%

\begin{frame}{Isogeny VDF ($\mathbb{F}_p$-version)}
  \begin{block}{\uncover<2>{Trusted} Setup}
    \begin{itemize}
    \item Pairing friendly \emph{supersingular} curve $E/\mathbb{F}_p$\\
      \uncover<2>{\textbf{with unknown endomorphism ring!!!}}
    \item Isogeny \emph{$\phi:E\to E'$} of degree \emph{$2^T$},
    \item Point \emph{$P\in E[(N,\pi-1)]$}, image \emph{$\phi(P)$}.
    \end{itemize}
  \end{block}

  \begin{block}{Evaluation}
    \begin{description}
    \item[Input:] random \emph{$Q\in E'[(N,\pi+1)]$},
    \item[Output:] \emph{$\hat\phi(Q)$}.
    \end{description}
  \end{block}

  \begin{block}{Verification}
    \large
    \[e_N(P,\hat\phi(Q)) \quad\overset{?}{=}\quad e_N(\phi(P),Q).\]
  \end{block}
\end{frame}

%%

\begin{frame}{Sequentiality?}
  \large\centering
  Wesolowski, Pietrzak: \hfill $x \longmapsto x^2$ \hspace{4em}

  \vfill
  Isogenies: \hfill $\displaystyle x \longmapsto x\frac{x\alpha_i-1}{x-\alpha_i}$ \hspace{4em}

  \vfill
  No speedup? Even with unlimited parallelism? Really?
  
  \medskip
  See Bernstein, Sorenson. \emph{Modular exponentiation via the
    explicit Chinese remainder theorem}.
\end{frame}

%%

\begin{frame}
  \centering
  \begin{tikzpicture}
    \begin{scope}[xscale=1.2,black!60]
      \def\crater{7}
      \foreach \i in {1,...,\crater} {
        \draw[fill] (360/\crater*\i:3cm) circle (5pt);
        \draw (360/\crater*\i : 3cm) -- (360/\crater*\i+360/\crater : 3cm);
        \foreach \j in {-1,1} {
          \draw[fill] (360/\crater*\i : 3cm) -- (360/\crater*\i + \j*360/\crater/4 : 4cm) circle (3pt);
          \foreach \k in {-1,0,1} {
            \draw[fill] (360/\crater*\i + \j*360/\crater/4 : 4cm) --
            (360/\crater*\i + + \j*360/\crater/4 + \k*360/\crater/6 : 4.5cm) circle (1pt);
          }
        }
      }
    \end{scope}
    
    \draw (0,1) node{\Huge\bf Thank you};
    \draw (0,-0.6) node{\large\url{https://defeo.lu/}};
    \draw (0,-1.3) node{\large\includegraphics[height=0.9em]{twitter.png}~\href{https://twitter.com/luca_defeo}{@luca\_defeo}};
  \end{tikzpicture}
\end{frame}

\end{document}


% LocalWords:  Isogeny abelian isogenies hyperelliptic supersingular Frobenius
% LocalWords:  isogenous


