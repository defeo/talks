\documentclass[12pt,aspectratio=169]{beamer}

\usepackage[utf8]{inputenc}
\usepackage[american]{babel}
\usepackage{amsmath,amsthm}
\usepackage{tikz}

\newcommand{\sm}[2]{\left(\protect\begin{matrix}#1\protect\\#2\protect\end{matrix}\right)}

\usepackage[nosfdefault]{comicneue}
\usepackage{sourcesanspro}
\usepackage[amssymb,amsfonts]{concmath}

\let\OLDitemize\itemize
\renewcommand\itemize{\OLDitemize\addtolength{\itemsep}{1em}}

\setbeamertemplate{navigation symbols}{}
\renewcommand{\emph}[1]{{\usebeamercolor[fg]{structure}#1}}

\title{Interpolating isogenies in quasi-optimal time}
\author{Luca De Feo}
\date[24 Nov 2023, François 3C]{24 Nov 2023, François 3C}
\institute{IBM Research Zürich}

\begin{document}

\frame[plain]{\titlepage}

%%

\begin{frame}{Schoof}
  \large
  \[\pi|E[\ell] \quad:\quad (X,Y) \quad\mapsto\quad (X^q, Y^q) \mod \phi_\ell(X)\]
\end{frame}

%%

\begin{frame}{Elkies}
  \large
  \[H < E[\ell]\]
  \pause
  \[\phi_H(X) \;\mid\; \phi_\ell(X)\]
  \pause
  \[\phi : E \to E/H\]
\end{frame}

%%

\begin{frame}{Finding $E/H$}
  \large
  \begin{gather*}
    \Phi_\ell(j(E), X)\\
    =\\
    (X-\alert{j_1}) \cdot (X-j_2) \cdot (X^n + \cdots) \cdots (X^n + \cdots)
  \end{gather*}
\end{frame}

%%

\begin{frame}{Finding $H$}
  \large\centering
  
  \begin{tikzpicture}
    \node (E) at (0,0) {$E$};
    \node (EH) at (5,0) {$E/H$};
    \draw[dashed,->] (E) edge node[above] {\normalsize $\ell$} (EH);
  \end{tikzpicture}
  
  \bigskip

  Now what?
\end{frame}

%%

\begin{frame}{Elkies / Bostan--Morain--Salvy--Schost}
  \large
  \begin{itemize}
  \item Set up a differential equation;
  \item Solve the differential equation (Newton iteration).
  \end{itemize}

  \bigskip
  \begin{description}
  \item[Complexity:] $\tilde{O}(\ell)$
  \item[Problem:] only works if $\ell < p$.
  \end{description}
\end{frame}

%%

\begin{frame}{Couveignes}
  \large\centering
  \begin{tikzpicture}
    \node (E) at (0,0) {$E[p^n]$};
    \node (EH) at (5,0) {$E'[p^n]$};
    \node (P) at (0,-1) {$P$};
    \node (P') at (5,-1) {$aP'$};
    \draw[dashed,->] (E) edge node[above] {\normalsize $\ell$} (EH);
  \end{tikzpicture}

  \bigskip
  \begin{itemize}
  \item Try all $a$ and interpolate, until one works.
  \item Complexity: need $p^n > 2\ell \quad\to\quad \tilde{O}(p^{2n}) = \tilde{O}(\ell^2)$
  \item For small $p$, obsolted by $p$-adic point-counting.
  \end{itemize}
\end{frame}

%%

\begin{frame}{Lercier--Sirvent}
  \large
  \begin{itemize}
  \item Lift Elkies' differential equation to $\mathbb{Q}_p$,
  \item Solve equation using BMSS,
  \item Reduce mod $p$.
  \item Complexity: $\tilde{O}(\ell^3)$ or $\tilde{O}(\ell^2)$.
  \end{itemize}
\end{frame}

%%

\begin{frame}{De Feo--Hugounenq--Plût--Schost} 
  \large\centering
  \begin{tikzpicture}
    \node (E) at (0,0) {$E[2^n]$};
    \node (EH) at (5,0) {$E'[2^n]$};
    \node (P) at (0,-1.5) {$\sm{P}{Q}$};
    \node (P') at (5,-1.5) {$\sm{a&b}{c&d}\sm{P'}{Q'}$};
    \draw[dashed,->] (E) edge node[above] {\normalsize $\ell$} (EH);
  \end{tikzpicture} 

  \bigskip
  \begin{description}
  \item[Complexity:] $2^{2n} > 2\ell \quad\to\quad \tilde{O}(2^{4n}\cdot\ell) = \tilde{O}(\ell^{3})$
  \end{description}
\end{frame}

%%

\begin{frame}{De Feo--Hugounenq--Plût--Schost} 
  \large\centering
  \begin{tikzpicture}
    \node (P) at (0,-1) {$\sm{P}{Q}$};
    \node (P') at (5,-1) {$\sm{a&}{&d}\sm{P'}{Q'}$};
    \draw[dashed,->] (P) edge node[above] {\normalsize $\ell$} (P');
  \end{tikzpicture} 

  \bigskip

  \begin{itemize}
  \item If $2$ splits in $\mathbb{Z}[\pi]$, use volcanoes to find
    eigenspaces.
  \item Complexity: $2^{2n} > 2\ell \quad\to\quad \tilde{O}(2^{2n}\cdot \ell) = \tilde{O}(\ell^{2})$
  \item Non-split case in Cyril's thesis.
  \end{itemize}
\end{frame}

%%

\begin{frame}{Facepalm: don't forget Weil}
  \large\centering
  \begin{tikzpicture}
    \node (P) at (0,-1) {$\sm{P}{Q}$};
    \node (P') at (5,-1) {$\sm{a&}{&d}\sm{P'}{Q'}$};
    \draw[dashed,->] (P) edge node[above] {\normalsize $\ell$} (P');
  \end{tikzpicture} 

  
  \[e(P,Q)^\ell \quad=\quad e(aP',dQ') \quad=\quad e(P',Q')^{ad}\]
  
  \bigskip
  \begin{description}
  \item[Complexity:] $2^{2n} > 2\ell \quad\to\quad \tilde{O}(2^{n}\cdot\ell) = \alert{\tilde{O}(\ell^{1.5})}$
  \end{description}
\end{frame}

%%

\begin{frame}{Shooting at isogenies with a cannon}
  \large\centering
  \begin{tikzpicture}
    \node (P) at (0,-1) {$\sm{P}{Q}$};
    \node (P') at (5,-1) {$\sm{a&}{&d}\sm{P'}{Q'}$};
    \draw[dashed,->] (P) edge node[above] {\normalsize $\ell$} (P');
  \end{tikzpicture}

  \bigskip
  
  \begin{itemize}
  \item SIDH attacks verify existence of $\ell$-isogenies in $\log(\ell)^{O(1)}$.
  \item Attack $2^n$ times, interpolate once!
  \item Complexity: $2^{2n} > 2\ell \quad\to\quad \tilde{O}(2^{n}\cdot\mathsf{SIDH} + \ell) = \alert{\tilde{O}(\ell)}$
  \end{itemize}
\end{frame}

%%

\begin{frame}
  \Huge\centering
  Happy birthday, F.~!
\end{frame}

%%

\begin{frame}{A small puzzle for the road}
  \large
  \begin{itemize}
  \item Given a semiprime \emph{$N = pq$},
  \item Knowing \emph{$\ell \mid p-1$} and \emph{$\ell \mid q-1$}
  \item \dots with \emph{$\ell > N^{1/3}$},
  \item Given \emph{$x$} such that \emph{$x^\ell = 1 \mod N$}
  \item \dots and \emph{$\gcd(x, N) = 1$}~!
  \item Can you find \emph{$p,q$}?
  \end{itemize}
  
\end{frame}

\end{document}


% LocalWords:  Isogeny abelian isogenies hyperelliptic supersingular Frobenius
% LocalWords:  isogenous


