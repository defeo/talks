\documentclass[aspectratio=169]{beamer}

\usepackage[utf8]{inputenc}
\usepackage[american]{babel}
\usepackage{amsmath,amsthm}
\usepackage{unicode}
\usepackage{sourcesanspro}
\usepackage[amssymb,amsfonts]{concmath}
\usepackage{multirow}
\usepackage{tikz}
\usetikzlibrary{matrix,decorations,decorations.text,calc,arrows,snakes,shapes,positioning}
\usepackage{ulem}

\mode<presentation>{%
  \usetheme{Boadilla}
  \setbeamertemplate{frametitle}[default][right]
}
\beamertemplatenavigationsymbolsempty
\usecolortheme[RGB={0,98,255}]{structure}
\definecolor{ibmblue}{RGB}{0,98,255}
\setbeamercolor*{title}{fg=ibmblue}
\usefonttheme[onlymath]{serif}
\renewcommand{\emph}[1]{{\usebeamercolor[fg]{structure}#1}}


\newcommand{\F}{\mathbb{F}}
\newcommand{\cyc}[1]{{〈 #1 〉}}
\newcommand{\bl}[1]{\textcolor{blue}{#1}}
\newcommand{\rd}[1]{\textcolor{red}{#1}}


\title{Supersingular Isogeny Key Encapsulation}
\author{presented by Luca De Feo}
\date[Jun 8, 2021, NIST PQC Conference]{June 8, 2021, Third PQC Standardization Conference}
\institute{IBM Research}

\begin{document}

\frame[plain]{
  \begin{center}
    {\usebeamerfont{title}{SIKE}\par}
    {\usebeamerfont{subtitle}{\inserttitle}\par}
    \vspace{2.5\baselineskip}
    
    \begin{tabular}{l l l}
      David Jao, UW & Reza Azarderakhsh, FAU & Matthew Campagna, Amazon \\
      Craig Costello, MSR & \underline{Luca De Feo, IBM} & Basil Hess, IBM \\
      Aaron Hutchinson, UW & Amir Jalali, LinkedIn & Koray Karabina, NRC \\
      Brian Koziel, TI & Brian LaMacchia, MSR & Patrick Longa, MSR \\
      Michael Naehrig, MSR & Geovandro Pereira, UW & Joost Renes, NXP \\
      Vladimir Soukharev, ISG & David Urbanik, UofT
    \end{tabular}

    \vspace{1.5\baselineskip}
    \url{https://sike.org/}\par
    \vspace{1\baselineskip}
    \insertdate\par
  \end{center}
}

%%

\begin{frame}{SIKE}
  \vfill
  \textbf{S}upersingular \textbf{I}sogeny \textbf{K}ey \textbf{E}ncapsulation (SIKE)
  \begin{itemize}
  \item IND-CCA2 KEM
  \item Based on \textbf{S}upersingular \textbf{I}sogeny \textbf{D}iffie-\textbf{H}ellman (SIDH)
  \item Uses Hofheinz et al. transformation (TCC 2017) on SIDH to achieve CCA security
  \end{itemize}

  \bigskip
  The SIKE protocol specifies:
  \begin{itemize}
  \item Parameter sets
  \item Key/ciphertext formats
  \item Encapsulation/decapsulation mechanisms
  \item Choice of symmetric primitives (hash functions, etc.)
  \end{itemize}
\end{frame}

%%

\begin{frame}{SIDH: simplified overview}
  \vfill  
  \hfill\begin{minipage}{0.4\linewidth}
    \emph{0.} Starting supersingular curve\\
    \phantom{0.} {$E \;:\; y^2 = x^3 + 6x^2 + x$} over $\F_{p^2}$;
  \end{minipage}\hfill\strut

  \bigskip
  
  \begin{minipage}{0.45\linewidth}
    \emph{1.} Alice chooses a kernel $\bl{A}\subset E(\F_{p^2})$\\
    \phantom{1.} and sends $E/\bl{A}$ to Bob;    
  \end{minipage}
  \hfill
  \begin{minipage}{0.45\linewidth}
    \begin{flushright}
      \emph{2.} Bob chooses a kernel $\rd{B}\subset E(\F_{p^2})$\\
      \phantom{2.} and sends $E/\rd{B}$ to Alice;
    \end{flushright}
  \end{minipage}

  \bigskip
  
  \begin{center}
    \emph{3.} Shared secret: $E/\cyc{\bl{A},\rd{B}} = (E/\bl{A})/\bl{\phi_A}(\rd{B})
    = (E/\rd{B})/\rd{\phi_B}(\bl{A})$.
  \end{center}

  \begin{center}
    \begin{tikzpicture}
      \begin{scope}
        \large
        \node[matrix of nodes, ampersand replacement=\&, column sep=2cm, row sep=1cm] (diagram) {
          |(E)| $g$ \& |(Es)| $g^{\bl{a}}$ \\
          |(Ep)| {$g^{\rd{b}}$} \& |(Eps)| {$g^{\bl{a}\rd{b}}$}\\
        };
        \node at (-.1,0) {\Large DH};
        \path[->,blue] (E) edge (Es);
        \path[->,blue] (Ep) edge (Eps);
        \path[->,red] (E) edge (Ep);
        \path[->,red] (Es) edge (Eps);
      \end{scope}
      \begin{scope}[xshift=8cm]
        \large
        \node[matrix of nodes, ampersand replacement=\&, column sep=2cm, row sep=1cm] (diagram) {
          |(E)| $E$ \& |(Es)| $E/\bl{A}$ \\
          |(Ep)| {$E/\rd{B}$} \& |(Eps)| {$E/\cyc{\bl{A},\rd{B}}$}\\
        };
        \node at (-.3,0) {\Large SIDH};
        \path[->,blue] (E) edge node[auto] {$\phi_A$} (Es);
        \path[->,blue] (Ep) edge (Eps);
        \path[->,red] (E) edge node[auto,swap] {$\phi_B$} (Ep);
        \path[->,red] (Es) edge (Eps);
      \end{scope}
    \end{tikzpicture}
  \end{center}
\end{frame}

%%

\begin{frame}{SIDH: detailed overview}
  \begin{center}
    Curves represented by public triplets of \textit{torsion points}
  \end{center}
  \begin{minipage}{0.45\linewidth}
      ``Alice's'' \textit{torsion basis}:\\
      $E^{\bl{A}} := \bl{(P_A, Q_A, R_A:=P_A-Q_A)}$
  \end{minipage}
  \hfill
  \begin{minipage}{0.45\linewidth}
    \begin{flushright}
      ``Bob's'' \textit{torsion basis}:\\
      $E^{\rd{B}} := \rd{(P_B, Q_B, R_B:=P_B-Q_B)}$
    \end{flushright}
  \end{minipage}

  \begin{center}
    \begin{tikzpicture}
      \node at (-2,0) {\bf Alice};
      \node at (9,0) {\bf Bob};
      \node (A) at (-2,-1) {$\bl{A\subset\cyc{P_A,Q_A}}$};
      \node (fA) at (-2,-2) {$\bl{\phi_A}$};
      \node (B) at (9,-1) {$\rd{B\subset\cyc{P_B,Q_B}}$};
      \node (fB) at (9,-2) {$\rd{\phi_B}$};
      \draw[<->] (A) edge[blue] (fA) (B) edge[red] (fB);
      \draw
      (0,-1) edge[-latex] node[above]
      {$E^{\rd{B}}/\bl{A}
        := \bigl(\bl{\phi_A}(\rd{P_B}), \bl{\phi_A}(\rd{Q_B}), \bl{\phi_A}(\rd{R_B})\bigr)$}
      (7,-1)
      (7,-2.5) edge[-latex] node[above]
      {$E^{\bl{A}}/\rd{B}
        := \bigl(\rd{\phi_B}(\bl{P_A}), \rd{\phi_B}(\bl{Q_A}), \rd{\phi_B}(\bl{R_A})\bigr)$
      } (0,-2.5);
    \end{tikzpicture}

    \bigskip
    Shared secret:\\
    $j\bigl(E/\cyc{\bl{A},\rd{B}}\bigr) =
    j\bigl((E^{\rd{B}}/\bl{A})/\bl{\phi_A}(\rd{B})\bigr) =
    j\bigl((E^{\bl{A}}/\rd{B})/\rd{\phi_B}(\bl{A})\bigr)$.
  \end{center}
\end{frame}

%%

\begin{frame}{Changes for SIKE in third round}
  \begin{itemize}
  \item New optimized ARMv8, Cortex M4, and VHDL implementations.
  \item Key compression:
    \begin{itemize}
    \item Changed format of compressed ciphertexts (12.5\% larger than in round 2).
    \item Major improvements in speed and memory usage.
    \end{itemize}
  \end{itemize}
\end{frame}

%%

\begin{frame}{Changes for SIKE in second round}
  \begin{itemize}
  \item \emph{New} parameter sets: \emph{SIKEp434}, SIKEp503,
    \emph{SIKEp610}, SIKEp751, \sout{SIKEp964};
  \item Updated security analysis.
  \item Starting curve changed;
  \item Introduced key compression: $\approx 40\%$ smaller public keys
    and ciphertexts;
  \end{itemize}
\end{frame}

%%

\begin{frame}{Parameter sets}
  \centering
  \begin{tabular}{c || c c c}
    \hline
    Scheme & prime $p$ & $\log_2 p$ & Security level\\
    \hline\hline
    \texttt{SIKEp434} & $2^{216}3^{137}-1$ & 434 & NIST 1\\
    \texttt{SIKEp503} & $2^{250}3^{159}-1$ & 503 & NIST 2\\
    \texttt{SIKEp610} & $2^{305}3^{192}-1$ & 610 & NIST 3\\
    \texttt{SIKEp751} & $2^{372}3^{239}-1$ & 751 & NIST 5\\
    \hline
  \end{tabular}
\end{frame}

%%

\begin{frame}{Performance}
  \vfill
  \centering
  \begin{tabular}{l || c c | c c }
    \hline
    Scheme & Public key & ciphertext & Encaps & Decaps \\
           & \multicolumn{2}{c|}{\small bytes} & \multicolumn{2}{c}{\small $10^6$ cycles (\texttt{x86\_64} asm)} \\
    \hline\hline
    \texttt{SIKEp434} & 330 & 346 & \phantom{0}9.7 & 10.3 \\
    \texttt{SIKEp434\_compressed} & 197 & 236 & 15.1 & 11.0 \\
    \texttt{SIKEp503} & 378 & 402 & 13.6 & 14.4 \\
    \texttt{SIKEp503\_compressed} & 225 & 280 & 21.2 & 15.7 \\
    \texttt{SIKEp610} & 462 & 486 & 27.3 & 27.4 \\
    \texttt{SIKEp610\_compressed} & 274 & 336 & 37.5 & 29.2 \\
    \texttt{SIKEp751} & 564 & 596 & 40.7 & 43.9 \\
    \texttt{SIKEp751\_compressed} & 335 & 410 & 63.3 & 46.6 \\
    \hline
  \end{tabular}

  \bigskip
  Memory footprint of compression 3--10$\times$ smaller compared to Round 2.
\end{frame}

%%

\begin{frame}{Additional implementations}
  \vfill
  \begin{tabular}{c || c || c c | c c}
    \hline
    \multirow{6}*{\parbox{4cm}{\centering ARM implementations\\($10^6$ cycles)}}
    & Scheme & \multicolumn{2}{c|}{Cortex M4 (ARMv7)\footnote{M. Anastasova, R. Azarderakhsh, M. Mozaffari Kermani, ``Fast Strategies for the Implementation of SIKERound 3 on ARM Cortex-M4'', \url{https://ia.cr/2021/115}.}} & \multicolumn{2}{c}{Cortex A72 (ARMv8)} \\
    & & Encaps & Decaps & Encaps & Decaps\\
    \cline{2-6}
    & \texttt{SIKEp434} & 69 & 74 & 28 & 30 \\
    & \texttt{SIKEp503} & 97 & 104 & 40 & 42 \\
    & \texttt{SIKEp610} & 198 & 199 & 90 & 91 \\
    & \texttt{SIKEp751} & 299 & 321 & 136 & 146 \\
    \hline
  \end{tabular}

  \bigskip
  
  \begin{tabular}{c || c || c c | c c}
    \hline
    \multirow{6}*{\parbox{4cm}{\centering VHDL implementation\\(FPGA, ms)}}
    & Scheme & \multicolumn{2}{c|}{Xilinx Artix-7} & \multicolumn{2}{c}{Xilinx Kintex UltraScale+} \\
    & & Encaps & Decaps & Encaps & Decaps\\
    \cline{2-6}
    & \texttt{SIKEp434} & 7.01 & 7.42 & 3.09 & 3.28 \\
    & \texttt{SIKEp503} & 8.81 & 9.25 & 3.75 & 3.93 \\
    & \texttt{SIKEp610} & 14.43 & 14.22 & 6.02 & 5.94 \\
    & \texttt{SIKEp751} & 17.37 & 18.39 & 7.43 & 7.87 \\
    \hline
  \end{tabular}
\end{frame}

%%

\begin{frame}{Recent developments}
  \vfill
  \begin{itemize}
  \item SIKE's speed has greatly improved over the last 10 years.
  \item Improvements, especially in software, become harder to come by.
  \item {[BI'21]} applies a \emph{Polynomial Modular Number System
      (PMNS)} representation to finite fields in SIKE:
    \begin{itemize}
    \item Does not appear to be competitive for SIKE's proposed
      parameters;
    \item Suggests new level 5 parameter, \texttt{p736}, which is
      1.17$\times$ faster.
    \item {[TWLLWG'20]} had explored similar ideas previously, but had
      not demonstrated a speed-up.
    \end{itemize}
  \end{itemize}
  
  \footnotetext{[BI'21] C. Bouvier, L. Imbert. ``An Alternative
    Approach for SIDH Arithmetic'', PKC 2021,
    \url{https://ia.cr/2020/1385}.}%
  \footnotetext[2]{[TWLLWG'20] J. Tian, P. Wang, Z. Liu, J. Lin,
    Z. Wang, J. Großschädl ``Faster Software Implementation of the
    SIKE Protocol Based on a New Data Representation'',
    \url{https://ia.cr/2020/660}.}
\end{frame}

%%

\begin{frame}{Recent developments}
  \vfill
  \begin{itemize}
  \item Best attack is the generic van Oorschot-Wiener (vOW) parallel
    collision finding algorithm.
  \item Current parameter selection penalizes SIKE: \textbf{memory is
      assumed to be free}.
  \item {[LWS'21]} uses a budget-based cost model to derive a more
    realistic security estimation:
    \begin{itemize}
    \item Takes into account processing (ASICs) and memory costs
      needed for cryptanalysis,
    \item Suggests new smaller parameters fit NIST levels more closely,
    \end{itemize}
  \end{itemize}

  \begin{center}
    \begin{tabular}{c | c c c c | c c c c }
      \hline
      & \multicolumn{4}{c|}{SIKE round 3} & \multicolumn{4}{c}{[LWS'21]} \\
      NIST level & $\log p$ & public key & Encaps & Decaps & $\log p$ & public key & Encaps & Decaps \\
      && \small bytes &\multicolumn{2}{c|}{\small $10^6$ cycles} && \small bytes & \multicolumn{2}{c}{\small $10^6$ cycles} \\
      \hline\hline
      1 & 434 & 330 & 9.7 & 10.3 & 377 & 288 B & \phantom{0}7.3 & \phantom{0}7.2\\
      3 & 610 & 462 & 27.3 & 27.4 & 546 & 414 B & 19.9 & 19.9\\
      5 & 751 & 564 & 40.7 & 43.9 & 697 & 528 B & 33.3 & 35.0\\
      \hline
    \end{tabular}
  \end{center}
  
  \footnotetext{[LWS'21] P. Longa, W. Wang, J. Szefer: ``The Cost to
    Break SIKE: A Comparative Hardware-Based Analysis with AES and
    SHA-3'', CRYPTO 2021. \url{https://eprint.iacr.org/2020/1457}.}
\end{frame}

%%

\begin{frame}{Summary}
  \begin{columns}[t]
    \begin{column}{0.45\textwidth}
      SIKE advantages:
      \begin{itemize}
      \item Smallest public key size. Key compression has become almost free.
      \item Straightforward parameter selection.
      \item No decryption error, Gaussians, rejection sampling, etc.
      \item Generic attacks are well understood.
      \item Only KEM proposal not based on lattices / codes / LW[ER].
      \end{itemize}
    \end{column}
    \begin{column}{0.45\textwidth}
      SIKE disadvantages:
      \begin{itemize}
      \item Slow.
      \item Non-generic attacks may one day pose a threat (they are currently far from it).
      \end{itemize}

      Work in progress:
      \begin{itemize}
      \item Side channel attacks, cryptanalysis.
      \item Do not miss Craig Costello's talk tomorrow!
      \end{itemize}
    \end{column}
  \end{columns}
\end{frame}

\end{document}
