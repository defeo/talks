\documentclass{beamer}

\usepackage[T1]{fontenc}
\usepackage[utf8]{inputenc}
\usepackage[american]{babel}
\usepackage{amsmath,amsthm}
\usepackage{../share/unicode}
\usepackage{tikz}
\usetikzlibrary{matrix,decorations,decorations.text,calc,arrows,snakes,shapes,positioning}

\usepackage[nosfdefault]{comicneue}
\usepackage{sourcesanspro}
\usepackage[amssymb,amsfonts]{concmath}
\usefonttheme[onlymath]{serif}
\usepackage{ulem}

\mode<presentation>{%
  \usetheme{Boadilla}
}
\beamertemplatenavigationsymbolsempty

\renewcommand{\emph}[1]{{\usebeamercolor[fg]{structure}#1}}

\newenvironment<>{goodblock}[1]{%
  \begin{actionenv}#2%
      \def\insertblocktitle{#1}%
      \par%
      \mode<presentation>{%
        \setbeamercolor{block title}{fg=green!60!black,bg=green!50!white}
       \setbeamercolor{block body}{bg=green!20!white}
     }%
      \usebeamertemplate{block begin}}
    {\par\usebeamertemplate{block end}\end{actionenv}}
\newenvironment<>{mehblock}[1]{%
  \begin{actionenv}#2%
      \def\insertblocktitle{#1}%
      \par%
      \mode<presentation>{%
        \setbeamercolor{block title}{fg=yellow!50!black,bg=yellow!50!white}
       \setbeamercolor{block body}{bg=yellow!20!white}
     }%
      \usebeamertemplate{block begin}}
    {\par\usebeamertemplate{block end}\end{actionenv}}
\newenvironment<>{badblock}[1]{%
  \begin{actionenv}#2%
      \def\insertblocktitle{#1}%
      \par%
      \mode<presentation>{%
        \setbeamercolor{block title}{fg=red!60!black,bg=red!50!white}
       \setbeamercolor{block body}{bg=red!20!white}
     }%
      \usebeamertemplate{block begin}}
    {\par\usebeamertemplate{block end}\end{actionenv}}


%\let\footcite\footnote

\newcommand{\F}{\mathbb{F}}
\newcommand{\bl}[1]{\textcolor{blue}{#1}}
\newcommand{\rd}[1]{\textcolor{red}{#1}}
\newcommand{\gn}[1]{\textcolor{green}{#1}}
\newcommand{\myedge}[3]{
  \draw[#3] (360/\crater*#1 : \diam) to[bend right] (360/\crater*#2 : \diam);
}
\newcommand{\from}{\overset{\$}{\leftarrow}}

\title{How to prove a secret isogeny}
\author[Luca De Feo]{
  Luca De Feo\\[1em]
  {\footnotesize based on joint work with}\\
  J. Burdges, S. Galbraith, S. Masson, C. Petit, A. Sanso}
\date[\url{https://defeo.lu/docet}]{June 4, 2019, CTCrypt, Svetlogorsk\\
  \medskip
  Slides online at \url{https://defeo.lu/docet}}
\institute[UVSQ]{Université Paris Saclay -- UVSQ, France}

\begin{document}

\frame[plain]{\titlepage}

%%

\begin{frame}{Elliptic curves}

  Let \emph{$E \;:\; y^2 = x^3 + ax + b$} be an elliptic curve\dots

  \begin{center}
    \begin{tikzpicture}[domain=-2.4566:4,samples=100]
      \begin{scope}
        \draw plot (\x,{0.5*sqrt(\x*\x*\x-4*\x+5)});
        \draw plot (\x,{-0.5*sqrt(\x*\x*\x-4*\x+5)});
      \end{scope}
      
      \begin{scope}[yscale=1/2]
        \draw[thin,gray,-latex] (0,-7) -- (0,7);
        \draw[thin,gray,-latex] (-3,0) -- (4,0);
        
        \draw (-3,1) -- (4,8/3+3);
        \begin{scope}[every node/.style={draw,circle,inner sep=1pt,fill},cm={1,2/3,0,0,(0,3)}]
          \node at (-2.287980,0) {};
          \node at (-0.535051,0) {};
          \node at (3.267475,0) {};
        \end{scope}
        \begin{scope}[every node/.style={yshift=0.3cm},cm={1,2/3,0,0,(0,3)}]
          \node at (-2.287980,0) {$P$};
          \node at (-0.535051,0) {$Q$};
          \node at (3.267475,0) {$R$};
        \end{scope}
        \draw[dashed] (3.267475,3.267475*2/3+3) -- (3.267475,-3.267475*2/3-3) 
        node[draw,circle,inner sep=1pt,fill] {}
        node[xshift=-0.1cm,anchor=east] {$P+Q$};
      \end{scope}
    \end{tikzpicture}
  \end{center}
\end{frame}

%%

\begin{frame}
  \frametitle{What's \alt<2->{\xout{scalar multiplication} an
      isogeny}{scalar multiplication}?}

  \begin{overlayarea}{\textwidth}{4em}
    \Large
    \[
      \alt<3->{\phi}{[n]}
      \;:\; P \mapsto
      \alt<3->{\phi(P)}{\underbrace{P + P + \cdots + P}_{n\text{ times}}}\]
  \end{overlayarea}
  
  \begin{itemize}
  \item A map \emph{$E\to \alt<4->{\xout{E} E'}{E\phantom{\xout{}}}$},
  \item a \emph{group morphism},
  \item with \emph{finite kernel}\\
    \alt<5->{(\xout{the torsion group $E[n]\simeq(ℤ/nℤ)^2$} any
      finite subgroup \emph{$H\subset E$})}{(the torsion group
      \emph{$E[n]\simeq(ℤ/nℤ)^2$})},
  \item \emph{surjective} (in the algebraic closure),
  \item given by \emph{rational maps} of degree \alt<6->{\xout{$n^2$}
      \emph{$\#H$}}{\emph{$n^2$}}.
  \end{itemize}

  \medskip
  
  \begin{uncoverenv}<7->
    (Separable) isogenies $\Leftrightarrow$ finite subgroups:
    \alert{\[0 \to H \to E \overset{\phi}{\to} E' \to 0\]}
  \end{uncoverenv}
\end{frame}

%% 

\begin{frame}{Isogenies: an example over $\F_{11}$}
  \begin{tikzpicture}[scale=0.4]
    \begin{scope}
      \node[anchor=center] at (0,7) {$E \;:\; y^2 = x^3 + x$};

      \uncover<-1>{
        \draw[thin,gray] (0,-6) -- (0,6);
        \draw[thin,gray] (-6,0) -- (6,0);
      }

      \foreach \x/\y in {0/0,5/3,-4/3,-3/5,-2/1,-1/3} {
        \draw[blue,fill] (\x,\y) circle (0.2) node(E_\x_\y){}
        (\x,-\y) circle (0.2) node(E_\x_-\y){};
      }

      \uncover<2->{\draw[red,fill] (0,0) circle (0.3);}
    \end{scope}

    \draw[black!10!white,thick] (8,-7) -- +(0,14);
    
    \begin{scope}[shift={(16,0)}]
      \node at (0,7) {$E' \;:\; y^2 = x^3 - 4x$};

      \uncover<-1>{
        \draw[thin,gray] (0,-6) -- (0,6);
        \draw[thin,gray] (-6,0) -- (6,0);
      }

      \foreach \x/\y in {0/0,2/0,3/2,4/2,6/4,-2/0,-1/5} {
        \draw[color=blue,fill] (\x,\y) circle (0.2) node(F_\x_\y){}
        (\x,-\y) circle (0.2) node(F_\x_-\y){};
      }
    \end{scope}

    \begin{scope}[color=red,-latex,dashed]
      \begin{uncoverenv}<2->
        \path
        (E_5_3) edge (F_3_2)
        (E_-4_3) edge (F_4_-2)
        (E_-3_5) edge (F_4_2)
        (E_-2_1) edge (F_3_-2)
        (E_-1_3) edge (F_-2_0);
      \end{uncoverenv}
      \begin{uncoverenv}<2->
        \path
        (E_5_-3) edge (F_3_-2)
        (E_-4_-3) edge (F_4_2)
        (E_-3_-5) edge (F_4_-2)
        (E_-2_-1) edge (F_3_2)
        (E_-1_-3) edge (F_-2_0);
      \end{uncoverenv}
    \end{scope}
  \end{tikzpicture}
  
  \begin{columns}
    \begin{column}{0.5\textwidth}
      \[\phi(x,y) = \left(\frac{x^2 + 1}{x},\quad y\frac{x^2-1}{x^2}\right)\]
    \end{column}
    \begin{column}{0.5\textwidth}
      \begin{itemize}
      \item<2-> Kernel generator in \alert{red}.
      \item<2-> This is a degree $2$ map.
      \item<2-> Analogous to $x\mapsto x^2$ in $\F_q^*$.
      \end{itemize}
    \end{column}
  \end{columns}
\end{frame}

%%

\begin{frame}{Up to \emph{isomorphism}}
  \begin{center}
    \begin{tikzpicture}[domain=-2.4566:4,samples=100]
      \newcount\zoomout
      \animate<15-21>
      \animatevalue<15-20>{\zoomout}{0}{10}
      \begin{uncoverenv}<-20>
        \begin{scope}[scale=1-0.09*\zoomout]
          \begin{scope}
            \draw[thin,gray,-latex] (0,-4) -- (0,4);
            \draw[thin,gray,-latex] (-4.2,0) -- (7,0);
          \end{scope}
          
          \newcount\xstretch
          \newcount\ystretch
          \newcount\slant
          \animate<1-13>
          \animatevalue<1-5>{\xstretch}{0}{4}
          \animatevalue<5-9>{\ystretch}{0}{4}
          \animatevalue<9-13>{\slant}{0}{4}      
          \begin{scope}[yscale=0.55-0.05*\the\ystretch,xscale=1+0.1*\the\xstretch,xslant=0.02*\slant]
            \draw plot (\x,{sqrt(\x*\x*\x-4*\x+5)});
            \draw plot (\x,{-sqrt(\x*\x*\x-4*\x+5)});

            \begin{uncoverenv}<-18>
              \draw (-3,1) -- (4,8/3+3);
              \begin{scope}[every node/.style={draw,circle,inner sep=1pt,fill},cm={1,2/3,0,0,(0,3)}]
                \node at (-2.287980,0) {};
                \node at (-0.535051,0) {};
                \node at (3.267475,0) {};
              \end{scope}
              \begin{scope}[every node/.style={yshift=0.3cm},cm={1,2/3,0,0,(0,3)}]
                \node at (-2.287980,0) {$P$};
                \node at (-0.535051,0) {$Q$};
                \node at (3.267475,0) {$R$};
              \end{scope}
              \draw[dashed] (3.267475,3.267475*2/3+3) -- (3.267475,-3.267475*2/3-3) 
              node[draw,circle,inner sep=1pt,fill] {}
              node[xshift=-0.1cm,anchor=east] {$P+Q$};
            \end{uncoverenv}
          \end{scope}

          \begin{uncoverenv}<14>
            \node[anchor=west] at (-4,-3) {\Large\alert{$y^2=x^3+ax+b \quad\longrightarrow\quad j\equiv 1728\frac{4a^3}{4a^3+27b^2}$}};
          \end{uncoverenv}
        \end{scope}
      \end{uncoverenv}
      
      \begin{uncoverenv}<21->
        \draw[fill] (0,0) circle (2pt) node[anchor=north] {$j=1728$};
        \uncover<22->{
          \draw (0,0) edge[bend left,->] node[auto] {$\phi$} (7,0);
          \draw[fill] (7.1,0) circle (2pt) node[anchor=north] {$j=287496$};
        }
      \end{uncoverenv}
    \end{tikzpicture}
  \end{center}  
\end{frame}

%%

\begin{frame}{Isogeny graphs}
  
  \vspace{-2mm}

  \begin{columns}
    \begin{column}{0.65\textwidth}
      We look at the graph of elliptic curves with isogenies \emph{up
        to isomorphism}.  We say two \alert{isogenies} $\phi,\phi'$
      are \alert{isomorphic} if:
    \end{column}
    \begin{column}{0.3\textwidth}
      \begin{center}
        \begin{tikzpicture}[node distance=4em]
          \node(E){$E$}; 
          \node(E1)[right of=E]{$E'$};
          \node(E2)[below of=E1]{$E'$};
          \scriptsize
          \path[->] (E) edge node[auto]{$\phi$} (E1);
          \path[->] (E) edge node[auto,swap]{$\phi'$} (E2);
          \path[<->] (E1) edge node[rotate=270] {\large$\widetilde{}$} (E2);
        \end{tikzpicture}
      \end{center}
    \end{column}
  \end{columns}

  \emph{Example:} Finite field, ordinary case, graph of isogenies of degree $3$.

  \begin{center}
    \begin{tikzpicture}[]
      \begin{scope}
        \def\crater{7}
        \foreach \i in {1,...,\crater} {
          \draw[fill] (360/\crater*\i:1cm) circle (5pt);
          \draw (360/\crater*\i : 1cm) -- (360/\crater*\i+360/\crater : 1cm);
          \foreach \j in {-1,1} {
            \draw[fill] (360/\crater*\i : 1cm) -- (360/\crater*\i + \j*360/\crater/4 : 2cm) circle (3pt);
            \foreach \k in {-1,0,1} {
              \draw[fill] (360/\crater*\i + \j*360/\crater/4 : 2cm) --
              (360/\crater*\i + + \j*360/\crater/4 + \k*360/\crater/6 : 2.5cm) circle (1pt);
            }
          }
        }
      \end{scope}
    \end{tikzpicture}
  \end{center}
\end{frame}

%%

\begin{frame}{The graph of isogenies of \alert{prime} degree \alert{$\ell\ne p$}}
  \medskip
  
  All graphs are \emph{undirected} (dual isogeny theorem).

  \bigskip
  
  \begin{tabular}{p{0.17\textwidth} p{0.8\textwidth}}
    \raggedright
    \emph{Ordinary case (isogeny volcanoes)}
    & \vspace{-1em}
      \begin{itemize}
      \item Nodes can have degree \emph{$0,1,2$} or \emph{$\ell+1$}.
        \begin{itemize}
        \item  For $\sim 50\%$ of the primes $\ell$, graphs are just isolated
          points;
        \item For other $\sim 50\%$, graphs are $2$-regular;
        \item other cases only happen for finitely many $\ell$'s.
        \end{itemize}
      \end{itemize}
    \\
    \raggedright
    \emph{Supersingular case ($\F_p$)}
    & \vspace{-1em}
      \begin{itemize}
      \item If $\ell=2$ nodes have degree $1$, $2$ or $3$;
      \item For $\sim 50\%$ of $\ell$, graphs are isolated points;
      \item For other $\sim 50\%$, graphs are $2$-regular;
      \end{itemize}
    \\
    \raggedright
    \emph{Supersingular case ($\F_{p^2}$)}
    & \vspace{-1em}
      \begin{itemize}
      \item The graph is \emph{$\ell+1$-regular}.
      \item There is a \alert{unique (finite) connected component} made
        of all supersingular curves with the same number of points.
      \end{itemize}
  \end{tabular}
\end{frame}

%%

\begin{frame}{Isogeny graphs taxonomy}
  \begin{columns}[t]
    \begin{column}{0.5\textwidth}
      \centering
      \textbf{Complex Multiplication (CM) graphs}

      \begin{tikzpicture}[scale=0.5]
        \def\crater{12}
        \def\jumpa{-8}
        \def\jumpb{9}
        \def\diam{2.5cm}

        \foreach \i in {1,...,\crater} {
          \draw[blue] (360/\crater*\i : \diam) to[bend right] (360/\crater*\i+360/\crater : \diam);
          \draw[red] (360/\crater*\i : \diam) to[bend right] (360/\crater*\i+\jumpa*360/\crater : \diam);
          \draw[green] (360/\crater*\i : \diam) to[bend right=50] (360/\crater*\i+\jumpb*360/\crater : \diam);
        }
        \foreach \i in {1,...,\crater} {
          \draw[fill] (360/\crater*\i: \diam) circle (2pt);
        }
      \end{tikzpicture}

      \begin{itemize}
      \item Ordinary / Supersingular (\emph{$\F_p$})
      \item Superposition of \emph{isogeny cycles}\\
        (one color per degree)
      \item Isomorphic to \emph{Cayley graph} of a \emph{quadratic
          class group}
      \item Large automorphism group
      \item Typical size \emph{$O(\sqrt{p})$}
      \item Used in: \textbf{CSIDH}
      \end{itemize}
    \end{column}
    \begin{column}{0.5\textwidth}
      \centering
      \textbf{\textit{Full} supersingular graphs}

      \begin{tikzpicture}[scale=0.62]
        \begin{scope}[every node/.style={fill,black,circle,inner sep=1pt}]
          \node at (0,0)  (1){};
          \node at (0,4) (20){};
          \node at (2,1)  (16z){};
          \node at (-2,1)  (81z){};
          \node at (-1,2) (77z){};
          \node at (1,2)  (20z){};
          \node at (-2,3)  (85z){};
          \node at (2,3)  (12z){};
        \end{scope}

        \begin{scope}[red]
          \path (1) edge (85z) edge (81z) edge (12z) edge (16z);
          \path (20) edge (85z) edge (77z) edge (20z) edge (12z);
          \path (81z) edge (85z) edge (77z) edge (16z);
          \path (85z) edge (12z);
          \path (12z) edge (16z);
          \path (16z) edge (20z);
          \path (20z) edge[bend right=10] (77z) edge[bend left=10] (77z);
        \end{scope}
      \end{tikzpicture}

      \begin{itemize}
      \item Supersingular (\emph{$\F_{p^2}$})
      \item One isogeny degree
      \item \emph{$(ℓ+1)$}-regular
      \item Tiny automorphism group
      \item Size \emph{$≈p/12$}
      \item Used in: \textbf{SIDH}
      \end{itemize}
    \end{column}
  \end{columns}
\end{frame}

%%

\begin{frame}{Post-quantum isogeny primitives}
  \begin{block}{\textbf{SIDH} (Jao, De Feo 2011)}
    \begin{itemize}
    \item Pronounce \emph{S--I--D--H};
    \item Based on isogeny walks in the \emph{full supersingular
        graph} over $\F_{p^2}$;
    \item Basis for the NIST KEM candidate \emph{SIKE};
    \item Better asymptotic quantum security;
    \item Short keys, slow.
    \end{itemize}
  \end{block}
  
  \begin{block}{\textbf{CSIDH} (Couveignes 1996; Rostovtsev, Stolbunov 2006;
      Castryck, Lange, Martindale, Panny, Renes 2018)}
    \begin{itemize}
    \item Pronounce \emph{Sea--Side};
    \item Based on isogeny walks in the \emph{supersingular CM graph}
      over $\F_p$;
    \item Straightforward generalization of Diffie--Hellman;
    \item More ``natural'' security assumption;
    \item Shorter keys, slower.
    \end{itemize}
  \end{block}
\end{frame}

%%

\begin{frame}{CSIDH key exchange}
  \begin{columns}
    \begin{column}{0.47\textwidth}
      \begin{itemize}
      \item A set of \emph{supersingular elliptic} curves over $\F_p$;
      \item<2-> A \emph{group action} by a commutative class group
        \emph{$G$};
      \item<3-> \emph{Small degree generators} of $G$:\\
        \small
        \bl{degree 2}, \rd{degree 3}, \gn{degree 5}, \ldots
      \end{itemize}

      \begin{uncoverenv}<4->
        \emph{Key exchange:}
        \begin{itemize}
        \item<4-> Alice picks secret $a = \bl{g_2^{a_2}}\rd{g_3^{a_3}}\gn{g_5^{a_5}}\cdots$,
        \item<5-> Bob picks secret $b = \bl{g_2^{b_2}}\rd{g_3^{b_3}}\gn{g_5^{b_5}}\cdots$,
        \item<6-> They exchange \emph{$E_A=a*E_1$} and \emph{$E_B=b*E_1$},
        \item<7-> Shared secret is \emph{$E_{AB}=(ab)*E_1 = a*E_B = b*E_A$}.
        \end{itemize}
      \end{uncoverenv}
    \end{column}
    \begin{column}{0.53\textwidth}
      \begin{tikzpicture}
        \begin{scope}
          \def\crater{12}
          \def\jumpa{-8}
          \def\jumpb{9}
          \def\diam{2.5cm}

          \uncover<2>{
            \draw[shorten >=3,->] (360/\crater : \diam) to[bend right] node[auto,swap]{$g$} (180 : \diam);
            \draw[shorten >=3,->] (180 : \diam) to[bend right] node[auto,swap]{$g^{-1}$} (360/\crater : \diam);
          }
          
          \foreach \i in {1,...,\crater} {
            \uncover<3>{\draw[blue] (360/\crater*\i : \diam) to[bend right] (360/\crater*\i+360/\crater : \diam);}
            \uncover<3>{\draw[red] (360/\crater*\i : \diam) to[bend right] (360/\crater*\i+\jumpa*360/\crater : \diam);}
            \uncover<3>{\draw[green] (360/\crater*\i : \diam) to[bend right=50] (360/\crater*\i+\jumpb*360/\crater : \diam);}
          }
          \foreach \i in {1,...,\crater} {
            \draw[fill] (360/\crater*\i: \diam) circle (2pt);
            \uncover<-3>{
              \draw (360/\crater*\i: \diam + 0.4cm) node{$E_{\i}$};
            }
          }

          \uncover<4>{
            % Alice 1
            \myedge{0}{1}{blue}\myedge{1}{2}{blue}\myedge{2}{6}{red}\myedge{6}{3}{green}
          }
          \uncover<5>{
            % Bob 1
            \begin{scope}[dashed,thick]
              \myedge{0}{4}{red}\myedge{4}{8}{red}\myedge{8}{5}{green}\myedge{5}{6}{blue}
            \end{scope}
          }
          \uncover<7->{
            % Alice 2
            \myedge{6}{7}{blue}\myedge{7}{8}{blue}\myedge{8}{0}{red}\myedge{0}{9}{green}
          }
          \uncover<7->{
            % Bob 2
            \begin{scope}[dashed,thick]
              \myedge{3}{7}{red}\myedge{7}{11}{red}\myedge{11}{8}{green}\myedge{8}{9}{blue}
            \end{scope}
          }
          
          \uncover<4->{\draw (0 : \diam + 0.4cm) node {$E$};}
          \uncover<4->{\draw (360/\crater*3 : \diam + 0.4cm) node {$E_A$};}
          \uncover<5->{\draw (360/\crater*6 : \diam + 0.4cm) node {$E_B$};}
          \uncover<7->{\draw (360/\crater*9 : \diam + 0.4cm) node {$E_{AB}$};}
        \end{scope}
      \end{tikzpicture}
    \end{column}
  \end{columns}
\end{frame}

%% 

\begin{frame}{SIDH key exchange}
  
  \begin{description}
  \item[Good news:] there is no action of a commutative class group.
  \item[Bad news:] there is no action of a commutative class group.
  \item[Idea:] Let \bl{Alice} and \rd{Bob} walk in two
    \emph{different isogeny graphs} on the \emph{same vertex set}.
  \end{description}

  \begin{columns}
    \begin{column}{0.7\textwidth}
      \centering
      \begin{tikzpicture}[scale=1.4]
        \begin{scope}[every node/.style={fill,black,circle,inner sep=2pt}]
          \node at (0,0)  (1){};
          \node at (0,4) (20){};
          \node at (2,1)  (16z){};
          \node at (-2,1)  (81z){};
          \node at (-1,2) (77z){};
          \node at (1,2)  (20z){};
          \node at (-2,3)  (85z){};
          \node at (2,3)  (12z){};
        \end{scope}

        \begin{uncoverenv}<1,3>
          \begin{scope}[blue,every loop/.style={looseness=50}]
            \path (1) edge (20) edge (16z) edge (81z);
            \path (20) edge[loop left] (20) edge[loop right] (20);
            \path (16z) edge (81z) edge (77z);
            \path (81z) edge (20z);
            \path (77z) edge (20z) edge (85z);
            \path (20z) edge (12z);
            \path (12z) edge[bend right=10] (85z) edge[bend left=10] (85z);
          \end{scope}
        \end{uncoverenv}
        
        \begin{uncoverenv}<2->
          \begin{scope}[red]
            \path (1) edge (85z) edge (81z) edge (12z) edge (16z);
            \path (20) edge (85z) edge (77z) edge (20z) edge (12z);
            \path (81z) edge (85z) edge (77z) edge (16z);
            \path (85z) edge (12z);
            \path (12z) edge (16z);
            \path (16z) edge (20z);
            \path (20z) edge[bend right=10] (77z) edge[bend left=10] (77z);
          \end{scope}
        \end{uncoverenv}
      \end{tikzpicture}
    \end{column}
    \begin{column}{0.3\textwidth}
      \small
      \emph{Figure:} \bl{$2$}- and \rd{$3$}-isogeny graphs on $\F_{97^2}$.
    \end{column}
  \end{columns}
\end{frame}

%%

\begin{frame}{SIDH key exchange}

  \begin{itemize}
  \item Fix small primes \bl{$\ell_A$}, \rd{$\ell_B$};
  \item \emph{No canonical labeling} of the \bl{$\ell_A$}- and
    \rd{$\ell_B$}-isogeny graphs; \emph{however\dots}
  \end{itemize}

  \begin{center}
    \bf
    Walk of length \bl{$e_A$}\\
    $=$\\
    Isogeny of degree \bl{$\ell_A^{e_A}$}\\
    $=$\\
    Kernel \bl{$\langle P\rangle\subset E[\ell_A^{e_A}]$}
  \end{center}
  
  \begin{center}
    \begin{tikzpicture}
      \begin{scope}
        \draw (0,1.2) node[anchor=east,blue] {$\ker\phi=〈P〉\subset E[\ell_A^{e_A}]$};
        \draw (0,0.4) node[anchor=east,red] {$\ker\psi=〈Q〉\subset E[\ell_B^{e_B}]$};
        \draw (0,-0.4) node[anchor=east,blue] {$\ker\phi' = 〈\rd{\psi(P)}〉$};
        \draw (0,-1.2) node[anchor=east,red] {$\ker\psi' = 〈\bl{\phi(Q)}〉$};
      \end{scope}
      \begin{scope}[xshift=4.5cm,coils/.style={-angle 90,decorate,decoration={coil,aspect=0,amplitude=1pt}}]
        \large
        \node[matrix of nodes, ampersand replacement=\&, column sep=3cm, row sep=1.5cm] (diagram) {
          |(E)| $E$ \& |(Es)| $E/〈\bl{P〉}$ \\
          |(Ep)| {$E/〈\rd{Q}〉$} \& |(Eps)| {$E/〈\bl{P,\rd{Q}}〉$}\\
        };
        \path[->,blue] (E) edge[coils] node[auto] {$\phi$} (Es);
        \path[->,blue] (Ep) edge[coils] node[auto,swap] {$\phi'$} (Eps);
        \path[->,red] (E) edge[coils] node[auto,swap] {$\psi$} (Ep);
        \path[->,red] (Es) edge[coils] node[auto] {$\psi'$} (Eps);
      \end{scope}
    \end{tikzpicture}
  \end{center}
\end{frame}

%%

\begin{frame}{Security assumptions}
  \begin{goodblock}{Isogeny walk problem}
    \begin{description}
    \item[Input] Two isogenous elliptic curves \emph{$E,E'$} over
      \emph{$\F_q$}.
    \item[Output] A path \emph{$E\to E'$} in an isogeny graph.
    \end{description}
  \end{goodblock}

  \begin{mehblock}{SIDH problem (1)}
    \begin{description}
    \item[Input]
      Elliptic curves \emph{$E,E'$} over \emph{$\F_q$},
      isogenous of degree \bl{$\ell_A^{e_A}$}.
    \item[Output] The unique path \emph{$E\to E'$} of length
      \bl{$e_A$} in the \bl{$\ell_A$}-isogeny graph.
    \end{description}    
  \end{mehblock}
  
  \begin{badblock}{SIDH problem (2)}
    \begin{description}
    \item[Input]
      \begin{itemize}
      \item Elliptic curves \emph{$E,E'$} over \emph{$\F_q$},
        isogenous of degree \bl{$\ell_A^{e_A}$};
      \item The action of the isogeny on $E[\rd{\ell_B^{e_B}}]$.
      \end{itemize}
    \item[Output] The unique path \emph{$E\to E'$} of length
      \bl{$e_A$} in the \bl{$\ell_A$}-isogeny graph.
    \end{description}    
  \end{badblock}
\end{frame}

%%

\begin{frame}{Why prove a secret isogeny?}
  \begin{description}
  \item[Public:] Curves \emph{$E, E'$}
  \item[Secret:] An isogeny walk \emph{$E\to E'$}
  \end{description}

  \begin{block}{Why?}
    \begin{itemize}
    \item For interactive identification;
    \item For signing messages;
    \item For validating public keys (esp. CSIDH);
    \item More\dots
    \end{itemize}
  \end{block}

  \begin{block}{Some properties}
    \begin{tabular}{l c c c c}
      & \multicolumn{2}{c}{\small Zero knowledge}\\
      & \small Statistical & \small Computational & \small Quantum resistance & \small Succinctness\\
      \hline
      CSIDH & \checkmark & & \checkmark & \\
      SIDH & & \checkmark & \checkmark &\\
      Pairings & & & & \checkmark\\
    \end{tabular}
  \end{block}
\end{frame}

%%

\begin{frame}{A $\Sigma$-protocol from Diffie--Hellman\footnote{Kids, do
      not try this at home! Use Schnorr!}}
  \begin{columns}
    \begin{column}{0.55\textwidth}
      \begin{itemize}
      \item<1-> A key pair \emph{$(s, g^s)$};
      \item<2-> Commit to a \emph{random element $g^r$};
      \item<3-> Challenge with bit \emph{$b\in\{0,1\}$};
      \item<4-> Respond with \emph{$c = \alert<6->{r - b\cdot s\mod\#G}$};
      \item<5-> Verify that \emph{$g^c(g^s)^b = g^r$}.
      \end{itemize}

      \begin{block}{Zero-knowledge}<6->
        \centering
        Does not leak because:\\
        \alert{$c$ is uniformly distributed} and independent from $s$.
      \end{block}

      \begin{uncoverenv}<7->
        Unlike Schnorr, compatible with\\
        \emph{group action Diffie--Hellman}.
      \end{uncoverenv}
    \end{column}  
    \begin{column}{0.40\textwidth}
      \centering
      \begin{tikzpicture}
        \node (g) at (0,0) {\alt<-6>{$g$}{$E_1$}};
        \node (gs) at (3,0) {\alt<-6>{$g^s$}{$E_s$}};
        \path[->] (g) edge node[auto]{\alt<-6>{$s$}{$g^s$}} (gs);
        \uncover<2->{
          \node (gr) at (1.5,-3) {\alt<-6>{$g^r$}{$E_r$}};
          \path[->] (g) edge node[auto,swap]{\alt<-6>{$r$}{$g^r$}} (gr);
        }
        \uncover<4->{
          \path[dashed,->] (gs) edge node[auto]{\alt<-6>{$r-s$}{$g^{r-s}$}} (gr);
        }
      \end{tikzpicture}
    \end{column}  
  \end{columns}
\end{frame}

%%

\begin{frame}{The trouble with groups of unknown structure}
  \begin{columns}
    \begin{column}{0.55\textwidth}
      In CSIDH secrets look like:
      $g^{\vec{s}} = \bl{g_2^{s_2}}\rd{g_3^{s_3}}\gn{g_5^{s_5}}\cdots$
      \begin{itemize}
      \item the elements $g_i$ are fixed,
      \item the secret is the exponent vector
        \emph{$\vec{s}=(s_2,s_3,\dots)\in[-B,B]^n$},
      \item secrets must be sampled in a box \emph{$[-B,B]^n$} ``large
        enough''\dots
      \end{itemize}

      \begin{block}{The leakage}<2->%
        With \emph{$\vec{s},\vec{r}\from[-B,B]^n$}, the distribution of
        \emph{$\vec{r}-\vec{s}$} \alert{depends on the long term secret $\vec{s}$!}
      \end{block}
    \end{column}
    \begin{column}{0.42\textwidth}
      \centering
      \begin{tikzpicture}[yscale=0.3,xscale=0.2]
        \draw (-2,2) node{$+B$} (-2,-2) node{$-B$};
        \draw (0,0) -- (18,0);
        \draw[dotted] (0,2) -- (18,2) (0,-2) -- (18,-2);

        \draw (8,-4) node {\Large$-$};
        
        \draw (-2,-6) node{$+B$} (-2,-10) node{$-B$};
        \draw (0,-8) -- (18,-8);
        \draw[dotted] (0,-6) -- (18,-6) (0,-10) -- (18,-10);

        \uncover<2->{
          \draw (8,-12) node {\Large$=$};
        
          \draw (-2,-15) node{$+B$} (-2,-19) node{$-B$};
          \draw (0,-17) -- (18,-17);
          \draw[dotted] (0,-15) -- (18,-15) (0,-19) -- (18,-19);
        }
      
        \foreach \i in {0,...,18} {
          \pgfmathparse{round(2.2*sin(140*\i))}
          \let\s\pgfmathresult
          \pgfmathparse{round(2.2*cos(125*\i))}
          \let\r\pgfmathresult
          \draw[very thick] (\i,0) -- (\i,\r);
          \draw[very thick] (\i,-8) -- (\i,-8+\s);
          \uncover<2->{
            \draw[very thick] (\i,-17) -- (\i,-17+\r-\s);
          }
        }
      \end{tikzpicture}
    \end{column}
  \end{columns}
\end{frame}

%%

\begin{frame}{The two fixes}
  \begin{block}{Compute the group structure and stop whining}
    \emph{CSI-FiSh}: Beullens, Kleinjung and Vercauteren 2019
    (\href{https://ia.cr/2019/498}{eprint:2019/498})
    \begin{itemize}
    \item Already suggested by Couveignes (1996) and Stolbunov (2006).
    \item Computationally intensive (\emph{subexponential parameter generation}).
    \item Decent parameters, e.g.: \emph{263 bytes, 390 ms, @NIST-1.} 
    \item[--] Technically not post-quantum.
   \end{itemize}
  \end{block}

  \begin{block}{Do like the lattice people}
    \emph{SeaSign}: D. and Galbraith 2019
    \begin{itemize}
    \item Use \emph{Fiat--Shamir with aborts} (Lyubashevsky 2009).
    \item[--] Huge increase in signature size and time.
    \item Compromise signature size/time with public key size (still
      slow).
    \end{itemize}
  \end{block}
\end{frame}

%%

\begin{frame}{Rejection sampling}
  \begin{columns}
    \begin{column}{0.55\textwidth}
      \begin{itemize}
      \item Sample \emph{long term secret $\vec{s}$} in the usual box \emph{$[-B,B]^n$},
      \item Sample \emph{ephemeral $\vec{r}$} in a larger box
        \emph{$[-(\delta+1)B,(\delta+1)B]^n$},
      \item Throw away \emph{$\vec{r}-\vec{s}$} if it is out of the box
        \emph{$[-\delta B,\delta B]^n$}.
      \end{itemize}

      \begin{block}{Zero-knowledge}
        \emph{Theorem:} $\vec{r}-\vec{s}$ is uniformly distributed in
        \emph{$[-\delta B,\delta B]^n$}.
      \end{block}

      \emph{Problem:} set $\delta$ so that rejection probability is
      low.
    \end{column}
    \begin{column}{0.42\textwidth}
      \centering
      \begin{tikzpicture}[yscale=0.3,xscale=0.2]
        \draw (-3,3) node{$+(\delta+1)B$} (-3,-3) node{$-(\delta+1)B$};
        \draw (0,0) -- (18,0);
        \draw[dotted] (0,3) -- (18,3) (0,-3) -- (18,-3);

        \draw (8,-5) node {\Large$-$};
        
        \draw (-2,-7.5) node{$+B$} (-2,-8.5) node{$-B$};
        \draw (0,-8) -- (18,-8);
        \draw[dotted] (0,-7.5) -- (18,-7.5) (0,-8.5) -- (18,-8.5);

        \draw (8,-12) node {\Large$=$};
        
        \draw (-2,-14.5) node{$+\delta B$} (-2,-19.5) node{$-\delta B$};
        \draw (0,-17) -- (18,-17);
        \draw[dotted] (0,-14.5) -- (18,-14.5) (0,-19.5) -- (18,-19.5);
      
        \foreach \i in {0,...,18} {
          \pgfmathparse{round(60*random()-30)/10}
          \let\r\pgfmathresult
          \pgfmathparse{round(2*random()-1)/2}
          \let\s\pgfmathresult
          \draw[very thick] (\i,0) -- (\i,\r);
          \draw[very thick] (\i,-8) -- (\i,-8+\s);
          \draw[very thick] (\i,-17) -- (\i,-17+\r-\s);
        }
      \end{tikzpicture}
    \end{column}
  \end{columns}
\end{frame}

%%

\begin{frame}{Performance}
  \begin{itemize}
  \item For $\lambda$-bit security, protocol must be \emph{repeated
      $\lambda$ times} in parallel;
  \item \emph{$\delta = \lambda n$} for a rejection probability
    \emph{$\le 1/3$};
  \item Signature size \emph{$\approx \lambda n$ coefficients $\in[-\delta B,\delta B]$};
  \item Sign/verify time linear in \emph{$\lVert \vec{r}-\vec{s}\rVert_\infty\approx \lambda^2n^2B$}.
  \end{itemize}

  \begin{block}{CSIDH instantiation (NIST-1)}
    \begin{description}
    \item[Parameters:] $\lambda=128, n=74, B=5$;
    \item[PK size:] 64 B
    \item[SK size:] 32 B
    \item[Signature:] 20 KiB
    \item[Verify time:] \alert{10 hours}
    \item[Sign time:] $3\times$ verify
    \end{description}
  \end{block}

\end{frame}

%%

\begin{frame}{Key/signature size compromise}
  \begin{columns}
    \begin{column}{0.5\textwidth}
      \begin{block}{}
        \begin{itemize}
        \item One key pair \emph{$(\vec{s},E_s)$};
        \item Challenge \emph{$b\in\{0,1\}$};
        \item Reveal \emph{$\vec{r}-b\vec{s}$};
        \item[$\to$] \emph{$\lambda$} iterations;
        \item[$\to$]<3-> Sample \emph{$r\from[-\lambda nB,\lambda nB]$}.
        \end{itemize}
      \end{block}

      \begin{block}{Compromise: \alert{$\mathbf{t}$}-bit challenges}<2->
        \begin{itemize}
        \item \alert{$\mathbf{2^t}$} key pairs \emph{$(\vec{s_i},E_i)$};
        \item Challenge \emph{$b\in\{0,2^t\}$};
        \item Reveal \emph{$\vec{r}-\vec{s_b}$};
        \item[$\to$] \emph{$\lambda\alert{\mathbf{/t}}$} iterations;
        \item[$\to$]<4-> Sample
          \emph{$r\from[-\lambda nB\alert{\mathbf{/t}},\lambda
            nB\alert{\mathbf{/t}}]$}.
        \end{itemize}
      \end{block}
    \end{column}  
    \begin{column}{0.47\textwidth}
      \centering
      \begin{tikzpicture}
        \node (E) at (0:0) {$E_1$};
        \uncover<1>{
          \node (Es) at (45:3) {$E_s$};
          \path[->] (E) edge node[above]{\small$\vec{s}$} (Es);
        }
        \uncover<2->{
          \foreach \i in {1,...,4} {
            \node (E\i) at (-30*\i + 75 : 3) {$E_\i$};
            \path[->] (E) edge node[above]{\small$\vec{s_\i}$} (E\i);
          }
        }
        \node (Er) at (-50:5) {$E_r$};
        \path[dashed,->] (E) edge[bend right] node[left]{\small$\vec{r}$} (Er);
        \uncover<2->{
          \path[dashed,->] (E2) edge[bend left] node[right]{\small$\vec{r}-\vec{s_2}$} (Er);
        }
      \end{tikzpicture}
    \end{column}  
  \end{columns}
\end{frame}

%%

\begin{frame}{Public key compression}
  \begin{center}
    \begin{tikzpicture}[xscale=0.9]
      \node (E) at (0,0) {$E_1$};
      \foreach \i in {1,...,4} {
        \node (E\i) at (3,2.5-\i) {$E_\i$};
        \path[->] (E) edge (E\i);
        \uncover<2->{
          \node (H0\i) at (5,2.5-\i) {$H(E_\i)$};
          \path[->] (E\i) edge (H0\i);
        }
      }
      \uncover<2->{
        \foreach \l in {1,2} {
          \foreach \i in {1,...,2^(2-\l)} {
            \pgfmathparse{int(\l-1)}
            \let\pl\pgfmathresult
            \pgfmathparse{int(2*\i-1)}
            \let\ca\pgfmathresult
            \pgfmathparse{int(2*\i)}
            \let\cb\pgfmathresult
            \node (H\l\i) at (5 + 3*\l, 2 + 0.5*2^\l - \i*2^\l) {$H(\bullet,\bullet)$};
            \path[->]
            (H\pl\ca) edge (H\l\i)
            (H\pl\cb) edge (H\l\i);
          }
        }
        \node at (12.3,0) {$=pk$};
        \uncover<3->{
          \path[->,blue,very thick]
          (E) edge (E2)
          (H01) edge (H11)
          (H12) edge (H21);
        }
      }
    \end{tikzpicture}
  \end{center}
  \medskip
  \begin{itemize}
  \item<2-> Construct Merkle tree on top of public keys, \emph{root is the new public key};
  \item<3-> Include Merkle proof in the signature.
  \end{itemize}
\end{frame}

%%

\begin{frame}{SeaSign Performance (NIST-1)}
  \small
  \begin{tabular}{l  c  c  c }
    & \parbox{7em}{\bf\centering $t=1$ bit\\ challenges}
    & \parbox{7em}{\bf\centering $t=16$ bits\\ challenges}
    & \bf PK compression \\
    \hline
    Sig size
    & \alert{20 KiB} & \gn{978 B} & 3136 B\\
    PK size
    & 64 B & \alert{4 MiB} & 32 B\\
    SK size
    & 32 B & 16 B & \alert{1 MiB} \\
    Est. keygen time
    & 30 ms & \alert{30 mins} & \alert{30 mins} \\
    Est. sign time
    & \alert{30 hours} & 6 mins & 6 mins \\
    Est. verify time
    & \alert{10 hours} & 2 mins & 2 mins \\
    \hline
    Asymptotic sig size
    & $O(\lambda^2\log(\lambda))$ & $O(\lambda t\log(\lambda))$ & $O(\lambda^2 t)$\pause\\[2em]
    \multicolumn{4}{c}{\bf Recent speed/size compromises by Decru, Panny and Vercauteren}\\
    \hline
    {Sig size}
    & 36 KiB & 2 KiB & ---\\
    {Est. sign time}
    & 30 mins & 80 s & ---\\
    {Est. verify time}
    & 20 mins & 20 s & ---
  \end{tabular}
\end{frame}

%%

\begin{frame}{A $Σ$-protocol for SIDH}
  \begin{columns}
    \begin{column}{0.6\textwidth}
      \begin{tikzpicture}
        \large
        \node[matrix of nodes, ampersand replacement=\&, column sep=3cm, row sep=1.5cm] (diagram) {
          |(E)| $E$ \& |(Es)| $E/〈\bl{S}〉$ \\
          |(Ep)| {\uncover<2->{$E/〈\rd{P}〉$}} \& |(Eps)| {\uncover<2->{$E/〈\rd{P,\bl{S}}〉$}}\\
        };
        \path[blue] (E) edge node[auto] {$\phi$} (Es);
        \uncover<2->{\path[blue] (Ep) edge node[auto,swap] {\alt<5>{$\phi'$}{\phantom{$\phi'$}?}} (Eps);}
        \uncover<2->{\path[red] (E) edge node[auto,swap] {\alt<3,6>{$\psi$}{?}} (Ep);}
        \uncover<2->{\path[red] (Eps) edge node[auto,swap] {\alt<4,6>{$\psi'$}{?}} (Es);}
      \end{tikzpicture}
    \end{column}
    \begin{column}{0.35\textwidth}
      \begin{mehblock}{$\frac{1}{3}$-soundness}
        Secret \bl{$\phi$} of degree \bl{$\ell_A^{e_A}$}.
      \end{mehblock}
    \end{column}
  \end{columns}

  \begin{enumerate}
  \item<2-> Choose a random point \rd{$P\in E[\ell_B^{e_B}]$}, compute the diagram;
  \item<2-> Publish the curves $E/〈\rd{P}〉$ and $E/〈\rd{P,\bl{S}}〉$;
  \item<3-> The verifier challenges to reveal \emph{one out of the 3}
    sides
    \begin{itemize}
    \item<3-> Isogenies \rd{$\psi,\psi'$} (degree \rd{$\ell_B^{e_B}$})
      unrelated to secret;
    \item<5-> Isogeny \bl{$\phi'$} conjectured to not reveal useful
      information on \bl{$\phi$}.
    \end{itemize}
  \end{enumerate}

  \begin{badblock}{Improving to $\frac{1}{2}$-soundness}<6->
    \begin{itemize}
    \item Reveal \rd{$\psi,\psi'$} simultaneously;
    \item Reveals action of \bl{$\phi$} on $E[\rd{\ell_B^{e_B}}]\quad⇒\quad$
      Stronger security assumption.
    \end{itemize}
  \end{badblock}
\end{frame}

%%

\begin{frame}{SIDH signature performance (NIST-1)}
  According to Yoo, Azarderakhsh, Jalali, Jao and Vladimir Soukharev
  2017:
  \begin{description}
  \item[Size:] $\approx 100 KB$,
  \item[Time:] seconds.
  \end{description}

  \pause
  
  \begin{goodblock}{Galbraith, Petit and Silva 2017}
    \begin{itemize}
    \item Concept similar to \emph{CSI-FiSh}: exploits \emph{known
        structure of endomorphism ring};
    \item Statistical zero knowledge (under heuristic assumptions);
    \item Based on the generic isogeny walk problem\\
      (requires \emph{special starting curve}, though);
    \item Size/performance comparable to Yoo \textit{et al.} (and possibly
      slower).
    \end{itemize}
  \end{goodblock}
\end{frame}

%%

\begin{frame}{Weil pairing and isogenies}
  \begin{block}{Theorem}
    Let \emph{$ϕ:E→E'$} be an isogeny and \emph{$\hat{ϕ}:E'→E$} its dual. \\
    Let \emph{$e_N$} be the Weil pairing of $E$ and \emph{$e_N'$} that
    of $E'$. %
    Then, for
    \[e_N(P,\hat{ϕ}(Q)) = e_N'(ϕ(P),Q),\]
    for any \emph{$P∈E[N]$} and \emph{$Q∈E'[N]$}.
  \end{block}
  
  \begin{block}{Corollary}
    \[e_N'(ϕ(P),ϕ(Q)) = e_N(P,Q)^{\deg ϕ}.\]
  \end{block}
\end{frame}

%%

\begin{frame}{Refresher: Boneh--Lynn--Shacham (BLS) signatures}
  \begin{block}{}
    \begin{description}
    \item[Setup:]
      \begin{itemize}
      \item Elliptic curve $E/\F_p$, s.t $N|\#E(\F_p)$ for a large prime $N$,
      \item (Weil) pairing $e_N:E[N]×E[N]\to\F_{p^k}$ for some
        small embedding degree $k$,
      \item A decomposition $E[N]=X_1 × X_2$, with $X_1=〈P〉$.
      \item A hash function $H:\{0,1\}^*→X_2$.
      \end{itemize}
    \item[Private key:] $s∈ℤ/Nℤ$.
    \item[Public key:] $sP$.
    \item[Sign:] $m ↦ sH(m)$.
    \item[Verifiy:] $e_N(P,sH(m)) = e_N(sP,H(m))$.
    \end{description}
  \end{block}

  \centering
  \begin{tikzpicture}[ampersand replacement=\&]
    \matrix(m)[matrix of math nodes,row sep=3em,column sep=4em]{
      X_1 × X_2 \& X_1 × X_2\\
      X_1 × X_2 \& \F_{p^k}\\
    };
    \draw[->]
    (m-1-1) edge node[above]{\small$[s]\times 1$} (m-1-2)
    (m-1-1) edge node[left]{\small$1\times [s]$} (m-2-1)
    (m-1-2) edge node[right]{\small$e_N$} (m-2-2)
    (m-2-1) edge node[below]{\small$e_N$} (m-2-2);
  \end{tikzpicture}
\end{frame}

%%

\begin{frame}{US patent 8,250,367 (Broker, Charles and Lauter 2012)}
  \begin{block}{Signatures from isogenies + pairings}
    \begin{itemize}
    \item Replace the secret \emph{$[s]:E→E$} with an isogeny \emph{$ϕ:E→E'$};
    \item Define decompositions
      \begin{align*}
        E[N] = X_1 × X_2, \qquad E'[N] = Y_1 × Y_2,
      \end{align*}
      s.t. \emph{$ϕ(X_1) = Y_1$} and \emph{$ϕ(X_2) = Y_2$};
    \item Define a hash function \emph{$H:\{0,1\}^*→Y_2$}.
    \end{itemize}
  \end{block}
  
  \centering
  \begin{tikzpicture}[ampersand replacement=\&]
    \matrix(m)[matrix of math nodes,row sep=3em,column sep=4em]{
      X_1 × Y_2 \& Y_1 × Y_2\\
      X_1 × X_2 \& \F_{p^k}\\
    };
    \draw[->]
    (m-1-1) edge node[above]{\small$ϕ\times 1$} (m-1-2)
    (m-1-1) edge node[left]{\small$1\times\hat{ϕ}$} (m-2-1)
    (m-1-2) edge node[right]{\small$e_N'$} (m-2-2)
    (m-2-1) edge node[below]{\small$e_N$} (m-2-2);
  \end{tikzpicture}
\end{frame}

%%

\begin{frame}{Pairing proofs: what for?}
  \begin{itemize}
  \item Non-interactive, not post-quantum, not zero knowledge;
  \item Useful for (partially) validating SIDH public keys;
  \item \alert{Succinct:} proof size, verification time independent of
    walk length!
  \end{itemize}

  \begin{block}{Application: Verifiable Delay Functions}
    D., Masson, Petit and Sanso 2019
    (\href{https://ia.cr/2019/166}{eprint:2019/166}):
    \begin{itemize}
    \item Similar to \emph{time-lock puzzles};
    \item No secret: everything is public;
    \item Generating proof takes configurable
      \emph{\textit{sequential} time $T$};
    \item Verifying proof takes time \emph{independent from $T$};
    \item Security assumptions very different and new!
    \item Applications to blockchains: randomness beacons, consensus
      protocols, \dots
    \end{itemize}
  \end{block}
\end{frame}

%%

\begin{frame}{Conclusion}
  \begin{itemize}
  \item \emph{Different} isogeny graphs enable different \emph{styles
      of proofs}, different \emph{security assumptions}.
  \item Post-quantum isogeny signatures are still \emph{far from
      practical}.
  \item \emph{Practical} isogeny signatures do exists (CSI-FiSh); you
    can start using them now if you are an isogeny hippie, but they
    \emph{do not scale}.
  \item Pairing-based proofs are usable, but not interesting for
    signatures: look into \emph{succinctness}, instead!
  \item Tons of open questions on classical and quantum security, on
    security proofs, and on constructions.
  \item Proofs can be \emph{chained} easily: useful for multi-party
    supersingular curve generation (work in progress with J. Burdges).
  \item \emph{The isogenista dream:} a one-pass post-quantum signature
    scheme based on walks in isogeny graphs.
  \end{itemize}
\end{frame}

%%

\begin{frame}
  \centering
  \begin{tikzpicture}
    \begin{scope}[xscale=1.2,black!60]
      \def\crater{7}
      \foreach \i in {1,...,\crater} {
        \draw[fill] (360/\crater*\i:3cm) circle (5pt);
        \draw (360/\crater*\i : 3cm) -- (360/\crater*\i+360/\crater : 3cm);
        \foreach \j in {-1,1} {
          \draw[fill] (360/\crater*\i : 3cm) -- (360/\crater*\i + \j*360/\crater/4 : 4cm) circle (3pt);
          \foreach \k in {-1,0,1} {
            \draw[fill] (360/\crater*\i + \j*360/\crater/4 : 4cm) --
            (360/\crater*\i + + \j*360/\crater/4 + \k*360/\crater/6 : 4.5cm) circle (1pt);
          }
        }
      }
    \end{scope}
    
    \draw (0,1) node{\Huge\bf Thank you};
    \draw (0,-0.6) node{\large\url{https://defeo.lu/}};
    \draw (0,-1.3) node{\large\includegraphics[height=0.9em]{../share/assets/twitter.png}~\href{https://twitter.com/luca_defeo}{@luca\_defeo}};
  \end{tikzpicture}
\end{frame}

\end{document}


% LocalWords:  Isogeny abelian isogenies hyperelliptic supersingular Frobenius
% LocalWords:  isogenous


