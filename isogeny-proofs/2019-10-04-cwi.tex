\documentclass[aspectratio=169]{beamer}

\usepackage[T1]{fontenc}
\usepackage[utf8]{inputenc}
\usepackage[american]{babel}
\usepackage{amsmath,amsthm}
\usepackage{../share/unicode}
\usepackage{tabularx}
\usepackage{tikz}
\usetikzlibrary{matrix,decorations,decorations.text,calc,arrows,snakes,shapes,positioning}

\usepackage[nosfdefault]{comicneue}
\usepackage{sourcesanspro}
\usepackage[amssymb,amsfonts]{concmath}
\usefonttheme[onlymath]{serif}
\usepackage{ulem}

\mode<presentation>{%
  \usetheme{Boadilla}
}
\beamertemplatenavigationsymbolsempty

\renewcommand{\emph}[1]{{\usebeamercolor[fg]{structure}#1}}

\newenvironment<>{goodblock}[1]{%
  \begin{actionenv}#2%
      \def\insertblocktitle{#1}%
      \par%
      \mode<presentation>{%
        \setbeamercolor{block title}{fg=green!60!black,bg=green!50!white}
       \setbeamercolor{block body}{bg=green!20!white}
     }%
      \usebeamertemplate{block begin}}
    {\par\usebeamertemplate{block end}\end{actionenv}}
\newenvironment<>{mehblock}[1]{%
  \begin{actionenv}#2%
      \def\insertblocktitle{#1}%
      \par%
      \mode<presentation>{%
        \setbeamercolor{block title}{fg=yellow!50!black,bg=yellow!50!white}
       \setbeamercolor{block body}{bg=yellow!20!white}
     }%
      \usebeamertemplate{block begin}}
    {\par\usebeamertemplate{block end}\end{actionenv}}
\newenvironment<>{badblock}[1]{%
  \begin{actionenv}#2%
      \def\insertblocktitle{#1}%
      \par%
      \mode<presentation>{%
        \setbeamercolor{block title}{fg=red!60!black,bg=red!50!white}
       \setbeamercolor{block body}{bg=red!20!white}
     }%
      \usebeamertemplate{block begin}}
    {\par\usebeamertemplate{block end}\end{actionenv}}


%\let\footcite\footnote

\newcommand{\F}{\mathbb{F}}
\newcommand{\bl}[1]{\textcolor{blue}{#1}}
\newcommand{\rd}[1]{\textcolor{red}{#1}}
\newcommand{\gn}[1]{\textcolor{green}{#1}}
\newcommand{\myedge}[3]{
  \draw[#3] (360/\crater*#1 : \diam) to[bend right] (360/\crater*#2 : \diam);
}
\newcommand{\from}{\overset{\$}{\leftarrow}}

\title{How to prove a secret isogeny}
\author{Luca De Feo}
\institute{IBM Research Zürich}
\date[Oct 4, 2019, CWI]{October 4, 2019, CWI, Amsterdam\\[1em]
  {\footnotesize based on joint work with}\\
  J. Burdges, S. Galbraith, S. Masson, C. Petit, A. Sanso}

\begin{document}

\frame[plain]{\titlepage}

%%

\begin{frame}{Why prove a secret isogeny?}
  \begin{description}
  \item[Public:] Curves \emph{$E, E'$}
  \item[Secret:] An isogeny walk \emph{$E\to E'$}
  \end{description}

  \begin{block}{Why?}
    \begin{itemize}
    \item For interactive identification;
    \item For signing messages;
    \item For validating public keys (esp. SIDH);
    \item More\dots
    \end{itemize}
  \end{block}

  \begin{block}{Some properties}
    \newcolumntype{C}{>{\centering\arraybackslash}X}
    \begin{tabularx}{\columnwidth}{l *{4}C}
      & \multicolumn{2}{c}{\small Zero knowledge}\\
      & \small Statistical & \small Computational & \small Quantum resistance & \small Succinctness\\
      \hline
      CSIDH & \checkmark & & \checkmark/{\footnotesize sort of} & \\
      SIDH & & \checkmark & \checkmark &\\
      Pairings & & & & \checkmark\\
    \end{tabularx}
  \end{block}
\end{frame}

%%

\begin{frame}{Security assumptions in Isogeny-based Cryptography}
  \begin{goodblock}{Isogeny walk problem}
    \begin{description}
    \item[Input] Two isogenous elliptic curves \emph{$E,E'$} over
      \emph{$\F_q$}.
    \item[Output] A path \emph{$E\to E'$} in an isogeny graph.
    \end{description}
  \end{goodblock}

  \begin{mehblock}{SIDH problem (1)}
    \begin{description}
    \item[Input]
      Elliptic curves \emph{$E,E'$} over \emph{$\F_q$},
      isogenous of degree \bl{$\ell_A^{e_A}$}.
    \item[Output] The unique path \emph{$E\to E'$} of length
      \bl{$e_A$} in the \bl{$\ell_A$}-isogeny graph.
    \end{description}    
  \end{mehblock}
  
  \begin{badblock}{SIDH problem (2)}
    \begin{description}
    \item[Input]
      \begin{itemize}
      \item Elliptic curves \emph{$E,E'$} over \emph{$\F_q$},
        isogenous of degree \bl{$\ell_A^{e_A}$};
      \item The action of the isogeny on $E[\rd{\ell_B^{e_B}}]$.
      \end{itemize}
    \item[Output] The unique path \emph{$E\to E'$} of length
      \bl{$e_A$} in the \bl{$\ell_A$}-isogeny graph.
    \end{description}    
  \end{badblock}
\end{frame}

%%

\begin{frame}{A $\Sigma$-protocol from Diffie--Hellman\footnote{Kids, do
      not try this at home! Use Schnorr!}}
  \begin{columns}
    \begin{column}{0.55\textwidth}
      \begin{itemize}
      \item<1-> A key pair \emph{$(s, g^s)$};
      \item<2-> Commit to a \emph{random element $g^r$};
      \item<3-> Challenge with bit \emph{$b\in\{0,1\}$};
      \item<4-> Respond with \emph{$c = \alert<6->{r - b\cdot s\mod\#G}$};
      \item<5-> Verify that \emph{$g^c(g^s)^b = g^r$}.
      \end{itemize}

      \begin{block}{Zero-knowledge}<6->
        \centering
        Does not leak because:\\
        \alert{$c$ is uniformly distributed} and independent from $s$.
      \end{block}

      \begin{uncoverenv}<7->
        Unlike Schnorr, compatible with\\
        \emph{group action Diffie--Hellman}.
      \end{uncoverenv}
    \end{column}  
    \begin{column}{0.40\textwidth}
      \centering
      \begin{tikzpicture}
        \node (g) at (0,0) {\alt<-6>{$g$}{$E_1$}};
        \node (gs) at (3,0) {\alt<-6>{$g^s$}{$E_s$}};
        \path[->] (g) edge node[auto]{\alt<-6>{$s$}{$g^s$}} (gs);
        \uncover<2->{
          \node (gr) at (1.5,-3) {\alt<-6>{$g^r$}{$E_r$}};
          \path[->] (g) edge node[auto,swap]{\alt<-6>{$r$}{$g^r$}} (gr);
        }
        \uncover<4->{
          \path[dashed,->] (gs) edge node[auto]{\alt<-6>{$r-s$}{$g^{r-s}$}} (gr);
        }
      \end{tikzpicture}
    \end{column}  
  \end{columns}
\end{frame}

%%

\begin{frame}{The trouble with groups of unknown structure}
  \begin{columns}
    \begin{column}{0.5\textwidth}
      In CSIDH secrets look like:
      $g^{\vec{s}} = \bl{g_2^{s_2}}\rd{g_3^{s_3}}\gn{g_5^{s_5}}\cdots$
      \begin{itemize}
      \item the elements $g_i$ are fixed,
      \item the secret is the exponent vector
        \emph{$\vec{s}=(s_2,s_3,\dots)\in[-B,B]^n$},
      \item secrets must be sampled in a box \emph{$[-B,B]^n$} ``large
        enough''\dots
      \end{itemize}

      \begin{block}{The leakage}<2->%
        With \emph{$\vec{s},\vec{r}\from[-B,B]^n$}, the distribution of
        \emph{$\vec{r}-\vec{s}$} \alert{depends on the long term secret $\vec{s}$!}
      \end{block}
    \end{column}
    \begin{column}{0.45\textwidth}
      \centering
      \begin{tikzpicture}[scale=0.3]
        \draw (-2,2) node{$+B$} (-2,-2) node{$-B$};
        \draw (0,0) -- (18,0);
        \draw[dotted] (0,2) -- (18,2) (0,-2) -- (18,-2);

        \draw (8,-4) node {\Large$-$};
        
        \draw (-2,-6) node{$+B$} (-2,-10) node{$-B$};
        \draw (0,-8) -- (18,-8);
        \draw[dotted] (0,-6) -- (18,-6) (0,-10) -- (18,-10);

        \uncover<2->{
          \draw (8,-12) node {\Large$=$};
        
          \draw (-2,-15) node{$+B$} (-2,-19) node{$-B$};
          \draw (0,-17) -- (18,-17);
          \draw[dotted] (0,-15) -- (18,-15) (0,-19) -- (18,-19);
        }
      
        \foreach \i in {0,...,18} {
          \pgfmathparse{round(2.2*sin(140*\i))}
          \let\s\pgfmathresult
          \pgfmathparse{round(2.2*cos(125*\i))}
          \let\r\pgfmathresult
          \draw[very thick] (\i,0) -- (\i,\r);
          \draw[very thick] (\i,-8) -- (\i,-8+\s);
          \uncover<2->{
            \draw[very thick] (\i,-17) -- (\i,-17+\r-\s);
          }
        }
      \end{tikzpicture}
    \end{column}
  \end{columns}
\end{frame}

%%

\begin{frame}{The two fixes}
  \begin{block}{Compute the group structure and stop whining}
    \emph{CSI-FiSh}: Beullens, Kleinjung and Vercauteren 2019
    \begin{itemize}
    \item Already suggested by Couveignes (1996) and Stolbunov (2006).
    \item Computationally intensive (\emph{subexponential parameter generation}).
    \item Decent parameters, e.g.: \emph{263 bytes, 390 ms, @NIST-1.} 
    \item[--] Technically not post-quantum (signing requires solving ApproxCVP).
   \end{itemize}
  \end{block}

  \begin{block}{Do like the lattice people}
    \emph{SeaSign}: D. and Galbraith 2019
    \begin{itemize}
    \item Use \emph{Fiat--Shamir with aborts} (Lyubashevsky 2009).
    \item[--] Huge increase in signature size and time.
    \item Compromise signature size/time with public key size (still
      slow).
    \end{itemize}
  \end{block}
\end{frame}

%%

\begin{frame}{Rejection sampling}
  \begin{columns}
    \begin{column}{0.5\textwidth}
      \begin{itemize}
      \item Sample \emph{long term secret $\vec{s}$} in the usual box \emph{$[-B,B]^n$},
      \item Sample \emph{ephemeral $\vec{r}$} in a larger box
        \emph{$[-(\delta+1)B,(\delta+1)B]^n$},
      \item Throw away \emph{$\vec{r}-\vec{s}$} if it is out of the box
        \emph{$[-\delta B,\delta B]^n$}.
      \end{itemize}

      \begin{block}{Zero-knowledge}
        \emph{Theorem:} $\vec{r}-\vec{s}$ is uniformly distributed in
        \emph{$[-\delta B,\delta B]^n$}.
      \end{block}

      \emph{Problem:} set $\delta$ so that rejection probability is
      low.
    \end{column}
    \begin{column}{0.5\textwidth}
      \centering
      \begin{tikzpicture}[scale=0.3]
        \draw (-3,3) node{$+(\delta+1)B$} (-3,-3) node{$-(\delta+1)B$};
        \draw (0,0) -- (18,0);
        \draw[dotted] (0,3) -- (18,3) (0,-3) -- (18,-3);

        \draw (8,-5) node {\Large$-$};
        
        \draw (-2,-7.5) node{$+B$} (-2,-8.5) node{$-B$};
        \draw (0,-8) -- (18,-8);
        \draw[dotted] (0,-7.5) -- (18,-7.5) (0,-8.5) -- (18,-8.5);

        \draw (8,-12) node {\Large$=$};
        
        \draw (-2,-14.5) node{$+\delta B$} (-2,-19.5) node{$-\delta B$};
        \draw (0,-17) -- (18,-17);
        \draw[dotted] (0,-14.5) -- (18,-14.5) (0,-19.5) -- (18,-19.5);
      
        \foreach \i in {0,...,18} {
          \pgfmathparse{round(60*random()-30)/10}
          \let\r\pgfmathresult
          \pgfmathparse{round(2*random()-1)/2}
          \let\s\pgfmathresult
          \draw[very thick] (\i,0) -- (\i,\r);
          \draw[very thick] (\i,-8) -- (\i,-8+\s);
          \draw[very thick] (\i,-17) -- (\i,-17+\r-\s);
        }
      \end{tikzpicture}
    \end{column}
  \end{columns}
\end{frame}

%%

\begin{frame}{Performance}
  \begin{itemize}
  \item For $\lambda$-bit security, protocol must be \emph{repeated
      $\lambda$ times} in parallel;
  \item \emph{$\delta = \lambda n$} for a rejection probability
    \emph{$\le 1/3$};
  \item Signature size \emph{$\approx \lambda n$ coefficients $\in[-\delta B,\delta B]$};
  \item Sign/verify time linear in \emph{$\lVert \vec{r}-\vec{s}\rVert_\infty\approx \lambda^2n^2B$}.
  \end{itemize}

  \begin{block}{CSIDH instantiation (NIST-1)}
    \begin{description}
    \item[Parameters:] $\lambda=128, n=74, B=5$;
    \item[PK size:] 64 B
    \item[SK size:] 32 B
    \item[Signature:] 20 KiB
    \item[Verify time:] \alert{10 hours}
    \item[Sign time:] $3\times$ verify
    \end{description}
  \end{block}

\end{frame}

%%

\begin{frame}{Key/signature size compromise}
  \begin{columns}
    \begin{column}{0.5\textwidth}
      \begin{block}{}
        \begin{itemize}
        \item One key pair \emph{$(\vec{s},E_s)$};
        \item Challenge \emph{$b\in\{0,1\}$};
        \item Reveal \emph{$\vec{r}-b\vec{s}$};
        \item[$\to$] \emph{$\lambda$} iterations;
        \item[$\to$]<3-> Sample \emph{$r\from[-\lambda nB,\lambda nB]$}.
        \end{itemize}
      \end{block}

      \begin{block}{Compromise: \alert{$\mathbf{t}$}-bit challenges}<2->
        \begin{itemize}
        \item \alert{$\mathbf{2^t}$} key pairs \emph{$(\vec{s_i},E_i)$};
        \item Challenge \emph{$b\in\{0,2^t\}$};
        \item Reveal \emph{$\vec{r}-\vec{s_b}$};
        \item[$\to$] \emph{$\lambda\alert{\mathbf{/t}}$} iterations;
        \item[$\to$]<4-> Sample
          \emph{$r\from[-\lambda nB\alert{\mathbf{/t}},\lambda
            nB\alert{\mathbf{/t}}]$}.
        \end{itemize}
      \end{block}
    \end{column}  
    \begin{column}{0.47\textwidth}
      \centering
      \begin{tikzpicture}
        \node (E) at (0:0) {$E_1$};
        \uncover<1>{
          \node (Es) at (45:3) {$E_s$};
          \path[->] (E) edge node[above]{\small$\vec{s}$} (Es);
        }
        \uncover<2->{
          \foreach \i in {1,...,4} {
            \node (E\i) at (-30*\i + 75 : 3) {$E_\i$};
            \path[->] (E) edge node[above]{\small$\vec{s_\i}$} (E\i);
          }
        }
        \node (Er) at (-50:5) {$E_r$};
        \path[dashed,->] (E) edge[bend right] node[left]{\small$\vec{r}$} (Er);
        \uncover<2->{
          \path[dashed,->] (E2) edge[bend left] node[right]{\small$\vec{r}-\vec{s_2}$} (Er);
        }
      \end{tikzpicture}
    \end{column}  
  \end{columns}
\end{frame}

%%

\begin{frame}{Public key compression}
  \begin{center}
    \begin{tikzpicture}
      \node (E) at (0,0) {$E_1$};
      \foreach \i in {1,...,4} {
        \node (E\i) at (3,2.5-\i) {$E_\i$};
        \path[->] (E) edge (E\i);
        \uncover<2->{
          \node (H0\i) at (5,2.5-\i) {$H(E_\i)$};
          \path[->] (E\i) edge (H0\i);
        }
      }
      \uncover<2->{
        \foreach \l in {1,2} {
          \foreach \i in {1,...,2^(2-\l)} {
            \pgfmathparse{int(\l-1)}
            \let\pl\pgfmathresult
            \pgfmathparse{int(2*\i-1)}
            \let\ca\pgfmathresult
            \pgfmathparse{int(2*\i)}
            \let\cb\pgfmathresult
            \node (H\l\i) at (5 + 3*\l, 2 + 0.5*2^\l - \i*2^\l) {$H(\bullet,\bullet)$};
            \path[->]
            (H\pl\ca) edge (H\l\i)
            (H\pl\cb) edge (H\l\i);
          }
        }
        \node at (12.3,0) {$=pk$};
        \uncover<3->{
          \path[->,blue,very thick]
          (E) edge (E2)
          (H01) edge (H11)
          (H12) edge (H21);
        }
      }
    \end{tikzpicture}
  \end{center}
  \medskip
  \begin{itemize}
  \item<2-> Construct Merkle tree on top of public keys, \emph{root is the new public key};
  \item<3-> Include Merkle proof in the signature.
  \end{itemize}
\end{frame}

%%

\begin{frame}{SeaSign Performance (NIST-1)}
  \centering
  \begin{tabular}{l  c  c  c }
    & \bf $t=1$ bit challenges
    & \bf $t=16$ bits challenges
    & \bf PK compression \\
    \hline
    Sig size
    & \alert{20 KiB} & \gn{978 B} & 3136 B\\
    PK size
    & 64 B & \alert{4 MiB} & 32 B\\
    SK size
    & 32 B & 16 B & \alert{1 MiB} \\
    Est. keygen time
    & 30 ms & \alert{30 mins} & \alert{30 mins} \\
    Est. sign time
    & \alert{30 hours} & 6 mins & 6 mins \\
    Est. verify time
    & \alert{10 hours} & 2 mins & 2 mins \\
    \hline
    Asymptotic sig size
    & $O(\lambda^2\log(\lambda))$ & $O(\lambda t\log(\lambda))$ & $O(\lambda^2 t)$\\[2em]
    \multicolumn{4}{c}{\bf Speed/size compromises by Decru, Panny and Vercauteren 2019}\\
    \hline
    {Sig size}
    & 36 KiB & 2 KiB & ---\\
    {Est. sign time}
    & 30 mins & 80 s & ---\\
    {Est. verify time}
    & 20 mins & 20 s & ---
  \end{tabular}
\end{frame}

%%

\begin{frame}{A $Σ$-protocol for SIDH}
  \begin{columns}
    \begin{column}{0.6\textwidth}
      \begin{tikzpicture}
        \large
        \node[matrix of nodes, ampersand replacement=\&, column sep=3cm, row sep=1.3cm] (diagram) {
          |(E)| $E$ \& |(Es)| $E/〈\bl{S}〉$ \\
          |(Ep)| {\uncover<2->{$E/〈\rd{P}〉$}} \& |(Eps)| {\uncover<2->{$E/〈\rd{P,\bl{S}}〉$}}\\
        };
        \path[blue] (E) edge node[auto] {$\phi$} (Es);
        \uncover<2->{\path[blue] (Ep) edge node[auto,swap] {\alt<5>{$\phi'$}{\phantom{$\phi'$}?}} (Eps);}
        \uncover<2->{\path[red] (E) edge node[auto,swap] {\alt<3,6>{$\psi$}{?}} (Ep);}
        \uncover<2->{\path[red] (Eps) edge node[auto,swap] {\alt<4,6>{$\psi'$}{?}} (Es);}
      \end{tikzpicture}
    \end{column}
    \begin{column}{0.35\textwidth}
      \begin{mehblock}{$\frac{1}{3}$-soundness}
        Secret \bl{$\phi$} of degree \bl{$\ell_A^{e_A}$}.
      \end{mehblock}
    \end{column}
  \end{columns}

  \begin{enumerate}
  \item<2-> Choose a random point \rd{$P\in E[\ell_B^{e_B}]$}, compute the diagram;
  \item<2-> Publish the curves $E/〈\rd{P}〉$ and $E/〈\rd{P,\bl{S}}〉$;
  \item<3-> The verifier challenges to reveal \emph{one out of the 3}
    sides
    \begin{itemize}
    \item<3-> Isogenies \rd{$\psi,\psi'$} (degree \rd{$\ell_B^{e_B}$})
      unrelated to secret;
    \item<5-> Isogeny \bl{$\phi'$} conjectured to not reveal useful
      information on \bl{$\phi$}.
    \end{itemize}
  \end{enumerate}

  \begin{badblock}{Improving to $\frac{1}{2}$-soundness}<6->
    \begin{itemize}
    \item Reveal \rd{$\psi,\psi'$} simultaneously;
    \item Reveals action of \bl{$\phi$} on $E[\rd{\ell_B^{e_B}}]\quad⇒\quad$
      Stronger security assumption.
    \end{itemize}
  \end{badblock}
\end{frame}

%%

\begin{frame}{SIDH signature performance (NIST-1)}
  According to Yoo, Azarderakhsh, Jalali, Jao and Vladimir Soukharev
  2017:
  \begin{description}
  \item[Size:] $\approx 100 KB$,
  \item[Time:] seconds.
  \end{description}

  \pause
  
  \begin{goodblock}{Galbraith, Petit and Silva 2017}
    \begin{itemize}
    \item Concept similar to \emph{CSI-FiSh}: exploits \emph{known
        structure of endomorphism ring};
    \item Statistical zero knowledge (under heuristic assumptions);
    \item Based on the generic isogeny walk problem\\
      (requires \emph{special starting curve}, though);
    \item Size/performance comparable to Yoo \textit{et al.} (and possibly
      slower).
    \end{itemize}
  \end{goodblock}
\end{frame}

%%

\begin{frame}{Weil pairing and isogenies}
  \begin{block}{Theorem}
    Let \emph{$ϕ:E→E'$} be an isogeny and \emph{$\hat{ϕ}:E'→E$} its dual. \\
    Let \emph{$e_N$} be the Weil pairing of $E$ and \emph{$e_N'$} that
    of $E'$. %
    Then, for
    \[e_N(P,\hat{ϕ}(Q)) = e_N'(ϕ(P),Q),\]
    for any \emph{$P∈E[N]$} and \emph{$Q∈E'[N]$}.
  \end{block}
  
  \begin{block}{Corollary}
    \[e_N'(ϕ(P),ϕ(Q)) = e_N(P,Q)^{\deg ϕ}.\]
  \end{block}
\end{frame}

%%

\begin{frame}{Refresher: Boneh--Lynn--Shacham (BLS) signatures}
  \begin{block}{}
    \begin{description}
    \item[Setup:]
      \begin{itemize}
      \item Elliptic curve $E/\F_p$, s.t $N|\#E(\F_p)$ for a large prime $N$,
      \item (Weil) pairing $e_N:E[N]×E[N]\to\F_{p^k}$ for some
        small embedding degree $k$,
      \item A decomposition $E[N]=X_1 × X_2$, with $X_1=〈P〉$.
      \item A hash function $H:\{0,1\}^*→X_2$.
      \end{itemize}
    \item[Private key:] $s∈ℤ/Nℤ$.
    \item[Public key:] $sP$.
    \item[Sign:] $m ↦ sH(m)$.
    \item[Verifiy:] $e_N(P,sH(m)) = e_N(sP,H(m))$.
    \end{description}
  \end{block}

  \centering
  \begin{tikzpicture}[ampersand replacement=\&]
    \matrix(m)[matrix of math nodes,row sep=3em,column sep=4em]{
      X_1 × X_2 \& X_1 × X_2\\
      X_1 × X_2 \& \F_{p^k}\\
    };
    \draw[->]
    (m-1-1) edge node[above]{\small$[s]\times 1$} (m-1-2)
    (m-1-1) edge node[left]{\small$1\times [s]$} (m-2-1)
    (m-1-2) edge node[right]{\small$e_N$} (m-2-2)
    (m-2-1) edge node[below]{\small$e_N$} (m-2-2);
  \end{tikzpicture}
\end{frame}

%%

\begin{frame}{US patent 8,250,367 (Broker, Charles and Lauter 2012)}
  \begin{block}{Signatures from isogenies + pairings}
    \begin{itemize}
    \item Replace the secret \emph{$[s]:E→E$} with an isogeny \emph{$ϕ:E→E'$};
    \item Define decompositions
      \begin{align*}
        E[N] = X_1 × X_2, \qquad E'[N] = Y_1 × Y_2,
      \end{align*}
      s.t. \emph{$ϕ(X_1) = Y_1$} and \emph{$ϕ(X_2) = Y_2$};
    \item Define a hash function \emph{$H:\{0,1\}^*→Y_2$}.
    \end{itemize}
  \end{block}
  
  \centering
  \begin{tikzpicture}[ampersand replacement=\&]
    \matrix(m)[matrix of math nodes,row sep=3em,column sep=4em]{
      X_1 × Y_2 \& Y_1 × Y_2\\
      X_1 × X_2 \& \F_{p^k}\\
    };
    \draw[->]
    (m-1-1) edge node[above]{\small$ϕ\times 1$} (m-1-2)
    (m-1-1) edge node[left]{\small$1\times\hat{ϕ}$} (m-2-1)
    (m-1-2) edge node[right]{\small$e_N'$} (m-2-2)
    (m-2-1) edge node[below]{\small$e_N$} (m-2-2);
  \end{tikzpicture}
\end{frame}

%%

\begin{frame}{Pairing proofs: what for?}
  \begin{itemize}
  \item<1-> Non-interactive, not post-quantum, not zero knowledge;
  \item<2-> Useful for (partially) validating SIDH public keys;
  \item<3-> \alert{Succinct:} proof size, verification time independent
    of walk length!
  \end{itemize}
\end{frame}

%%

{
  \setbeamercolor{background canvas}{bg=black}
  \begin{frame}[plain]
    \begin{tikzpicture}[remember picture,overlay]
      \node(pic)[at=(current page.center)] {
        \includegraphics[width=\paperwidth]{hodl.jpg}
      };
    \end{tikzpicture}
  \end{frame}
}

%%

\begin{frame}{Distributed lottery}
  Participants \textbf{A, B, \dots, Z} want to agree on a random
  winning ticket.

  \begin{block}{Flawed protocol}
    \begin{itemize}
    \item Each participant \emph{$x$} broadcasts a random string
      \emph{$s_x$};
    \item Winning ticket is \emph{$H(s_A, \dots, s_Z)$}.
    \end{itemize}
  \end{block}

  \pause

  \begin{block}{Fixes}
    \begin{itemize}
    \item Make the hash function \textbf{sloooooooooooooooooooooooooooow};
    \item<3> Make it possible to verify \emph{$w = H(s_A, \dots, s_Z)$} \textbf{fast}.
    \end{itemize}
  \end{block}
\end{frame}

%%

\begin{frame}{Verifiable Delay Functions (Boneh, Bonneau, Bünz, Fisch 2018)}
  \begin{block}{Wanted}
    Function (family) \emph{$f:X\to Y$} s.t.:

    \begin{itemize}
    \item Evaluating $f(x)$ takes \emph{long time}:
      \begin{itemize}
      \item \emph{uniformly} long time,
      \item on almost all random inputs $x$,
      \item even after having seen many values of $f(x')$,
      \item even given \emph{massive number of processors};
      \end{itemize}
    \item Verifying $y=f(x)$ is \emph{efficient}:
      \begin{itemize}
      \item ideally, exponential separation between evaluation and
        verification.
      \end{itemize}
    \end{itemize}
  \end{block}

  \centering
  \pause\vfill
  \textbf{Exercise}
  \pause\vfill
  Think of a function you like with these properties
  \pause\vfill
  Got it?
  \pause\vfill
  \textbf{You're probably wrong!}
\end{frame}

%%

\begin{frame}{Sequentiality}
  Ideal functionality:

  \[y = f(x) = \underbrace{H(H(\cdots(H(x))))}_{T \text{ times}}\]

  \begin{itemize}
  \item \emph{Sequential} assuming hash output ``unpredictability'',
  \item but how do you verify?
  \end{itemize}
\end{frame}

%%

\begin{frame}{VDFs from groups of unknown order}
  \begin{block}{Setup}
    A group of \emph{unknown order}, e.g.:
    \begin{itemize}
    \item \emph{$\mathbb{Z}/N\mathbb{Z}$} with $N=pq$ an RSA modulus, $p,q$ \emph{unknown}\\
      (e.g., generated by some trusted authority),
    \item \emph{Class group} of imaginary quadratic order.
    \end{itemize}
  \end{block}

  \begin{block}{Evaluation}
    \begin{columns}
      \begin{column}{0.4\textwidth}
        With \emph{delay parameter} $T$:

        \vspace{1em}
        Conjecturally, fastest algorithm is repeated squaring.
      \end{column}
      \begin{column}{0.3\textwidth}
        \begin{align*}
          f:G &\longrightarrow G\\
          x &\longmapsto x^{2^T}
        \end{align*}
      \end{column}
    \end{columns}
  \end{block}

  \begin{block}{Verification (Wesolowski 2019, Pietrzak 2019)}
    \centering\pause\bf\Large
    Aha!
  \end{block}
\end{frame}

%%

\begin{frame}{Isogeny VDF}
  Assume $\deg\phi = 2^T$
  \vfill
  \[e_N(\phi(P),\phi(Q)) = e_N(P,Q)^{2^T}\]
  \vfill
  \begin{description}
  \item[Right side:] known group structure:
    \emph{$2^T \;\;\to\;\; 2^T\bmod p^k-1$};
  \item[Left side:] can evaluate $\phi$ in \emph{less than $T$ steps}?
  \end{description}
\end{frame}

%%

\begin{frame}{Isogeny VDF ($\mathbb{F}_p$-version)}
  \begin{block}{\uncover<2>{Trusted} Setup}
    \begin{itemize}
    \item Pairing friendly \emph{supersingular} curve $E/\mathbb{F}_p$\\
      \uncover<2>{\textbf{with unknown endomorphism ring!!!}}
    \item Isogeny \emph{$\phi:E\to E'$} of degree \emph{$2^T$},
    \item Point \emph{$P\in E[(N,\pi-1)]$}, image \emph{$\phi(P)$}.
    \end{itemize}
  \end{block}

  \begin{block}{Evaluation}
    \begin{description}
    \item[Input:] random \emph{$Q\in E'[(N,\pi+1)]$},
    \item[Output:] \emph{$\hat\phi(Q)$}.
    \end{description}
  \end{block}

  \begin{block}{Verification}
    \large
    \[e_N(P,\hat\phi(Q)) \quad\overset{?}{=}\quad e_N(\phi(P),Q).\]
  \end{block}
\end{frame}

%%

\begin{frame}{Sequentiality?}
  \large\centering
  \hspace{4em} Wesolowski, Pietrzak: \hfill $x \longmapsto x^2$ \hspace{4em}

  \vfill
  \hspace{4em} Isogenies: \hfill $\displaystyle x \longmapsto x\frac{x\alpha_i-1}{x-\alpha_i}$ \hspace{4em}

  \vfill
  No speedup? Even with unlimited parallelism? Really?
  
  \medskip
  See: Bernstein, Sorenson.\\
  \emph{Modular exponentiation via the explicit Chinese remainder
    theorem}.
\end{frame}

%%

\begin{frame}{Conclusion}
  \begin{itemize}
  \item \emph{Different} isogeny graphs enable different \emph{styles
      of proofs}, different \emph{security assumptions}.
  \item Post-quantum isogeny signatures are still \emph{far from
      practical}.
  \item \emph{Practical} isogeny signatures do exists (CSI-FiSh); you
    can start using them now if you are an isogeny hippie, but they
    \emph{do not scale}.
  \item Pairing-based proofs are usable, but not interesting for
    signatures: look into \emph{succinctness}, instead!
  \item Proofs can be \emph{chained} easily: useful for multi-party
    supersingular curve generation (work in progress with J. Burdges).
  \end{itemize}
\end{frame}

%%

\begin{frame}{Some open questions}
  \begin{itemize}
  \item Classical and quantum security of the various assumptions:
    \begin{itemize}
    \item Is the SIDH assumption \emph{qualitatively} than the CSIDH
      one?
    \item What's the exact quantum security of CSIDH?
    \item Is starting from a special curve a weakness?
    \end{itemize}
  \item \emph{Hash into the supersingular set:} can you generate
    random supersingular curves without learning their endomorphism
    ring?
  \item \emph{Ordinary pairing friendly curves with large graphs:} all
    known constructions of pairing-friendly curves produce small class
    groups. Large class groups would provide an ideal setting for
    VDFs.
  \item \emph{The isogenista dream:} a one-pass, post-quantum,
    efficient signature scheme based on walks in isogeny graphs.
  \end{itemize}
\end{frame}

%%

\begin{frame}
  \centering
  \begin{tikzpicture}[remember picture,overlay]
    \begin{scope}[xscale=1.7,opacity=0.8]
      \def\crater{12}
      \def\jumpa{-8}
      \def\jumpb{9}
      \def\diam{5cm}

      \foreach \i in {1,...,\crater} {
        \draw[blue] (360/\crater*\i : \diam) to[bend right] (360/\crater*\i+360/\crater : \diam);
        \draw[red] (360/\crater*\i : \diam) to[bend right] (360/\crater*\i+\jumpa*360/\crater : \diam);
        \draw[green] (360/\crater*\i : \diam) to[bend right=50] (360/\crater*\i+\jumpb*360/\crater : \diam);
      }
    \end{scope}
    
    \draw (0,1) node{\Huge\bf Thank you};
    \draw (0,-0.6) node{\large\url{https://defeo.lu/}};
    \draw (0,-1.3) node{\large\includegraphics[height=0.9em]{twitter.png}~\href{https://twitter.com/luca_defeo}{@luca\_defeo}};
  \end{tikzpicture}
\end{frame}

\end{document}


% LocalWords:  Isogeny abelian isogenies hyperelliptic supersingular Frobenius
% LocalWords:  isogenous
